\documentclass[12pt]{article}
\usepackage{fullpage}
\usepackage{alltt}
\usepackage[dvips]{graphics}
%*************************************************************%
%
%    Ensemble, 2_00
%    Copyright 2004 Cornell University, Hebrew University
%           IBM Israel Science and Technology
%    All rights reserved.
%
%    See ensemble/doc/license.txt for further information.
%
%*************************************************************%
\newcommand {\ensemble}		{Ensemble}
\newcommand {\horus}		{Horus}
\newcommand {\caml}		{Objective Caml}
\newcommand {\chk}		{$\surd$}

\newcommand{\hide}[1]           {~}
\newcommand{\note}[1]           {{\bf [#1]}}
\newcommand{\todo}[1]           {{\note{TODO: #1}}}
\newcommand{\mlval}[1]          {{\bf {#1}}}
\newcommand{\cval}[1]           {{\tt {#1}}}

\newcommand {\incgraphics}[1]   {\includegraphics{fig/#1}}

\newcommand {\sourcecode}[1]    {{\bf {#1}}}
%\newcommand {\sourcelayer}[1]  {{\htmladdnormallink{#1}{../../layers/#1}}}
%\newcommand {\sourceutil}[1]   {{\htmladdnormallink{#1}{../../util/#1}}}
\newcommand {\sourcelayer}[1]   {\sourcecode{layers/#1}}
\newcommand {\sourcedemo}[1]    {\sourcecode{demo/#1}}
\newcommand {\sourceutil}[1]    {\sourcecode{util/#1}}
\newcommand {\sourcetrans}[1]   {\sourcecode{trans/#1}}
\newcommand {\sourcetype}[1]    {\sourcecode{type/#1}}
\newcommand {\sourceappl}[1]    {\sourcecode{appl/#1}}
\newcommand {\sourcece}[1]      {\sourcecode{ce/#1}}
\newcommand {\sourcehot}[1]     {\sourcecode{hot/#1}}

\newcommand {\ECast}[0]         {\mlval{ECast}}
\newcommand {\ESend}[0]         {\mlval{ESend}}
\newcommand {\ESubCast}[0]      {\mlval{ESubCast}}
\newcommand {\ECastUnrel}[0]    {\mlval{ECastUnrel}}
\newcommand {\ESendUnrel}[0]    {\mlval{ESendUnrel}}
\newcommand {\EMergeRequest}[0] {\mlval{EMergeRequest}}
\newcommand {\EMergeGranted}[0] {\mlval{EMergeGranted}}
\newcommand {\EOrphan}[0]       {\mlval{EOrphan}}
\newcommand {\EAccount}[0]      {\mlval{EAccount}}
\newcommand {\EAsync}[0]        {\mlval{EAsync}}
\newcommand {\EBlock}[0]        {\mlval{EBlock}}
\newcommand {\EBlockOk}[0]      {\mlval{EBlockOk}}
\newcommand {\EDump}[0]         {\mlval{EDump}}
\newcommand {\EElect}[0]        {\mlval{EElect}}
\newcommand {\EExit}[0]         {\mlval{EExit}}
\newcommand {\EFail}[0]         {\mlval{EFail}}
\newcommand {\EGossipExt}[0]     {\mlval{EGossipExt}}
\newcommand {\EGossipExtDir}[0]    {\mlval{EGossipExtDir}}
\newcommand {\EInit}[0]     {\mlval{EInit}}
\newcommand {\ELeave}[0]     {\mlval{ELeave}}
\newcommand {\ELostMessage}[0]     {\mlval{ELostMessage}}
\newcommand {\EMergeDenied}[0]     {\mlval{EMergeDenied}}
\newcommand {\EMergeFailed}[0]     {\mlval{EMergeFailed}}
\newcommand {\EMigrate}[0]         {\mlval{EMigrate}}
\newcommand {\EPresent}[0]         {\mlval{EPresent}}
\newcommand {\EPrompt}[0]          {\mlval{EPrompt}}
\newcommand {\EProtocol}[0]        {\mlval{EProtocol}}
\newcommand {\ERekey}[0]           {\mlval{ERekey}}
\newcommand {\ERekeyPrcl}[0]       {\mlval{ERekeyPrcl}}
\newcommand {\EStable}[0]          {\mlval{EStable}}
\newcommand {\EStableReq}[0]       {\mlval{EStableReq}}
\newcommand {\ESuspect}[0]         {\mlval{ESuspect}}
\newcommand {\ESystemError}[0]     {\mlval{ESystemError}}
\newcommand {\ETimer}[0]           {\mlval{ETimer}}
\newcommand {\EView}[0]            {\mlval{EView}}
\newcommand {\EXferDone}[0]        {\mlval{EXferDone}}
\newcommand {\ESyncInfo}[0]        {\mlval{ESyncInfo}}
\newcommand {\ESecureMsg}[0]       {\mlval{ESecureMsg}}
\newcommand {\EChannelList}[0]     {\mlval{EChannelList}}
\newcommand {\EFlowBlock}[0]       {\mlval{EFlowBlock}}
\newcommand {\EAuth}[0]            {\mlval{EAuth}}
\newcommand {\ESecChannelList}[0]  {\mlval{ESecChannelList}}
\newcommand {\ERekeyCleanup}[0]    {\mlval{ERekeyCleanup}}
\newcommand {\ERekeyCommit}[0]     {\mlval{ERekeyCommit}}

\newcommand {\Dn}[1]               {\mlval{Dn(E{#1})}}
\newcommand {\Up}[1]               {\mlval{Up(E{#1})}}

\newlength{\figurewidth}
\newsavebox{\figurebox}
\newenvironment{codebox}{
\figurewidth\hsize
\addtolength{\figurewidth}{-4\fboxsep}
\addtolength{\figurewidth}{-4\fboxrule}

\begin{alltt}
\sbox{\figurebox}\bgroup
\begin{minipage}{\figurewidth}
}{
\end{minipage}
\egroup
\fbox{\usebox{\figurebox}}
\end{alltt}
}

% Ohad.
% A macro for putting scaled figures in boxes.
\newcommand{\putfigfbox}[2]     {\fbox { \scalebox{#1}{\includegraphics{#2}} } }




\title{\ensemble\ Reference Manual}
\author{Mark Hayden (hayden@pa.dec.com)
\thanks{Thanks to Takako Hickey, Roy Friedman, Robbert van Renesse,
Zhen Xiao, and Ohad Rodeh for descriptions of their contributions.} \\
\small{Copyright \copyright\ 1997 Cornell University}}

\begin{document}
\maketitle

\begin{abstract}
\ensemble\ is a reliable group communication toolkit implemented in the
\caml\ programming language.  The purposes of this implementation are:
\begin{itemize}
\item 
to provide a concise and clear ``reference'' implementation of the \ensemble\
architecture and protocols
\item 
to abstract protocol layer implementations as far as possible from the
runtime system
\item
to support the application of formal methods to real implementations of
distributed communication protocols
\item 
to provide a flexible platform for ease of experimentation
\end{itemize}
Throughout, we attempt to follow a design that supports a simple
compilation of the protocols to C.
\end{abstract}

\newpage
\tableofcontents
\newpage

\section{Introduction}

This document is attempting to serve several goals.  It is intended to be:
\begin{itemize}
\item
A description/motivation of the \ensemble\ architecture.
\item
An aid for learning about the \ensemble\ protocol layers and their workings.
\item
An informal specification of individual protocol layers and common
compositions of the layers.
\item
Documentation of possible alternate designs/implementations of the system.
\item
A depository of specification and verification information developed for
protocol layers.
\item
A depository of descriptions of potential projects.
\item
Source for combined hypertext code/text documentation of \ensemble.
\end{itemize}

\paragraph{Documentation TODO list:}
\begin{itemize}
\item transports
\item mark all ML values with mlval\{\}
\item add more detail in the event fields section
\item add ``common problems'' section
\begin{itemize}
\item layers acknowledging messages
\item \UpInit\ vs. \UpView
\end{itemize}
\item layer programming tutorial
\end{itemize}

\newpage
\part{The \ensemble\ Architecture}
\section{Identifiers}

\ensemble\ uses a variety of identifiers for a variety of different purposes.  Here
we summarize the important ones and describe what they are used for.  Their type
definitions can be found in the \sourcecode{type} directory in the file corresponding
to the name.  Look in \sourcetype{README} for an up to date listing of these files.
Most of the different identifiers support a similar interface for a variety of
operations.  Several of the identifiers are opaque, in the sense that the interface
hides the actual structure of the identifier.  All identifiers have a
\mlval{string\_of\_id} function defined which gives a human-readable description of
their contents.

\paragraph{Changes from \horus}
\begin{itemize}
\item
None of the identifiers are defined as fixed length sequences of bytes.  This is
something that remains to be done.
\item
\horus\ EID's have been split into endpoint and group identifiers.
\item
We have removed addressing information entirely from endpoint and group identifiers.
\item
Entity identifiers have been eliminated.  Connection, stack, and protocol identifiers
have been added to support a variety of features new to \ensemble.
\end{itemize}

\subsection{Endpoint Identifiers}
Endpoint identifiers are unique names for communication endpoints.  A single
process can create any number of local endpoint identifiers, each of which is
guaranteed to be unique (within some limits).  A process can have multiple
endpoints in a single group.  An endpoint can be a member of multiple groups.
However, the endpoints in a group must be distinct.

\subsection{Group Identifiers}
Group identifiers serve as unique names of communication groups.  They do not contain
addressing information.  The exception to this rule is that groups communicating via
Deering multicast choose a random multicast address by taking a hash of the group
address.  Processes can create any number of local group identifiers, each of which
is guaranteed to be unique (within some limits).

\subsection{View Identifiers}
View identifiers are unique identifiers of group views.  Whenever communication
protocols proceed through a view change, the resulting view is given a new view
identifier.  These are made unique by pairing the coordinator of the group with a
logical timestamp that is advanced whenever a view change happens.  Although two
partitions of a group may share the same time stamp, they will have different
coordinators.

\subsection{Connection Identifiers}
Connection identifiers are used to route messages to the precise destinations.  They
specify the exact destination endpoint or group, the view identifier, the protocol
stack to deliver to, the type of protocol being used to send the message, as well as
several other bits of information.  Typically, endpoint or group identifiers are used
to send messages to the correct processes and connection identifiers are used to
route messages to the exact destination of a message within a process.  Messages
usually contain a connection identifier as a ``header'' of the message but do not
contain endpoint identifiers or group identifiers, except as subfields of a
connection identifier.

\subsection{Protocol Identifiers}
There is a one-to-one relationship between the standard protocol stacks of \ensemble\
and protocol identifiers.  Applications select the protocol to use by specifying the
protocol id of that stack.  Having identifiers for protocols is convenient because
they can be passed around in messages and have equality comparisons made on them,
whereas the actual protocol stacks cannot.

\subsection{Mode identifiers}  
Each communication domain has a corresponding mode identifier used to specify that
domain.

\subsection{Stack Identifiers}
Stack identifiers are used to distinguish between the various domains that
a protocol stack may be receiving messages through.  Each of the various
kinds of ``channels'' that protocol stacks use has a separate identifier.
Currently, these are:
\begin{description}
\item
[Primary :] This is the primary communication channel for a protocol stack.
This is normally where most messages are received.
\item
[Bypass :] Messages sent via the optimized bypass protocols use this id. 
\item
[Merge :] Messages sent by group merge protocols use this id.
\item
[ProtoArb :] This is used for protocol-arbitration protocols (i.e.., protocols
that decide which protocol to use.
\item
[Authenticate :] This stack id is reserved for authentication protocols.
\end{description}

\section{The Event Module}
\label{event:module}

Events in \ensemble\ are used for \emph{intra-stack} communication, as opposed
to inter-stack communication, for which messages are used.  Currently, the
event module is the only \ensemble-specific module that all layers use.
Events contain a well-known set of fields which all layers use for
communicating between themselves using a common \emph{event protocol}.
Learning this protocol is one of the harder parts of understanding \ensemble.
In this section we describe the operations supported for events; in
section~\ref{event:protocol} we describe the meaning of the various event
types and their fields.

We repeatedly refer the reader to the source code of the event module source
files, both \sourcetype{event.mli} and \sourcetype{event.ml}.  This is done to
ensure that information in this documentation does not fall out of date due to
small changes in the event module.

Note that certain of the operations invalidate events passed as arguments to
the function.  This means that no further operations are accessing on the event
should be done after the function call.  The purpose of this limitation is to
allow multiple implementations of the event module with different memory
allocation mechanisms.  The default implementation of events is purely
functional and these rules can be violated without causing problems.  Other
implementations of the event module require that events be manipulated
according to these rules, and yet other implementations trace event
modifications to check that the rules are not violated.  What this means is
that protocol designers do not need to be concerned with allocation and
deallocation issues, except in the final stages of development.

\subsection{Fields}
Events are ML records with fixed sets of fields.  We refer to
\sourcetype{event.mli} for their type definitions and fields.

\subsubsection{Extension fields}
Events have a special field called the extension field.  Uncommon fields are
included in up events as a linked list of extensions to this field.  The list
of valid extensions is defined in \sourcetype{event.mli} by the type definition
\mlval{fields}.

\subsubsection{Event Types}
Events have a ``type'' field (called \mlval{typ} to avoid clashes with the
\mlval{type} keyword) in them which can take values from a fixed set of
enumerated constants.  For the enumerations of the type fields for events, we
refer to \sourceappl{event.mli} for the type definitions for \mlval{typ}.

\subsubsection{Field Specifiers}
Events have defined for them a variant record called \mlval{field}.  These are
called field specifiers.  There is a one-to-one relation between the fields in
up and down events and the variants in the fields specifiers.  As will be seen
shortly, lists of field specifiers are passed to event constructor and modifier
functions to specify the fields in an event to be modified and their values.
This allows changes to an event to be specified incrementally.

\subsection{Constructors}
Events are constructed with the \mlval{createupDef} function.

\begin{codebox}
  (* Constructor *)
val create	: debug -> typ -> field list -> t
\end{codebox}

Create takes 3 arguments:
\begin{itemize}
\item
The string name of the module or location where this operation is being
performed.  This is used only for debugging purposes and usually the value
\mlval{name} (defined to be the name of the current module) is used for
this argument.
\item
The type of the event, which is a \mlval{typ} enumeration.
\item
A list of field specifiers for changing the values of the fields in the new
events.  Unmodified fields should not be accessed.  For example, if an empty
list is passed as a field specifier then only the type field of the event will
be available in the event.
\end{itemize}
The return value of the constructor functions is a valid event.

\subsection{Special Constructors}
\sourcetype{event.ml} defines some special case constructors for either
performance or ease-of-coding reasons.  All of these constructors are defined
using the \mlval{create} function or could be defined using them.

\subsection{Modifiers}
Events are modified with the \mlval{set} function.

\begin{codebox}
  (* Modifier *)
val set		: debug -> t -> field list -> t
\end{codebox}

\mlval{set} takes 3 arguments:
\begin{itemize}
\item
The string name where this modification is taking place.  Used only for
debugging purposes.
\item
The event which is being modified.  The event passed as an argument to this
function is invalidated: no further references should be made to the event.
\item
A field specifier list.  See the arguments description for Constructors.
\end{itemize}
The return value of \mlval{set} is a new event with the same fields as the
original event, except for the changes in the specifier list.

\subsection{Copiers}
Events are copied with the \mlval{copy} function.

\begin{codebox}
  (* Copier *)
val copy	: debug -> t -> t
\end{codebox}

Copy takes two arguments:
\begin{itemize}
\item
The name where this modification is taking place.  Used only for debugging
purposes.
\item
The event to be copied.
\end{itemize}
The return value is a new event with its fields set to the same values as
the original.

\subsection{Destructors}
Events are released with the \mlval{free} function.
\begin{codebox}
  (* Destructor *)
val free	: debug -> t -> unit
\end{codebox}

Free functions takes two arguments:
\begin{itemize}
\item
The name where this modification is taking place.  Used only for debugging
purposes.
\item
The event to be deallocated.  This event becomes invalidated by this
function call.  No further references to the event should be made.
\end{itemize}
The return value is the \mlval{unit} value.

%*************************************************************%
%
%    Ensemble, 2_00
%    Copyright 2004 Cornell University, Hebrew University
%           IBM Israel Science and Technology
%    All rights reserved.
%
%    See ensemble/doc/license.txt for further information.
%
%*************************************************************%
\newcommand {\eventtype}[2]    {\item {#1:} #2}

\newenvironment{EventType}{%
\begin{itemize}
}{\end{itemize}
}

\newcommand {\chainentry}[2]    {#1 & #2 \\ \hline}

\newenvironment{ChainTable}{%
\begin{quote}\begin{tabular}{|l|l|} \hline
}{\end{tabular}\end{quote}
}

\section{Event protocol: Intra-stack communication}
\label{event:protocol}
\ensemble\ embodies two forms of communication.  The first is
communication between protocol stacks in a group, using messages sent
via some communication transport.  The second is intra-stack
communication between protocol layers sharing a protocol stack (see
fig~\ref{comm:event}), using \ensemble\ events (see
page~\pageref{event:module} for a overview of \ensemble events).  One
use of events is for passing information directly related to messages
(i.e., broadcast messages are usually associated with \ECast\ events).
However, events also are used for notifying layers of group membership
changes, telling other layers about suspected failed members,
synchronizing protocol layers for view changes, passing acknowledgment
and stability information, alarm requests and timeouts, etc\ldots.  In
order for a set of protocol layers to harmoniously implement a higher
level protocol, they have to agree on what these various events
``mean,'' and in general follow what is called here the \ensemble\
\emph{event protocol}.

The layering in \ensemble\ is advantageous because it allows complex protocols
to be decomposed into a set of smaller, more understandable protocols.
However, layering also introduces the event protocol which complicates the
system through the addition of intra-stack communication (the event
protocol) to inter-stack communication (normal message communication).

\emph{Be aware that this information may become out of date.  Although the
``spirit'' of the information presented here is unlikely to change in
drastic ways, always consider the possibility that this information does
not exactly match that in \sourcetype{event.ml} and \sourcetype{event.mli}.
Please alert us when such inconsistencies are discovered so they may be
corrected.}

\begin{figure}[tb]
\begin{center}
\resizebox{6in}{!}{\incgraphics{comm}}
%\scalebox{0.7}{\includegraphics{fig/comm}}
\end{center}
\caption{\em Events are used for intra-stack communication: layers can only
communicate with other layers by modifying events; layers never read or
modify other layer's message headers.  Messages are used for inter-stack
communication: only messages are sent between group members; events are
never sent between members.}
\label{comm:event}
\end{figure}

The documentation of the event protocol proceeds as follows.
\begin{itemize}
\item
``types'' of events are listed along with a summary of their meaning
\item
\todo{the types that usually have a message associated with them are
identified}
\item
event fields are described along with a summary of their usage
\item
\todo{a table is given showing the event types for which the various event
fields have defined values}
\item
the several \emph{event chains} which occur in protocol stacks are listed
(event chains are sequences of event micro-protocols that tend to occur in
\ensemble\ protocol stacks)
\end{itemize}

\subsection{Event Types}

\begin{figure}
\begin{codebox}

    (* These events should have messages associated with them. *)
  | ECast				(* Multicast message *)
  | ESend				(* Pt2pt message *)
  | ESubCast				(* Multi-destination message *)
  | ECastUnrel				(* Unreliable multicast message *)
  | ESendUnrel				(* Unreliable pt2pt message *)
  | EMergeRequest			(* Request a merge *)
  | EMergeGranted			(* Grant a merge request *)
  | EOrphan				(* Message was orphaned *)

    (* These types do not have messages. *)
  | EAccount				(* Output accounting information *)
(*| EAck			      *)(* Acknowledge message *)
  | EAsync				(* Asynchronous application event *)
  | EBlock				(* Block the group *)
  | EBlockOk				(* Acknowledge blocking of group *)
  | EDump				(* Dump your state (debugging) *)
  | EElect				(* I am now the coordinator *)
  | EExit				(* Disable this stack *)
  | EFail				(* Fail some members *)
  | EGossipExt				(* Gossip message *)
  | EGossipExtDir			(* Gossip message directed at particular address *)
  | EInit				(* First event delivered *)
  | ELeave				(* A member wants to leave *)
  | ELostMessage			(* Member doesn't have a message *)
  | EMergeDenied			(* Deny a merge request *)
  | EMergeFailed			(* Merge request failed *)
  | EMigrate				(* Change my location *)
  | EPresent                            (* Members present in this view *)
  | EPrompt				(* Prompt a new view *)
  | EProtocol				(* Request a protocol switch *)
  | ERekey				(* Request a rekeying of the group *)
  | ERekeyPrcl				(* The rekey protocol events *)
  | ERekeyPrcl2				(*                           *)
  | EStable				(* Deliver stability down *)
  | EStableReq				(* Request for stability information *)
  | ESuspect				(* Member is suspected to be faulty *)
  | ESystemError			(* Something serious has happened *)
  | ETimer				(* Request a timer *)
  | EView				(* Notify that a new view is ready *)
  | EXferDone				(* Notify that a state transfer is complete *)
  | ESyncInfo
      (* Ohad, additions *)
  | ESecureMsg				(* Private Secure messaging *)
  | EChannelList			(* passing a list of secure-channels *)
  | EFlowBlock				(* Blocking/unblocking the application for flow control*)
(* Signature/Verification with PGP *)
  | EAuth

  | ESecChannelList                     (* The channel list held by the SECCHAN layer *)
  | ERekeyCleanup
  | ERekeyCommit 
\end{codebox}
\caption{Event typ type definition.  Taken from \sourcetype{event.mli}.}
\label{fig:enum}
\end{figure}

This section describes the different types of events.  See
fig~\ref{fig:enum} for the source code of enumerated types. The
behavior of a layer depends not only on the event type and its
fields, but also on the \emph{direction} from which it arrives. For
example, an \ESend\ event travels in the sender stack from the
application down, and at the receiver from the bottom, up to the
application. The sender and receiver layers behave quite differently
depending on whether the message is sent or received. In what follows,
we sometimes specifically include the event direction. Detailed Descriptions: 

\begin{EventType}
\eventtype{\Up{Block}}{The coordinator is blocking the view. 
Is a reply to \Dn(Block); replied with \Dn{BlockOk}.}

\eventtype{\Dn{Block}}{The group is being blocked. Is a reply to
\Up{Suspect} and \Up{MergeRequest}; replied with \Up{Block}.}

\eventtype{\Up{BlockOk}}{The coordinator gets one of these events when the
group is blocked.  Is a reply to \Dn{BlockOk}; replied with \Dn{View} or
\Dn{MergeRequest}.}

\eventtype{\Dn{BlockOk}}{Is a reply to \Up{Block}; replied with
\Up{BlockOk} (but usually only at the coordinator).}

\eventtype{\Up{Cast}}{A member (whose rank is specified by the origin field)
broadcast a message to all members in the group.  Usually the broadcast is
delivered at all members except the sender. }

\eventtype{\Dn{Cast}}{A message is being broadcast.  
Replied with \Up{Cast} at all members but sender.}

\eventtype{\Up{Send}}{Another member sent us a pt2pt message.}
\eventtype{\Dn{Send}}{A message is being sent pt2pt.  Results in an
\Up{Send} at the destination.}

\eventtype{\ESubCast}{A message that will be multicast to a subset of
the group.}
\eventtype{\ECastUnrel}{A message that will be unreliablely multicasted}
\eventtype{\ESendUnrel}{A message that will be unreliablely sent point-to-point}

\eventtype{\Up{MergeRequest}}{Some other partition want to merge with us.
Replied with \Dn{MergeGranted} or \Dn{MergeDenied}.}

\eventtype{\Up{MergeGranted}}{Notification that a merge is ready to
proceed.}

\eventtype{\Dn{MergeGranted}}{Done by the coordinator after an
\Up{MergeRequest} to tell the other coordinator that the merge is
progressing.  Results in an \Up{MergeGranted} at the merging
coordinator.}

\eventtype{\Up{MergeDenied}}{This is notification that the coordinator of a
partition we tried to merge with has explicitly denied the merge.  Is a
reply to \Dn{MergeRequest}; replied with \Dn(View).}

\eventtype{\Dn{MergeDenied}}{Done by the coordinator after an \Up{MergeRequest}
to tell another coordinator that its request has been denied.  Is a reply
to \Up{MergeRequest}.}

\eventtype{\Up{MergeFailed}}{This is notification that some problem occurred
in an attempt to merge with another partition of our group.  Is a reply to
\Dn{MergeRequest}; replied with \Dn(View).}

\eventtype{\Up{Orphan}}{A message has lost its parent.  Usually it is OK to
ignore this message: it is just being delivered in case we are interested.}

\eventtype{\EAccount}{Used to control accounting
information. Periodically this event is passed through the stack, and
each layer can record its current information/performance.}

\eventtype{\EAsync}{Used to handle asynchronous events. Unused currently.}

\eventtype{\Up{Dump}}{A layer wants the stack to dump its state.  Usually the
top-most layer will bounce this down as a \Dn{Dump} and the members will dump
their state on the \Dn{Dump} event.}

\eventtype{\Dn{Dump}}{Dump your state.  Pass it on.  Is a reply to \Up{Dump}.}

\eventtype{\Up{Elect}}{This member has been elected to be coordinator of the
group. Usually means that this member will be generating failure and view
events and managing the group for the rest of the view.}

\eventtype{\Up{Exit}}{This protocol stack has been disabled (because of a
previous \Dn{Leave} event).  Layers should pass this event up and then do
nothing else.}

\eventtype{\Up{Fail}}{This is notification that some members have been marked
as failed.  This does not necessarily mean we will get no more messages
from the failed members.  The COM layer drops them, but messages from those
members retransmitted by other members are still delivered.  Usually, it
also means the coordinator has started or will start a new view soon (but
this is not necessary).  Is a reply to \Dn{Fail}.}

\eventtype{\Dn{Fail}}{Some members are being failed.  Is a reply to
\Up{Suspect}; replied with \Up{Fail}.}

\eventtype{\Up{GossipExt}}{A gossip message has arrived.  Note that
the data is in the extension fields.}

\eventtype{\Dn{GossipExt}}{Transmit a gossip message.  Note that data
is carried in extension fields of events and does not use the normal
header pushing mechanism.}

\eventtype{\Up{GossipExtDir}}{The same, but send a gossip message to a
specified address.}

\eventtype{\Up{Init}}{This is the first event that any layer receives.  It
should be passed up the stack.}

\eventtype{\Up{Leave}}{Some member (specified by the origin field) is leaving
the group.}

\eventtype{\Dn{Leave}}{We are leaving the group.  Replied with \Up{Exit} event.
Depending on when this is seen it can mean different things.  Before a new
view change it means we are really leaving the group.  After a new view, it
usually means that we have a new protocol stack taking part in the next
view and the \Dn{Leave} is just garbage collecting this protocol stack because
its view is over.}


\eventtype{\Up{LostMessage}}{This is notification that some protocol layer
below does not have a message it thinks it should have.  What this means is
usually highly-protocol-stack-specific.  Sometimes replied with \Dn{Fail}.}

\eventtype{\EMigrate}{Used to support migration of an endpoint between
transport addresses.}

\eventtype{\EPresent}{A notification that all group members currently
in the group are live and started ``talking''. This is useful in large
groups, where it takes time for \emph{all} members to synchronize and
agree on the view.}

\eventtype{\EPrompt}{Ask for a new view}

\eventtype{\EProtocol}{Ask for a new protocol}

\eventtype{\ERekey}{Request a rekey}

\eventtype{\ERekeyPrcl}{Used for inter-layer communication between the
rekeying layers. Used to separate out their event types.}

\eventtype{ERekeyPrcl2}{dito.}

\eventtype{\Up{Stable}}{This event contains stability information.  If we are
buffering broadcast messages, we can use this to decide which messages are
safe to drop.}

\eventtype{\EStableReq}{Ask for stability information}

\eventtype{\Up{Suspect}}{This is notification that some other layer (or
possibly some other member) thinks that some members should be kicked out
of the group.  Replied with \Dn{Fail} and often \Dn{Block}.}

\eventtype{\Dn{Suspect}}{Some members are suspected to be failed.
Replied with an \Up{Suspect} with the same members failed.}

\eventtype{\Up{SystemError}}{Something serious has happened.  Do whatever you
feel like because the world is about to fall apart.}

\eventtype{\Up{Timer}}{A timer has expired.  Pass it on.}

\eventtype{\Dn{Timer}}{A request for a timer alarm.  Replied with an
\Up{Timer} at or after the requested time.}

\eventtype{\Up{View}}{A new view is ready.  Note that this does not affect our
protocol: usually a different instance of our protocol stack will be
created to take care of the next view.  This event is not delivered at the
beginning of a view.  The \Up{View} event signals the end of a view.
\Up{Init} events are delivered at the beginning of a view.}

\eventtype{\Dn{View}}{A new view is prepared.  Usually followed by an
\Up{View}.  Usually does not affect the current protocol stack, but
later results in the creation of a new protocol stack for the new
view.}


\eventtype{\EXferDone}{Used for the state-transfer layer. Tells the
Xfer-layer that this endpoint has completed its state-transfer
protocol.}

\eventtype{\ESyncInfo}{Used inside the virtual-synchrony protocol.}

\eventtype{\ESecureMsg}{Send a secure private message on an encrypted
point-to-point channel to another member.}

\eventtype{\EChannelList}{Used to debug the secure-channel layer. The
event lists the set of current open secure-channels.}

\eventtype{\ESecChannelList}{dito.}

\eventtype{\EFlowBlock}{Blocking/unblocking the application for flow
control.}

\eventtype{\EAuth}{Signature/Verification with PGP.}

\eventtype{\ERekeyCleanup}{Erase all current open secure
point-to-point connections. This initializes the secure-channel cache.}

\eventtype{\ERekeyCommit}{Commit the current tentative group-key as
the key for the upcoming view}.

\end{EventType}


\subsection{Event fields}
Here we describe all the fields that appear in the events. The
type definitions appear in fig~\ref{event-fields} and
fig~\ref{fig:extensions}.  
%Default values for the fields appear in
%fig~\ref{fig:defaults}.

\begin{figure}
\begin{codebox}
type field =
      (* Common fields *)
  | Type        of typ            (* type of the message*)
  | Peer        of rank           (* rank of sender/destination *)
  | Iov	        of Iovecl.t       (* payload of message *)
  | ApplMsg                       (* was this message generated by an appl? *)

      (* Uncommon fields *)
  | Address     of Addr.set	  (* new address for a member *)
  | Failures    of bool Arrayf.t  (* failed members *)
  | Presence    of bool Arrayf.t  (* members present in the current view *)
  | Suspects    of bool Arrayf.t  (* suspected members *)
  | SuspectReason of string	  (* reasons for suspicion *)
  | Stability   of seqno Arrayf.t (* stability vector *)
  | NumCasts    of seqno Arrayf.t (* number of casts seen *)
  | Contact     of Endpt.full * View.id option (* contact for a merge *)

      (* HEAL gossip *)  
  | HealGos     of Proto.id * View.id * Endpt.full * View.t * Hsys.inet list
  | SwitchGos   of Proto.id * View.id * Time.t  (* SWITCH gossip *)
  | ExchangeGos	of string		(* EXCHANGE gossip *)

      (* INTER gossip *)
  | MergeGos    of (Endpt.full * View.id option) * seqno * typ * View.state
  | ViewState   of View.state	(* state of next view *)
  | ProtoId     of Proto.id	(* protocol id (only for down events) *)
  | Time        of Time.t	(* current time *)
  | Alarm       of Time.t	(* for alarm requests *)
  | ApplCasts   of seqno Arrayf.t
  | ApplSends   of seqno Arrayf.t
  | DbgName     of string

      (* Flags *)
  | NoTotal                     (* message is not totally ordered*)
  | ServerOnly	                (* deliver only at servers *)
  | ClientOnly	                (* deliver only at clients *)
  | NoVsync
  | ForceVsync
  | Fragment	                (* Iovec has been fragmented *)

      (* Debugging *)
  | History     of string       (* debugging history *)

      (* Private Secure Messaging *)
  | SecureMsg of Buf.t
  | ChannelList of (rank * Security.key) list
	
      (* interaction between Mflow, Pt2ptw, Pt2ptwp and the application *)
  | FlowBlock of rank option * bool

      (* Signature/Verification with Auth *)
  | AuthData of Addr.set * Auth.data

      (* The channel list held by the SECCHAN layer *)
  | SecChannelList of Trans.rank list
  | SecStat of int              (* PERF figures for SECCHAN layer *)
  | RekeyFlag of bool           (* Do a cleanup or not *)
\end{codebox}
\caption{Fields for events.  Taken from \sourcetype{event.mli}}
\label{fig:extensions}
\end{figure}

\subsubsection{Event Fields}
\label{event-fields}
\begin{EventType}
\eventtype{Typ}{The type of the event.}

\eventtype{Flags}{A bitfield specifying a set of potential flags for the
event.}

\eventtype{Peer}{rank of sender/destination}

\eventtype{Iov}{Iovec list containing raw application data.}

\eventtype{ApplMsg}{was this message generated by an appl? This
sometimes requires a different treatment than other, system generated 
messages.}

\eventtype{Address}{A address for a member, used, for example, in 
sending gossip messages to specific endpoints.}

\eventtype{Failures}{List of ranks of members that have failed.}

\eventtype{Presence}{The list of members present in the current
view. Used in large groups.}

\eventtype{Suspects}{List of ranks of members that are suspected to have
failed or be faulty in some way.}

\eventtype{SuspectReason}{String containing ``reason'' for suspecting
members.  Used for debugging purposes.}

\eventtype{Stability}{Vector of number of broadcasts for each member in
the group that are stable.}

\eventtype{NumCasts}{Vector of number of known broadcasts for members in
the group.}

\eventtype{Contact}{Endpoint of contact used for communication to endpoints
outside of group.  Usually only in merge events.}

\eventtype{HealGos} {The type of gossip messages sent between HEAL layers.}

\eventtype{SwitchGos}{dito, for SWITCH layers.}

\eventtype{ExchangeGos}{dito, for EXCHANGE layers.}

\eventtype{MergeGos}{dito, for MERGE layers.}

\eventtype{ViewState}{Used to pass around view state for new views}

\eventtype{ProtoId}{A new protocol id that we are switching to.}

\eventtype{Time}{Time that the event/message was received.}

\eventtype{Alarm}{Requested time for an alarm.}

\eventtype{ApplCasts}{The sequence numbers of the latest multicast
messages sent in the group.}

\eventtype{ApplSends}{dito, for send messages}

\eventtype{NoTotal}{Flag: specifying that this message should not be
totally ordered even if total a totally-ordered stack is in use. Used
to send Ensemble control messages which should not be delayed. Fail
and View messages are sent this way.}

\eventtype{ServerOnly}{Flag: Deliver only at servers.}

\eventtype{ClientOnly}{Flag: Deliver only at clients.}

\eventtype{NoVsync}{Flag: This message is not subject to virtual synchrony}
\eventtype{ForceVsync}{Flag: this message \emph{must} be subject to
virtual synchrony}

\eventtype{Fragment}{Flag: This event contains a fragment of an Iovec.}

\eventtype{SecureMsg}{Send this buffer through a secure-channel
point-to-point to another member.}

\eventtype{ChannelList}{Describe the secure channel list.}

\eventtype{FlowBlock}{Used for interaction the flow control layers: Mflow, Pt2ptw, Pt2ptwp and the application}

\eventtype{AuthData}{Send a buffer for PGP to check, and receive a
checked buffer from PGP}

\end{EventType}

\subsection{Event fields and the ``types'' for which they are defined}
\note{TODO}

\subsection{Event Chains}
We describe here common event sequences, or chains, in \ensemble.
Event chains are sequences of alternate up and down events that
ping-pong up and down the protocol stack bouncing between the two
end-layers of the chain.  The end layers are typically the the top and
bottom-most layers in the stack (eg., TOP and BOT).  The most common
exceptions to this are the message chains (Sends and Broadcasts),
which can have any layer for their top layer.

Note that these sequences are just prototypical.  Necessarily, there
are variations in which of layers see which parts of these sequences.
For example, consider the Failure Chain in a virtual synchrony stack
with the GMP layer.  The Failure Chain begins at the coordinator with
an \ESuspect\ event initiated at any layer in the stack.  The BOT
layer bounces this up as an \ESuspect\ event.  The top-most layer
usually responds with a \EFail\ event.  The \EFail\ event passes
down through all the layers until it gets to the GMP layer.  The GMP layer
at the coordinator both passes the \EFail\ event to the layer below
and passes down a \ECast\ event (thereby beginning a Broadcast
Chain\ldots).  At the coordinator, the \EFail\ event bounces off of
the BOT layer as an \EFail\ event and then passes up to the top of the
protocol stack.  At the other members, an \ECast\ event will be received
at the GMP layer.  The message is marked as a ``Fail'' message, so the
GMP layers generate and send down an \EFail\ event (just like the one
at the coordinator) and this is also bounced off the BOT layer as an
\EFail\ event.  The lesson here is that the different layers in the
different members of the group all essentially saw the same Failure
Chain, but exact sequencing was different.  For example, the layers
above the GMP layer at the members other than the coordinator did not
see a \EFail\ event. \todo{give diagram}

\todo{Leave Chain}

\subsubsection{Timer Chain}
Request for a timer, followed by an alarm timeout.
\begin{ChainTable}
\chainentry{\ETimer}{down: timeout requested, sent down to BOT.}
\chainentry{\ETimer}{up: alarm generated in BOT at or after requested
time, and sent up.}
\end{ChainTable}

\subsubsection{Send Chain}
Send a pt2pt message followed by stability detection.
\begin{ChainTable}
\chainentry{\ESend}{down: send a pt2pt message down.}
\chainentry{\ESend}{up: destinations receive the message}
\chainentry{\EStable}{message eventually becomes stable, and stability
information is bounced off BOT.}
\end{ChainTable}

\subsubsection{Broadcast Chain}
Broadcast of a message followed by stability detection.  
\begin{ChainTable}
\chainentry{\ECast}{down: broadcast a message}
\chainentry{\ECast}{up: other members receive the broadcast}
\chainentry{\EStable}{broadcast eventually becomes stable, and
stability information is bounced off BOT}
\end{ChainTable}

\subsubsection{Failure Chain}
Suspicion and ``failure'' of group members.
\begin{ChainTable}
\chainentry{\ESuspect}{down: suspicion of failures generated at any layer}
\chainentry{\ESuspect}{up: notification of suspicion of failures}
\chainentry{\EFail}{down: coord fails suspects}
\chainentry{\EFail}{up: all members get failure notice}
\end{ChainTable}

\subsubsection{Block Chain}
Blocking of a group prior to a membership change.
\begin{ChainTable}
\chainentry{\ESuspect/\EMergeRequest}{up: reasons for coord blocking}
\chainentry{\EBlock}{down: coord starts blocking}
\chainentry{\EBlock}{up: all members get block notice}
\chainentry{\EBlockOk}{down: all members reply to block notice}
\chainentry{\EBlockOk}{up: coord get block OK notice}
\chainentry{\EMergeRequest\/~\EView}{down: coord begins Merge or View chain}
\end{ChainTable}

\subsubsection{View Chain}
Installation of a new view, followed by garbage collection of the old view.
\begin{ChainTable}
\chainentry{\EView}{down: coord begins view chain (after failed merge or blocking)}
\chainentry{\EView}{up: all members get view notice}
\chainentry{\EExit}{down: protocol stacks are ready for garbage collection \note{todo}}
\chainentry{\EExit}{up: protocol stacks are garbage collected}
\end{ChainTable}

\subsubsection{Merge Chain (successful)}
Partition A merges with partition B, followed by garbage collection of the
old view.  We focus on partition A and only give a subset of events in
partition B.
\begin{ChainTable}
\chainentry{\EMergeRequest}{down: coord A begins merge chain (after blocking)}
\chainentry{\EMergeRequest}{up: coord B gets merge request}
\chainentry{\EMergeGranted}{down: coord B replies to merge request}
\chainentry{\EMergeGranted}{up: coord A gets merge OK notice}
\chainentry{\EView}{down: coord A installs new view for coord B}
\chainentry{\EView}{up: all members in group A get view notice}
\chainentry{\EExit}{down: protocol stacks are ready for garbage collection}
\chainentry{\EExit}{up: protocol stacks are garbage collected}
\end{ChainTable}
 \todo{\EExit\ above is currently \ELeave}

\subsubsection{Merge Chain (failed)}
Failed merge, followed by installation of a view.
\begin{ChainTable}
\chainentry{\EMergeRequest}{down: coord begins merge chain (after blocking)}
\chainentry{\EMergeFailed\ or}{}
\chainentry{\EMergeDenied}{up: coord detect merge problem}
\chainentry{\EView}{down: coord begins view chain}
\end{ChainTable}

%*************************************************************%
%
%    Ensemble, 1.10
%    Copyright 2001 Cornell University, Hebrew University
%    All rights reserved.
%
%    See ensemble/doc/license.txt for further information.
%
%*************************************************************%
\section{Layer Execution Model}

\subsection{Callbacks}
Layers are implemented as a set of callbacks that handle events passed to
the layer by \ensemble\ from some other protocol layer.  These callbacks can in
turn call callbacks that \ensemble\ has provided the layer for passing events
onto other layers.  Logically, a layer is initialized with one callback for
passing events to the layer above it, and another callback for passing
events to the layer below it.  After initialization, a layer returns two
callbacks to the system: one for handling events from the layer below it
and another for handling events from the layer above it.  In practice,
these ``logical callbacks'' are subdivided into several callbacks that
handle the cases where different kinds of messages are attached to events.

\begin{figure}[tb]
\begin{center}
\incgraphics{automata}
\end{center}
\caption{\em Layers are executed as I/O automata, with pairs FIFO event
queues connecting adjacent layers.}
\label{fig:automata}
\end{figure}

\subsection{Ordering Properties}
The system infrastructure that handles scheduling of protocol layers and
the passing of events between protocol layers provides the following
guarantees:
\begin{description}
\item
[FIFO ordering] : The infrastructure guarantees that events passed between
two layers are delivered in order.  For instance, if layer A is stacked
above layer B, then all events layer A passes to layer B are guaranteed to
be delivered in FIFO order to layer B.  In addition, events that layer B
passes up to layer A are guaranteed to be delivered in FIFO order to layer
A.  Note that these ordering properties allow the scheduler some
flexibility in scheduling because they only specify the ordering of events
in a single channel between a pair of layers.
\item
[no concurrency] : The sytem infrastructure that hands events to layers
through the callbacks never invokes a layer's callbacks concurrently.  It
guarantees that at most one invocation of any callback is executing at a
time and that the current callback returns before another callback is made
to the protocol layer.  See fig~\ref{fig:automata} for a diagram of layer
automata.  Note that although a single layer may not be executed
concurrently, different layers \emph{may} be executed concurrently by a
scheduler.
\end{description}
The execution of a protocol stack can be visualized as a set of protocol
layers executing with a pair of event queues between each pair: one queue
for events going up and another for events going down.  The protocol layers
are then automata that repeatedly are scheduled to take pending events from
one of the adjacent incoming queues, execute it, and then deposit zero or
more events into two adjacent outgoing queues (see fig~\ref{fig:automata}).

%*************************************************************%
%
%    Ensemble, 1.10
%    Copyright 2001 Cornell University, Hebrew University
%    All rights reserved.
%
%    See ensemble/doc/license.txt for further information.
%
%*************************************************************%
\section{Layer Anatomy: what are the pieces of a layer?}

This is a description of the standard pieces of a \ensemble\ layer.  This
description is meant to serve as a general introduction the standard
``idioms'' that appear in layers.  Because all layers follow the same
general structure, we present a single documentation of that structure, so
that comments in a layer describe what is particular to that layer rather
than repeating the features each has in common with all the others.
Comments on additional information that would be useful here would be
appreciated.

\subsection{Design Goals}

A design goal of the protocol layers is to include as little
\ensemble-specific infrastructure is present in the layers.  For instance,
none of the layers embody notions of synchronization, messages operations,
of event scheduling.  In fact, the only \ensemble-specific modules used by
layers are the Event and the View modules.

%The hope is that this leads to a flexible design in which the
%implementations of the layers can be shifted between different
%infrastructures/architectures without modification to the layers.  The can
%be seen in the different architectures that are compatible with these
%layers.  They can be compiled with functional, imperative, or ``purifying''
%event implementations.  They can be compiled with an architecture where
%messages are tupled together (\sourcelayer{layers\_pair.*}) and also one
%where messages are composed together as lists
%(\sourcelayer{layer\_list.*}).  \note{Currently, only the pair version is
%kept up to date, called \sourcelayer{layer.*}.} Finally, they are
%independent of the scheduling infrastructure.  As long as the
%infrastructure satisfies a few fixed assumptions, the layers can be
%scheduled in many different ways and can even be compiled into a
%multithreaded implementation.

%The idea is to build a single implementation of a layer, but allow it to be
%``presented'' in different architectures.  We do this by placing different
%wrappers around the same core implementation of a layer.  In fact, the
%layers are never used directly: they always must have some wrapper placed
%around them (by {\bf layers\_*.ml}.  It is the versions of the layers built
%in the {\bf layers\_*} modules which are the actual layers that are
%exported from this directory.

\subsection{Notes}

Some general notes on layers:
\begin{itemize}
\item
All layers are in single files.
%\item
%See the {\bf TEMPLATE} layer for an example of a layer that does nothing
%looks like {\bf[todo:needs updating]}.
%\item
%See the {\bf STABLE\_SIMP} layer for an example of a simple layer which has
%been fully commented. \note{where did I put that?}
\item
Usually the only objects exported by a layer are the type ``header'' and
the value ``l''.  When referred to outside of a layer, these values are
prefixed with the name of the layer followed by ``.''  and the name of
the object (either ``header'' or ``l'').  For instance the STABLE layer is
referenced by the name ``Stable.l''.
\end{itemize}


\subsection{Values and Types}
Listed below are the values and types commonly found in a layer, listed in
the usual order of occurrence.  For each object we give a description of
its typical use and whether or not it is exported out of the layer.

\begin{itemize}

\item
\mlval{name} : Local variable containing the name of the layer.
\hide{This is typically used for printing debugging messages and in
combination with the event debugging event module
(\sourcemuts{event\_check.ml}).}

\item
\mlval{failwith} : Typically this standard \caml\ function is redefined to
prefix the failure message with the name of the layer.  Sometimes it is
also redefined to dump the state of a layer.

\item
\mlval{header} : Exported type of the header a layer puts on messages.
Layers that do not put headers on messages do not have this type defined.
The type is exported abstractly so it is opaque outside the layer.  This
type is usually a disjoint union (variant record) of several different
types.  Some example variants are:
\newcommand{\stfield}[2]  {\item \mlval{#1} : #2}
\begin{itemize}
\stfield{NoHdr}{Almost always one of the values is defined to be
  \mlval{NoHdr} which is used for messages for which the layer does not put
  on a (non-trivial) header}
\stfield{Data}{Put on application data messages from the layer above}
\stfield{Gossip}{Put on gossip messages (such as by a stability layer)}
\stfield{Ack/Nak}{Acknowledgement or negative acknowledgements}
\stfield{Retrans}{Retransmissions of application data messages}
\stfield{View}{List of members in a new view.}
\stfield{Fail}{List of members being failed.}
\stfield{Suspect}{List of members to be suspected.}
\stfield{Block}{Prepare to synchronize this group.}
\end{itemize}

\item
\mlval{nohdr} : exported variable.  This is a variable that only occurs in a few
layers.  It is always defined to be the value \mlval{NoHdr} of the header type
of the layer.  It is exported so that the rest of the system can do
optimizations when layers put trivial headers on a message.

\item
\mlval{normCastHdr} : exported variable.  This is a variable that is used only
by special layers that may need to generate a valid header for this layer.
This variable should have no relevence to the execution of a layer.

\item
\mlval{state} : Local type that contains the state of an instance of a
layer.  Some layers do not yet use this, but eventually all of them will.
\note{layers that do not use a state varable have the state split up
amongst several local state variables}.  The state then is referenced
through the local variable \mlval{s}.  A field in a state record is
referred to by the \caml\ syntax, \mlval{s.field}, where \mlval{field} is
the name of a field in the state record.

Some example field names used in layer states:

\begin{itemize}
\stfield{time}{time of last \UpTimer\ event seen}
\stfield{next\_sweep/next\_gossip}{
  next time that I want to do something (such as retransmit
  messages, synchronize clocks, ...)}
\stfield{sweep}{time between sweeping}
\stfield{buf}{buffer of some sort}
\stfield{blocking/blocked}{boolean for whether group is currently blocking}
\stfield{ltime}{the current ``logical time stamp'' (usually taken from
  the \mlval{view\_id})}
\stfield{max\_ltime}{the largest logical time stamp seen by this member}
\stfield{seqno}{some sequence number}
\stfield{failed}{
  information on which members have failed (either a
  list of ranks or a boolean vector indexed by rank)
}
\stfield{elected}{do I think I am the coordinator?}
\end{itemize}

General notes on fields:
\begin{itemize}
\item{\em fields with type vect}: usually array with one entry for each member, indexed by rank
\item{\em fields with type map}: usually mapping of members eid to some state on the member
\item{\em fields with ``up'' (or ``dn'') in the name}: refers to some info kept on
  events that are going up (or down)
\item{\em fields with ``cast'' (or ``send'') in the name}: refers to state
  kept about broadcasts (or sends)
\item{\em fields with ``buf'' in the name}: refers to some buffer
\item{\em fields with ``dbg'' in the name}: fields used only for debugging puposes
\item{\em fields with ``acct'' in the name}: fields used only for keeping
  track of tracing or accounting data for the layer
\end{itemize}

\item
\mlval{dump} : Local function that takes a value of type \mlval{state} and
prints some representation for debugging purposes

\item
\mlval{member} : Local type that only occurs in some layers.  Defines
state kept for each member in a group.  Typically, the layer's state will
have an array of member objects indexed by rank (and this field is usually
called \mlval{members}).  The notes above fields in state records generally
apply to the fields in member types as well.

\item
\mlval{init} : Initialization function for this layer.  Takes two
arguments.  The first is a tuple containing arguments specific to this
layer.  The second is a view state record containing arguments general to
the protocol stack this layer is in.  This function does any initialization
required and returns a \mlval{state} record for a layer instance.

Some example names of initialization parameters for layers:
\begin{itemize}
\stfield{sweep}{float value of how often to carry out some action
(such as retransmitting messages or pinging other members)}
\stfield{timeout}{amount of time to use for some timeout (such as how
long to wait before kicking a non-responsive member out of a group)}
\stfield{window}{size of window to use for some buffer}
\end{itemize}

\item
\mlval{hdlrs} : Function initializing handlers for this layer.  Takes two
arguments.  The first is a \mlval{state} record for this layer.  The second
is a record containing all the handlers the layer is to  use for passing
events out of the layer.  Return value is a set of handlers to be used for
passing events into this layer.

\end{itemize}

%*************************************************************%
%
%    Ensemble, 1_42
%    Copyright 2003 Cornell University, Hebrew University
%           IBM Israel Science and Technology
%    All rights reserved.
%
%    See ensemble/doc/license.txt for further information.
%
%*************************************************************%
\section{Event Handlers: Standard}

Logically, a protocol has two incoming event handlers (one each above and
below) and two outgoing event handlers (one each above and below).  In
practice, because some events have messages and others do not, these
handlers are split up into several extra handlers.  The breakdown of the 4
logical handlers into 10 actual handlers is done for compatibility with the
ML typechecker.  Typechecking is used extensively to guarantee that layers
recieve messages of the same type they send.  This is a very useful
property because it prevents a large class of programming errors.

\begin{table}[tb]
\begin{center}
\begin{tabular}{|l|c|c|c|c|c|} \hline
name	& in/	& up/	& above/	& message?	& header? \\
	& out	& dn	& below		&		& 	\\ \hline \hline
upnm	& out 	& up	& above		& no		& no	\\ \hline
up	& out	& up	& above		& yes		& no	\\ \hline
dnnm	& out	& dn	& below		& no		& no	\\ \hline
dnlm	& out	& dn	& below 	& no		& yes	\\ \hline
dn	& out	& dn	& below		& yes		& yes	\\ \hline \hline
upnm	& in 	& up	& below		& no		& no	\\ \hline
uplm	& in	& up	& below		& no		& yes	\\ \hline
up	& in	& up	& below		& yes		& yes	\\ \hline
dnnm	& in 	& dn 	& above		& no		& no	\\ \hline
dn	& in	& dn	& above		& yes		& no	\\ \hline
\end{tabular}
\end{center}
\caption{\em The 10 standard event handlers.}
\label{table:handlers}
\end{table}

\begin{figure}[tb]
\begin{center}
\resizebox{!}{4in}{\incgraphics{handlers}}
%\resizebox{!}{4in}{\includegraphics*[0,0][500,700]{fig/handlers}}
\end{center}
\caption{\em Diagram of the 10 standard event handlers.  Note that the
ABOVE layer has a similar interface above it as the BELOW layer.  Likewise
with the interface beneath the BELOW layer.}
\label{fig:handlers}
\end{figure}

In the standard configuration, each layer has 10 handlers.  A handler is
uniquely specified by a set of characteristics: whether it is an incoming
or outgoing handler, a handler for up events or down events, a handler for
communication with the layer above or for the layer below, whether it has
an associated message, and whether it has an associated header.  See
table~\ref{table:handlers} for a enumeration of the 10 handlers.  Of the 10
handlers, 5 are outgoing and 5 are incoming; 5 are up event handlers and 5
are down event handlers; 4 are for event communication with the layer below
and 6 are for event communication with the layer above.  These are depicted
in fig~\ref{fig:handlers}.

The names of the handlers have two parts.  The first specifies the sort of
event the handler is called with (``up'' or ``dn'').  The second specifies
the sort of message that is associated with the event and may be either
``'' (nothing, the default case), ``lm'' (for local message), or ``nm''
(for no message), which correspond to:
\begin{description}
\item
[nothing:] Events with associated messages, where the message was created
by a layer above this layer.  This layer was not the first layer to push a
header onto the message and will not be the last layer to pop its header
off the message.
\item
[``lm'':] Events with associated messeges, where the
message was created by this layer.  This was the first layer to push a
header onto the message and is the last layer to pop its header off of the
message.
\item
[``nm'':] Corresponds to events without associated messages.
These handlers always take a single argument which is either an up event or
a down event.
\end{description}

%*************************************************************%
%
%    Ensemble, 1_42
%    Copyright 2003 Cornell University, Hebrew University
%           IBM Israel Science and Technology
%    All rights reserved.
%
%    See ensemble/doc/license.txt for further information.
%
%*************************************************************%
%%%%%%%%%%%%%%%%%%%%%%%%%%%%%%%%%%%%%%%%%%%%%%%%%%%%%%%%%%%%
%This is the new security description, by Ohad Rodeh.
%%%%%%%%%%%%%%%%%%%%%%%%%%%%%%%%%%%%%%%%%%%%%%%%%%%%%%%%%%%%
\section{The Ensemble Security Architecture (by Ohad Rodeh)}

This section describes the \ensemble\ security architecture. We
believe that \ensemble\ completely supports the fortress security
model. Only trusted, authorized members are allowed into the group.
Once a member is allowed into a group, it is completely trusted.
\ensemble\ is not secure against attacks from members that have been
admitted into the group: any group member can break the protocols by
sending bad messages.

The goal of our architecture is to secure group messages from
tampering and eavesdropping. To this end, all group messages are
signed and (possibly) encrypted. While it is possible to use public
key cryptography for this task, we find this approach unacceptably
expensive. Since all group members are mutually trusted, we share a
symmetric encryption key, and a MAC~\footnote{MAC, Message Authentication
Code. This is typically a keyed hash function.} key among them. These
keys are used to seal all group messages, making the seal/unseal
operation very fast\footnote{symmetric encryption/MAC is roughly 1000
times faster than equivalent public key operations.}. As a shorthand,
we shall refer to the key-pair as the {\it group key}. Using a group key
raises two challenges:

\begin{description}
\item[ A rekeying mechanism:] allowing secure replacement of the current
group key once it is deemed insecure, or if there is danger that it
was leaked to the adversary. Dissemination of the new key should be
performed without relying on the old (compromised) group key.

\item[Secure key agreement in a group:] i.e., a protocol that creates
secure agreement among group members on a mutual group key. 
\end{description}

We focus on benign failures and assume that authenticated members will
not be corrupted.  Byzantine fault tolerant systems suffer from
poor performance since they use costly protocols and make extensive use of
public key cryptography. We believe that our failure model is
sufficient for the needs of most practical applications.  

The user may specify a security policy for an application. The policy
specifies for each address\footnote{An Ensemble address is comprised
of a set of identifiers, for example an IP address and a PGP principal
name. Generally, an address includes an identifier for each
communication medium the endpoint is using \{UDP,TCP,MPI,ATM,..\}.}
whether it is trusted or not. Each application maintains its
own policy, it is up to Ensemble to enforce it and
to allow only mutually trusted members into the same subgroup. A policy
allows an application to specify the members that it trusts and
exclude untrusted members from its subgroup.

\subsection{Cryptographic Infrastructure}
Our design supports the use of a variety of authentication and
encryption mechanisms. \ensemble\ has been interfaced with the OpenSSL
(see {\tt http://www.openssl.org/}) cryptographic library, the PGP authentication engine, and the Kerberos
centralized authentication system (this is out of date). By default,
messages are signed using MD5, encrypted using RC4, and authentication
is performed using PGP. Because these three functionalities are
carried out independently any combination of supported authentication,
signature, and encryption systems can be used.  A future goal is to
allow multiple systems to be supported concurrently.  Under such a
system, processes would be able to compare the systems they have
support for and select any system that both have support for.

\subsection{Rekeying}
Ensemble rekeying uses the notion of {\it secure channels}. A secure
channel between endpoints $p$ and $q$ is essentially a symmetric
encryption key $k_{pq}$ agreed upon between $p$ and $q$. This key is
known only to $p$ and $q$ and is different than the group
key. Whenever confidential information needs to be passed between $p$
and $q$ it is encrypted using $k_{pq}$ and sent using Ensemble
reliable point-to-point messaging. 

The basic rekeying protocol supported uses a binary tree structure.
In order to rekey the group, a complete binary tree spanning the group
is created. Member 0 is the father of 1 and 2, 1 is the father of 3
and 4, etc.. The leader chooses a new key $k_{new}$ and sends it securely to 1
and 2; member number 1 sends $k_{new}$ securely down to 3 and 4, etc.. When a tree
leaf receives a new key it sends up a clear-text acknowledgment. When
acknowledgments reach the leader (0) it prompts the group for a view
change in which the new key will be used. 

$k_{new}$ is disseminated confidentially using secure channels. We
cannot use the old key to protect $k_{new}$ since the old key is
assumed to be compromised. Secure channels are created upon
demand by Ensemble, they are then cached for future use. Creating a secure
channel is a costly operation taking hundreds of milliseconds even on
fast CPUs. It is performed in the background so as not to block the
application.

Recently, we have added faster rekeying protocols to the system. A
complete implementation of the dWGL algorithm has been added, in the
form of several layers. There are two new algorithms rekey\_dt, and
rekey\_diam. There are described in the reference manual. 

\subsection{A secure stack}
The Security architecture is comprised of 5 layers:
\begin{description}
\item
[Exchange:] secure key agreement. This layer is responsible for
securely handing the group key to new joining group components. Component
leaders mutually authenticate and check authorization policies prior
to handing the group key securely between them.
\item
[Encrypt:] chain-encryption of all user messages.
\item 
[Secchan:] create and manage a cache of secure channels. 
\item 
[PerfRekey:] handling common rekeying tasks. For example, after a new
key has been disseminated to the group, acknowledgments must be
collected from all group members.
\item
[Rekey\_dt:] Binary tree rekeying. Rekeying a group is very fast once secure
channels have been setup. We logged an average rekey operation for a 20 member
group at 100 milliseconds. Rekey\_dt assumes that the Secchan and
PerfRekey layers are in the stack. 
\end{description}

The regular and secure Ensemble stacks are depicted in
Figure~\ref{fig:sec-stack}. The Top and Bottom layer cap the stack
from both sides. The membership layers compute the current set of live
and connected machines, the Appl\_top layer interfaces with the
application and provides reliable send and receive capabilities for
point-to-point and multicast messages. The RFifo layers provide
reliable per-source fifo messaging. The Exchange and Rekey layers are
related to the membership layers since the group key is a part of the
view information. The Encrypt layer encrypts all user messages hence
it is below the Appl\_top layer.

\begin{table}[h]
\centerline{
\begin{tabular}{|l|l|l|} \hline
Regular     & security additions              \\ \hline \hline
Top         &				      \\ \hline
 	    & Exchange                        \\ \hline
	    & Rekey\_dt                       \\ \hline
	    & PerfRekey                       \\ \hline
            & Secchan                         \\ \hline
Gmp         &                                 \\ \hline 
Top\_appl   & Interface to the application    \\ \hline
            & Encrypt			      \\ \hline
Rfifo       &			   	      \\ \hline
Bottom      &				      \\ \hline 
\end{tabular}
}
\caption{The Ensemble stack. On the left is the default stack that
includes an application interface, the membership algorithm and a
reliable-fifo module. To the right is a secure stack with the
Exchange, Encrypt, Rekey\_dt, and Secchan layers in place.}
\label{fig:sec-stack}
\end{table}


\subsection{Security events}
There are three security events to note:
\begin{itemize}
\item 
{ERekey:} By this event the application requests a Rekey operation. 			
\item 
{ESecureMsg:} This event is used by the Rekey layer to send private messages to
other processes. The Secchan layer catches this event
and sends the message securely to its destination.
\item 
{ERekeyPrcl:} this event is used in the communication between
all rekeying layers. 
\end{itemize}

The Vs\_key field was added to the view state was to allow for group keys.
It holds the current group key.

\subsection{Using Security}
Ensemble has three security properties:
\begin{enumerate}
\item 
{Rekey:} Add rekeying to the stack.
\item 
{OptRekey:} Use the dWGL algorithm for rekeying. 
\item
{Auth:} Authenticate all messages.
\item 
{Privacy:} Encrypt all user messages.
\end{enumerate}

An application wishing for strong security should choose all of the
above properties in its stack and perform a {\it Control Rekey} action
once every several hours. Note that there are two flavors to
application Rekey-ing:
\begin{itemize}
\item{Rekey false:} The default, as above.
\item{Rekey true:} Cleanup prior to rekeying. For performance
considerations, \ensemble\ keeps cached key-ing material and secure
channels. These should be cleared up every couple of hours to prevent
an adversary from using cryptanalysis to discover the group key. 
\end{itemize}

An example command line, for application appl, with pgp user name James\_Joyce:
\begin{codebox}
appl -add\_prop Auth -add\_prop Privacy -key 01234567012345670123456701234567 
     -pgp James\_Joyce
\end{codebox}

In order to add authorization to the stack, thereby controlling which
members are allowed to join a group, one must do:

\begin{codebox}
  val policy_function : Addr.set -> bool
  val interface : Appl_intf.New.t

  let state = Layer.new_state interface in
  let state = Layer.set_exchange (Some policy_function) state in 
  Appl.config_new_full state (ls,vs)
\end{codebox}

Instead of simply:
\begin{codebox}
  Appl.config_new interface state (ls,vs)
\end{codebox}

Authorization is not linked to the Security architecture, regular
stacks can perform authorization. Control of joining members is
delegated to the group leader that checks its authorization list and
allows/disallows join. Every view change the authorization list is checked and
existing members that are not authorized are removed. 

In practice, if an application changes its authorization list
dynamically, it must perform a {\emph Prompt} and a {\emph Rekey}
whenever such a change occurs.

\subsection{Checking that things work}
To check that PGP has been installed correctly, that \ensemble\ can
talk to it without fault, and the cryptographic support is running
correctly, one can use the armadillo demo program. 

In order to set up PGP, one must create principals and corresponding
public and private keys. These are installed by PGP in its local
key repository. The basic PGP key-generation command is:
\begin{codebox}
zigzag ~/ensemble/demo> pgp -kg
\end{codebox}

To work with the armadillo demo, you'll need to create principals in
the group $o1, o2, ...$. Armadillo creates a set of endpoints, and
then runs a test between them. To this end, the program has a ``-n''
flag that describes the number of endpoints to use. For example, the
command line {\tt armadillo -n 2 ...} tells armadillo that use a two
members configuration. These members will have principal names $o1$
and $o2$ respectively.

To view the set of principals in the repository do:
\begin{codebox}
zigzag ~/ensemble/demo> pgp -kv
pub   512/2F045569 1998/06/15 o2
pub   512/A2358EED 1998/06/15 o1
2 matching keys found.
\end{codebox}

To check that PGP runs correctly do: 
\begin{codebox}
zigzag ~/ensemble/demo> armadillo -prog pgp 
PGP works
check_background
got a ticket
background PGP works
\end{codebox}

If something is broken, the PGP execution trace can be viewed using:
\begin{codebox}
zigzag ~/ensemble/demo> armadillo -prog pgp  -trace PGP 
\end{codebox}

If more information is required use the flags {\tt -trace PGP1 -trace PGP2}.
The default version of PGP that \ensemble\ works with is 2.6. If,
however, you'd like to use a different version, set your environment
variable $ENS\_PGP\_VERSION$ to the version number. Versions 5.0 and
6.5 are also supported. 

To check that OpenSSL is installed correctly, one can do:
\begin{codebox}
zigzag ~/ensemble/demo> armadillo -prog perf
\end{codebox}

For a wider scale test use the {\it exchange} test. This is a test
that creates a set of endpoints, with principal names: {\it o1, o2,
...}, and merges them securely together into one group. Each group
merge requires that group-leaders properly authenticate themselves
using PGP. The test is started with all members in components containing
themselves, and ends when a single secure component is created. 
Note that it will keep running until reaching the timeout. The timeout
is set by default to 20 seconds. 
To invoke the test do:
\begin{codebox}
zigzag ~/ensemble/demo> armadillo -prog exchange -n 2 -real_pgp
\end{codebox}

If something goes wrong, a trace of the authentication protocol is
available through {\tt -trace EXCHANGE}. 

The {\tt -real\_pgp} flag tells armadillo not to simulate PGP. 
Simulation is the default mode for armadillo, since we use it to
test communication protocol correctness. 

To check that rekeying works do: 
\begin{codebox}
zigzag ~/ensemble/demo> armadillo -prog rekey  -n 5
\end{codebox}


To test security with two separate processes do the following:
\begin{codebox}
zigzag ~/ensemble/demo> gossip &
zigzag ~/ensemble/demo> mtalk -key 11112222333344441111222233334444 
                  -add_prop Auth -pgp o1
zigzag ~/ensemble/demo> mtalk -key 01234567012345670123456701234567 
                 -add_prop Auth -pgp o2
\end{codebox}

The two mtalk processes should authenticate each other and merge.

The three command line arguments specify:
\begin{itemize}
\item {\tt -key 111122223333444111122223333444} : The initial security key of the
      system. Should be a 16 byte string.
\item {\tt -add\_prop Auth}: Add the authentication protocol.
      Otherwise, stacks with different keys will not be able to
      merge. 
\item {\tt -pgp o1}: Specify the principal name for the system.
\end{itemize}

\subsection{Using security from HOT and EJava}
The security options have been added to the HOT interface. For a
demonstration program look at hot\_sec\_test.c in the hot subdirectory. 
The only steps one needs to make are: (1) Set the program's principal
name (2) Set the security bit. Both of these options are specified in the
join-options structure. For example, in hot\_sec\_test.c:
\begin{codebox}
static void join(
		 int i,
		 char **argv
) {
  state *s ;
  s = (state *) hot_mem_Alloc(memory, sizeof(*s)) ;
  memset(s,0,sizeof(*s)) ;
  
  s->status = BOGUS;
  s->magic = HOT_TEST_MAGIC;

  ...

  strcpy(s->jops.transports, "UDP");
  strcpy(s->jops.group_name, "HOT_test");
  
  ...

  sprintf(s->jops.princ, "Pgp(o%d)",i);
  s->jops.secure = 1;

  ...
  
  /* Join the group.
   */
  err = hot_ens_Join(&s->jops, &s->gctx);
  if (err != HOT_OK) {
    hot_sys_Panic(hot_err_ErrString(err));
  }

}
\end{codebox}

EJava is interfaced with HOT, so they share a similar interface. Note
that the outboard mode, supported by both interface is {\bf
insecure}. The messages passing on the TCP connection between the
client and server are neither MACed nor encrypted. Therefore, they can
be used securely only when situated on a single machine. 


\newpage
\part{The \ensemble\ Protocols}
%*************************************************************%
%
%    Ensemble, 2_00
%    Copyright 2004 Cornell University, Hebrew University
%           IBM Israel Science and Technology
%    All rights reserved.
%
%    See ensemble/doc/license.txt for further information.
%
%*************************************************************%
%\newcommand {\layerdef}[1]     {\subsection{{#1} Protocol Layer}}
%\newcommand {\stackdef}[1]     {\subsection{{#1} Protocol Stack}}
%\newcommand {\parameters}[0]   {\paragraph{Parameters} ~\par}
%\newcommand {\generates}[0]    {\paragraph{Generated Events} ~\par}
%\newcommand {\properties}[0]   {\paragraph{Properties} ~\par}
%\newcommand {\protocol}[0]     {\paragraph{Protocol} ~\par}
%\newcommand {\notes}[0]        {\paragraph{Notes} ~\par}
%\newcommand {\sources}[0]      {\paragraph{Sources} ~\par}
%\newcommand {\testing}[0]      {\paragraph{Testing} ~\par}

\newenvironment{Layer}[1]       {\subsection{{#1}}}{}
\newenvironment{Stack}[1]       {\subsection{{#1}}}{}

\newenvironment{GenEvent}{%
\paragraph{Generated Events} ~\par
\begin{quote}\begin{tabular}{|l|} \hline
}{\end{tabular}\end{quote}
}
\newcommand {\genevent}[1]      {#1 \\ \hline}

\newenvironment{Protocol}{%
\paragraph{Protocol} ~\par
\begin{quote}
}{\end{quote}}

\newenvironment{Sources}{%
\paragraph{Sources} ~\par
\begin{quote}\begin{tabular}{|l|} \hline
}{\end{tabular}\end{quote}
}
\newcommand {\sourcesfile}[1]   {layers/#1 \\ \hline}

\newenvironment{Parameters}{%
\paragraph{Parameters}~\par\begin{itemize}
}{\end{itemize}}

\newenvironment{Properties}{%
\paragraph{Properties} ~\par\begin{itemize}
}{\end{itemize}}

\newenvironment{Notes}{%
\paragraph{Notes} ~\par\begin{itemize}
}{\end{itemize}}

\newenvironment{Testing}{%
\paragraph{Testing} ~\par\begin{itemize}
}{\end{itemize}}

\section{Layers and Stacks}

We document a subset of the \ensemble\ layers and stacks (compositions of
layers) in this section.  This documentation is intended to be largely
independent of the implementation language.  They are currently listed in
order, bottom-up, of their use in the VSYNC layer.

Each layer (or stack) has these items in its documentation:

\begin{Layer}{ANYLAYER}
The name of the layer follwed by a general description of its purpose.

\begin{Protocol}
A description of the protocol implemented by the layer.
\end{Protocol}

\begin{Parameters}
\item The list of parameters required to initialize the layer, along with
descriptions of their purpose.
\item \note{should also specify reasonable values}
\item If a layers takes no arguments, the documentation specifies
``None.''
\end{Parameters}

\begin{Properties}
\item A list of informal properties of the layer.
\end{Properties}

\begin{Notes}
\item General notes about the layer.
\end{Notes}

\paragraph{Sources} ~\par
The source files for the ML implementation of the layer.

\paragraph{Generated Events} ~\par
A list of event types generated by the layer.  In the future, this field
will contain more information, such as what event types are examined by
the layer (instead of being blindly passed on).  Hopefully, this
information will eventually be generated automatically.

\begin{Testing}
\item Information about the status of the layer regarding testing.
\item Testing information should always be documented: if the layer has
  not been tested, that should be stated.
\item What testing has been completed on the layer (along with version
  information).
\item What infrastracture is in place for testing the layer.
\item Known bugs for a layer are listed in the ML source code.
\end{Testing}
\end{Layer}

\begin{Layer}{CREDIT}

This layer implements a credit based flow control.

\begin{Protocol}
On initialization, sender informs receivers how many credits it wants to
keep in stock.  Receivers sends credits whenever it finds that the sender
is low on credits, either explicitly through a sender's request or
implicitly through its local accounting.  A credit is one time use only.
Sender is allowed to send a message only if it has a credit available.  If
the sender does not have a credit, the message is buffered.  Buffered
messages are sent when new credits arrive.  Credits are piggybacked to data
messages whenever there is an opportunity of doing so to save bandwidth.
\end{Protocol}

\begin{Parameters}
\item 
rtotal: the total number of credits that this member can give out.  Should
be set according to the number of receive buffers that the machine the
member is running has.
\item 
ntoask: the number of credits that this member likes to keep in stock.
\item 
whentoask: the threshold number of credits remaining at sender before the
receiver consider sending out more credits.
\item 
pntoask: like ntoask for piggyback style of credit giving.
\item 
pwhentoask: like nwhentoask for piggyback style of credit giving.
\item 
sweep: frequency at which periodic sweep routine, which give out credits to
senders, should run.
\end{Parameters}

%\begin{Properties}
%\end{Properties}

\begin{Notes}
\item Future implementation should support dynamic credit adjustment.
\item Alternative flow control layers include RATE and WINDOW.
\end{Notes}

\begin{Sources}
\sourcesfile{credit.ml}
\end{Sources}

Last updated: Fri Mar 29, 1996

\end{Layer}

%*************************************************************%
%
%    Ensemble, 1_42
%    Copyright 2003 Cornell University, Hebrew University
%           IBM Israel Science and Technology
%    All rights reserved.
%
%    See ensemble/doc/license.txt for further information.
%
%*************************************************************%
\begin{Layer}{RATE}

This layer implements a sender rate based flow control.  Multicast messages
from each sender are sent at a rate not exceeding some prescribed value.

\begin{Protocol}
All the messages to be sent are buffered initially.  Buffered messages are
sent on periodic timeouts that are set based on the sender's rate.
\end{Protocol}

\begin{Parameters}
\item
rate\_n, rate\_t: the pair determines the rate.  At most \mlval{rate\_n}
messages are allowed to sent over any time period of \mlval{rate\_t}.  This
is ensured by having two consecutive messages sent with a inter-send time
of at least $(rate\_t / rate\_n)$ apart.
\end{Parameters}

%\begin{Properties}
%\end{Properties}

\begin{Notes}
\item Future implementation should support dynamic rate adjustment.
\item Alternative flow control layers include CREDIT and WINDOW.
\end{Notes}

\begin{Sources}
\sourcesfile{rate.ml}
\end{Sources}

\emph{This layer and its documentation were written by Takako Hickey.}
\end{Layer}

\begin{Layer}{BOTTOM} 

Not surprisingly, the BOTTOM layer is the bottommost layer in a \ensemble\
protocol stack.  It interacts directly with the communication transport by
sending/receiving messages and scheduling/handling timeouts.  The properties
implemented are all \emph{local} to the protocol stack in which the layer
exists: ie., a (dn)Fail event causes failed members to be removed from the local
view of the group, but no failure message to be sent out--it is assumed that
some other layer actually informs the other members of the failure.

\begin{Protocol}
None
\end{Protocol}

\begin{Parameters}
\item None
\end{Parameters}

\begin{Properties}
\item
Requires messages be appropriately fragmented for the transport in use.
\item
\Dn{Timer}\{time\} events cause an alarm to be scheduled with the transport so
that an \Up{Timer} event is later delivered some time after $time$.
\item
\Dn{Block}, \Dn{View}, \Dn{Stable}, and \Dn{Fail} events cause an \Up{Block},
\Up{View}, \Up{Stable}, and \Up{Fail} event (respectively) to be locally
``bounced'' up the protocol stack.  No communication results from these
events.
\item
In addition, \Dn{Fail} events cause further Send and Cast messages from the
failed members to be dropped.
\item
\Dn{View} events do not affect the membership in the current protocol stack.
The view in the resulting \Up{View} event is merely a proposal for the next
view of the group.  (It is expected that a new protocol stack will be
created for that view.)
\item
\Dn{Send}, \Dn{Cast} events cause messages to be sent (unreliably) to other
members in the group.  The resulting \Up{Send} and \Up{Cast} events are delivered
with the origin field set with the rank of the sender and the time field
set with the time that the messages was received (according to the
transport).
\item
\Dn{Merge}, \Dn{MergeDenied}, and \Dn{MergeGranted} causes messages to be sent
(unreliably) to members outside of the group.  These result in
\Up{MergeRequest}, \Up{MergeDenied}, and \Up{MergeGranted} messages at the
destination, respectively.
\item
\Dn{Leave} events disable the transport instance and bounce up an \Up{Exit}
event.  No further events are delivered after the \Up{Exit}. \note{currently,
this may not be true}
\end{Properties}

\begin{Sources}
\sourcesfile{bottom.ml}
\end{Sources}

\begin{GenEvent}
\genevent{\Up{Block}}
\genevent{\Up{Cast}}
\genevent{\Up{Exit}}
\genevent{\Up{Fail}}
\genevent{\Up{Stable}}
\genevent{\Up{MergeDenied}}
\genevent{\Up{MergeGranted}}
\genevent{\Up{MergeRequest}}
\genevent{\Up{Send}}
\genevent{\Up{Suspect}}
\genevent{\Up{Timer}}
\genevent{\Up{View}}
\end{GenEvent}

\begin{Testing}
\item see the VSYNC stack
\end{Testing}
\end{Layer}

%*************************************************************%
%
%    Ensemble, 2_00
%    Copyright 2004 Cornell University, Hebrew University
%           IBM Israel Science and Technology
%    All rights reserved.
%
%    See ensemble/doc/license.txt for further information.
%
%*************************************************************%
\begin{Layer}{CAUSAL}

The CAUSAL layer implements causally order multicast.  It assumes
reliable, FIFO ordered reliable messaging from layers below.

\begin{Protocol}
The protocol has two versions: full and compressed vectors.  First,
we explain the simple version which uses full vectors. Then, we
explain how these vectors are compressed.

Each outgoing message is appended with a \emph{causal vector}. This
vector contains the last causally delivered message from each member
in the group.  Each received message is checked for deliverability.
It may be delivered only if all messages which it causally follows,
according to its causal vector, have been delivered.  If it is not
yet deliverable, it is delayed in the layer until delivery is
possible.  A view change erases all delayed messages, since they can
never become deliverable.

Causal vectors become large with the group size, so they must be
compressed in order for this protocol to scale.  The compression we
use is derived from the Transis system.  We demonstrate with an
example: assume the membership includes three processes $p,q$ and
$r$. Process $p$ sends message $m_{p,1}$, $q$ sends $m_{q,1}$,
causally following $m_{p,1}$ and $r$ sends $m_{r,1}$ causally
following $m_{q,1}$. The causal vector for $m_{r,1}$ is
$[1|1|1]$. There is redundancy in the causal vector since it is clear
that $m_{r,1}$ follows $m_{r,0}$. Furthermore, since $m_{q,1}$
follows $m_{p,1}$ we may omit stating that $m_{r,1}$ follows
$m_{p,1}$. To conclude, it suffices to state that $m_{r,1}$ follows
$m_{q,1}$.  Using such optimizations causal vectors may be compressed
considerably.
\end{Protocol}

\begin{Sources}
\sourcesfile{causal.ml}
\end{Sources}

\begin{Testing}
\item
The CHK\_CAUSAL protocol layer checks for CAUSAL delivery.
\end{Testing}

\emph{This layer and its documentation were written by Ohad Rodeh.}
\end{Layer}

\begin{Layer}{ELECT}

This layer implements a leader election protocol.  It determines when a member
should become the coordinator.  Election is done by delivering an \DnElect\
event at the new coordinator.

\begin{Protocol}
When a member suspects all lower ranked members of being faulty, that member elects
itself as coordinator.
\end{Protocol}

\begin{Parameters}
\item None
\end{Parameters}

\begin{Properties}
\item
\UpSuspect\ events may cause a \DnElect\ event to be generated.
\end{Properties}

\begin{Sources}
\sourcesfile{elect.ml}
\end{Sources}

\begin{GenEvent}
\genevent{\DnElect}
\end{GenEvent}

\begin{Testing}
\item
see also the VSYNC stack
\end{Testing}
\end{Layer}


%*************************************************************%
%
%    Ensemble, 1.10
%    Copyright 2001 Cornell University, Hebrew University
%    All rights reserved.
%
%    See ensemble/doc/license.txt for further information.
%
%*************************************************************%
\begin{Layer}{ENCRYPT}

This layer encrypts application data for privacy.  Uses keys in the view state
record.  Authentication needs to be provided by the lower layers in the system.
The protocol headers are not encrypted.  This layer must reside above FIFO
layers for sending and receiving because it uses encryption contexts whereby
the encryption of a message is dependent on the previous messages from this
member.  These contexts are dropped at the end of a view.  A smarter protocol
would try to maintain them, as they improve the quality of the encryption.

\begin{Protocol}
Does chained encryption on the message payload in the \mlval{iov} field of
events.  Each member keeps track of the encryption state for all incoming and
outgoing point-to-point and multicast channels.  Messages marked
\mlval{Unreliable} are not encrypted (these should not be application
messages).
\end{Protocol}

\begin{Parameters}
\item None
\end{Parameters}

\begin{Properties}
\item
Guarantees (modulo encryption being broken) that only processes that know the
shared group key can read the contents of the application portion of data
messages.
\item
Requires FIFO ordering on point-to-point and multicast messages.
\end{Properties}

\begin{Sources}
\sourcesfile{encrypt.ml}
\end{Sources}

\begin{GenEvent}
\genevent{None}
\end{GenEvent}

\begin{Testing}
\item
see the VSYNC stack
\end{Testing}
\end{Layer}

%*************************************************************%
%
%    Ensemble, 1_42
%    Copyright 2003 Cornell University, Hebrew University
%           IBM Israel Science and Technology
%    All rights reserved.
%
%    See ensemble/doc/license.txt for further information.
%
%*************************************************************%
\begin{Layer}{HEAL}

This protocol is used to merge partitions of a group.

\begin{Protocol}
The coordinator occasionally broadcasts the existence of this partition via
\Dn{GossipExt} events.  These are delivered unreliably to coordinators of other
partitions.  If a coordinator decides to merge partitions, then it prompts a
view change and inserts the name of the remote coordinator in the \Up{BlockOk}
event.  The INTER protocol takes over from there.  Merge cycles are prevented
by only allowing merges to be made from smaller view id's to larger view id's.
\end{Protocol}

\begin{Parameters}
\item
heal\_wait\_stable : whether or not to wait for a first broadcast message to
become stable before starting the protocol.  This ensures that all the members
are in the group.
\end{Parameters}

\begin{Properties}
\item \todo{}
\end{Properties}

\begin{Sources}
\sourcesfile{heal.ml}
\end{Sources}

\begin{GenEvent}
\genevent{\Up{Prompt}}
\genevent{\Dn{GossipExt}}
\end{GenEvent}

\begin{Testing}
\item see the VSYNC stack
\end{Testing}
\end{Layer}

%*************************************************************%
%
%    Ensemble, 1_42
%    Copyright 2003 Cornell University, Hebrew University
%           IBM Israel Science and Technology
%    All rights reserved.
%
%    See ensemble/doc/license.txt for further information.
%
%*************************************************************%
\begin{Layer}{INTER}

This protocol handles view changes that involve more than one
partition (see also INTRA).

\begin{Protocol}
Group merges are the more complicated part of the group membership
protocol.  However, we constrain the problem so that:
\begin{itemize}
\item
Groups cannot be both merging and accepting mergers at the same time.
This eliminates the potential for cycles in the ``merge-graph.''
\item
A view (i.e. view\_id) can only attempt to merge once, and only if no
failures have occured.  Each merge attempt is therefore uniquely
identified by the view\_id of the merging group.  Note also that by
requiring no failures to have occured for a merge to happen, this
prevents a member from being failed in one view and then reappearing
in the next view.  There has to be an intermediate view without the
failed member: this is a desirable property.
\end{itemize}
The merge protocol works as follows:
\begin{enumerate}
\item
The merging coordinator blocks its group,
\item
The merging coordinator sends a merge request to the remote group's
coordinator.
\item
The remote coordinator blocks its group,
\item
The remote coordinator installs a new view (with the mergers in it) and
sends the view to the merging coordinator (through a merge-granted
message).
\item
The merging coordinator installs the view in its group.
\end{enumerate}
If the merging coordinator times out on the merged coordinator then it
immediately installs a new view in its partition (without the other members
even finding out about the merge attempt).
\end{Protocol}

\begin{Parameters}
\item None
\end{Parameters}

\begin{Properties}
\item
When another partition is merging, a View message is also sent to the
coordinator of the merging group, which then forwards the message to
the rest of its group.
\item 
Requires that \Dn{Merge} events only be delivered by the original
coordinator of views (in which no failures have yet occured).
Otherwise, the partition should first form a new view and then attempt
the merge.
\item
\Dn{Merge} causes a \Dn{Merge} event to be delivered to the layer below.  This
will be replied with either an \Up{View}, \Up{MergeFailed}, or \Up{MergeDenied}
event, depending on the outcome of the merge attempt.
\item
\Up{MergeRequest}'s are only delivered at the coordinator.  And only if the
group is not currently blocking and only if the mergers list does not
contain members that are/were in this view or in previous merge requests in
this view.
\end{Properties}

\begin{Sources}
\sourcesfile{inter.ml}
\end{Sources}

\begin{GenEvent}
\genevent{\Dn{Merge}}
\genevent{\Dn{MergeDenied}}
\genevent{\Dn{Suspect}}
\end{GenEvent}

\begin{Testing}
\item
see the VSYNC stack
\end{Testing}
\end{Layer}


\begin{Layer}{INTRA}

This layer manages group membership within a view (see also the INTER layer).
There are three related tasks here:
\begin{itemize}
\item
Forwarding of group membership events to the rest of the group
(without INTRA, normally \DnView\ and \DnFail\ events have only local
effect).
\item 
Filtering of group membership events from remote members (for
example, when two other group members think they are the coordinator
and fail each other, the INTRA layer choose one of them and ignores
the other member).
\item
Determining the view\_id of the following view.
\end{itemize}

\begin{Protocol}
This is a relatively simple group membership protocol.  We have done our
best to resist the temptation to ``optimize'' special cases under which the
group is ``unnecessarily'' partitioned.  We also constrain the conditions
under which operations such as merges can occur.  The implementation does
not ``touch'' any data messages: it only handles group membership changes.
Furthermore, this protocol does not use any timeouts.

Views and failures are forwarded via broadcast to the rest of the members.
Other members accept the view/failure if they are consistent with their
current representation of the group's state.  Otherwise, the view/failure
message is dropped and the sender is suspected of being problematic. 
\end{Protocol}

\begin{Parameters}
\item None
\end{Parameters}

\begin{Properties}
\item
\DnView\ events are passed on to the layer below.  They also cause a View message to
be broadcast to the other members.  On receipt of this View message, the other
members either accept it (and deliver a \DnView\ event to layer below) or mark the
sender of the View as problematic, and possibly deliver a \DnSuspect\ event to the
layer below.
\item
Requires FIFO, atomic broadcast delivery from layers below.
\item
\DnFail\ events are passed on to the layer below.  They also cause a Fail
message to be broadcast to the other members.  On receipt of this Fail
message, the other INTRA instances will either accept it (and deliver a
\DnFail\ event to the layer beneath them) or mark the sender of the Fail
message as problematic, and possibly deliver an \DnSuspect\ event to the
layer below.
\item
View and Fail messages from a particular coordinator are delivered in FIFO
order to the members.
\item
Not all members may see same set of \UpFail\ events.  However, the set of
failed members grows monotonicly with each failure notification.
\end{Properties}

\begin{Sources}
\sourcesfile{intra.ml}
\end{Sources}

\begin{GenEvent}
\genevent{\DnAck}
\genevent{\DnCast}
\genevent{\DnFail}
\genevent{\DnSuspect}
\genevent{\DnView}
\end{GenEvent}

\begin{Testing}
\item
see the VSYNC stack
\end{Testing}
\end{Layer}


%*************************************************************%
%
%    Ensemble, 1.10
%    Copyright 2001 Cornell University, Hebrew University
%    All rights reserved.
%
%    See ensemble/doc/license.txt for further information.
%
%*************************************************************%
\begin{Layer}{LEAVE}

This protocol has two tasks.  (1) When a member really wants to leave a group,
the LEAVE protocol tells the other members to suspect this member.  (2) The
leave protocol garbage collects old protocol stacks by initiating a \DnLeave\
after getting an \UpView\ and then getting an \UpStable\ where everything is
marked as being stable.

\begin{Protocol}
Both protocols are simple.

For leaving the group, a member broacasts a Leave message to the group
which causes the other members to deliver a \DnSuspect\ event.  Note that
the other members will get the Leave message only after receiving all the
prior broadcast messages.  This member should probably stick around,
however, until these messages have stabilized.

Garbage collection is done by waiting until all broadcast message are
stable before delivering a local \DnLeave\ event.
\end{Protocol}

\begin{Parameters}
\item
leave\_wait\_stable : whether or not to wait for the leave announcment to
become stable before leaving
\end{Parameters}

\begin{Properties}
\item \todo{}
\end{Properties}

\begin{Sources}
\sourcesfile{leave.ml}
\end{Sources}

\begin{GenEvent}
\genevent{\DnLeave}
\end{GenEvent}

\begin{Testing}
\item see the VSYNC stack
\end{Testing}
\end{Layer}

%*************************************************************%
%
%    Ensemble, 1.10
%    Copyright 2001 Cornell University, Hebrew University
%    All rights reserved.
%
%    See ensemble/doc/license.txt for further information.
%
%*************************************************************%
\begin{Layer}{MERGE}

This protocol provides reliable retransmissions of merge messages and failure
detection of remote coordinators when merging.

\begin{Protocol}
Simple retransmission protocol.  A hash table is used to detect copied
merge requests, which are dropped.
\end{Protocol}

\begin{Parameters}
\item
merge\_sweep: how often to retransmit merge requests
\item
merge\_timeout: how long before timing out on merges
\end{Parameters}

\begin{Properties}
\item
\DnMerge, \DnMergeGranted, and \DnMergeDenied\ events are buffered for later
retransmission.
\item
\UpMergeRequest, \UpMergeGranted, and \UpMergeDenied\ events are filtered
so that each event is delivered at most once by this layer (i.e., so that
retransmissions are dropped).
\item
After $timeout$ time (a parameter listed above) an \UpMergeFailed\ event is
delivered with the problems field set to be the contact of the \DnMerge\
(only) event.  (It is assumed that the merge process will normally be
complete before this timeout occurs.)
\end{Properties}

\begin{Notes}
\item
Removal of this protocol layer only makes the merges unreliable, and
stops the failure detection of the new coordinator.
\end{Notes}

\begin{Sources}
\sourcesfile{merge.ml}
\end{Sources}

\begin{GenEvent}
\genevent{\UpSuspect}
\genevent{\DnMerge}
\genevent{\DnTimer}
\end{GenEvent}

\begin{Testing}
\item see the VSYNC stack
\end{Testing}
\end{Layer}

\begin{Layer}{MFLOW} 

This layer implements window-based flow control for multicast messages.
Multicast messages from each sender are transmitted only if the number of send
credit left is greater than zero.  The protocol attempts to avoid situations
where all recievers send credit at the same time, so that a sender is not
flooded with credit messages.

\begin{Protocol}
Whenever the amount of send credits drops to zero, messages are buffered without
being sent.  On receipt of acknowledgement credit, the amount of send credits
are recalculated and buffered messages are sent based on the new credit.
\end{Protocol}

\begin{Parameters}
\item mflow\_window : the maximum amount on unacknowledged messages or the size of the
window.
\item mflow\_ack\_thresh : The acknowledge threshold.  After receiving this number of
bytes of data from a sender, the receiver acknowledged previous credit.
\end{Parameters}

\begin{Properties}
\item
This protocol bounds the number of unrecieved multicast messages a member has
sent.
\item
The amount of received credits are initialized to different values for
avoiding many members sending back acknowledge at the same time. 
\item
This protocol requires reliable multicast and point-to-point properties from
underlying protocol layers.
\end{Properties}

\begin{Notes}
\item
As opposed to most of the \ensemble\ protoocols, this protocol implements flow
control on bytes and not on messages.  It only considers the data in the
application payload portion of the message (the \mlval{iov} field of the
event).
\item
Because of the EBlockOk events, this layer needs to be below the
broadcast stability layer.
\end{Notes}

\begin{Sources}
\sourcesfile{mflow.ml}
\end{Sources}

\begin{Testing}
\item
Some testing has been carried out.
\end{Testing}

\emph{This layer and its documentation were written with Zhen Xiao.}
\end{Layer}

\begin{Layer}{MNAK}

The MNAK (Multicast NAK) layer implements a reliable, agreed, FIFO-ordered
broadcast protocol.  Broadcast messages from each sender are delivered in
FIFO-order at their destinations.  Messages from live members are delivered
reliably and messages from failed members are retransmitted by the coordinator
of the group.  When all failed members are marked as such, the protocol
guarantees that eventually all live members will have delivered the same set of
messages.

\begin{Protocol}
Uses a negative acknowledgment (NAK) protocol: when messages are detected
to be out of order (or the $NumCast$ field in an \UpStable\ event detects
missing messages), a NAK is sent.  The NAK is sent in one of three ways,
chosen in the following order:
\begin{enumerate}
\item Pt2pt to the sender, if the sender is not failed.
\item Pt2pt to the coordinator, if the reciever is not the coordinator.
\item Broadcast to the rest of the group if the receiver is the
  coordinator.
\end{enumerate}
All broadcast messages are buffered until stable.
\end{Protocol}

\begin{Parameters}
\item mnak\_allow\_lost : boolean that determines whether the MNAK layer will
check for lost messages.  Lost messages are only possible when using an
inaccurate stability protocol.
\end{Parameters}

\begin{Properties}
\item
Requires stability and $NumCast$ information (equivalent to that provided
by the STABLE layer).
\item
\DnCast\ events cause \UpCast\ events to be delivered at all other members
(but not locally) in the group in FIFO order.
\item
\UpCast\{ack = Ack(rank,seqno)\} events from this layer require a later
\DnAck\{dn\_ack = Ack(rank,seqno)\} event to be delivered from above.
\item
\UpStable\ events from the layer below cause stable messages to be garbage
collected and may cause NAK messages to be sent to other members in the
group.
\end{Properties}

\begin{Sources}
\sourcesfile{mnak.ml}
\end{Sources}

\begin{GenEvent}
\genevent{\DnCast}
\genevent{\DnSend}
\end{GenEvent}

\begin{Testing}
\item
The CHK\_FIFO protocol layer checks for FIFO  safety conditions.
\item
The FIFO application generates bursty communication in which a token traces
its way through each burst.  If a reliable communication layer drops the
token, communication comes to an abrupt halt.  This is intended to capture
the liveness conditions of FIFO layers.
\item 
see also the VSYNC stack
\end{Testing}
\end{Layer}

\begin{Layer}{PRIMARY}

Detect primary partition in a group. Usually a primary partition has
the majority of members or holds some important resources.
\begin{Protocol}
Upon \UpInit\ event, a member sends a message to the coordinator,
claiming that it is in the current view. When a view has the majority
of members, its coordinator prompts a view change to make itself the
primary partition if it is not yet.  When a new view is ready, it
decides whether it is primary and mark it as so.
\end{Protocol}

\begin{Parameters}
\item primary\_quorum: how many servers (non-client member) are
needed to form the primary partition.
\end{Parameters}

\begin{Properties}
\item
Guarantees no two primary partitions can have the same logical timestamp.
\item
Optimal in the normal case: no addtional view change is necessary.
\item
This protocol requires group membership management from underlying
protocol layers. 
\end{Properties}

\begin{Sources}
\sourcesfile{primary.ml}
\end{Sources}

\begin{GenEvent}
\genevent{\DnPrompt}
\genevent{\DnSend}
\end{GenEvent}

\begin{Testing}
\item
TODO
\end{Testing}

\emph{This layer and its documentation were written with Zhen Xiao.}
\end{Layer}

%*************************************************************%
%
%    Ensemble, 2_00
%    Copyright 2004 Cornell University, Hebrew University
%           IBM Israel Science and Technology
%    All rights reserved.
%
%    See ensemble/doc/license.txt for further information.
%
%*************************************************************%
\begin{Layer}{PT2PT} 

This layer implements reliable point-to-point message delivery.

\todo{finish this documentation}

\begin{Parameters}
\item
pt2pt\_sweep : how often to retransmit messages and send out acknowledgments
\item 
pt2pt\_ack\_rate : determines how many messages will be received before an
acknowledgement is generated.
\item 
pt2pt\_sync : boolean determining if point-to-point messages should be
synchronized with view changes
\end{Parameters}

\begin{Testing}
\item see the VSYNC stack
\end{Testing}
\end{Layer}

%*************************************************************%
%
%    Ensemble, 1.10
%    Copyright 2001 Cornell University, Hebrew University
%    All rights reserved.
%
%    See ensemble/doc/license.txt for further information.
%
%*************************************************************%
\begin{Layer}{PT2PTW} 

This layer implements window-based flow control for point to point messages.
Point-to-point messages from each sender are transmitted only if the window is
not yet full.

\begin{Protocol}
Whenever the amount of send credits drops to zero, messages are buffered
without being sent.  On receipt of acknowledgement credit, the amount of send
credits are recalculated and buffered messages are sent based on the new
credit.  Acknowledgements are sent whenever a speicified threshhold is passed.
\end{Protocol}

\begin{Parameters}
\item pt2ptw\_window : the maximum amount on unacknowledged messages or the size of the
window.
\item pt2ptw\_ack\_thresh : The acknowledge threshold.  After receiving this
number of bytes of data from a sender, the receiver acknowledges previous
credit.
\end{Parameters}

\begin{Properties}
\item
This protocol bounds the number of unrecieved point-to-point messages a member
can send.
\item
This protocol requires reliable point-to-point properties from underlying
protocol layers.
\end{Properties}

\begin{Notes}
\item
As opposed to most of the \ensemble\ protocols, this protocol implements flow
control on bytes and not on messages.  It only considers the data in the
application payload portion of the message (the \mlval{iov} field of the
event).
\end{Notes}

\begin{Sources}
\sourcesfile{pt2ptw.ml}
\end{Sources}

\begin{Testing}
\item
Some testing has been carried out.
\end{Testing}

Last updated: March 21, 1997

\end{Layer}

'\begin{Layer}{PT2PTWP} 

This layer implements an adaptive window-based flow control protocol 
for point-to-point communication between the group members. 

In this protocol the receiver's buffer space is shared between all group
members. This is accomplished by dividing the receiver's window among 
the senders according to the bandwidth of the data being received from each 
sender. Such way of sharing attempts to minimize the number of ack messages, 
i.e.\ to increase message efficiency.

\begin{Protocol}
In the following, the term acknowledgement is used with the meaning 
of flow control protocols and not that of reliable communication protocols.

This protocol uses \emph{credits} to measure the available buffer space 
at the receiver's side. Each sender maintains a \emph{window} per each 
destination, which is used to bound the unacknowledged data a process 
can send point-to-point to the given destination. For each message it 
sends, the process deducts a certain amount of credit based on the size 
of the message. Messages are transmitted only if the sender has enough 
credit for them. Otherwise, messages are buffered at the sender.

A receiver keeps track of the amount of unacknowledged data it has 
received from each sender. Whenever it decides to acknowledge a sender,
it sends a message containing new amount of credit for this sender.
On receipt of an acknowledgement message, sender recalculates the amount 
of credit for this receiver, and the buffered messages are sent based 
on the new credit.

The receiver measures the bandwidth of the data being received from each 
sender. It starts with zero bandwidth, and adjusts it periodically with 
timeout \emph{pt2ptwp\_sweep}. 

On receipt of a point-to-point message, the receiver checks if the sender 
has passed threshold of its window, i.e. if the amount of data in
point-to-point messages received from this sender since the last ack was
sent to it has exceeded a certain ratio, \emph{pt2ptwp\_ack\_thresh}, 
of the sender's window. If it is, an ack with some credit has to be sent 
to the sender. In order to adjust processes' windows according to their 
bandwidth, the receiver attempts to steal some credit from an appropriate 
process and add it to the sender's window. The receiver looks for a process 
with maximal \( \frac{window}{\sqrt{bandwidth}} \) ratio, decreases its window 
by certain amount of credit and increases the window of the sender appropriately. 
Then the receiver sends the sender ack with the new amount of credit. When the 
process from which the credit was stolen passes theshold of its new, smaller 
window, the receiver sends ack to it.
\end{Protocol}

\begin{Parameters}
\item pt2ptwp\_window : size of the receiver's window, reflects the receiver's 
buffer space.
\item pt2ptwp\_ack\_thresh : the ratio used by the receiver while deciding 
to acknowledge senders.
\item pt2ptwp\_min\_credit : minimal amount of credit each process must have.
\item pt2ptwp\_bw\_thresh : credit may be stolen for processes with greater 
bandwidth only.
\item pt2ptwp\_sweep : the timeout of periodical adjustment of bavdwidth.
\end{Parameters}

\begin{Properties}
\item
This protocol requires reliable point-to-point properties from underlying 
protocol layers.
\end{Properties}

\begin{Notes}
\item
As opposed to most of the \ensemble\ protocols, this protocol implements flow
control on bytes and not on messages. It only considers the data in the
application payload portion of the message (the \mlval{iov} field of the
event).
\end{Notes}

\begin{Sources}
\sourcesfile{pt2ptwp.ml}
\end{Sources}

\begin{Testing}
\item
Correctness and performance testing has been carried out.
\end{Testing}

\end{Layer}

\begin{Layer}{REALKEYS}
\label{layer:realkeys}
This layer is part of the dWGL suite. Together with OptRekey is
implements the dWGL protocol. This layer's task is to actually
perform the instructions passed to it from OptRekey, generate and
pass securely all group subkeys, and finally the group key.

\begin{Protocol}
When a Rekey operation is performed a complex set of layers and
protocols is set into motion. Eventually, each group member receives a
new keygraph and a set of instructions describing how to merge its
partial keytree with the rest of the group keytrees to achieve
a unified group tree. The head of the keytree is the group key.

The instructions are implemented in several stages by the subleaders:
\begin{enumerate}
\item Choose new keys, and send them securely to peer subleaders
using secure channels.
\item Get new keys through secure channels. Disseminate these keys
by encrypting them with the top subtree key, and sending pt-2-pt to the leader.
\item When the leader gets all 2nd stage messages, it bundles them
into a single multicast and sends to the group. 
\item A member $p$ that receives the multicast, extracts the set of
keys it should know. Member $p$ creates an \mlval{ERekeyPrcl} event
with the new group key attached. The event it send down to PerfRekey
notifing it that the protocol is complete.
\end{enumerate}

\end{Protocol}

\begin{Properties}
\item Requires VSYNC properties.
\end{Properties}

\begin{Sources}
\sourcesfile{realkeys.ml}
\sourcesfile{type/tdefs.ml,mli}
\end{Sources}

\begin{GenEvent}
\genevent{\mlval{ESecureMsg}}
\genevent{\DnCast}
\genevent{\DnSend}
\end{GenEvent}

\begin{Testing}
\item 
The armadillo program (in the demo subdirectory) tests the security properties
of \ensemble.
\end{Testing}

\end{Layer}




\begin{Layer}{REKEY}

This layers switches the group key upon request. There may be several 
reasons for switching the key:
\begin{itemize}
\item The key's lifetime has expired --- it is now possible that some
dedicated attacker has cracked it.
\item The key has been compromised. 
\item Application authorization policies have changed and previously trusted
members need to be excluded from the group. 
\end{itemize}

This layer also relies on the Secchan layer to create secure
channels when required. A secure channel is essentially a way
to pass confidential information between two endpoints. The Secchan
layer creates secure channels upon demand and caches them for future
use. This allows the new group key to be disseminated efficiently and
confidentially through the tree. 

\begin{Protocol}
When a member layer gets an \ERekeyPrcl\ event, it sends a message to the
coordinator to start the rekeying process.  The coordinator generates
a new key and sends it to its children using secure channels. The
children pass it down the tree. Once a member receives the new key is
passes it down to PerfRekey using an \mlval{ERekeyPrcl} event.

The PerfRekey layer is responsible for collecting acknowledgments from the
members and performing a view change with the new key once
dissemination is complete. 
\end{Protocol}

\begin{Parameters}
\item {rekey\_degree:} The degree of the dissemination tree. By
default it is 2.
\end{Parameters}

\begin{Properties}
\item
Guarantees during a view change, either all members switch to the new
shared key or none of them do.
\end{Properties}

\begin{Sources}
\sourcesfile{rekey.ml}
\end{Sources}

\begin{GenEvent}
\genevent{\DnCast}
\genevent{\DnSend}
\end{GenEvent}

\begin{Testing}
\item 
The armadillo.ml file in the demo directory tests the security properties
of \ensemble.
\end{Testing}

\emph{This layer was originally written by Mark Hadyen with Zhen Xiao.
Ohad Rodeh later rewrote the security layers and related infrastructure.}
\end{Layer}



%*************************************************************%
%
%    Ensemble, 1.10
%    Copyright 2001 Cornell University, Hebrew University
%    All rights reserved.
%
%    See ensemble/doc/license.txt for further information.
%
%*************************************************************%
\begin{Layer}{REKEY\_DT}
\label{layer:rekey_dt}

This is the default rekeying layer. The basic 
data structure used is a tree of secure channels. This tree
changes every view-change, therefore the name of the layer. 
Dynamic Tree REKEY. 

The basic problem in obtaining efficient rekeying is the high cost of
constructing secure channels. A secure channel is established using a
two-way handshake using a Diffie-Hellman exchange.  At the time of
writing, a PentiumIII 500Mhz can perform one side of a Diffie-Hellman
exchange (using the OpenSSL cryptographic library) in 40 milliseconds.
This is a heavyweight operation. 

To discuss the set of channels in a group, we shall view it as a graph
where the nodes are group members, and the edges are secure channels
connecting them. The strategy employed by REKEY\_DT is to use a tree
graph. When a rekey request is made by a user, in some view $V$, the
leader multicasts a tree structure that uses, as much as possible, the
existing set of edges. 

For example, if the view is composed of several previous
components, then the leader attempts to merge together existing
key-trees. If a single member joins, then it is located as close to the
root as possible, for better tree-balancing. If a member leaves, then
the tree may, in the worst case, split into three pieces. The leader
fuses them together using (at most) 2 new secure channels. 

The leader chooses a new key and passes it to its
children. The key is passed recursively down the tree until it reaches
the leaves. The leaf nodes send acknowledgments back to the leader.

This protocol has very good performance. It is even possible, that a
rekey will not require any new secure-channels. For example, in case
of member leave, where the node was a tree-leaf. 

\begin{Protocol}
When a member layer gets an \ERekeyPrcl\ event, it sends a message to the
coordinator to start the rekeying process.  The coordinator checks if
the view is composed of a single tree-component. If not, it multicasts
a {\it Start} message. All members that are tree-roots, sends their
tree-structure to the leader. The leader merges the trees together,
and multicasts the group-tree. It then chooses a new key and sends it
down the tree. 

Once a member receives the new key is passes it down to PerfRekey
using an \mlval{ERekeyPrcl} event.

The PerfRekey layer is responsible for collecting acknowledgments from the
members and performing a view change with the new key once
dissemination is complete. 
\end{Protocol}

\begin{Sources}
\sourcesfile{rekey\_dt.ml}
\end{Sources}

\begin{GenEvent}
\genevent{\DnCast}
\genevent{\DnSend}
\end{GenEvent}

\begin{Testing}
\item 
The armadillo.ml file in the demo directory tests the security properties
of \ensemble.
\end{Testing}

\end{Layer}



\begin{Layer}{SECCHAN}
\label{layer:secchan}
This layer is responsible for sending and receiving private messages 
to/from group members. Privacy is guaranteed through the creation and
maintenance of {\it secure channels}. 

A secure channel is, essentially, a symmetric key (unrelated to the group key)
agreed upon between two members. This key is used to encrypt any
confidential message sent between them. We allow layers above 
Secchan to send/receive confidential information using
{\it SecureMsg} events. When a SecureMsg($dst,data$) event arrives at
Secchan, a secure channel to member $dst$ is created (if one does
not already exist). Then, the $data$ is encrypted using the secure channel key
and reliably sent to {\it dst}.

This layer relies on an authentication engine - this is provided in
system independent form by the Auth module. Currently, PGP is used for
authentication. New random shared keys are generated by the
Security module. The Security module also provides hashing and
symmetric encryption functions. Currently RC4 is used for encryption
and MD5 is used for hashing. 

\begin{Protocol}
A secure channel between members $p$ and $q$ is created using the
following basic protocol:
\begin{enumerate}
\item 
Member $p$ chooses a new random symmetric key $k_{pq}$.  It creates a
ticket to $q$ that includes $k_{pq}$ using the Auth module ticket
facility. Essentially, Auth encrypts $k_{pq}$ with $q$'s public key and
signs it using $p$'s private key. Member $p$ then sends the ticket to $q$.
\item 
Member $q$ authenticates and decrypts the message, and sends an
acknowledgment ({\it Ack}) back to $p$.
\end{enumerate}

This two-phase protocol is used to prevent the occurrence of a {\it
double channel}. By this we mean the case where $p$ and $q$ open
secure channels to each other at the same time. We augment the Ack phase;
$q$ discards $p$'s ticket if:
\begin{enumerate} 
\item $q$ has already started opening a channel to $p$
\item $q$ has a larger\footnote{Polymorphic comparison is used here.} name
than $p$.
\end{enumerate}

Secchan also keeps the number of open channels, per member, below the
{\it secchan\_cache\_size} configuration parameter. Regardless, a
channel is closed if it's lifetime exceeds 8 hours (the setable {\it
secchan\_ttl} parameter).  A two-phase protocol is used to close a
channel. If members $p$ and $q$ share channel, assuming $p$ created
it, then $p$ sends a {\it CloseChan} message to $q$.  Member $q$
responds by sending a {\it CloseChanOk} to $p$.

It typically happens that many secure channels are created
simultaneously group wide. For example, in the first Rekey of a
group. If we tear down all these channels exactly 8 hours from their
inception, the group will experience an explosion of management
information. To prevent this, we stagger channel tear down times.
Upon creation, a channel's maximal lifetime is set to $8 hours + I
seconds$ where $I$ is a random integer in the range [0 ..{\it
secchan\_rand}] . {\it secchan\_rand} is set by default to 200 seconds,
which we view as enough.

\end{Protocol}

\begin{Properties}
\item Requires VSYNC properties.
\end{Properties}

\begin{Parameters}
\item{secchan\_cache\_size:} determines size of secure channel cache. 
\item{secchan\_ttl:} Time To Live of a channel.
\item{secchan\_rand:} Used to stagger channel refresh times.
\item {secchan\_causal\_flag:} for performance evaluations.
\end{Parameters}

\begin{Sources}
\sourcesfile{secchan.ml}
\sourcesfile{msecchan.ml}
\end{Sources}

\begin{GenEvent}
\genevent{EChannelList}
\genevent{ESecureMsg}
\genevent{\DnCast}
\genevent{\DnSend}
\end{GenEvent}

\begin{Testing}
\item 
The armadillo program (in the demo subdirectory) tests the security properties
of \ensemble.
\end{Testing}

\end{Layer}




%*************************************************************%
%
%    Ensemble, 1_42
%    Copyright 2003 Cornell University, Hebrew University
%           IBM Israel Science and Technology
%    All rights reserved.
%
%    See ensemble/doc/license.txt for further information.
%
%*************************************************************%
\begin{Layer}{SEQUENCER}

This layer implements a sequencer based protocol for total ordering.

\begin{Protocol}
One member of the group serves as the sequencer. Any member that wishes to
send messages, send them point-to-point to the sequencer. The sequencer then
delivers the message localy, and cast it to the rest of the group. Other
members, as soon as they receive a cast from the sequencer, they deliver
the message.

If a view change occurs, messages are tagged as unordered and are send as
such.
When the \Up{View} event arrives, indicating that the group has successfully
been flushed, these messages are delivered in a deterministic order everywhere
(according to the ranks of their senders, breaking ties using FIFO).
\end{Protocol}

\begin{Parameters}
\item None
\end{Parameters}

\begin{Properties}
\item
Requires VSYNC properties.
\end{Properties}

\begin{Sources}
\sourcesfile{sequencer.ml}
\end{Sources}

\begin{GenEvent}
\genevent{\Dn{Cast}}
\genevent{\Dn{Send}}
\end{GenEvent}

\begin{Testing}
\item
\todo{}
\end{Testing}

\emph{This layer and its documentation were written by Roy Friedman.}
\end{Layer}

\begin{Layer}{SLANDER}

This protocol is used to share suspicions between members of a partition.  This
way, if one member suspects another member of being faulty, the coordinator is
informed so that the faulty member is removed, even if the coordinator does not
detect the failure.  This ensures that partitions will occur even in the case
of asymmetric network failures.  Without the protocol, only when the
coordinator notices the faulty member will the member be removed.

\begin{Protocol}
The protocol works by broadcasting slander messages to other members whenever
it recieves a new Suspect event.  On the receipt of such a message, DnSuspect
events are generated.
\end{Protocol}

\begin{Parameters}
\item None
\end{Parameters} 

\begin{Properties}
\item
If any member suspects another member of being faulty, all members will
eventually suspect that member.
\item
\Up{Suspect} events may cause a Slander message to be generated.
\end{Properties}

\begin{Sources}
\sourcesfile{slander.ml}
\end{Sources}

\begin{GenEvent}
\genevent{\Dn{Suspect}}
\end{GenEvent}

\begin{Testing}
\item
see also the VSYNC stack
\end{Testing}

\emph{This layer and its documentation were written by Zhen Xiao.}
\end{Layer}

\begin{Layer}{STABLE}

This layer tracks the stability of broadcast messages and does
failure detection.  It keeps track of and gossips about an
acknowledgement matrix, from which it occasionally computes the
number of messages from each member that are stable and delivers this
information in an \Dn{Stable} event to the layer below (which will be
bounced back up by a layer such as the BOTTOM layer).

\begin{Protocol}
The stability protocol consists of each member keeping track of its
view of an acknowledgement matrix.  In this matrix, each entry,
(A,B), corresponds to the number of member B's messages member A has
acknowledged (the diagonal entries, (A,A), contain the number of
broadcast messages sent by member A).  The minimum of column A
(disregarding entries for failed members) is the number of broadcast
messages from A that are stable.  The vector of these minimums is
called the stability vector.  The maximum of column A (disregarding
entries of failed members) is the number of broadcast messages member
A has sent that are held by at least one live member.  The vector of
the maximums is called the $NumCast$ vector \note{there has got to be
a better name}.  Occasionally, each member gossips its row to the
other members in the group.  Occasionally, the protocol layer
recomputes the stability and $NumCast$ vectors and delivers them up
in an \Dn{Stable} event.
\end{Protocol}

\begin{Parameters}
\item 
stable\_sweep: how often to (1) gossip and (2) deliver stability (if
it has changed)
\item 
stable\_explicit\_ack: whether to request end-to-end acknowledgements
for messages
\end{Parameters}

\begin{Properties}
\item
Unless it is marked with the \mlval{Unreliable} option all
DnCast events are counted by the STABLE layer and require eventual
acknowledgement by the other members in the group in order to achieve
stability.
\item
\Dn{Stable} events from the stability layer have two extension fields
set.  The first is the \mlval{StableVect} extension, which is the
vector of stability number of messages from each of the members in
the group which are known to be stable.  The second is the
\mlval{NumCast} extension which is a vector with the number of
broadcast messages each member in the group is known to have sent.
\item
\Dn{Stable} events are never delivered before all live members have
acknowledged at least the number of messages noted in the stability
event. (safety)
\item
\Dn{Stable} event will eventually be delivered after live members have
acknowledged message \mlval{seqno} from member A, where the entry in
the stable vector for member A is at least \mlval{seqno+1}. (liveness)
\item
The stability vectors in \Dn{Stable} events from the STABLE layer are
monotonically increasing.
\end{Properties}

\begin{Notes}
\item
\mlval{NumCast} entries are not monotonicly increasing.  For example,
consider the case of member A broadcasting some messages (which are
all dropped by the network), then broadcasting its gossip information
(which are recieved), then failing.  The other members may deliver
some UpStable events with the number of known broadcasts from member
A, in which the dropped broadcasts are counted.  However, after the
other members detect member A's failure, the \mlval{NumCast} entry
for member A will be lowered to be the number of messages from A that
the live members have recieved, which will be lower than when A was
not failed.
\item
\Up{Cast} events do not need to be acknowledged individually: an
acknowledgment, \mlval{Ack(from,seqno)}, is taken to acknowledge all
of the first \mlval{seqno} messages from the member with rank
\mlval{from}.
\item
An attempt has been made to speed up stability detection during view
changes by sending extra gossip messages when failures have occurred.
\end{Notes}

\begin{Sources}
\sourcesfile{stable.ml}
\end{Sources}

\begin{GenEvent}
\genevent{\Up{Stable}}
\genevent{\Dn{Cast}}
\genevent{\Dn{Timer}}
\end{GenEvent}

\begin{Testing}
\item see the VSYNC stack
\end{Testing}
\end{Layer}

%*************************************************************%
%
%    Ensemble, 1.10
%    Copyright 2001 Cornell University, Hebrew University
%    All rights reserved.
%
%    See ensemble/doc/license.txt for further information.
%
%*************************************************************%
\begin{Layer}{SUSPECT}

This layer regularly pings other members to check for suspected
failures.  Suspected failures are announce in a \DnSuspect\ event to
the layer below (which will be bounced back up by a layer such as the
BOTTOM layer).

\begin{Protocol}
Simple pinging protocol.  Uses a sweep interval.  On each sweep, Ping
messages are broadcast unreliably to the entire group.  Also, the
number of sweep rounds since the last Ping was received from other
members is checked and if it exceed the \mlval{max\_idle} threshold
then a \DnSuspect\ event is generated.  \hide{Suspicions are repeated
until a Ping message is received from the suspected member.}
\end{Protocol}

\begin{Parameters}
\item 
suspect\_sweep : how often to Ping other members and check for suspicions
\item
suspect\_max\_idle : number of unacknowledged Ping messages before generating
failure suspicions.
\end{Parameters}

\begin{Properties}
\item
Suspicions are no guarantee that an actual failure has occured, only a guess.
\end{Properties}

\begin{Notes}
\item None
\end{Notes}

\begin{Sources}
\sourcesfile{suspect.ml}
\end{Sources}

\begin{GenEvent}
\genevent{\DnSuspect}
\genevent{\DnCast}
\genevent{\DnTimer}
\end{GenEvent}

\begin{Testing}
\item see the VSYNC stack
\end{Testing}
\end{Layer}

\begin{Layer}{SYNC}

This layer implements a protocol for blocking a group during view changes.  One
member initiates the SYNC protocol by delivering a \DnBlock\ event from above.
Other members will receive an \UpBlock\ event.  After replying with a
\DnBlockOk, the layers above the SYNC layer should not broadcast any further
messages.  Eventually, after all members have responded to the \UpBlock\ and
all broadcast messages are stable, the member that delivered the \DnBlock\
event will recieve an \UpBlockOk\ event.

\begin{Protocol}
This protocol is very inefficient and needs to be reimplemented at some
point.  The Block request is broadcast by the coordinator.  All members
respond with another broadcast.  When the coordinator gets all replies, it
delivers up an \UpBlockOk.
\end{Protocol}

\begin{Parameters}
\item None
\end{Parameters}

\begin{Properties}
\item
Requires FIFO, reliable broadcasts with stability detection.
\item
Expects at most one \DnBlock\ from above.
\item
Always delivers at most one \UpBlockOk\ event.  Only delivers an \UpBlockOk\ if
a \DnBlock\ was recieved from above.
\item
When at least one member recieves a \DnBlock\ event, all live members will
eventually deliver an \UpBlock\ event.
\item
Expects at most one \DnBlockOk\ event from above.  Expects a \DnBlockOk\ from
above only if an \UpBlock\ event was previously delivered by this layer.
\item
Expects a \DnBlock\ to the layers below will be replied with an \UpBlock\ from
below.
\item
When all members have delivered a \DnBlockOk\ event from above and all
broadcast messages have been acknowledged (by non-failed members),
eventually all members who delivered a \DnBlock\ event will receive an
\UpBlockOk\ event from this layer.
\end{Properties}

\begin{Sources}
\sourcesfile{sync.ml}
\end{Sources}

\begin{GenEvent}
\genevent{\UpBlockOk}
\genevent{\DnAck}
\genevent{\DnBlock}
\genevent{DnCast}
\end{GenEvent}

\begin{Testing}
\item
The CHK\_SYNC protocol layer checks for SYNC safety conditions.
\item 
see also the VSYNC stack
\end{Testing}
\end{Layer}

%*************************************************************%
%
%    Ensemble, 1_42
%    Copyright 2003 Cornell University, Hebrew University
%           IBM Israel Science and Technology
%    All rights reserved.
%
%    See ensemble/doc/license.txt for further information.
%
%*************************************************************%
\begin{Layer}{TOPS}

This layer implements a lexicographic total ordering protocol.  (This is
a variation on the protocol developed as part of the Transis project.)

\begin{Protocol}
The protocol works by lexigraphically ordering messages. For example,
if group members are $\{ A, B, C \}$ and they send messages $A_1,
B_1,$ and $C_1,$ then the ordering will be $A_1 < B_1 < C_1$. Since
the ordering is fixed the protocol can get stuck if a member does not
send a message every timeout. For example, if the application messages
are $A_1$ and $C_1$ then the protocol would wait indefinitely for
$B_1$ and never deliver $C_1$. Hence, every member multicasts a null
message every timeout to maintain liveness. Currently, the timeout is
hardcoded to one second.

This protocol is not normally used because it has high latency 
if members do not multicast messages often. 
\end{Protocol}

\begin{Parameters}
\item None
\end{Parameters}

\begin{Properties}
\item
Requires VSYNC properties, implements the AGREE property. 
\end{Properties}

\begin{Sources}
\sourcesfile{tops.ml}
\end{Sources}

\end{Layer}
%*************************************************************%
%
%    Ensemble, 1_42
%    Copyright 2003 Cornell University, Hebrew University
%           IBM Israel Science and Technology
%    All rights reserved.
%
%    See ensemble/doc/license.txt for further information.
%
%*************************************************************%
\begin{Layer}{TOTEM}

This layer implements the rotating token protocol for total ordering.  (This is
a variation on the protocol developed as part of the Totem project.)

\begin{Protocol}
The protocol here is fairly simple: As soon as the stack becomes valid, the
lowest ranked member starts rotating a token in the group. In order to send a
message, a process must wait for the token. When the token arrives, all
buffered messages are broadcast, and the token is passed to the next member.
The token must be passed on even if there are no buffered messages.

If a view change occurs, messages are tagged as unordered and are send as
such.
When the \Up{View} event arrives, indicating that the group has successfully
been flushed, these messages are delivered in a deterministic order everywhere
(according to the ranks of their senders, breaking ties using FIFO).
\end{Protocol}

\begin{Parameters}
\item None
\end{Parameters}

\begin{Properties}
\item
Requires VSYNC properties and local delivery.
\end{Properties}

\begin{Sources}
\sourcesfile{totem.ml}
\end{Sources}

\begin{GenEvent}
\genevent{\Dn{Cast}}
\end{GenEvent}

\begin{Testing}
\item
\todo{}
\end{Testing}

\emph{This layer and its documentation were written by Roy Friedman.}
\end{Layer}

%*************************************************************%
%
%    Ensemble, 2_00
%    Copyright 2004 Cornell University, Hebrew University
%           IBM Israel Science and Technology
%    All rights reserved.
%
%    See ensemble/doc/license.txt for further information.
%
%*************************************************************%
\begin{Layer}{WINDOW} 

This layer implements window-based flow control based on stability information.
Multicast messages from each sender are sent only if the number of
unacknowledged messages from the sender is smaller than the window.

\begin{Protocol}
Whenever the number of unstable messages goes above the window, messages
are buffered without being sent.  On receipt of a stability update, the
number of unstable messages are recalculated and buffered messages are sent
as allowed by the window.
\end{Protocol}

\begin{Parameters}
\item window\_window : the window size in number of messages
\end{Parameters}

\begin{Properties}
\item
Requires stability information in the form of \Up{Stable} events.
\end{Properties}

\begin{Notes}
\item 
Future implementation should support dynamic window adjustment.
\item 
Performance with the WINDOW layer depends in part with the frequency of
stability updates.  The WINDOW flow control works the best when the
frequency is based on the number of unstable messages rather than on
periodic timeouts.
\item 
Alternative flow control layers include RATE and CREDIT.
\end{Notes}

\begin{Sources}
\sourcesfile{window.ml}
\end{Sources}

\emph{This layer and its documentation were written by Takako Hickey.}
\end{Layer}

\begin{Layer}{XFER}

This protocol facilitates application based state-transfer. 
The view structure contains a boolean field {\tt xfer\_view}
conveying whether the current view is one where
state-transfer is taking place ({\tt xfer\_view = true}) or whether it 
is a regular view ({\tt xfer\_view = false}).

\begin{Protocol}
It is assumed that an application initiates state-transfer after a view
change occurs. In the initial view, {\tt xfer\_view = true}. 
In a fault free run, 
each application sends pt-2-pt and multicast messages, according
to its state-transfer protocol. Once the application-protocol is
complete, an {\tt XferDone} action is sent to Ensemble. 
This action is caught by the Xfer layer, where each member sends a pt-2-pt
message {\tt XferMsg} to the leader. When the leader
collects {\tt XferMsg} from all members, the state-transfer is
complete, and a new view is installed with the {\tt xfer\_view} field
set to false. 

When faults occur, and members fail during the state-transfer
protocol, new views are installed with {\tt xfer\_view} set to {\tt
true}. This informs applications that state-transfer was not
completed, and they can restart the protocol. 
\end{Protocol}

\begin{Notes}
\item 
This layer allows the application to choose
the state-transfer protocol it wishes to use, the only constrain being
the XferDone actions. 

\item 
In the normal case, (a fault free run) the protocol should take a 
single view to complete. 
\end{Notes}

\begin{Parameters}
\item None
\end{Parameters}

\begin{Properties}
\item
Requires VSYNC properties.
\end{Properties}

\begin{Sources}
\sourcesfile{xfer.ml}
\end{Sources}

\end{Layer}


\begin{Layer}{ZBCAST}

The ZBCAST layer implements a gossip-style probabilistically reliable
multicast protocol.  Unlike most other protocols in \ensemble, this
protocol admits a small, but non-zero probability of message loss: a
message might be garbage collected even though some operational member
in the group has not received it yet.  We found that doing so can
offer dramatic improvements in the performance and scalability of the
protocol.


\begin{Protocol}
This protocol is composed of two sub-protocols structured roughly as in the
Internet MUSE protocol.  The first protocol is an unreliable multicast
protocol which makes a best-effort attempt to efficiently deliver each
message to its destinations.  The second protocol is a 2-phase
anti-entropy protocol that operates in a series of unsynchronized
rounds.  During each round, the first phase detects message loss; the
second phase corrects such losses and runs only if needed.
\end{Protocol}

\begin{Parameters}
\item zbcast\_fanout : the fanout of gossip messages.  This determines
how many destinations a member gossips to during each round.

\item zbcast\_sweep: the interval of each round.

\item zbcast\_idle: how many rounds to wait after the last
retransmission request of a message before that message can be garbage
collected.

\item zbcast\_max\_polls: the maximum number of destinations a member
can poll for missing messages during one round.

\item zbcast\_max\_reqs: the maximum number of retransmission requests
a member can make during one round.

\item zbcast\_max\_entropy: the maximum amount of data a member can
retransmit during one round.

\item zbcast\_req\_limit: the threshold for message retransmission
request before that request is multicasted to the whole group.

\item zbcast\_reply\_limit: the threshold for message retransmission
before that retransmission is multicasted to the whole group.
\end{Parameters}

\begin{Properties}
\item
Under some conservative assumptions about the network properties,
message delivery can be proved to have a bimodal distribution under
this protocol: with a very small probability the message will be
delivered to a small number of destinations(including failed ones);
with very high probability the message will be delivered to almost all
destinations; and with vanishingly low probability the message will be
delivered to \emph{many} but not \emph{most} destinations.

\item
Using this protocol for multicast transmissions, virtual synchrony
cannot be guaranteed since it admits a non-zero probability of message
losses at some operational members.  Message losses (if any) are
reported to the application.  If the message loss is deemed to
compromise correct behavior, the application may decide to leave the
group and then rejoins them, triggering state transfer -- a separate
feature provided by \ensemble.

\item
This protocol needs multicast support from underlying layers.  If
IP-multicast is not available, GCAST protocol is needed to simulate
the effect of multicast by a series of unicasts.

\item
As its current implementation, this protocol requires \emph{groupd}
membership services.

\item
This protocol assumes that the application is able to control its
message transmission within a certain rate (rate-based flow control).
If the load injected into the network is heavier than what it can
sustain, the failure probability and latency guarantees of the
protocol may no longer hold.
\end{Properties}

\begin{Sources}
\sourcesfile{zbcast.ml}
\end{Sources}

\begin{GenEvent}
\genevent{\Dn{Cast}}
\genevent{\Dn{Send}}
\genevent{\Up{LostMessage}}
\end{GenEvent}

\begin{Testing}
\item
Extensive experiments have been conducted on a SP2 parallel machine
(used as a network of UNIX workstations) with group size ranging from
$8$ to $128$ nodes.  The protocol scales gracefully and maintains
stable throughput.
\item
The protocol has been tested on a multicast-capable LAN with 30
Solaris workstations.  One of the member is the sender and the rest
are receivers.  The sender is sending at a rate of 200 messages per
second.  Each message is $1000$ bytes.  Most of the receivers are able
to maintain a steady throughput of $200$ msgs/sec.  Message losses are
very rare.

We emphasize that in both tests we have reached the limit of the
largest group of machines which we have access to.  We believe that
our protocol can scale far more than what is indicated above.
\item
In the next step of our work, we will investigate its performance on WAN.
\end{Testing}

\emph{This layer and its documentation were written by Zhen Xiao.  It
is based on the \emph{PBCAST} protocol implemented by Mark Hayden.
This documentation is based the \emph{Bimodal Multicast} paper.}
\end{Layer}

%*************************************************************%
%
%    Ensemble, 1_42
%    Copyright 2003 Cornell University, Hebrew University
%           IBM Israel Science and Technology
%    All rights reserved.
%
%    See ensemble/doc/license.txt for further information.
%
%*************************************************************%
\begin{Stack}{VSYNC}

Virtual synchrony is decomposed into a set of 8 independent protocol
layers, listed in figure~\ref{vsynclayers}.  The layers in this stack are
decribed in the layer section.

\begin{table}[b]
\begin{center}
\begin{tabular}{|l|l|l|}			   \hline
name		& purpose			\\ \hline
LEAVE		& reliable group leave		\\ \hline
INTER		& inter-group view management	\\ \hline
INTRA		& intra-group view management	\\ \hline
ELECT		& leader election		\\ \hline
MERGE		& reliable group merge		\\ \hline
SYNC		& view change synchronization	\\ \hline
PT2PT		& FIFO, reliable pt2pt		\\ \hline
SUSPECT		& failure suspcions		\\ \hline
STABLE		& broadcast stability		\\ \hline
MNAK		& FIFO, agreed broadcast	\\ \hline
BOTTOM		& bare-bones communication	\\ \hline
\end{tabular}
\end{center}
\caption{Virtual synchrony protocol stack}
\label{vsynclayers}
\end{table}

\todo{here describe the overall protocol created by composing all
the protocol layers}

\begin{Parameters}
\item \todo{composition of parameters below}
\end{Parameters}

\begin{Protocol}
\todo{composition of protocols below}
\end{Protocol}

\begin{Properties}
\item
\note{some form of composition of properties in layers}
\end{Properties}

\begin{Notes}
\item
Causal ordering can be introduced by replacing the MNAK layer with a
causal implementation of same protocol.
\item
Weak virtual synchrony can be implemented by removing the SYNC layer
and adding application support for managing multiple live protocol
stacks.
\end{Notes}

\begin{Testing}
\item
Use the various testing code described in the component layers.
\item
Version of Jan 12, 1996, tested with $>100000$ random failure scenarios.
\item
Version of April 10, 1997, tested with $>100000$ random failure scenarios.
\item
Random testing is done nightly on debugged VSYNC protocol stack.
\end{Testing}
\end{Stack}


\appendix
%*************************************************************%
%
%    Ensemble, 2_00
%    Copyright 2004 Cornell University, Hebrew University
%           IBM Israel Science and Technology
%    All rights reserved.
%
%    See ensemble/doc/license.txt for further information.
%
%*************************************************************%
\section{Appendix: ML Does Not Allow Segmentation Faults}

Normally, \ensemble\ should never experience segmentation faults.  When they occur,
there are only a few possible causes.  We list these below along with fixes.  Please
inform us if you detect other sources of ``unsafety'' in \ensemble.
\begin{itemize}
\item
One of the \ensemble\ extensions to \caml\ written in C (in the
\sourcecode{socket} directory) may have a bug.  Most (all?) extensions have
equivalent ML implementations.  Use those and see if the bug goes away.
\end{itemize}


%*************************************************************%
%
%    Ensemble, 1.10
%    Copyright 2001 Cornell University, Hebrew University
%    All rights reserved.
%
%    See ensemble/doc/license.txt for further information.
%
%*************************************************************%
\newenvironment{FormatTable}{%
\begin{quote}\begin{tabular}{|l|l|} \hline
}{\end{tabular}\end{quote}
}
\newcommand {\formatentry}[2]	{#1 & #2 \\ \hline}

\section{\ensemble\ Membership Service TCP Interface}
\label{section:memership}

\note{This is intended as an appendix to the Maestro paper (Maestro:
A Group Structuring Tool For Applications With Multiple Quality of
Service Requirements).  It describes the exact TCP messaging
interface to the group membership service described in that paper.}

The description here is of the nuts-and-bolts TCP interface to the
\mlval{maestro} membership service service described in the
\ensemble\ tutorial.  Ensemble also supports a direct interface to
this service in ML.  Developers using ML should probably use this
interface instead.  See \sourceappl{maestro/*.mli} for the source
code for the interface to this service.

\subsection{Locating the service}
The membership service uses the environment variable
\mlval{ENS\_GROUPD\_PORT} to select a TCP port number to use.  Client
processes connect to this port in the normal fashion for TCP
services.  Client processes can join any number of groups over a
single connection to a server, so they normally only connect once to
the servers.

If you run \mlval{groupd} on all the hosts from which your clients
will be using the service, then processes can connect to the local
port on their host.  However, clients are not limited to using local
servers, and can connect to any membership server on their system.

If the TCP connection breaks, the membership service will fail the
member from all groups that it joined.  However, a client can
reconnect to the same server and rejoin the groups it was in.  If
client's membership server crashes, it can reconnect to a different
server.

\subsection{Communicating with the service}
Communication with the service is done through specially formatted
messages.  We describe the message types and their format here.

\begin{description}
\item
[messages:] Messages in both directions are formatted as follows.
Both directions of the TCP streams are broken into variable-length
messages.  All messages are multiples of 4 bytes long and all fields
are aligned on $4$ byte boundaries.  The first $4$ bytes of a message
is a header.  The header consists of an unsigned integer in network
byte order (NBO), giving the length of the body of the message (not
including the header).  The next message follows immediately after
the body.
\item
[integers:] Integers are unsigned and are formatted as $4$ bytes in
NBO.
\item
[strings:] Strings have a somewhat more complex format.  The total
size of a string in bytes is multiple of $4$.  The first $4$ bytes
are an integer length (unsigned, NBO).  The length is not necessarily
a multiple of $4$ bytes.  The body of the string immidiately follows
the length.  Immediately following the body are $0-3$ bytes of
padding to bring the total length to a multiple of $4$ bytes.
\item
[endpoint and group identifiers:] These types have the same format as
strings.  For non-Ensemble applications, the contents can contain
whatever the transport service you are using requires.  \ensemble\
only tests the contents of endpoint and group identifiers for
equality with other endpoints and groups.
\item
[lists:] Lists have two parts.  The first is an integer giving the
number of elements in the list.  Immediately following that are the
elements in the list, one after the other and adjacent to
one-another.  It is assumed that the application knows the formats of
the items in the list in order to break them up.
\end{description}

\begin{figure}[tb]
\begin{center}
\resizebox{!}{3in}{\incgraphics{member_state}}
\end{center}
\caption{\em Client state machine diagram of the client-server membership protocol.}
\label{fig:state}
\end{figure}

The actual messages sent between the client and the servers are composed of integers
and strings.  The first field of a message is an integer \emph{tag} value from which
the format of the remainder of the message can be determined.

\begin{itemize}
\item 

\mlval{Coord\_View} : A new view is being installed.  The view is a
list of Endpt.id's.  A member who just sent a Join message may not be
included in the view, in which case it should await the next View
message.  The ltime is the logical time of the view.  The first entry
in the view and the ltime uniquely identify the view.  The ltime's
that a member sees grow monotonicly.  In addition, a boolean value is
sent specifying whether this view is a primary view.  The primary bit
is based on the primary bit of the group daemon's being used.
\begin{FormatTable}
\formatentry{integer}{Coord\_View = 0}
\formatentry{group}{my group}
\formatentry{endpoint}{my endpoint}
\formatentry{integer}{logical time}
\formatentry{boolean}{primary view}
\formatentry{endpoint list}{view of the group}
\end{FormatTable}
\item 
\mlval{Coord\_Sync} : All members should "synchronize" (usually this
means waiting for messages to stabilize) and then reply with a SyncOk
message.  The next view will not be sent until all members have
replied.
\begin{FormatTable}
\formatentry{integer}{Coord\_Sync = 1}
\formatentry{group}{my group}
\formatentry{endpoint}{my endpoint}
\end{FormatTable}
\item 
\mlval{Coord\_Failed} : Fail a member is being reported as having
failed.  This is done because members may need to know about failures
in order to determine when they are synchronized.
\begin{FormatTable}
\formatentry{integer}{Coord\_Failed = 2}
\formatentry{group}{my group}
\formatentry{endpoint}{my endpoint}
\formatentry{endpoint list}{failed endpoints}
\end{FormatTable}
\item 
\mlval{Member\_Join} : Request to join the group.  Replied with a
View message.
\begin{FormatTable}
\formatentry{integer}{Member\_Join = 3}
\formatentry{group}{my group}
\formatentry{endpoint}{my endpoint}
\formatentry{bool}{logical time}
\end{FormatTable}
\item 
\mlval{Member\_Sync} : This member is synchronized.  Is a reply to a
Sync message.  Will be replied with a View message.
\begin{FormatTable}
\formatentry{integer}{Member\_Sync = 4}
\formatentry{group}{my group}
\formatentry{endpoint}{my endpoint}
\end{FormatTable}
\item 
\mlval{Member\_Fail} : Fail other members in the group (or leave the
group by failing self).
\begin{FormatTable}
\formatentry{integer}{Member\_Fail = 5}
\formatentry{group}{my group}
\formatentry{endpoint}{my endpoint}
\formatentry{endpoint list}{failed endpoints}
\end{FormatTable}
\item 
\mlval{Client\_Version} : This is sent by the client process to tell
the server its version.  If the server's version is incompatible, the
server will send an Error message and close the client's connection.
The value to use for the version field can be found by running any of
the Ensemble demonstration programs with the \mlval{-v} flag.
\begin{FormatTable}
\formatentry{integer}{Member\_Version= 6}
\formatentry{string}{service name (``ENSEMBLE:groupd'')}
\formatentry{string}{my version (``0.40'')}
\end{FormatTable}
\item
\mlval{Server\_Error} : This is sent by the server.  An error has 
occurred.  Usually this means the client's connection will be closed.
\begin{FormatTable}
\formatentry{integer}{Server\_Error = 7}
\formatentry{string}{explanation}
\end{FormatTable}

\end{itemize}

%*************************************************************%
%
%    Ensemble, 2_00
%    Copyright 2004 Cornell University, Hebrew University
%           IBM Israel Science and Technology
%    All rights reserved.
%
%    See ensemble/doc/license.txt for further information.
%
%*************************************************************%
\section{Bimodal Multicast (by Ken Birman, Mark Hayden, and Zhen
Xiao)}

There are many methods for making a multicast protocol
\emph{reliable}.  The majority protocols in \ensemble\ aim to provide
virtually synchronous properties.  However, these properties come with
a price in terms of the possibility of unstable or unpredictable
performance under stress and limited scalability.  This is
unacceptable to some applications where system stability and
scalability are viewed as inextricable from other aspects of
reliability.

This section describes a bimodal multicast protocol that not only has
much better scalability properties but also provides predictable
reliability even under highly perturbed conditions.  This work is
described in the \emph{Bimodal Multicast} paper
(ncstrl.cornell/TR98-1683) by Ken Birman, Mark Hayden, Oznur Ozkasap,
Zhen Xiao, Mihai Budiu and Yaron Minsky (this documentation is based
on that paper).  The original version of the protocol was implemented
by Mark Hayden.  It was reimplemented by Zhen Xiao with many new
optimizations and is described in the {\bf ZBCAST} layer.  In the
remainder of the section, we will refer to our new protocol as Zbcast.

\subsection{Protocol description}

Zbcast protocol consists of two stages: 

\begin{itemize}
 \item During the first stage, the protocol multicasts each message
 using an unreliable multicast primitive.  IP multicast can be used to
 serve this purpose if it is available.  Otherwise a randomized
 tree-dissemination protocol is used (the GCAST layer).  In the latter
 case, the protocol uses the \ensemble\ group membership manager to
 track membership information.

 \item The second stage uses an anti-entropy protocol to detect and
 correct message losses in the group.  The protocol consists of a
 series of rounds.  In each round, every member randomly choose
 another member and exchange its message histories with him.  A member
 compares the received message history with its own.  If it detects
 itself to be lacking a message during the exchange, then it solicits
 copies of that message from the original sender.

 A member records the round in which a message is received.  The
 message will be gossiped until it has not been requested for
 retransmission for a prespecified number of rounds.  At that point
 the member will garbage collect that message.  Due to the
 probabilistic nature of the protocol, it is possible for a message to
 be garbage collected by other members while some operational member
 has not received it yet.  In such cases, the member missing the
 message will report a message gap to the application.
 
\end{itemize}


\subsection{Usage}

To use Zbcast protocol, specify the ``Zbcast'' property on the command
line as follows(using perf demo as an example):
\begin{codebox}
  perf -prog 1-n -add_prop Zbcast -groupd
\end{codebox}
This assumes that IP-multicast is available in the underlying
network.  Remember to set the related environment variables:
\begin{codebox}
ENS_DEERING_PORT=38350
ENS_MODES=Deering:UDP
\end{codebox}

Otherwise the Gcast layer needs be linked into the stack:
\begin{codebox}
  perf -prog 1-n -add_prop Zbcast -add_prop Gcast -groupd
\end{codebox}
Note that in both cases we need \emph{groupd} to track group
membership information.  This is the state of art of the current
implementation and is not something intrinsic to the protocol.  If
sufficient needs arise, we are going to remove this restriction.

Message losses are reported to the application via \Up{LostMessage}
event.  The application can either ignore those messages
(i.e. multimedia applications) or leave the process groups and then
rejoin them, triggering state transfer.

\section{The Ensemble client/server interface (by Robbert vanRenesse)}

Ensemble provides client/server interfaces.  We document in this section the
interface as provided for OCaml programmers, but the interfaces are also available as
C interfaces (through the use of the Camouflage tool).  Clients do not have to load
Ensemble to be able to talk to an Ensemble service.  Also, the server interface is
available independent of Ensemble if so desired.  Replicated servers will choose to
use Ensemble.  The Ensemble client/server (ECS) interface is scalable: since it uses
TCP as its communication mechanism, clients can invoke services from all over the
world.

\subsection{Linking}

All programs that use ECS should be linked with \sourcecode{lib/librpc.cma} and one
of the \sourcecode{lib/hsys\_*.cmo's}, depending on the environment.  For UNIX,
\sourcecode{hsys\_unix.cmo} is appropriate, whereas for Windows,
\sourcecode{hsys\_ntunix.cmo} or \sourcecode{hsys\_ntskt.cmo} can be used.

\subsection{Initialization}

All ECS processes have to open the Rpc module.  It makes available the Sockio module
in which the ECS procedures live.  To use ECS from an Ensemble process, the following
command needs to be used for initialization:

\begin{codebox}
  Sockio.ensemble_register (Alarm.add_sock alarm) (Alarm.rmv_sock alarm)
\end{codebox}
where \mlval{alarm} is typically returned by \mlval{Transport.alarm}.  Ensemble
programs also have to invoke \mlval{Sockio.sweep()} regularly (say every two seconds)
to keep things rolling along.  To use ECS without Ensemble, the following command
should be executed instead:
\begin{codebox}
  Sockio.sockio_register()
\end{codebox}

In this case, the server needs to run a main loop that typically looks like this:
\begin{codebox}
  while true do
    Sockio.select 2.0
  done
\end{codebox}
(In this case, the \mlval{select} wakes up every two seconds, and
automatically invokes \mlval{sweep}.)

\subsection{Creating an RPC server}

An RPC server would then use

where \mlval{port} is an integer giving the TCP port that is used for the server
(should typically be above 1000, or 0 if you don't care), and \mlval{eval rid cmd} is
a function that is invoked for each request message (cmd) that arrives.  The function
is invoked with a Request Identifier (rid) which is used for responding to requests.
Two kinds of responses are possible:
\begin{codebox}
  Sockio.response rid reply;
  Sockio.except rid description;
\end{codebox}

The first sends a normal response back to the client.  The second sends an
exceptional response.  (A third one is in the making allowing for redirecting
requests to different servers.)  The following is a simple example of a file server
that supports read and write commands:
\begin{codebox}
  open Rpc
  
  type file_cmd
    = FILE_READ of string
    | FILE_WRITE of string * string
  
  let eval rid cmd =
    let dispatch = function
      | FILE_READ name ->
          Sockio.response rid (File.read name)
      | FILE_WRITE name data ->
          File.write name data;
          Sockio.response rid ()
    in
    try
      dispatch cmd
    with
      | Sys_error descr ->
          Sockio.except rid descr
      | _ ->
          Sockio.except rid "something went wrong"
\end{codebox}
(Typically, the type declaration would be in a separate \mlval{.mli} file for use by
both the client and the server.)

An Ensemble server using the Appl interface would possibly queue commands for later
use, and invoke the \mlval{async()} downcall to generate an upcall that dequeues the
command and handles it.

For advanced users: the Request Identifier contains an identifier that may be used in
a replicated service to detect retransmissions of the same request.  This identifier
may be retrieved using the function \mlval{Sockio.mid\_of\_rid}.  A printable version
of the result may be invoked by subsequently calling \mlval{Sockio.string\_of\_mid}.

\subsection{Creating an RPC Client}

A client first needs to generate a handle for the server:
\begin{codebox}
   let file_svr = Sockio.bind_to_service "file"
\end{codebox}

If you are running the Ensemble White Pages service, this is all that need be done to
contact the file service.  If you don't want to run the White Pages service, this
command need be followed by one or more commands to specify the location of the
server or servers:
\begin{codebox}
   Sockio.add_to_service file_svr "snotra.cs.cornell.edu" port;
   Sockio.add_to_service file_svr "tumeric.cs.cornell.edu" port;
\end{codebox}

When an RPC is done, any of these services will be contacted.  Services that are not
currently reachable will be ignored.

The client can now invoke RPCs using the following command:
\begin{codebox}
  Sockio.rpc file_svr (FILE_READ "test") success failure
\end{codebox}

This commands sends the given command to the file server, and invokes the command
\mlval{success reply} if the server invokes \mlval{Sockio.response reply}, or
alternatively the command \mlval{failure description} if the server invokes
\mlval{Sockio.except description}.  Unfortunately, due to some typing difficulties,
both \mlval{reply} and \mlval{description} are of type \mlval{Obj.t}, and have to be
cast to the proper type using \mlval{Obj.magic}.  For example, to print the file
"test", you may write the following:
\begin{codebox}
  let complete data =
    printf "\%s" (Obj.magic data); flush stdout
  and abort descr =
    printf "Problem: \%s" (Obj.magic descr); flush stdout
  in
  Sockio.rpc file_svr (FILE_READ "test") complete abort
\end{codebox}

RPCs can not fail because of problematic network connections or faulty servers.
\mlval{Sockio.rpc} will retransmit the request to its servers until it receives a
response, be it positive or negative.  It is therefore possible that a request gets
executed multiple times, but as mentioned in the previous section, the servers can
filter out retransmissions.  In the future, we may add a time-out facility to RPCs.

Also, RPCs are non-blocking, and the responses are returned through upcalls.  You can
use OCaml threads to remedy this.  Alternatively, you can use Robbert's Goal-Oriented
Programming Style (as will be described somewhere soon) to deal effectively with an
upcall driven environment.

\subsection{Non-RPC clients and servers}

ECS doesn't enforce an RPC style on clients and servers.  Alternatively, just a
connection can be set up over which messages may be transmitted.  The server, in
that case invokes

\begin{codebox}
  Sockio.server "Name" port callback
\end{codebox}

Each time a connection with a client is established, callback is invoked with two
function arguments.  The first is the send routine for this connection, which takes a
message argument (of arbitrary type).  The second is a close routine which can close
the connection.  The callback has two return a tuple with two functions.  The first
is a routine that may be used to deliver messages (it takes a message argument).  The
second is a routine that is invoked when a failure occurs.

A client can set up a connection to a server using the following:
\begin{codebox}
  Sockio.client "Name" callback
\end{codebox}

The callback is very similar to the server's one, except that the failure routine
returns a boolean value.  If true, the connection is reestablished (perhaps with a
different server).  If false, the connection remains closed.

%%*************************************************************%
%
%    Ensemble, 2_00
%    Copyright 2004 Cornell University, Hebrew University
%           IBM Israel Science and Technology
%    All rights reserved.
%
%    See ensemble/doc/license.txt for further information.
%
%*************************************************************%
\section{Sessvr and Procsvr: Remote execution service}

\mlval{sessvr} together with \mlval{procsvr} provide
coordinated execution of commands on multiple machines\footnote{this
description was written by Takako Hickey}.
\mlval{sessvr} forms the front end of the execution service.
It maintains a database of machines in a cluster, and serves
as the interface to clients of the cluster.
\mlval{procsvr} forms the back end of the execution service.
It handles the execution of programs on the machine it is running.
To assist \mlval{sessvr} in selecting machines, \mlval{procsvr}
periodically informs \mlval{sessvr} of its resource usage status.

A client submits to \mlval{sessvr} a set of programs to be executed
along with criteria used to select machines on which programs
are executed (e.g., machine architecture, load, etc.)
\mlval{sessvr} selects machines that are appropriate (to balance load
ties are broken randomly) and forwards the
request to the procsvr of each machine.
\mlval{procsvr} performs the request and returns the response to sessvr,
which forwards the request to the client.

The execution service currently tolerates message losses.
It will tolerate crash failures in near future.


\b{Execution service interface}

This section describes the rpc interface client must use to
communicate with the execution server.
An example client program is \mlval{dsh}.


\begin{itemize}
\item request: SessCreate of string\\
   response: SessCreateSuccess of string * string\\

Before requesting execsvr to execute any commands, a session must
be created using SessCreate.  The argument to SessCreate is a name hint.
A unique session name is generated from it and returned as the second
argument to SessCreateSuccess.


\item request: ProcCreate of string * command array array * string array
     * (dbop * attrval) list array * (string * string) list * sessop\_flag\\
   response: ProcCreateSuccess of string * sessproc array\\

\begin{center}
\begin{tabular}{|l|}	   \hline
dbop	 \\ \hline
DBeq \\
DBlt \\
DBgt \\
DBlteq \\
DBgteq \\
DBmin of int \\
DBrandom of int \\
DBinclude of string \\ \hline
\end{tabular}
\end{center}

\begin{center}
\begin{tabular}{|l|}	   \hline
dbval	 \\ \hline
String of string \\
Int of int \\
Float of float \\
Endpt of Ensemble.Endpt.id \\
Addr of Unix.inet\_addr \\
StrList of string list \\
Noval \\ \hline
\end{tabular}
\end{center}

\begin{center}
attrval = string * dbval
\end{center}
   
\begin{center}
\begin{tabular}{|l|l|}  \hline
sessop\_flag		& description \\ \hline
AllOrNothing of unit	& if not enough machines, abort \\
MultLimit of int	& run upto n programs per machine \\
Unlimited of unit	& infinite programs per machine \\ \hline
\end{tabular}
\end{center}
   
\begin{center}
\begin{tabular}{|l|l|l|}  \hline
sessproc fields & type & description \\ \hline
seprocname	& string &		process name \\
seprogram 	& string &		program  \\
seenv		& string array &	environment \\
seprocsvr	& Ensemble.Endpt.id & 	procsvr address \\ \hline
\end{tabular}
\end{center}

Once session is created, processes can be added to it using ProcCreate.
Arguments to ProcCreate are session name, programs, environment,
machine specification, process properties, and flag.  Machine
specification is an array of list of criteria.  Select operation on
machine database is performed on each list of criteria until enough
machines are selected.  Programs are arrays of arrays of commands.
Each component of the outer array contains commands to be executed in
parallel.  Each of these components contains an array of the size
same as that of machine specification and corrensponds to command
to be executed based on machine selection criteria.
This would, for example, allow execution of different version of
a program based on machine architecture.
Flag indicates what to do when not enough machines are selected.


\item request: SessWait of string\\
   response: SessWaitSuccess of string\\
   request: ProcWait of string * string\\
   response: ProcWaitSuccess of string * string\\

Processes created can be waited using SessWait or ProcWait operations.
SessWait on session waits on all processes and returns SessWaitSuccess
when the last process finishes.  ProcWait on a process in a session
waits for a specific process in a session.


\item request: SessSig of string * sess\_sig \\
   response: SessSigSuccess of string * sess\_sig \\
   request: ProcSig of string * string * sess\_sig \\
   response: ProcSigSuccess of string * string * sess\_sig \\

\begin{center}
\begin{tabular}{|l|}  \hline
sess\_sig \\ \hline
SessSigKill \\
SessSigSuspend \\
SessSigResume \\ \hline
\end{tabular}
\end{center}

Processes created can be killed using SessSig or ProcSig operations.
Currently only the kill signal is supported.
\end{itemize}

%\newcommand {\projectheader}[0]	 {\section{Projects}}
%\newcommand {\projectsection}[1] {\subsection{#1}}
%\newcommand {\project}[1]	 {\subsubsection{#1}}
%%*************************************************************%
%
%    Ensemble, 1.10
%    Copyright 2001 Cornell University, Hebrew University
%    All rights reserved.
%
%    See ensemble/doc/license.txt for further information.
%
%*************************************************************%
\documentclass{article}
\usepackage{fullpage}
\newcommand {\note}[1]		{{\bf [#1]}}
\newcommand {\sourcecode}[1]	{{\bf {#1}}}
\newcommand {\mlval}[1]		{{\bf {#1}}}
\newcommand {\hide}[1]

\newcommand {\ensemble}[0]	{Ensemble}
\newcommand {\horus}[0]		{Horus}
\newcommand {\caml}[0]		{Objective Caml}
\newcommand {\project}[1]	{\subsection{#1}}

\title{Some projects related to \ensemble}
\author{Mark Hayden}

\begin{document}
\maketitle
This is a list of potential projects related to \ensemble.

\section{Distributed Systems}

\project{Rendezvous/Instrumentation} 
From its start, the Isis toolkit has had mechanisms for ``rendezvous'ing''
with already-running Isis application processes to get state dumps.
Implementing a similar mechanism for \ensemble\ to allow connecting with
\ensemble\ processes and investigating the protocol states would be a very
useful tool.

\project{``Universal'' Load Generator}
Currently, performance tests are difficult to run on \ensemble\ because we
do not have any testing infrastructure.  I propose building a suite of test
programs for generating different kinds of communication loads on the
system, monitoring system load, and testing a variety of performance
metrics.  This would make \ensemble\ much more useful for doing protocol
research.  This is not just ``make work'': this could be good research
because performance evaluation of distributed systems is a recurring
problem.

\project{Buffer Pool Management} 
With protocol layer bypassing, the main costs of using virtual synchrony
with \ensemble\ are (1) extra communication to detect broadcast stability and
(2) loss of memory locality due to the buffering of messages while awaiting
stability.  This project would involve building a special buffer management
system in C to decrease the impact of (2).  The buffer management system
should optimize for buffering and deallocation operations, and not for
buffer-lookup operations: the normal case is for a message to be buffered
and then later dropped, without ever looking it up.  It should find ways to
make sure buffered messages are packed closely together in memory.  One
more note: with the protocol bypass code, the buffering operation is the
only operation that involves memory allocation: if we can do this in C we
should dramatically reduce the number of garbage collections.

\project{Common Transport}
The C and ML implementations of \ensemble\ should have very similar requirements
on their communication transports.  This project would involve building a
common transport system in C and then interfacing it with ML so that we
could run both C and ML protocol stacks in the same address space.  Should
also improve the performance of the ML implementation considerably.

\project{Weak Virtual Synchrony}
Protocol stack switches on view changes support an easy implementation of
weak virtual synchrony.  The idea is to not wait until the group is blocked
before starting the new view.  Instead, create the new view and associated
protocol stacks immediately.  The application can then send and recieve
messages in the new view immediately, but has to be able to handle
receiving of messages in the old view until that becomes stable.
Implementing this probably require a new protocol layer or two and
extensions to the application interface.

\project{Unimplemented Protocol Layers} 

Some of the original \horus\ protocol layers are currently unimplemented in
\ensemble.  These include LWG (lightweight groups), CAUSAL (causal ordering),
PRIMARY (primary views).

\project{Textbook Protocols}
Our protocols are getting surprisingly simple.  It would be interesting to
try to find some ``textbook'' protocols that directly translate to the
\ensemble\ framework.  The more interesting possibilities for this are probably
the MNAK and the GMP protocols.

\section{Distributed Applications}

\project{Predicate Detection Server}

Developing and verifying new protocols is difficult because a protocol's
state is distributed across several processes.  A predicate detection
system is a mechanism for collecting all of the states in the system and
testing predicates against the states in a meaningful fashion (for example,
only against consistent cuts of the system).  Implementing and using such a
server in ML is facilitated by the general purpose marshaling primitives.
Some ideas here:
\begin{itemize}
\item
Allow different predicate detection policies to be used to determine which
cuts of the system the predicate is detected against.
\item
Incorporate this with a rendezvous mechanism to allow predicate detection
to be started on already-running groups.
\end{itemize}

\project{Reliable/Load-balanced Web Server}
\note{todo}

\project{Connection ID Server}

This is an application that will have to be built at some point.  The new
version of the \ensemble\ transports use hashed connection identifiers for
message routing.  These are assumed to be unique, and if they are not big
problems result.  A connection ID server would allocate new IDs and ensure
that they are unique.  It probably should not have to hand out new IDs for
every group view, but instead provide ``seeds'' with which groups can
generate guaranteed unique IDs on their own.

\section{Programming Language Related Projects}
\project{Compiling to C}
The protocol layers use only a simple subset of the ML language.  We should
be able to compile the protocol layers directly to C in a rather
straightforward fashion.  Some goals: (1) maintain formatting of the
original source, including comments, (2) detect message allocation and
generate inlined marshaling and unmarshaling code.

\project{Compiling Bypass Layers}
Layering introduces several performance problems in \ensemble\ (as well as other
systems).  Under certain conditions, stacks of layer perform staticly
determined sets of operations on events.  Special bypass layers can be
built which test for these conditions and then optimize out almost all the
layering overhead that protocol stacks normally suffer from.  Experiments
doing this by hand show that:
\begin{enumerate}
\item
The fast-path code can be made very small.
\item 
When the whole stack is being bypassed, state updates can be delayed until
after messages are sent.
\item
Event queuing can be optimized out.
\item
Events can be ``virtualized'' to eliminate their allocation and
deallocation.
\item 
Small, fixed size headers can be statically determined and used at runtime.
\end{enumerate}
Building a compiler to automate this process would involve:
\begin{enumerate}
\item 
Building a state machine representation of all layers to be bypassed.
\item 
Determining event traces through the layers that are suitable for bypass.
\item 
Using event traces for code generation.
\end{enumerate}

\project{Annotating ML Programs}
ML's static scoping should allow all variable references to be determined
at compile time.  We should be able to take a set of ML modules and
generate html files for each in which all references have be converted to
hyper-text links to the definition of the value or type.
\end{document}

\end{document}
