%*************************************************************%
%
%    Ensemble, 1.10
%    Copyright 2001 Cornell University, Hebrew University
%    All rights reserved.
%
%    See ensemble/doc/license.txt for further information.
%
%*************************************************************%
\section{Partitionable Execution Service: dsh and PEX}

\mlval{dsh} is a remote execution shell similar to Unix
\mlval{rsh} but
allows programs (distributed or not) to be started on multiple
machines\footnote{This description was written by Takako M. Hickey}.
A user can also specify kinds of machines he or she wants
programs to run.
 
\mlval{dsh} is actually an example client program for partitionable
execution service (PEX).  PEX allows clients to manage
multiple related long running processes via sessions.
PEX replicates sessions for high-availability
and ensures consistency of replica, so that processes will not be
lost due to machine failures or network partitions.
PEX consists of session servers (\mlval{sessvr}), which manage sessions,
and process servers (\mlval{procsvr}), which handle execution of
processes on individual machines.
 
To use \mlval{dsh}, PEX must also be setup.
One set of PEX can be shared by multiple users if they start
\mlval{dsh} with the same set of environment variables (described
shortly) as PEX.
Currently commands submitted to PEX via \mlval{dsh}
are run under the same environment as that of PEX.
 
PEX is an on-going project.  A separate research paper is available.


\subsection{Starting PEX}

All programs related to PEX (\mlval{groupd, sessvr, procsvr, pexmgr,
dsh}) must be started with environmental variables
\mlval{ENS\_ID, ENS\_MODES, ENS\_PORT, ENS\_GROUPD\_HOSTS,
ENS\_GROUPD\_PORT, ENS\_GOSSIP\_HOSTS, ENS\_GOSSIP\_PORT,
ENS\_EXECSVR\_HOSTS, ENS\_EXECSVR\_PORT, ENS\_DBDEFAULT}
set the same.
\mlval{ENS\_EXECSVR\_HOSTS} and \mlval{ENS\_EXECSVR\_PORT}
specify where to find \mlval{sessvr}.
\mlval{ENS\_DBDEFAULT} is an optional machine database file described
later.  e.g.,

\begin{verbatim}
%setenv ENS_ID              shared
%setenv ENS_MODES           UDP
%setenv ENS_PORT            9100
%setenv ENS_GROUPD_HOSTS    eclair0:eclair1
%setenv ENS_GROUPD_PORT     9200
%setenv ENS_GOSSIP_HOSTS    eclair0:eclair1
%setenv ENS_GOSSIP_PORT     9300
%setenv ENS_EXECSVR_HOSTS   eclair0:eclair1
%setenv ENS_EXECSVR_PORT    9400
%setenv ENS_DBDEFAULT       ~/ensemble/execsvr/dbinput
\end{verbatim}
 
To start PEX, start group membership service
\mlval{groupd} on any machine just like for any Ensemble application,
then start \mlval{sessvr} on any machines (the minimum of one, but
more than one will provide fault-tolerance of machine that runs
the server) and \mlval{procsvr} on each of the machines that
processes may run.

\begin{verbatim}
eclair0% groupd
eclair0% sessvr -groupd
eclair1% sessvr -groupd
eclair0% procsvr -groupd
eclair1% procsvr -groupd
eclair2% procsvr -groupd
\end{verbatim}

Once PEX is started, \mlval{dsh} can be used to run
programs.  For example to run two programs producer and consumer on
two ``least loaded'' machines (as defined by machine database at
\mlval{sessvr}) do:
\begin{verbatim}
% dsh -c "producer,consumer" -leastload 2
\end{verbatim}


\subsection{Using dsh}
The syntax of \mlval{dsh} is as follows:
\begin{verbatim}
dsh -c commands [machspec] [-p procspec]
\end{verbatim}

Use a comma to separate multiple commands to be executed in
parallel, and use a colon to separate multiple commands to
be executed sequentially.

machspec is 0 or more of:
\begin{center}
\begin{tabular}{|l|l|}					   \hline
option			& description			\\ \hline
-m machine\_list	& specific list of machines	\\
-name machine\_name		& specific machine	\\
-arch machine\_architecture	& machine architecture	\\
-os os\_type			& operating system type	\\
-data dataname			& has specific data	\\
-leastload n			& n least loaded	\\
-loadbount load			& bounded load	\\
-random n			& random n	\\ \hline
\end{tabular}
\end{center}


Some machines may have restrictions on types of programs it
allows to run.  To use machines with restrictions process must provide
specification of its properties as a list of attribute-value pairs
separated by semicolons.  e.g.,
\begin{verbatim}
-p "project=ensemble;owner=takako"
\end{verbatim}


\subsection{Editing machine database}
\mlval{sessvr} maintains a database of machines in the cluster
it serves.  Minimally, each entry contains information about
machine name, address, and load.  More information can be added either
at the startup of \mlval{sessvr} via database initialization file
or at runtime via a database editor \mlval{dbedit}.

The database initialization file contains a series of attribute-value
pairs.  Entries are separated by specification of the key attribute
``name''.  For example, the following groups the first three lines
into one entry.
\begin{verbatim}
name=eclair0
machtype=sparc
ostype=sunos4
name=eclair1
\end{verbatim}
The default initialization file is \mlval{execsvr/dbinput}.
A different file can be specified via environmental variable
\mlval{ENS\_DBDEFAULT}.


\mlval{dbedit} allows inspection and editing of the machine database
while \mlval{sessvr} is running.

Supported operations are:
\begin{center}
\begin{tabular}{|l|l|}				   \hline
command					& description	\\ \hline \hline
g key=val				& get entry \\
d key=val				& delete entry \\
c key=val;attr1=val1;attr2=val2;...	& change entry \\
h					& help \\ \hline
\end{tabular}
\end{center}

Currently all attributes are assumed to be of type string,
and no mechanism is provided to add in machine restrictions.

\subsection{Managing PEX}
To find out status of machines running PEX, execute
\mlval{pexmgr}.
\begin{verbatim}
% pexmgr
\end{verbatim}

To shutdown PEX, start \mlval{pexmgr}
with \mlval{-shutdown} option.
\begin{verbatim}
% pexmgr -shutdown
\end{verbatim}
