\documentclass{article}
\usepackage{fullpage}
\newcommand {\note}[1]		{{\bf [#1]}}
\newcommand {\sourcecode}[1]	{{\bf {#1}}}
\newcommand {\mlval}[1]		{{\bf {#1}}}
\newcommand {\hide}[1]

\newcommand {\ensemble}[0]	{Ensemble}
\newcommand {\horus}[0]		{Horus}
\newcommand {\caml}[0]		{Objective Caml}
\newcommand {\project}[1]	{\subsection{#1}}

\title{Some projects related to \ensemble}
\author{Mark Hayden}

\begin{document}
\maketitle
This is a list of potential projects related to \ensemble.

\section{Distributed Systems}

\project{Rendezvous/Instrumentation} 
From its start, the Isis toolkit has had mechanisms for ``rendezvous'ing''
with already-running Isis application processes to get state dumps.
Implementing a similar mechanism for \ensemble\ to allow connecting with
\ensemble\ processes and investigating the protocol states would be a very
useful tool.

\project{``Universal'' Load Generator}
Currently, performance tests are difficult to run on \ensemble\ because we
do not have any testing infrastructure.  I propose building a suite of test
programs for generating different kinds of communication loads on the
system, monitoring system load, and testing a variety of performance
metrics.  This would make \ensemble\ much more useful for doing protocol
research.  This is not just ``make work'': this could be good research
because performance evaluation of distributed systems is a recurring
problem.

\project{Buffer Pool Management} 
With protocol layer bypassing, the main costs of using virtual synchrony
with \ensemble\ are (1) extra communication to detect broadcast stability and
(2) loss of memory locality due to the buffering of messages while awaiting
stability.  This project would involve building a special buffer management
system in C to decrease the impact of (2).  The buffer management system
should optimize for buffering and deallocation operations, and not for
buffer-lookup operations: the normal case is for a message to be buffered
and then later dropped, without ever looking it up.  It should find ways to
make sure buffered messages are packed closely together in memory.  One
more note: with the protocol bypass code, the buffering operation is the
only operation that involves memory allocation: if we can do this in C we
should dramatically reduce the number of garbage collections.

\project{Common Transport}
The C and ML implementations of \ensemble\ should have very similar requirements
on their communication transports.  This project would involve building a
common transport system in C and then interfacing it with ML so that we
could run both C and ML protocol stacks in the same address space.  Should
also improve the performance of the ML implementation considerably.

\project{Weak Virtual Synchrony}
Protocol stack switches on view changes support an easy implementation of
weak virtual synchrony.  The idea is to not wait until the group is blocked
before starting the new view.  Instead, create the new view and associated
protocol stacks immediately.  The application can then send and recieve
messages in the new view immediately, but has to be able to handle
receiving of messages in the old view until that becomes stable.
Implementing this probably require a new protocol layer or two and
extensions to the application interface.

\project{Unimplemented Protocol Layers} 

Some of the original \horus\ protocol layers are currently unimplemented in
\ensemble.  These include LWG (lightweight groups), CAUSAL (causal ordering),
PRIMARY (primary views).

\project{Textbook Protocols}
Our protocols are getting surprisingly simple.  It would be interesting to
try to find some ``textbook'' protocols that directly translate to the
\ensemble\ framework.  The more interesting possibilities for this are probably
the MNAK and the GMP protocols.

\section{Distributed Applications}

\project{Predicate Detection Server}

Developing and verifying new protocols is difficult because a protocol's
state is distributed across several processes.  A predicate detection
system is a mechanism for collecting all of the states in the system and
testing predicates against the states in a meaningful fashion (for example,
only against consistent cuts of the system).  Implementing and using such a
server in ML is facilitated by the general purpose marshaling primitives.
Some ideas here:
\begin{itemize}
\item
Allow different predicate detection policies to be used to determine which
cuts of the system the predicate is detected against.
\item
Incorporate this with a rendezvous mechanism to allow predicate detection
to be started on already-running groups.
\end{itemize}

\project{Reliable/Load-balanced Web Server}
\note{todo}

\project{Connection ID Server}

This is an application that will have to be built at some point.  The new
version of the \ensemble\ transports use hashed connection identifiers for
message routing.  These are assumed to be unique, and if they are not big
problems result.  A connection ID server would allocate new IDs and ensure
that they are unique.  It probably should not have to hand out new IDs for
every group view, but instead provide ``seeds'' with which groups can
generate guaranteed unique IDs on their own.

\section{Programming Language Related Projects}
\project{Compiling to C}
The protocol layers use only a simple subset of the ML language.  We should
be able to compile the protocol layers directly to C in a rather
straightforward fashion.  Some goals: (1) maintain formatting of the
original source, including comments, (2) detect message allocation and
generate inlined marshaling and unmarshaling code.

\project{Compiling Bypass Layers}
Layering introduces several performance problems in \ensemble\ (as well as other
systems).  Under certain conditions, stacks of layer perform staticly
determined sets of operations on events.  Special bypass layers can be
built which test for these conditions and then optimize out almost all the
layering overhead that protocol stacks normally suffer from.  Experiments
doing this by hand show that:
\begin{enumerate}
\item
The fast-path code can be made very small.
\item 
When the whole stack is being bypassed, state updates can be delayed until
after messages are sent.
\item
Event queuing can be optimized out.
\item
Events can be ``virtualized'' to eliminate their allocation and
deallocation.
\item 
Small, fixed size headers can be statically determined and used at runtime.
\end{enumerate}
Building a compiler to automate this process would involve:
\begin{enumerate}
\item 
Building a state machine representation of all layers to be bypassed.
\item 
Determining event traces through the layers that are suitable for bypass.
\item 
Using event traces for code generation.
\end{enumerate}

\project{Annotating ML Programs}
ML's static scoping should allow all variable references to be determined
at compile time.  We should be able to take a set of ML modules and
generate html files for each in which all references have be converted to
hyper-text links to the definition of the value or type.
\end{document}
