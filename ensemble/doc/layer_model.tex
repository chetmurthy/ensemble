%*************************************************************%
%
%    Ensemble, 1_42
%    Copyright 2003 Cornell University, Hebrew University
%           IBM Israel Science and Technology
%    All rights reserved.
%
%    See ensemble/doc/license.txt for further information.
%
%*************************************************************%
\section{Layer Execution Model}

\subsection{Callbacks}
Layers are implemented as a set of callbacks that handle events passed to
the layer by \ensemble\ from some other protocol layer.  These callbacks can in
turn call callbacks that \ensemble\ has provided the layer for passing events
onto other layers.  Logically, a layer is initialized with one callback for
passing events to the layer above it, and another callback for passing
events to the layer below it.  After initialization, a layer returns two
callbacks to the system: one for handling events from the layer below it
and another for handling events from the layer above it.  In practice,
these ``logical callbacks'' are subdivided into several callbacks that
handle the cases where different kinds of messages are attached to events.

\begin{figure}[tb]
\begin{center}
\includegraphics{fig/automata}
\end{center}
\caption{\em Layers are executed as I/O automata, with pairs FIFO event
queues connecting adjacent layers.}
\label{fig:automata}
\end{figure}

\subsection{Ordering Properties}
The system infrastructure that handles scheduling of protocol layers and
the passing of events between protocol layers provides the following
guarantees:
\begin{description}
\item
[FIFO ordering] : The infrastructure guarantees that events passed between
two layers are delivered in order.  For instance, if layer A is stacked
above layer B, then all events layer A passes to layer B are guaranteed to
be delivered in FIFO order to layer B.  In addition, events that layer B
passes up to layer A are guaranteed to be delivered in FIFO order to layer
A.  Note that these ordering properties allow the scheduler some
flexibility in scheduling because they only specify the ordering of events
in a single channel between a pair of layers.
\item
[no concurrency] : The sytem infrastructure that hands events to layers
through the callbacks never invokes a layer's callbacks concurrently.  It
guarantees that at most one invocation of any callback is executing at a
time and that the current callback returns before another callback is made
to the protocol layer.  See fig~\ref{fig:automata} for a diagram of layer
automata.  Note that although a single layer may not be executed
concurrently, different layers \emph{may} be executed concurrently by a
scheduler.
\end{description}
The execution of a protocol stack can be visualized as a set of protocol
layers executing with a pair of event queues between each pair: one queue
for events going up and another for events going down.  The protocol layers
are then automata that repeatedly are scheduled to take pending events from
one of the adjacent incoming queues, execute it, and then deposit zero or
more events into two adjacent outgoing queues (see fig~\ref{fig:automata}).
