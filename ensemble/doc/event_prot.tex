\newcommand {\eventtype}[2]    {\item {#1:} #2}

\newenvironment{EventType}{%
\begin{itemize}
}{\end{itemize}
}

\newcommand {\chainentry}[2]    {#1 & #2 \\ \hline}

\newenvironment{ChainTable}{%
\begin{quote}\begin{tabular}{|l|l|} \hline
}{\end{tabular}\end{quote}
}

\section{Event protocol: Intra-stack communication}
\label{event:protocol}
\ensemble\ embodies two forms of communication.  The first is
communication between protocol stacks in a group, using messages sent
via some communication transport.  The second is intra-stack
communication between protocol layers sharing a protocol stack (see
fig~\ref{comm:event}), using \ensemble\ events (see
page~\pageref{event:module} for a overview of \ensemble events).  One
use of events is for passing information directly related to messages
(i.e., broadcast messages are usually associated with \ECast\ events).
However, events also are used for notifying layers of group membership
changes, telling other layers about suspected failed members,
synchronizing protocol layers for view changes, passing acknowledgment
and stability information, alarm requests and timeouts, etc\ldots.  In
order for a set of protocol layers to harmoniously implement a higher
level protocol, they have to agree on what these various events
``mean,'' and in general follow what is called here the \ensemble\
\emph{event protocol}.

The layering in \ensemble\ is advantageous because it allows complex protocols
to be decomposed into a set of smaller, more understandable protocols.
However, layering also introduces the event protocol which complicates the
system through the addition of intra-stack communication (the event
protocol) to inter-stack communication (normal message communication).

\emph{Be aware that this information may become out of date.  Although the
``spirit'' of the information presented here is unlikely to change in
drastic ways, always consider the possibility that this information does
not exactly match that in \sourcetype{event.ml} and \sourcetype{event.mli}.
Please alert us when such inconsistencies are discovered so they may be
corrected.}

\begin{figure}[tb]
\begin{center}
\resizebox{6in}{!}{\incgraphics{comm}}
%\scalebox{0.7}{\includegraphics{fig/comm}}
\end{center}
\caption{\em Events are used for intra-stack communication: layers can only
communicate with other layers by modifying events; layers never read or
modify other layer's message headers.  Messages are used for inter-stack
communication: only messages are sent between group members; events are
never sent between members.}
\label{comm:event}
\end{figure}

The documentation of the event protocol proceeds as follows.
\begin{itemize}
\item
``types'' of events are listed along with a summary of their meaning
\item
\todo{the types that usually have a message associated with them are
identified}
\item
event fields are described along with a summary of their usage
\item
\todo{a table is given showing the event types for which the various event
fields have defined values}
\item
the several \emph{event chains} which occur in protocol stacks are listed
(event chains are sequences of event micro-protocols that tend to occur in
\ensemble\ protocol stacks)
\end{itemize}

\subsection{Event Types}

\begin{figure}
\begin{codebox}

    (* These events should have messages associated with them. *)
  | ECast				(* Multicast message *)
  | ESend				(* Pt2pt message *)
  | ESubCast				(* Multi-destination message *)
  | ECastUnrel				(* Unreliable multicast message *)
  | ESendUnrel				(* Unreliable pt2pt message *)
  | EMergeRequest			(* Request a merge *)
  | EMergeGranted			(* Grant a merge request *)
  | EOrphan				(* Message was orphaned *)

    (* These types do not have messages. *)
  | EAccount				(* Output accounting information *)
(*| EAck			      *)(* Acknowledge message *)
  | EAsync				(* Asynchronous application event *)
  | EBlock				(* Block the group *)
  | EBlockOk				(* Acknowledge blocking of group *)
  | EDump				(* Dump your state (debugging) *)
  | EElect				(* I am now the coordinator *)
  | EExit				(* Disable this stack *)
  | EFail				(* Fail some members *)
  | EGossipExt				(* Gossip message *)
  | EGossipExtDir			(* Gossip message directed at particular address *)
  | EInit				(* First event delivered *)
  | ELeave				(* A member wants to leave *)
  | ELostMessage			(* Member doesn't have a message *)
  | EMergeDenied			(* Deny a merge request *)
  | EMergeFailed			(* Merge request failed *)
  | EMigrate				(* Change my location *)
  | EPresent                            (* Members present in this view *)
  | EPrompt				(* Prompt a new view *)
  | EProtocol				(* Request a protocol switch *)
  | ERekey				(* Request a rekeying of the group *)
  | ERekeyPrcl				(* The rekey protocol events *)
  | ERekeyPrcl2				(*                           *)
  | EStable				(* Deliver stability down *)
  | EStableReq				(* Request for stability information *)
  | ESuspect				(* Member is suspected to be faulty *)
  | ESystemError			(* Something serious has happened *)
  | ETimer				(* Request a timer *)
  | EView				(* Notify that a new view is ready *)
  | EXferDone				(* Notify that a state transfer is complete *)
  | ESyncInfo
      (* Ohad, additions *)
  | ESecureMsg				(* Private Secure messaging *)
  | EChannelList			(* passing a list of secure-channels *)
  | EFlowBlock				(* Blocking/unblocking the application for flow control*)
(* Signature/Verification with PGP *)
  | EAuth

  | ESecChannelList                     (* The channel list held by the SECCHAN layer *)
  | ERekeyCleanup
  | ERekeyCommit 
\end{codebox}
\caption{Event typ type definition.  Taken from \sourcetype{event.mli}.}
\label{fig:enum}
\end{figure}

This section describes the different types of events.  See
fig~\ref{fig:enum} for the source code of enumerated types. The
behavior of a layer depends not only on the event type and its
fields, but also on the \emph{direction} from which it arrives. For
example, an \ESend\ event travels in the sender stack from the
application down, and at the receiver from the bottom, up to the
application. The sender and receiver layers behave quite differently
depending on whether the message is sent or received. In what follows,
we sometimes specifically include the event direction. Detailed Descriptions: 

\begin{EventType}
\eventtype{\Up{Block}}{The coordinator is blocking the view. 
Is a reply to \Dn(Block); replied with \Dn{BlockOk}.}

\eventtype{\Dn{Block}}{The group is being blocked. Is a reply to
\Up{Suspect} and \Up{MergeRequest}; replied with \Up{Block}.}

\eventtype{\Up{BlockOk}}{The coordinator gets one of these events when the
group is blocked.  Is a reply to \Dn{BlockOk}; replied with \Dn{View} or
\Dn{MergeRequest}.}

\eventtype{\Dn{BlockOk}}{Is a reply to \Up{Block}; replied with
\Up{BlockOk} (but usually only at the coordinator).}

\eventtype{\Up{Cast}}{A member (whose rank is specified by the origin field)
broadcast a message to all members in the group.  Usually the broadcast is
delivered at all members except the sender. }

\eventtype{\Dn{Cast}}{A message is being broadcast.  
Replied with \Up{Cast} at all members but sender.}

\eventtype{\Up{Send}}{Another member sent us a pt2pt message.}
\eventtype{\Dn{Send}}{A message is being sent pt2pt.  Results in an
\Up{Send} at the destination.}

\eventtype{\ESubCast}{A message that will be multicast to a subset of
the group.}
\eventtype{\ECastUnrel}{A message that will be unreliablely multicasted}
\eventtype{\ESendUnrel}{A message that will be unreliablely sent point-to-point}

\eventtype{\Up{MergeRequest}}{Some other partition want to merge with us.
Replied with \Dn{MergeGranted} or \Dn{MergeDenied}.}

\eventtype{\Up{MergeGranted}}{Notification that a merge is ready to
proceed.}

\eventtype{\Dn{MergeGranted}}{Done by the coordinator after an
\Up{MergeRequest} to tell the other coordinator that the merge is
progressing.  Results in an \Up{MergeGranted} at the merging
coordinator.}

\eventtype{\Up{MergeDenied}}{This is notification that the coordinator of a
partition we tried to merge with has explicitly denied the merge.  Is a
reply to \Dn{MergeRequest}; replied with \Dn(View).}

\eventtype{\Dn{MergeDenied}}{Done by the coordinator after an \Up{MergeRequest}
to tell another coordinator that its request has been denied.  Is a reply
to \Up{MergeRequest}.}

\eventtype{\Up{MergeFailed}}{This is notification that some problem occurred
in an attempt to merge with another partition of our group.  Is a reply to
\Dn{MergeRequest}; replied with \Dn(View).}

\eventtype{\Up{Orphan}}{A message has lost its parent.  Usually it is OK to
ignore this message: it is just being delivered in case we are interested.}

\eventtype{\EAccount}{Used to control accounting
information. Periodically this event is passed through the stack, and
each layer can record its current information/performance.}

\eventtype{\EAsync}{Used to handle asynchronous events. Unused currently.}

\eventtype{\Up{Dump}}{A layer wants the stack to dump its state.  Usually the
top-most layer will bounce this down as a \Dn{Dump} and the members will dump
their state on the \Dn{Dump} event.}

\eventtype{\Dn{Dump}}{Dump your state.  Pass it on.  Is a reply to \Up{Dump}.}

\eventtype{\Up{Elect}}{This member has been elected to be coordinator of the
group. Usually means that this member will be generating failure and view
events and managing the group for the rest of the view.}

\eventtype{\Up{Exit}}{This protocol stack has been disabled (because of a
previous \Dn{Leave} event).  Layers should pass this event up and then do
nothing else.}

\eventtype{\Up{Fail}}{This is notification that some members have been marked
as failed.  This does not necessarily mean we will get no more messages
from the failed members.  The COM layer drops them, but messages from those
members retransmitted by other members are still delivered.  Usually, it
also means the coordinator has started or will start a new view soon (but
this is not necessary).  Is a reply to \Dn{Fail}.}

\eventtype{\Dn{Fail}}{Some members are being failed.  Is a reply to
\Up{Suspect}; replied with \Up{Fail}.}

\eventtype{\Up{GossipExt}}{A gossip message has arrived.  Note that
the data is in the extension fields.}

\eventtype{\Dn{GossipExt}}{Transmit a gossip message.  Note that data
is carried in extension fields of events and does not use the normal
header pushing mechanism.}

\eventtype{\Up{GossipExtDir}}{The same, but send a gossip message to a
specified address.}

\eventtype{\Up{Init}}{This is the first event that any layer receives.  It
should be passed up the stack.}

\eventtype{\Up{Leave}}{Some member (specified by the origin field) is leaving
the group.}

\eventtype{\Dn{Leave}}{We are leaving the group.  Replied with \Up{Exit} event.
Depending on when this is seen it can mean different things.  Before a new
view change it means we are really leaving the group.  After a new view, it
usually means that we have a new protocol stack taking part in the next
view and the \Dn{Leave} is just garbage collecting this protocol stack because
its view is over.}


\eventtype{\Up{LostMessage}}{This is notification that some protocol layer
below does not have a message it thinks it should have.  What this means is
usually highly-protocol-stack-specific.  Sometimes replied with \Dn{Fail}.}

\eventtype{\EMigrate}{Used to support migration of an endpoint between
transport addresses.}

\eventtype{\EPresent}{A notification that all group members currently
in the group are live and started ``talking''. This is useful in large
groups, where it takes time for \emph{all} members to synchronize and
agree on the view.}

\eventtype{\EPrompt}{Ask for a new view}

\eventtype{\EProtocol}{Ask for a new protocol}

\eventtype{\ERekey}{Request a rekey}

\eventtype{\ERekeyPrcl}{Used for inter-layer communication between the
rekeying layers. Used to separate out their event types.}

\eventtype{ERekeyPrcl2}{dito.}

\eventtype{\Up{Stable}}{This event contains stability information.  If we are
buffering broadcast messages, we can use this to decide which messages are
safe to drop.}

\eventtype{\EStableReq}{Ask for stability information}

\eventtype{\Up{Suspect}}{This is notification that some other layer (or
possibly some other member) thinks that some members should be kicked out
of the group.  Replied with \Dn{Fail} and often \Dn{Block}.}

\eventtype{\Dn{Suspect}}{Some members are suspected to be failed.
Replied with an \Up{Suspect} with the same members failed.}

\eventtype{\Up{SystemError}}{Something serious has happened.  Do whatever you
feel like because the world is about to fall apart.}

\eventtype{\Up{Timer}}{A timer has expired.  Pass it on.}

\eventtype{\Dn{Timer}}{A request for a timer alarm.  Replied with an
\Up{Timer} at or after the requested time.}

\eventtype{\Up{View}}{A new view is ready.  Note that this does not affect our
protocol: usually a different instance of our protocol stack will be
created to take care of the next view.  This event is not delivered at the
beginning of a view.  The \Up{View} event signals the end of a view.
\Up{Init} events are delivered at the beginning of a view.}

\eventtype{\Dn{View}}{A new view is prepared.  Usually followed by an
\Up{View}.  Usually does not affect the current protocol stack, but
later results in the creation of a new protocol stack for the new
view.}


\eventtype{\EXferDone}{Used for the state-transfer layer. Tells the
Xfer-layer that this endpoint has completed its state-transfer
protocol.}

\eventtype{\ESyncInfo}{Used inside the virtual-synchrony protocol.}

\eventtype{\ESecureMsg}{Send a secure private message on an encrypted
point-to-point channel to another member.}

\eventtype{\EChannelList}{Used to debug the secure-channel layer. The
event lists the set of current open secure-channels.}

\eventtype{\ESecChannelList}{dito.}

\eventtype{\EFlowBlock}{Blocking/unblocking the application for flow
control.}

\eventtype{\EAuth}{Signature/Verification with PGP.}

\eventtype{\ERekeyCleanup}{Erase all current open secure
point-to-point connections. This initializes the secure-channel cache.}

\eventtype{\ERekeyCommit}{Commit the current tentative group-key as
the key for the upcoming view}.

\end{EventType}


\subsection{Event fields}
Here we describe all the fields that appear in the events. The
type definitions appear in fig~\ref{fig:fields} and
fig~\ref{fig:extensions}.  Default values for the fields appear in
fig~\ref{fig:defaults}.

\begin{figure}
\begin{codebox}
type field =
      (* Common fields *)
  | Type        of typ            (* type of the message*)
  | Peer        of rank           (* rank of sender/destination *)
  | Iov	        of Iovecl.t       (* payload of message *)
  | ApplMsg                       (* was this message generated by an appl? *)

      (* Uncommon fields *)
  | Address     of Addr.set	  (* new address for a member *)
  | Failures    of bool Arrayf.t  (* failed members *)
  | Presence    of bool Arrayf.t  (* members present in the current view *)
  | Suspects    of bool Arrayf.t  (* suspected members *)
  | SuspectReason of string	  (* reasons for suspicion *)
  | Stability   of seqno Arrayf.t (* stability vector *)
  | NumCasts    of seqno Arrayf.t (* number of casts seen *)
  | Contact     of Endpt.full * View.id option (* contact for a merge *)

      (* HEAL gossip *)  
  | HealGos     of Proto.id * View.id * Endpt.full * View.t * Hsys.inet list
  | SwitchGos   of Proto.id * View.id * Time.t  (* SWITCH gossip *)
  | ExchangeGos	of string		(* EXCHANGE gossip *)

      (* INTER gossip *)
  | MergeGos    of (Endpt.full * View.id option) * seqno * typ * View.state
  | ViewState   of View.state	(* state of next view *)
  | ProtoId     of Proto.id	(* protocol id (only for down events) *)
  | Time        of Time.t	(* current time *)
  | Alarm       of Time.t	(* for alarm requests *)
  | ApplCasts   of seqno Arrayf.t
  | ApplSends   of seqno Arrayf.t
  | DbgName     of string

      (* Flags *)
  | NoTotal                     (* message is not totally ordered*)
  | ServerOnly	                (* deliver only at servers *)
  | ClientOnly	                (* deliver only at clients *)
  | NoVsync
  | ForceVsync
  | Fragment	                (* Iovec has been fragmented *)

      (* Debugging *)
  | History     of string       (* debugging history *)

      (* Private Secure Messaging *)
  | SecureMsg of Buf.t
  | ChannelList of (rank * Security.key) list
	
      (* interaction between Mflow, Pt2ptw, Pt2ptwp and the application *)
  | FlowBlock of rank option * bool

      (* Signature/Verification with Auth *)
  | AuthData of Addr.set * Auth.data

      (* Information passing between optimized rekey layers *)
  | Tree    of bool * Tree.z
  | TreeAct of Tree.sent
  | AgreedKey of Security.key

      (* The channel list held by the SECCHAN layer *)
  | SecChannelList of Trans.rank list
  | SecStat of int              (* PERF figures for SECCHAN layer *)
  | RekeyFlag of bool           (* Do a cleanup or not *)
\end{codebox}
\caption{Fields for events.  Taken from \sourcetype{event.mli}}
\label{fig:extensions}
\end{figure}

\subsubsection{Event Fields}
\begin{EventType}
\eventtype{Typ}{The type of the event.}

\eventtype{Flags}{A bitfield specifying a set of potential flags for the
event.}

\eventtype{Peer}{rank of sender/destination}

\eventtype{Iov}{Iovec list containing raw application data.}

\eventtype{ApplMsg}{was this message generated by an appl? This
sometimes requires a different treatment than other, system generated 
messages.}

\eventtype{Address}{A address for a member, used, for example, in 
sending gossip messages to specific endpoints.}

\eventtype{Failures}{List of ranks of members that have failed.}

\eventtype{Presence}{The list of members present in the current
view. Used in large groups.}

\eventtype{Suspects}{List of ranks of members that are suspected to have
failed or be faulty in some way.}

\eventtype{SuspectReason}{String containing ``reason'' for suspecting
members.  Used for debugging purposes.}

\eventtype{Stability}{Vector of number of broadcasts for each member in
the group that are stable.}

\eventtype{NumCasts}{Vector of number of known broadcasts for members in
the group.}

\eventtype{Contact}{Endpoint of contact used for communication to endpoints
outside of group.  Usually only in merge events.}

\eventtype{HealGos} {The type of gossip messages sent between HEAL layers.}

\eventtype{SwitchGos}{dito, for SWITCH layers.}

\eventtype{ExchangeGos}{dito, for EXCHANGE layers.}

\eventtype{MergeGos}{dito, for MERGE layers.}

\eventtype{ViewState}{Used to pass around view state for new views}

\eventtype{ProtoId}{A new protocol id that we are switching to.}

\eventtype{Time}{Time that the event/message was received.}

\eventtype{Alarm}{Requested time for an alarm.}

\eventtype{ApplCasts}{The sequence numbers of the latest multicast
messages sent in the group.}

\eventtype{ApplSends}{dito, for send messages}

\eventtype{NoTotal}{Flag: specifying that this message should not be
totally ordered even if total a totally-ordered stack is in use. Used
to send Ensemble control messages which should not be delayed. Fail
and View messages are sent this way.}

\eventtype{ServerOnly}{Flag: Deliver only at servers.}

\eventtype{ClientOnly}{Flag: Deliver only at clients.}

\eventtype{NoVsync}{Flag: This message is not subject to virtual synchrony}
\eventtype{ForceVsync}{Flag: this message \emph{must} be subject to
virtual synchrony}

\eventtype{Fragment}{Flag: This event contains a fragment of an Iovec.}

\eventtype{SecureMsg}{Send this buffer through a secure-channel
point-to-point to another member.}

\eventtype{ChannelList}{Describe the secure channel list.}

\eventtype{FlowBlock}{Used for interaction the flow control layers: Mflow, Pt2ptw, Pt2ptwp and the application}

\eventtype{AuthData}{Send a buffer for PGP to check, and receive a
checked buffer from PGP}

\end{EventType}

\subsection{Event fields and the ``types'' for which they are defined}
\note{TODO}

\subsection{Event Chains}
We describe here common event sequences, or chains, in \ensemble.
Event chains are sequences of alternate up and down events that
ping-pong up and down the protocol stack bouncing between the two
end-layers of the chain.  The end layers are typically the the top and
bottom-most layers in the stack (eg., TOP and BOT).  The most common
exceptions to this are the message chains (Sends and Broadcasts),
which can have any layer for their top layer.

Note that these sequences are just prototypical.  Necessarily, there
are variations in which of layers see which parts of these sequences.
For example, consider the Failure Chain in a virtual synchrony stack
with the GMP layer.  The Failure Chain begins at the coordinator with
an \ESuspect\ event initiated at any layer in the stack.  The BOT
layer bounces this up as an \ESuspect\ event.  The top-most layer
usually responds with a \EFail\ event.  The \EFail\ event passes
down through all the layers until it gets to the GMP layer.  The GMP layer
at the coordinator both passes the \EFail\ event to the layer below
and passes down a \ECast\ event (thereby beginning a Broadcast
Chain\ldots).  At the coordinator, the \EFail\ event bounces off of
the BOT layer as an \EFail\ event and then passes up to the top of the
protocol stack.  At the other members, an \ECast\ event will be received
at the GMP layer.  The message is marked as a ``Fail'' message, so the
GMP layers generate and send down an \EFail\ event (just like the one
at the coordinator) and this is also bounced off the BOT layer as an
\EFail\ event.  The lesson here is that the different layers in the
different members of the group all essentially saw the same Failure
Chain, but exact sequencing was different.  For example, the layers
above the GMP layer at the members other than the coordinator did not
see a \EFail\ event. \todo{give diagram}

\todo{Leave Chain}

\subsubsection{Timer Chain}
Request for a timer, followed by an alarm timeout.
\begin{ChainTable}
\chainentry{\ETimer}{down: timeout requested, sent down to BOT.}
\chainentry{\ETimer}{up: alarm generated in BOT at or after requested
time, and sent up.}
\end{ChainTable}

\subsubsection{Send Chain}
Send a pt2pt message followed by stability detection.
\begin{ChainTable}
\chainentry{\ESend}{down: send a pt2pt message down.}
\chainentry{\ESend}{up: destinations receive the message}
\chainentry{\EStable}{message eventually becomes stable, and stability
information is bounced off BOT.}
\end{ChainTable}

\subsubsection{Broadcast Chain}
Broadcast of a message followed by stability detection.  
\begin{ChainTable}
\chainentry{\ECast}{down: broadcast a message}
\chainentry{\ECast}{up: other members receive the broadcast}
\chainentry{\EStable}{broadcast eventually becomes stable, and
stability information is bounced off BOT}
\end{ChainTable}

\subsubsection{Failure Chain}
Suspicion and ``failure'' of group members.
\begin{ChainTable}
\chainentry{\ESuspect}{down: suspicion of failures generated at any layer}
\chainentry{\ESuspect}{up: notification of suspicion of failures}
\chainentry{\EFail}{down: coord fails suspects}
\chainentry{\EFail}{up: all members get failure notice}
\end{ChainTable}

\subsubsection{Block Chain}
Blocking of a group prior to a membership change.
\begin{ChainTable}
\chainentry{\ESuspect/\EMergeRequest}{up: reasons for coord blocking}
\chainentry{\EBlock}{down: coord starts blocking}
\chainentry{\EBlock}{up: all members get block notice}
\chainentry{\EBlockOk}{down: all members reply to block notice}
\chainentry{\EBlockOk}{up: coord get block OK notice}
\chainentry{\EMergeRequest\/~\EView}{down: coord begins Merge or View chain}
\end{ChainTable}

\subsubsection{View Chain}
Installation of a new view, followed by garbage collection of the old view.
\begin{ChainTable}
\chainentry{\EView}{down: coord begins view chain (after failed merge or blocking)}
\chainentry{\EView}{up: all members get view notice}
\chainentry{\EExit}{down: protocol stacks are ready for garbage collection \note{todo}}
\chainentry{\EExit}{up: protocol stacks are garbage collected}
\end{ChainTable}

\subsubsection{Merge Chain (successful)}
Partition A merges with partition B, followed by garbage collection of the
old view.  We focus on partition A and only give a subset of events in
partition B.
\begin{ChainTable}
\chainentry{\EMergeRequest}{down: coord A begins merge chain (after blocking)}
\chainentry{\EMergeRequest}{up: coord B gets merge request}
\chainentry{\EMergeGranted}{down: coord B replies to merge request}
\chainentry{\EMergeGranted}{up: coord A gets merge OK notice}
\chainentry{\EView}{down: coord A installs new view for coord B}
\chainentry{\EView}{up: all members in group A get view notice}
\chainentry{\EExit}{down: protocol stacks are ready for garbage collection}
\chainentry{\EExit}{up: protocol stacks are garbage collected}
\end{ChainTable}
 \todo{\EExit\ above is currently \ELeave}

\subsubsection{Merge Chain (failed)}
Failed merge, followed by installation of a view.
\begin{ChainTable}
\chainentry{\EMergeRequest}{down: coord begins merge chain (after blocking)}
\chainentry{\EMergeFailed\ or}{}
\chainentry{\EMergeDenied}{up: coord detect merge problem}
\chainentry{\EView}{down: coord begins view chain}
\end{ChainTable}
