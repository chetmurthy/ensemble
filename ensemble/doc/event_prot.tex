\newcommand {\eventtype}[2]	{\item {#1:} #2}

\newenvironment{EventType}{%
\begin{itemize}
}{\end{itemize}
}

\newcommand {\chainentry}[2]	{#1 & #2 \\ \hline}

\newenvironment{ChainTable}{%
\begin{quote}\begin{tabular}{|l|l|} \hline
}{\end{tabular}\end{quote}
}

\section{Event protocol: Intra-stack communication}
\label{event:protocol}
\ensemble\ embodies two forms of communication.  The first is communication
between protocol stacks in a group, using messages sent via some
communication transport.  The second is intra-stack communication between
protocol layers sharing a protocol stack (see fig~\ref{comm:event}), using
\ensemble\ events (see page~\pageref{event:module} for a overview of \ensemble
events).  One use of events is for passing information directly related to
messages (i.e., broadcast messages are usually associated with \UpCast\ or
\DnCast\ events).  However, events also are used for notifying layers of
group membership changes, telling other layers about suspected failed
members, synchronizing protocol layers for view changes, passing
acknowledgment and stability information, alarm requests and timeouts,
etc\ldots.  In order for a set of protocol layers to harmoniously implement
a higher level protocol, they have to agree on what these various events
``mean,'' and in general follow what is called here the \ensemble\ \emph{event
protocol}.

The layering in \ensemble\ is advantageous because it allows complex protocols
to be decomposed into a set of smaller, more understandable protocols.
However, layering also introduces the event protocol which complicates the
system through the addition of intra-stack communication (the event
protocol) to inter-stack communication (normal message communication).

\emph{Be aware that this information may become out of date.  Although the
``spirit'' of the information presented here is unlikely to change in
drastic ways, always consider the possibility that this information does
not exactly match that in \sourcetype{event.ml} and \sourcetype{event.mli}.
Please alert us when such inconsistencies are discovered so they may be
corrected.}

\begin{figure}[tb]
\begin{center}
\resizebox{6in}{!}{\incgraphics{comm}}
\end{center}
\caption{\em Events are used for intra-stack communication: layers can only
communicate with other layers by modifying events; layers never read or
modify other layer's message headers.  Messages are used for inter-stack
communication: only messages are sent between group members; events are
never sent between members.}
\label{comm:event}
\end{figure}

The documentation of the event protocol proceeds as follows.
\begin{itemize}
\item
``types'' of events are listed along with a summary of their meaning
\item
\todo{the types that usually have a message associated with them are
identified}
\item
event fields are described along with a summary of their usage
\item
\todo{a table is given showing the event types for which the various event
fields have defined values}
\item
the several \emph{event chains} which occur in protocol stacks are listed
(event chains are sequences of event micro-protocols that tend to occur in
\ensemble\ protocol stacks)
\end{itemize}

\note{Some possible changes (sugested by Takako): \UpMergeRequest\ to
UpMerge, \DnMergeGranted\ to DnMergeOk, \UpMergeGranted\ to UpMergeOk.}

\subsection{Up Event Types}

\begin{figure}
\begin{codebox}

  (* Event types *)
type typ =
  | EAccount				(* Output accounting information *)
  | EAck				(* Acknowledge message *)
  | EAlive				(* added by Ohad *)
  | EAsync				(* Asynchronous application event *)
  | EBlock				(* Block the group *)
  | EBlockOk				(* Acknowledge blocking of group *)
  | ECast				(* Multicast a message *)
  | EDump				(* Dump your state (debugging) *)
  | EElect				(* I am now the coordinator *)
  | EExit				(* Finished with this stack *)
  | EFail				(* Fail some members *)
  | EGossipExt				(* Gossip message *)
  | EInit				(* First event delivered *)
  | EInvalid				(* Erroneous event type *)
  | ELeave				(* A member wants to leave *)
  | ELostMessage			(* Member doesn't have a message *)
  | EMerge				(* Request a merge *)
  | EMergeDenied			(* Deny a merge request *)
  | EMergeFailed			(* Merge request failed *)
  | EMergeGranted			(* Grant a merge request *)
  | EMergeRequest			(* Other group wants to merge *)
  | EMigrate				(* Change my location *)
  | EOrphan				(* Message was orphaned *)
  | EPrompt				(* Prompt a new view *)
  | EProtocol				(* Request a protocol switch *)
  | ESend				(* Received pt2pt message *)
  | EStable				(* Deliver stability down *)
  | ESuspect				(* Member is suspected to be faulty *)
  | ESystemError			(* Something serious has happened *)
  | ETimer				(* Request a timer *)
  | ETransView				(* added by Ohad *)
  | EView				(* Notify that a new view is ready *)
  | EXferDone				(* Notify that a state transfer is complete *)
\end{codebox}
\caption{Event typ type definition.  Taken from \sourcetype{event.mli}.}
\label{fig:upenum}
\end{figure}

This section describes the different types of events.  See
fig~\ref{fig:upenum} for the source code of enumerated types.  Detailed
Descriptions:
\begin{EventType}
\eventtype{\UpBlock}{The coordinator is blocking the view.  Is a reply to
\DnBlock; replied with \DnBlockOk.}

\eventtype{\UpBlockOk}{The coordinator gets one of these events when the
group is blocked.  Is a reply to \DnBlockOk; replied with \DnView\ or
\DnMerge.}

\eventtype{\UpCast}{A member (whose rank is specified by the origin field)
broadcast a message to all members in the group.  Usually the broadcast is
delivered at all members except the sender.  Replied with a \DnAck\ event (by
the layer that initiated the \DnCast\ event).}

\eventtype{\UpDump}{A layer wants the stack to dump its state.  Usually the
top-most layer will bounce this down as a \DnDump\ and the members will dump
their state on the \DnDump\ event.}

\eventtype{\UpElect}{This member has been elected to be coordinator of the
group.  Usually means that this member will be generating failure and view
events and managing the group for the rest of the view.}

\eventtype{\UpExit}{This protocol stack has been disabled (because of a
previous \DnLeave\ event).  Layers should pass this event up and then do
nothing else.}

\eventtype{\UpFail}{This is notification that some members have been marked
as failed.  This does not necessarily mean we will get no more messages
from the failed members.  The COM layer drops them, but messages from those
members retransmitted by other members are still delivered.  Usually, it
also means the coordinator has started or will start a new view soon (but
this is not necessary).  Is a reply to \DnFail.}

\eventtype{\UpInit}{This is the first event that any layer receives.  It
should be passed up the stack.}

\eventtype{\UpInvalid}{It is usually an error to get an event of this type.}

\eventtype{\UpLeave}{Some member (specified by the origin field) is leaving
the group.}

\eventtype{\UpLostMessage}{This is notification that some protocol layer
below does not have a message it thinks it should have.  What this means is
usually highly-protocol-stack-specific.  Sometimes replied with \DnFail.}

\eventtype{\UpMergeDenied}{This is notification that the coordinator of a
partition we tried to merge with has explicitly denied the merge.  Is a
reply to \DnMerge; replied with \DnView.}

\eventtype{\UpMergeFailed}{This is notification that some problem occurred
in an attempt to merge with another partition of our group.  Is a reply to
\DnMerge; replied with \DnView.}

\eventtype{\UpMergeGranted}{Notification that a merge is ready to
proceed. \note{I'm not sure what should be done with this}}

\eventtype{\UpMergeRequest}{Some other partition want to merge with us.
Replied with \DnMergeGranted\ or \DnMergeDenied.}

\eventtype{\UpOrphan}{A message has lost its parent.  Usually it is OK to
ignore this message: it is just being delivered in case we are interested.}

\eventtype{\UpSend}{Another member sent us a pt2pt message.  Usually it is
replied with \DnAck.}

\eventtype{\UpStable}{This event contains stability information.  If we are
buffering broadcast messages, we can use this to decide which messages are
safe to drop.}

\eventtype{\UpSuspect}{This is notification that some other layer (or
possibly some other member) thinks that some members should be kicked out
of the group.  Replied with \DnFail\ and often \DnBlock.}

\eventtype{\UpSystemError}{Something serious has happened.  Do whatever you
feel like because the world is about to fall apart.}

\eventtype{\UpTimer}{A timer has expired.  Pass it on.}

\eventtype{\UpGossip}{A gossip message has arrived.  Note that the data is
in the extension fields.}

\eventtype{\UpView}{A new view is ready.  Note that this does not affect our
protocol: usually a different instance of our protocol stack will be
created to take care of the next view.  This event is not delivered at the
beginning of a view.  The \UpView\ event signals the end of a view.
\UpInit\ events are delivered at the beginning of a view.}
\end{EventType}

\subsection{Down Event Types}

This section describes the different types of down events. Detailed
descriptions:
\begin{EventType}
\eventtype{\DnAck}{An acknowledgment of a message event.  Only up events
that have the ack field set to something other than \mlval{NoAck} need to
be acknowledged.  Usually, the layer that initiated a message event is the
one that should acknowledge it.}

\eventtype{\DnBlock}{The group is being blocked.  Is a reply to \UpSuspect\ and
\UpMergeRequest; replied with \UpBlock.}

\eventtype{\DnBlockOk}{Is a reply to \UpBlock; replied with \UpBlockOk\ (but
usually only at the coordinator).}

\eventtype{\DnCast}{A message is being broadcast.  Replied with \UpCast\ at all
members but sender.}

\eventtype{\DnDump}{Dump your state.  Pass it on.  Is a reply to UpDump.}

\eventtype{\DnFail}{Some members are being failed.  Is a reply to \UpSuspect;
replied with \UpFail.}

\eventtype{\DnGossip}{Transmit a gossip message.  Note that data is carried
in extension fields of events and does not use the normal header pushing
mechanism.}

\eventtype{\DnInvalid}{It is usually an error to pass events of this type around.}

\eventtype{\DnLeave}{We are leaving the group.  Replied with \UpExit\ event.
Depending on when this is seen it can mean different things.  Before a new
view change it means we are really leaving the group.  After a new view, it
usually means that we have a new protocol stack taking part in the next
view and the \DnLeave\ is just garbage collecting this protocol stack because
its view is over.\todo{split these uses into two events: \DnLeave\ and
\DnExit}}

\eventtype{\DnMerge}{We are going to try to merge with another partition.
Usually done only by the coordinator.  Is a reply to \UpBlockOk; replied
with \UpView\ (sent by the remote coordinator).}

\eventtype{\DnMergeDenied}{Done by the coordinator after an \UpMergeRequest
to tell another coordinator that its request has been denied.  Is a reply
to \UpMergeRequest.}

\eventtype{\DnMergeGranted}{Done by the coordinator after an \UpMergeRequest
to tell the other coordinator that the merge is progressing.  Results in an
\UpMergeGranted\ at the merging coordinator.}

\eventtype{\DnSend}{A message is being sent pt2pt.  Results in an \UpSend\ at the
destination.}

\eventtype{\DnStable}{\note{do note use}}

\eventtype{\DnSuspect}{Some members are suspected to be failed.  Replied with an
\UpSuspect\ with the same members failed.}

\eventtype{\DnTimer}{A request for a timer alarm.  Replied with an \UpTimer\ at or
after the requested time.}

\eventtype{\DnView}{A new view is prepared.  Usually followed by an \UpView.
Usually does not affect the current protocol stack, but later results in the
creation of a new protocol stack for the new view.}
\end{EventType}

\subsection{Events and Messages}

\subsection{Event fields}
Here we describe all the fields that appear in up and down events.  The
type definitions appear in fig~\ref{fig:fields} and
fig~\ref{fig:extensions}.  Default values for the fields appear in
fig~\ref{fig:defaults}.

\begin{figure}
\begin{codebox}

type field =
  | Type	of typ
  | Origin	of rank
  | Ack	        of acknowledgement
  | Ranks       of rank list
  | Iov	        of Iovec.tl
  | Address     of Addr.tl		(* new address for a member *)
  | Failures	of rank list		(* failed members *)
  | Suspects	of rank list            (* suspected members *)
  | SuspectReason of string		(* reasons for suspicion *)
  | Stability	of seqno array		(* stability vector *)
  | NumCasts	of seqno array		(* number of casts seen *)
  | Mergers	of View.state		(* list of merging members *)
  | Contact	of Endpt.full * View.id option (* contact for a merge *)
  | HealGos	of Proto.id * View.id * Endpt.full * View.t (* HEAL gossip *)
  | SwitchGos	of Proto.id * View.id * Time.t  (* SWITCH gossip *)
  | ExchangeGos	of string		(* EXCHANGE gossip *)
  | History	of string		(* debugging history *)
  | ViewState	of View.state		(* state of next view *)
  | ProtoId	of Proto.id		(* protocol id (only for down events) *)
  | Time        of Time.t		(* current time *)
  | Alarm       of Time.t		(* for alarm requests *)
  | Unreliable				(* message is unreliable *)
  | NoTotal				(* message is not totally ordered*)
  | ServerOnly				(* deliver only at servers *)
  | ClientOnly				(* deliver only at clients *)
  | Causal				(* causal ordering *)
  | Agreed				(* agreed ordering *)
  | Safe				(* safe delivery *)
  | AckVct        of seqno option array
  | Transitional  of bool array
\end{codebox}
\caption{Fields for events.  Taken from \sourcetype{event.mli}}
\label{fig:extensions}
\end{figure}

\subsubsection{Up Event Fields}
\begin{EventType}
\eventtype{Typ}{The type of the event.}

\eventtype{Origin}{The rank of the member who initiated the event.}

\eventtype{Flags}{A bitfield specifying a set of potential flags for the
event.}

\eventtype{Ack}{Contains information about how to acknowledge this event.
Typically, every event with the \mlval{ack} field not set to \mlval{NoAck}
should be acknowledged exactly once by exactly one layer.  Usually,
acknowledgement is made by the layer which iniitated the event.}

\eventtype{Iov}{Iovec list containing raw application data.}

\eventtype{Time}{Time that the event/message was received.}

\eventtype{Ranks}{A potentially empty list of ranks.}

\eventtype{Alarm}{Requested time for an alarm.}

\eventtype{Iov}{Iovec list containing raw application data.}

\eventtype{Suspects}{List of ranks of members that are suspected to have
failed or be faulty in some way.}

\eventtype{SuspectReason}{String containing ``reason'' for suspecting
members.  Used for debugging purposes.}

\eventtype{Failures}{List of ranks of members that have failed.}

\eventtype{ViewId}{View id of \emph{next} view.}

\eventtype{Stability}{Vector of number of broadcasts for each member in
the group that are stable.}

\eventtype{NumCasts}{Vector of number of known broadcasts for members in
the group.}

\eventtype{Contact}{Endpoint of contact used for communication to endpoints
outside of group.  Usually only in merge events.}
\end{EventType}

\subsection{Event fields and the ``types'' for which they are defined}
\note{TODO}

\subsection{Event Chains}
We describe here common event sequences, or chains, in \ensemble.  Event chains
are sequences of alternate up and down events that ping-pong up and down
the protocol stack bouncing between the two end-layers of the chain.  The
end layers are typically the the top and bottom-most layers in the stack
(eg., TOP and COM).  The most common exceptions to this are the message
chains (Sends and Broadcasts), which can have any layer for their top
layer.

Note that these sequences are just prototypical.  Necessarily, there are
variations in which of layers see which parts of these sequences.  For
example, consider the Failure Chain in a virtual synchrony stack with the
GMP layer.  The Failure Chain begins at the coordinator with an \DnSuspect\
event initiated at any layer in the stack.  The COM layer bounces this up
as an \UpSuspect\ event.  The top-most layer usually responds with a \DnFail\
event.  The \DnFail\ event passes through all the layers until it gets to
the GMP layer.  The GMP layer at the coordinator both passes the \DnFail\
event to the layer below and passes a \DnCast\ event (thereby beginning a
Broadcast Chain\ldots).  At the coordinator, the \DnFail\ event bounces off
of the COM layer as an \UpFail\ event and then passes up to the top of the
protocol stack.  At the other members, an \UpCast\ event will be received at
the GMP layer.  The message is marked as a ``Fail'' message, so the GMP
layers generate a \DnFail\ event (just like the one at the coordinator) and
this is also bounced off the COM layer as an \UpFail\ event.  The lesson here
is that the different layers in the different members of the group all
essentially saw the same Failure Chain, but exact sequencing was different.
For example, the layers above the GMP layer at the members other than the
coordinator did not see a \DnFail\ event. \todo{give diagram}

\todo{Leave Chain}

\subsubsection{Timer Chain}
Request for a timer, followed by an alarm timeout.
\begin{ChainTable}
\chainentry{\DnTimer}{timeout requested}
\chainentry{\UpTimer}{alarm generated at or after requested time}
\end{ChainTable}

\subsubsection{Send Chain}
Send a pt2pt message followed by stability detection.
\begin{ChainTable}
\chainentry{\DnSend}{send a pt2pt message}
\chainentry{\UpSend}{destinations receive the message}
\end{ChainTable}
\hide{\chainentry{\DnAck}{destinations acknowledge the message}
\chainentry{\UpStable}{message eventually becomes stable}}

\subsubsection{Broadcast Chain}
Broadcast of a message followed by stability detection.  The \DnAck
may be optional.
\begin{ChainTable}
\chainentry{\DnCast}{broadcast a message}
\chainentry{\UpCast}{other members receive the broadcast}
\chainentry{\DnAck}{other members acknowledge the broadcast}
\chainentry{\UpStable}{broadcast eventually becomes stable}
\end{ChainTable}

\subsubsection{Failure Chain}
Suspicion and ``failure'' of group members.
\begin{ChainTable}
\chainentry{\DnSuspect}{suspicion of failures generated at any layer}
\chainentry{\UpSuspect}{notification of suspicion of failures}
\chainentry{\DnFail}{coord fails suspects}
\chainentry{\UpFail}{all members get failure notice}
\end{ChainTable}

\subsubsection{Block Chain}
Blocking of a group prior to a membership change.
\begin{ChainTable}
\chainentry{\UpSuspect/\UpMergeRequest}{reasons for coord blocking}
\chainentry{\DnBlock}{coord starts blocking}
\chainentry{\UpBlock}{all members get block notice}
\chainentry{\DnBlockOk}{all members reply to block notice}
\chainentry{\UpBlockOk}{coord get block OK notice}
\chainentry{\DnMerge/\DnView}{coord begins Merge or View chain}
\end{ChainTable}

\subsubsection{View Chain}
Installation of a new view, followed by garbage collection of the old view.
\begin{ChainTable}
\chainentry{\DnView}{coord begins view chain (after failed merge or blocking)}
\chainentry{\UpView}{all members get view notice}
\chainentry{\DnExit}{protocol stacks are ready for garbage collection \note{todo}}
\chainentry{\UpExit}{protocol stacks are garbage collected}
\end{ChainTable}

\subsubsection{Merge Chain (successful)}
Partition A merges with partition B, followed by garbage collection of the
old view.  We focus on partition A and only give a subset of events in
partition B.
\begin{ChainTable}
\chainentry{\DnMerge}{coord A begins merge chain (after blocking)}
\chainentry{\UpMergeRequest}{coord B gets merge request}
\chainentry{\DnMergeGranted}{coord B replies to merge request}
\chainentry{\UpMergeGranted}{coord A gets merge OK notice}
\chainentry{\DnView}{coord A installs new view for coord B}
\chainentry{\UpView}{all members in group A get view notice}
\chainentry{\DnExit}{protocol stacks are ready for garbage collection}
\chainentry{\UpExit}{protocol stacks are garbage collected}
\end{ChainTable}
 \todo{\DnExit\ above is currently \DnLeave}

\subsubsection{Merge Chain (failed)}
Failed merge, followed by installation of a view.
\begin{ChainTable}
\chainentry{\DnMerge}{coord begins merge chain (after blocking)}
\chainentry{\UpMergeFailed\ or}{}
\chainentry{\UpMergeDenied}{coord detect merge problem}
\chainentry{\DnView}{coord begins view chain}
\end{ChainTable}
