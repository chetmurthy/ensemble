\section{\ensemble\ ML Application Interface}
\label{section:applintf}
\todo{add example handlers from mtalk}

We present a simple interface for building single-group applications.  This
interface is intended to make small applications easy to build, and to protect
users from complications in the internals of the system.

The interface is implemented as a set of callbacks the application
provides to \ensemble.  The application is notified through these
callbacks (in a similar fashion to callbacks with Motif widgets) of
events that occur in the system, such as message receipts and
membership changes.

The interface for a member of a group is always in one of two states,
\emph{blocked} or \emph{unblocked}.  While unblocked, only the
\mlval{recv\_send}, \mlval{recv\_cast}, and \mlval{heartbeat}
callbacks are enabled.  This is the normal state of the system. While
block, the application \emph{should} refrain from sending messages. However,
it can send messages, causing the system to fail with the notification
``sending while blocked''.

Messages are sent by returning from these callbacks lists of actions to
take.  An action is usually a message send: either a \mlval{Cast} (group
broadcast) or a \mlval{Send} (point-to-point message).  Thus, messages are
delivered by callbacks from \ensemble\ and further messages are sent by
returning values from these callbacks.

\subsection{Compilation}
Compiling ML applications is easy.  You can use \sourcedemo{Makefile} as a
skeleton for your own applications.

\subsection{Interface Definition and Initialization}
Below is the full ML interface type definition for the application
interface described here.  A group member is initialized by creating
an interface record which defines a set of callback handlers for the
application.  This is then passed to one of the \ensemble\ stack
initialization functions exported by \sourcecode{appl/appl.mli}.

\begin{codebox}
(* Some type aliases.
 *)
type rank	= int
type view 	= Endpt.id list
type origin 	= rank
type dests 	= rank array

type control =
  | Leave
  | Prompt
  | Suspect of rank list

  | XferDone
  | Rekey of bool 
  | Protocol of Proto.id
  | Migrate of Addr.set
  | Timeout of Time.t            (* not supported *)

  | Dump
  | Block of bool                (* not for casual use *)
  | No_op

type ('cast_msg,'send_msg) action =
  | Cast of 'cast_msg
  | Send of dests * 'send_msg
  | Send1 of rank * 'send_msg
  | Control of control



\end{codebox}
\begin{codebox}
(* APPL_INTF.New.full: The record interface for applications.  An
 * application must define all the following callbacks and
 * put them in a record.
 *)

  type cast_or_send = C | S
  type blocked = U | B

  type 'msg naction = ('msg,'msg) action

  type 'msg handlers = {
    flow_block : rank option * bool -> unit ;
    block : unit -> 'msg naction array ;
    heartbeat : Time.t -> 'msg naction array ;
    receive : origin -> blocked -> cast_or_send -> 'msg -> 'msg naction array ;
    disable : unit -> unit
  } 

  type 'msg full = {
    heartbeat_rate : Time.t ;
    install : View.full -> ('msg naction array) * ('msg handlers) ;
    exit : unit -> unit
  } 
\}
\end{codebox}

\subsection{Actions}
Some callbacks allow a (possibly empty) array of actions to be 
returned.  There are 4 different kinds of actions:
\begin{description}
\item
[Cast(msg)] : Causes \mlval{msg} to be broadcast to the group.
\item
[Send(dests,msg)] : Causes \mlval{msg} to be sent to a subset of the
group specified in \mlval{dests}.  \mlval{dests} is an array of ranks.
\item
[Send1(dest,msg)] : Same as \mlval{Send}, but sends \mlval{msg} to a
single destination. This is slightly more efficient for single destinations.
\item
[Control c] : This bundles together all control actions. There 
are several of these:
\begin{description}
\item
[Leave] : Causes the member to leave the group.  There should always
be at most one \mlval{Leave} action returned in an action array.
\item
[Prompt] : Ask the system to perform a view-change immediately.
\item
[XferDone] : Signals that this member has completed its state
transfer.  If a state transfer layer is in the protocol stack, this
will trigger a new non-state transfer view after all members have
taken an \mlval{XferDone} action.
\item
[Rekey opt] : Ask the system to rekey itself. This should be done in case
the current key may have been compromised, for example, if a
previously trusted member should be expelled. The \mlval{opt}
parameter describes whether previously constructed pt-2-pt session
keys can be used to optimize this operation, or whether this is
disallowd. For the casual user, the optimized version (opt = false)
should be used.
\item
[Protocol(protocol)] : Requests a protocol switch.  If the stack supports
protocol switches, a new view will be triggered.
\item
[Dump] : Causes some debugging output to be printed by the stack in use.
The output depends greatly on the protocol stack.
\item
The rest of the actions are not intended for the casual user, they
are either not supported, badly supported, or used by system internals.
\end{description}
\end{description}


\subsection{The install callback}
Whenever a new view is installed, the application install callback is 
called. This handler describes several callbacks:
\begin{codebox}
  type 'msg handlers = {
    flow_block : rank option * bool -> unit ;
    block : unit -> 'msg naction array ;
    heartbeat : Time.t -> 'msg naction array ;
    receive : origin -> blocked -> cast_or_send -> 'msg -> 'msg naction array ;
    disable : unit -> unit
  } 
\end{codebox}

\mlval{flow\_block source onoff} is called whenever there are flow control
issues. The \mlval{onoff} value describes whether communication on the
specific channel can resume, or should be held back momentarily until 
communication problems are resolved. If the \mlval{source} is None,
then the problematic channel is multicast, if it is
\mlval{Some(rank)} then there are issues with the point-to-point
connection between this endpoint, and endpoint \mlval{rank}.

\mlval{block ()} is called to notify the application to stop sending
messages, because a view change is pending. It is an error to send
messages from now on, until a new view is installed, and
\mlval{install} will be called again.

\mlval{heartbeat current\_time} is regularly called by \ensemble\ when
the application is unblocked.  The expected rate of heartbeats is
specified through the \mlval{heartbeat\_rate} field of the interface
record.  The return values for all of these callbacks is an action
array.

\mlval{receive origin bk cs msg} is called when a message
has been received. The callback is made with the origin of the 
message, the current block state (bk), if this is a Cast of Send
message (cs) and the message itself.

The install callback is called with the current view state, it returns
a set of 5 handlers, and also a set of actions to be performed
immediatly. It is wrapped up in a structure bundling the heartbeat
rate, exit function (see below), and itself.

%*************************************************************%
%
%    Ensemble, 1_42
%    Copyright 2003 Cornell University, Hebrew University
%           IBM Israel Science and Technology
%    All rights reserved.
%
%    See ensemble/doc/license.txt for further information.
%
%*************************************************************%
\subsection{View state}

Several callbacks receive as an argument a pair of records with
information about the new view.  The information is split into two
parts, a \mlval{View.state} and a \mlval{View.local} record.  The
first contains information that is common to all the members in the
view, such as the \mlval{view} of the group.  The same record is
delivered to all members.  The second record contains information
local to the member that receives it.  These fields include the
\mlval{Endpt.id} of the member and its \mlval{rank} in the view.  It
also contains information that is derived from the \mlval{View.state}
record, such as \mlval{nmembers} with is merely the length of the
\mlval{view} field.

\begin{codebox}
(* VIEW.STATE: a record of information kept about views.
 * This value should be common to all members in a view.
 *)
type state = {
  (* Group information.
   *)
  version       : Version.id ;		(* version of Ensemble *)
  group		: Group.id ;		(* name of group *)
  proto_id	: Proto.id ;		(* id of protocol in use *)
  coord         : rank ;		(* initial coordinator *)
  ltime         : ltime ;		(* logical time of this view *)
  primary       : primary ;		(* primary partition? (only w/some protocols) *)
  groupd        : bool ;		(* using groupd server? *)
  xfer_view	: bool ;		(* is this an XFER view? *)
  key		: Security.key ;	(* keys in use *)
  prev_ids      : id list ;             (* identifiers for prev. views *)
  params        : Param.tl ;		(* parameters of protocols *)
  uptime        : Time.t ;		(* time this group started *)

  (* Per-member arrays.
   *)
  view 		: t ;			(* members in the view *)
  clients	: bool Arrayf.t ;	(* who are the clients in the group? *)
  address       : Addr.set Arrayf.t ;	(* addresses of members *)
  out_of_date   : ltime Arrayf.t	; (* who is out of date *)
  lwe           : Endpt.id Arrayf.t Arrayf.t ; (* for light-weight endpoints *)
  protos        : bool Arrayf.t  	(* who is using protos server? *)
}
\end{codebox}

\begin{codebox}
(* VIEW.LOCAL: information about a view that is particular to 
 * a member.
 *)
type local = {
  endpt	        : Endpt.id ;		(* endpoint id *)
  addr	        : Addr.set ;		(* my address *)
  rank 	        : rank ;		(* rank in the view *)  
  name		: string ;		(* my string name *)
  nmembers 	: nmembers ;		(* # members in view *)
  view_id 	: id ;			(* unique id of this view *)
  am_coord      : bool ;  		(* rank = vs.coord? *)
  falses        : bool Arrayf.t ;       (* all false: used to save space *)
  zeroes        : int Arrayf.t ;        (* all zero: used to save space *)
  loop          : rank Arrayf.t ;      	(* ranks in a loop, skipping me *)
  async         : (Group.id * Endpt.id) (* info for finding async *)
}  

(* LOCAL: create local record based view state and endpt.
 *)
val local : debug -> Endpt.id -> state -> local
\end{codebox}

Most of the fields are moderately self-explanatory.  If
\mlval{xfer\_view} is true, then this view is only for state transfer
and all members should take an \mlval{XferDone} action when the state
transfer is complete.  The view field is defined as \mlval{View.t},
which is:
\begin{codebox}
(* VIEW.T: an array of endpt id's.
 *)
type t = Endpt.id Arrayf.t
\end{codebox}



\subsection{Asynchronous operation}
The application can only send messages when handling a callback.
Under some circumstances (such as when receiving input from another
source), it is necessary to send messages immediately rather than
waiting for the next regularly scheduled heartbeat to occur.  Call the
function \mlval{Appl.async} with the group and endpoint of the group.
This returns a function that can be called whenever an immediate
hearbeat is desired.  \note{This replaces the previous
\mlval{heartbeat\_now} callback.}
\begin{codebox}
  let async = Appl.async (group,endpt) in
  async ()
\end{codebox}


\subsection{Exit notice}
Called when the member has left the group (through a previous \mlval{Leave}
action).  This is the last callback the group member will receive.
\begin{codebox}
  exit                  : unit -> unit ;
\end{codebox}

\subsection{Properties}
\label{sec:properties}

The \ensemble\ \mlval{Property} module is used to construct protocols based on
desired properties the application wants.  You can look at \sourceappl{property.mli}
for the various properties that are supported by \ensemble:
\begin{codebox}
type id =
  | Agree				(* agreed (safe) delivery *)
  | Gmp					(* group-membership properties *)
  | Sync				(* view synchronization *)
  | Total				(* totally ordered messages *)
  | Heal				(* partition healing *)
  | Switch		  (* protocol switching *)
  | Auth				(* authentication *)
  | Causal			(* causally ordered broadcasts *)
  | Subcast			(* subcast pt2pt messages *)
  | Frag				(* fragmentation-reassembly *)
  | Debug				(* adds debugging layers *)
  | Scale				(* scalability *)
  | Xfer				(* state transfer *)
  | Cltsvr			(* client-server management *)
  | Suspect			(* failure detection *)
  | Evs					(* extended virtual synchrony *)
  | Flow				(* flow control *)
  | Migrate			(* process migration *)
  | Privacy			(* encryption of application data *)
  | Rekey				(* support for rekeying the group *)
  | Primary			(* primary partition detection *)
  | Local				(* local delivery of messages *)
  | Slander			(* members share failure suspiciions *)

    (* The following are not normally used.
     *)
  | Drop				(* randomized message dropping *)
  | Pbcast			(* Hack: just use pbcast prot. *)
  | Dbg         (* on-line modification of network topology *)
\end{codebox}

Here is a short description of some of the properties:
\begin{itemize}
\item {Gmp:} Group Membership Properties.
\item {Sync:} Synchronizes messages on view changes to ensure view synchrony.
\item {Total:} Broadcast messages are totally ordered in the group.
\item {Heal:} Group partitions are healed.
\item {Switch:} Allows on-the-fly protocol switching.
\item {Auth:} Allows only authenticated and authorized members into
the group. Creates secure agreement in the group on a mutual group
key. This key is used to sign and verify, using keyed-MD5, all group
messages. This protects the group from outisde attack. 
\item {Rekey:} Allows rekeying the group.  
\item {Privacy:} Encrypts all user messages. 
\item {Causal:} Broadcasts are causally ordered.
\item {Subcast:} Point-to-point messages are sent using filtered broadcasts.
Guarantees FIFO ordering between broadcasts and point-to-point messages.
\item {Frag:} Message fragmentation.  Allows messages of any size to be sent.
\item {Debug:} Inserts a variety of ``assertion'' protocols that check that
other properties are being met.
\item {Scale:} Switches some protocols with more scalable versions.
\item {Xfer:} Causes the state transfer field (\mlval{xfer}) of view states to
be set.
\item {Cltsvr:} Causes the clients field of view states to be set according to
whether members are ``clients'' or ``servers''.
\item {Suspect:} Members watch other members for suspected failures.
\end{itemize}

The \mlval{Property.choose} function selects a protocol stack based on a list
of desired properties (you can examine the implementation to see exactly how
this is done):
\begin{codebox}
(* Create protocol with desired properties.
 *)
val choose : id list -> Proto.id
\end{codebox}

The default properties used for \ensemble\ applications is \mlval{Property.vsync}.
This is one of a variety of predefined protocol property lists defined in the
\mlval{Property} module:
\begin{codebox}
let vsync = [Gmp;Sync;Heal;Migrate;Switch;Frag;Suspect;Flow]
let total = vsync @ [Total]
let scale = vsync @ [Scale]
let fifo = [Frag]
\end{codebox}


In order to set the properties used by an application, you would use the
following code:
\begin{codebox}
  (* Choose default view state.
   *)
  let vs = Appl.default_info "my-appl" in

  (* Select desired properties.
   *)
  let properties = [ (* list of properties *) ] in

  (* Choose corresponding protocol stack.
   *)
  let proto_id = Property.choose properties in

  (* Set proto_id of the view state record.
   *)
  let vs = View.set vs [Vs_proto_id proto_id] in

  (* Configure the application
   *)
  Appl.config_new my_interface vs ;
\end{codebox}

As described in the reference manual, each of these protocols are derived by
combining a set of protocol layers together to get a full protocol stack with
application-level properties.  Anyway, here we describe the behavior of the
\mlval{vsync} protocol stack.
\begin{itemize}
\item
The first callback a protocol stack receives is an
\mlval{install} with a singleton view.
\item
All members in the same partition of a group receive the same
\mlval{View.state} records (excepting the \mlval{rank} field, of
course).
\item
\mlval{Send} messages are delivered reliably and in FIFO order.  It is
an error for a member to send a message to itself.
\item
\mlval{Cast} messages are delivered reliably and in FIFO order.  FIFO
order for \mlval{Cast} messages means that members receive the
messages in the order they were sent by the sender.  \mlval{Cast}
messages are usually not delivered to the sender (the primary
exceptions are stacks with total-ordering layers in them).
\item
There is no ordering relationship \emph{between} \mlval{Send} and
\mlval{Cast} messages.
\item
Messages are delivered in the same view they were sent in (the
protocol stack ``blocks'' so that the protocols can flush all the
current messages out of the system before advancing to the next view).
\item
\mlval{Cast} messages are delivered atomically.  This means that
either all members (excepting the sender) or none will receive a
\mlval{Cast} message.  If the sender of a \mlval{Cast} message fails,
other members who received the message will retransmit it for the
failed member.  When there is more than one member in a group, a
\mlval{Cast} message may be delivered to no members only if the sender
fails.
\item
All members that receive the same consecutive views (they get the same
\mlval{install upcalls} will have delivered the same set of
\mlval{Cast} messages between the upcalls (but not necessarily in the
same order).  Thus views can be considered as synchronization points
where all members agree on what has been done so far.
\end{itemize}

\subsection{Initializing \ensemble\ Applications}

This is a description of how simple applications are initialized with
\ensemble.  The source code presented here is extracted from the
\mlval{mtalk} demo, which is distributed with \ensemble.  The source
can be found in \sourcedemo{mtalk.ml} which compiles and links with
the \ensemble\ library to form the \sourcedemo{mtalk} executable.

An application consists of two parts, initialization and an interface.
The initialization involves setting up \ensemble\ and the
communication framework.  An interface consists of a set of callback
handlers that manage application events that \ensemble\ generates for
messages and membership changes.  The initialization code tends to be
similar across applications, and the handlers tend to contain most of
the application-specific functionality.  We present a sample set of
initialization code, which can easily be adapted for other simple
applications.  We do not describe the callback handlers here; they are
described in section~\ref{section:applintf}.  For specific examples,
see \sourcedemo{mtalk.ml} and \sourcedemo{rand.ml}.

\begin{codebox}
let run () =
  (*
   * Parse command line arguments.
   *)
  Arge.parse [
    (*
     * Extra arguments can go here.
     *)
  ] (Arge.badarg name) "mtalk: multiperson talk program" ;

  (*
   * Get default transport and alarm info.
   *)
  let view_state = Appl.default_info "mtalk" in

  let alarm = Alarm.get_hack () in
\end{codebox}
The initialization must do several things, all of which can be
contained in a single function, as shown here with the function
\mlval{run}.  First parse the command-line arguments as is done above.
In addition to arguments provided by the applicatoin, this parses the
standard \ensemble\ arguments.  Then, \mlval{default\_info} is called.
This initializes a \mlval{View.state} record (which contains all the
information other modules need to initialize your application).

\begin{codebox}    
  (*
   * Choose a string name for this member.  Usually
   * this is "userlogin@host".
   *)
  let name =
    try
      let host = gethostname () in

      (* Get a prettier name if possible.
       *)
      let host = string_of_inet (inet_of_string host) in
      sprintf "%s@%s" (getlogin ()) host
    with _ -> view_state.name
  in

  (*
   * Initialize the application interface.
   *)
  let interface = intf name alarm in
\end{codebox}    
Next we initialize the interface record that contains the
application's handlers and which does the actual work of the
application.  How the interface is initialized is application
dependent.  For example, \mlval{interface} will usually require
several arguments.  In the \mlval{mtalk} application, the interface
takes the endpoint identifier of the application and a string name to
use for this member of the talk group.  Other applications will use
different arguments.

\begin{codebox}    
  (*
   * Initialize the protocol stack, using the interface and
   * view state chosen above.  
   *)
  Appl.config_new interface view_state ;
\end{codebox}    
The code above initializes the protocol stack.  In this case we use
the \mlval{vsync} protocol properties, which provide FIFO,
virtually-synchronous communication and an automatic merging facility
for healing partitions.  There are several different sets of
properties by the \sourceappl{property.mli} module, each of which
provides different properties or performance characteristics (for
more information about properties, see section~\ref{sec:properties}).

\begin{codebox}    
  (*
   * Enter a main loop
   *)
  Appl.main_loop ()
  (* end of run function *)


(* Run the application, with exception handlers to catch any
 * problems that might occur.
 *)
let _ = Appl.exec ["mtalk"] run
\end{codebox}    
The initialization is complete and we enter a main loop.  The main
loop never returns.  The final code calls the \mlval{run} function
with some standard exception handlers to catch any exceptions that
should not, but may, occur.

This is all that is required for initializing simple, single-group Ensemble
applications.  
%The main part of the work required for an application is in
%building the handlers for sending and receiving messages, described in the
%following section.
