%*************************************************************%
%
%    Ensemble, 1_42
%    Copyright 2003 Cornell University, Hebrew University
%           IBM Israel Science and Technology
%    All rights reserved.
%
%    See ensemble/doc/license.txt for further information.
%
%*************************************************************%
\section{Outboard messaging}

The {\tt ce\_outboard} program allows clients to work with a remote Ensemble server
that implements the CE library calls. The description here is of the
nuts-and-bolts TCP interface to the {\tt ce\_outboard} service.


\subsection{Locating the service}
The CE outboard service uses TCP port 5002. Client
processes connect to this port on the local-host in the normal fashion for TCP
services. Client processes can join any number of groups and perform
any number of operations over a single connection to a server. If {\tt
  ce\_outboard} is not running on the local machine then processes
will not be able to connect to it. Currently, we do not support 
connecting to a remote service. 

If the TCP connection breaks, the CE outboard service will fail the
member from all groups that it joined.  However, a client can
reconnect to the same server and rejoin the groups it was in. 

\subsection{Communicating with the service}
Communication with the service is done through specially formatted
messages.  We describe the message types and their format here.

\begin{description}
\item
[messages:] Messages in both directions are formatted as follows.
Both directions of the TCP streams are broken into variable-length
{\it packets}.  A packet has a header of size $8$ consisting of two
integers in network byte order (NBO). The first integer gives the
length of the message header, the second integer gives the length of
message body. The next message follows immediately after
the body.
\item
[integers:] Integers are unsigned and are formatted as $4$ bytes in
NBO.
\item
[booleans:] Booleans are formatted the same as integers where $1$ equals true and
$0$ equals false.
\item
[strings:] Strings have a somewhat more complex format.
The first $4$ bytes are an integer length (unsigned, NBO).  
The body of the string immidiately follows
the length.
\item
[endpoint and group identifiers:] These types have the same format as
strings.  For non-Ensemble applications, the contents can contain
whatever the transport service you are using requires.  \ensemble\
only tests the contents of endpoint and group identifiers for
equality with other endpoints and groups.
\item
[time:] Time is formatted as two integers; the first counts 10s of seconds, the
second counts 10s of microseconds.
\item
[lists:] Lists have two parts.  The first is an integer giving the
number of elements in the list.  Immediately following that are the
elements in the list, one after the other and adjacent to
one-another.  It is assumed that the application knows the formats of
the items in the list in order to break them up.
\item
[iovec:] Message body, a seqence of bytes that is bulk data.
\end{description}

The actual messages sent between the client and the server are
composed of integers and strings.  The first field of a message is an
integer that denotes the group id, the second field is an integer
\emph{tag} value from which the format of the remainder of the message
can be determined.

%\vspace{2.5mm}
%\hrule{}
%\vspace{1.5mm}

\subsection{Messages from server to client}

\begin{itemize}
\item{View} : A new view is being installed. The view is a
list of Endpt.id's.  A member who just sent a Join message may not be
included in the view, in which case it should await the next View
message.  The ltime is the logical time of the view.  The first entry
in the view and the ltime uniquely identify the view.  The ltime's
that a member sees grow monotonicly.  In addition, a boolean value is
sent specifying whether this view is a primary view.  The primary bit
is based on the primary bit of the group daemon's being used.

\begin{FormatTable}
\formatentry{integer}{group-id}
\formatentry{integer}{View = 0}
\formatentry{endpoint}{my endpoint}
\formatentry{string}{my address}
\formatentry{integer}{rank}
\formatentry{string}{my name}
\formatentry{int}{number of members}
\formatentry{view\_id}{current view id}
\formatentry{boolean}{am I the coordinator''}
\formatentry{string}{Ensemble verion number}
\formatentry{string}{group}
\formatentry{string}{protocol stack}
\formatentry{string}{group coordinator rank}
\formatentry{int}{logical time of the view}
\formatentry{boolean}{is this the primary partition?}
\formatentry{boolean}{are we using the group-daemon?}
\formatentry{boolean}{is this a state-transfer view?}
\formatentry{string}{The group key}
\formatentry{list}{previous view-ids, those that merged to form this view}
\formatentry{string}{parameters}
\formatentry{time}{uptime of this endpoint}
\formatentry{list}{view of this group}
\formatentry{list}{addresses of group members}
\formatentry{iovec}{an empty message body}
\end{FormatTable}

\item{Cast} : A mulitcast message has been received on this group. 
\begin{FormatTable}
\formatentry{integer}{group-id}
\formatentry{integer}{Cast = 1}
\formatentry{integer}{origin}
\formatentry{iovec}{message body}
\end{FormatTable}

\item{Send} : A point-to-point message has been received for this
  endpoint. 
\begin{FormatTable}
\formatentry{integer}{group-id}
\formatentry{integer}{Send = 2}
\formatentry{integer}{origin}
\formatentry{iovec}{message body}
\end{FormatTable}

\item{Heartbeat} : A periodic heartbeat. Is about to be {\bf deprecated}. 
\begin{FormatTable}
\formatentry{integer}{group-id}
\formatentry{integer}{Heartbeat = 3}
\formatentry{time}{the current time}
\formatentry{iovec}{empty}
\end{FormatTable}

\item{Block} : Block the endpoint as preperation for a view
  change. After receiving the Block message the client needs to
  (asynchronously) reply with a BlockOk message. After sending a
  BlockOk no messages can be sent in this view. 
\begin{FormatTable}
\formatentry{integer}{group-id}
\formatentry{integer}{Block = 4}
\formatentry{iovec}{empty}
\end{FormatTable}

\item{Exit} : this endpoint is no longer valid. After receiving this
  message the client can discard any outstanding state for this
  endpoint. 
\begin{FormatTable}
\formatentry{integer}{group-id}
\formatentry{integer}{Exit = 5}
\formatentry{iovec}{empty}
\end{FormatTable}

\end{itemize}


%\vspace{2.5mm}
%\hrule{}
%\vspace{1.5mm}

\subsection{Messages from client to server}
\begin{itemize}
\item{Join} : Join a group with a set of options. The use-properties
  flag indicates whether to use the properties field to derive a stack
  or use the raw set of layers requested in the protocol field. The
  group-id must be unique across this socket-connection and is
  chosen by the client. 
\begin{FormatTable}
\formatentry{integer}{group-id}
\formatentry{integer}{Join = 0}
\formatentry{integer}{heartbeat rate}
\formatentry{string}{transports}
\formatentry{string}{protocol}
\formatentry{string}{group-name}
\formatentry{string}{properties}
\formatentry{bool}{use the properties?}
\formatentry{bool}{use the groupd service?}
\formatentry{string}{parameters}
\formatentry{string}{is this a light-weight client?}
\formatentry{string}{use a debugging stack?}
\formatentry{string}{principal}
\formatentry{string}{group security key}
\formatentry{bool}{secure stack?}
\end{FormatTable}

\item{Cast} : Multicast a message in this group.
\begin{FormatTable}
\formatentry{integer}{group-id}
\formatentry{integer}{Cast = 1}
\formatentry{iovec}{message body}
\end{FormatTable}

\item{Send} : Send a point-to-point message to a subset of the
  group. The client must make sure that the target ranks are within
  the group, the server will exit with an error otherwise. 
\begin{FormatTable}
\formatentry{integer}{group-id}
\formatentry{integer}{Send = 2}
\formatentry{list}{target ranks}
\formatentry{iovec}{message body}
\end{FormatTable}


\item{Send1} : Send a point-to-point message to a single destination. 
The client must make sure that the destination is within
  the group, the server will exit with an error otherwise. 
\begin{FormatTable}
\formatentry{integer}{group-id}
\formatentry{integer}{Send1 = 3}
\formatentry{iovec}{message body}
\end{FormatTable}

\item{Suspect} : Suspect a list of members. The client must make sure
  that the suspected members are valid member ranks; the server will exit with an error otherwise. 
\begin{FormatTable}
\formatentry{integer}{group-id}
\formatentry{integer}{Suspect = 4}
\formatentry{list}{suspect rank list}
\formatentry{iovec}{message body}
\end{FormatTable}

\item{XferDone} : Indicate that state-trasfer has completed. 
\begin{FormatTable}
\formatentry{integer}{group-id}
\formatentry{integer}{XferDone = 5}
\formatentry{iovec}{empty}
\end{FormatTable}

\item{Protocol} : Request a stack change. Ask for a specific set of
  layers. 
\begin{FormatTable}
\formatentry{integer}{group-id}
\formatentry{integer}{Protocol = 6}
\formatentry{string}{protocol}
\formatentry{iovec}{empty}
\end{FormatTable}

\item{Property} : Request a stack change. Ask for a set of
  properties. Ensemble will derive a stack from the set of
  properties. 
\begin{FormatTable}
\formatentry{integer}{group-id}
\formatentry{integer}{Property = 7}
\formatentry{string}{protocol}
\formatentry{iovec}{empty}
\end{FormatTable}

\item{Leave} : Request to leave the group. The endpoint is valid until
  the Exit message is received from the server. 
\begin{FormatTable}
\formatentry{integer}{group-id}
\formatentry{integer}{Leave = 8}
\formatentry{iovec}{empty}
\end{FormatTable}

\item{Prompt} : Ask for an immediate view change. 
\begin{FormatTable}
\formatentry{integer}{group-id}
\formatentry{integer}{Prompt = 9}
\formatentry{iovec}{empty}
\end{FormatTable}

\item{Rekey} : Rekey the group. 
\begin{FormatTable}
\formatentry{integer}{group-id}
\formatentry{integer}{Rekey = 10}
\formatentry{iovec}{empty}
\end{FormatTable}

\item{BlockOk} : A response to the Block message. After sending the
  BlockOk no new messages can be sent in the view. 
\begin{FormatTable}
\formatentry{integer}{group-id}
\formatentry{integer}{BlockOk = 11}
\formatentry{iovec}{empty}
\end{FormatTable}
\end{itemize}

