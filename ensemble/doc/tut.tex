\documentclass[11pt]{article}
\usepackage{fullpage}
\usepackage{alltt}
\usepackage{hyperref}

%%%%%%%%%%%%%%%%%%%%%%%%%%%%%%%%%%%%%%%%%%%%%%%%%%
%%pdf-latex stuff
%
\newif\ifpdf
\ifx\pdfoutput\undefined
   \pdffalse
\else
   \pdfoutput=1
   \pdftrue
\fi

\ifpdf
  \usepackage[pdftex]{graphicx}
  \pdfcompresslevel=9
  \DeclareGraphicsExtensions{.jpg,.pdf,.mps,.png}
\else
  \usepackage[dvips]{graphics}
\fi
%%%%%%%%%%%%%%%%%%%%%%%%%%%%%%%%%%%%%%%%%%%%%%%%%%

%*************************************************************%
%
%    Ensemble, 2_00
%    Copyright 2004 Cornell University, Hebrew University
%           IBM Israel Science and Technology
%    All rights reserved.
%
%    See ensemble/doc/license.txt for further information.
%
%*************************************************************%
\newcommand {\ensemble}		{Ensemble}
\newcommand {\horus}		{Horus}
\newcommand {\caml}		{Objective Caml}
\newcommand {\chk}		{$\surd$}

\newcommand{\hide}[1]           {~}
\newcommand{\note}[1]           {{\bf [#1]}}
\newcommand{\todo}[1]           {{\note{TODO: #1}}}
\newcommand{\mlval}[1]          {{\bf {#1}}}
\newcommand{\cval}[1]           {{\tt {#1}}}

\newcommand {\incgraphics}[1]   {\includegraphics{fig/#1}}

\newcommand {\sourcecode}[1]    {{\bf {#1}}}
%\newcommand {\sourcelayer}[1]  {{\htmladdnormallink{#1}{../../layers/#1}}}
%\newcommand {\sourceutil}[1]   {{\htmladdnormallink{#1}{../../util/#1}}}
\newcommand {\sourcelayer}[1]   {\sourcecode{layers/#1}}
\newcommand {\sourcedemo}[1]    {\sourcecode{demo/#1}}
\newcommand {\sourceutil}[1]    {\sourcecode{util/#1}}
\newcommand {\sourcetrans}[1]   {\sourcecode{trans/#1}}
\newcommand {\sourcetype}[1]    {\sourcecode{type/#1}}
\newcommand {\sourceappl}[1]    {\sourcecode{appl/#1}}
\newcommand {\sourcece}[1]      {\sourcecode{ce/#1}}
\newcommand {\sourcehot}[1]     {\sourcecode{hot/#1}}

\newcommand {\ECast}[0]         {\mlval{ECast}}
\newcommand {\ESend}[0]         {\mlval{ESend}}
\newcommand {\ESubCast}[0]      {\mlval{ESubCast}}
\newcommand {\ECastUnrel}[0]    {\mlval{ECastUnrel}}
\newcommand {\ESendUnrel}[0]    {\mlval{ESendUnrel}}
\newcommand {\EMergeRequest}[0] {\mlval{EMergeRequest}}
\newcommand {\EMergeGranted}[0] {\mlval{EMergeGranted}}
\newcommand {\EOrphan}[0]       {\mlval{EOrphan}}
\newcommand {\EAccount}[0]      {\mlval{EAccount}}
\newcommand {\EAsync}[0]        {\mlval{EAsync}}
\newcommand {\EBlock}[0]        {\mlval{EBlock}}
\newcommand {\EBlockOk}[0]      {\mlval{EBlockOk}}
\newcommand {\EDump}[0]         {\mlval{EDump}}
\newcommand {\EElect}[0]        {\mlval{EElect}}
\newcommand {\EExit}[0]         {\mlval{EExit}}
\newcommand {\EFail}[0]         {\mlval{EFail}}
\newcommand {\EGossipExt}[0]     {\mlval{EGossipExt}}
\newcommand {\EGossipExtDir}[0]    {\mlval{EGossipExtDir}}
\newcommand {\EInit}[0]     {\mlval{EInit}}
\newcommand {\ELeave}[0]     {\mlval{ELeave}}
\newcommand {\ELostMessage}[0]     {\mlval{ELostMessage}}
\newcommand {\EMergeDenied}[0]     {\mlval{EMergeDenied}}
\newcommand {\EMergeFailed}[0]     {\mlval{EMergeFailed}}
\newcommand {\EMigrate}[0]         {\mlval{EMigrate}}
\newcommand {\EPresent}[0]         {\mlval{EPresent}}
\newcommand {\EPrompt}[0]          {\mlval{EPrompt}}
\newcommand {\EProtocol}[0]        {\mlval{EProtocol}}
\newcommand {\ERekey}[0]           {\mlval{ERekey}}
\newcommand {\ERekeyPrcl}[0]       {\mlval{ERekeyPrcl}}
\newcommand {\EStable}[0]          {\mlval{EStable}}
\newcommand {\EStableReq}[0]       {\mlval{EStableReq}}
\newcommand {\ESuspect}[0]         {\mlval{ESuspect}}
\newcommand {\ESystemError}[0]     {\mlval{ESystemError}}
\newcommand {\ETimer}[0]           {\mlval{ETimer}}
\newcommand {\EView}[0]            {\mlval{EView}}
\newcommand {\EXferDone}[0]        {\mlval{EXferDone}}
\newcommand {\ESyncInfo}[0]        {\mlval{ESyncInfo}}
\newcommand {\ESecureMsg}[0]       {\mlval{ESecureMsg}}
\newcommand {\EChannelList}[0]     {\mlval{EChannelList}}
\newcommand {\EFlowBlock}[0]       {\mlval{EFlowBlock}}
\newcommand {\EAuth}[0]            {\mlval{EAuth}}
\newcommand {\ESecChannelList}[0]  {\mlval{ESecChannelList}}
\newcommand {\ERekeyCleanup}[0]    {\mlval{ERekeyCleanup}}
\newcommand {\ERekeyCommit}[0]     {\mlval{ERekeyCommit}}

\newcommand {\Dn}[1]               {\mlval{Dn(E{#1})}}
\newcommand {\Up}[1]               {\mlval{Up(E{#1})}}

\newlength{\figurewidth}
\newsavebox{\figurebox}
\newenvironment{codebox}{
\figurewidth\hsize
\addtolength{\figurewidth}{-4\fboxsep}
\addtolength{\figurewidth}{-4\fboxrule}

\begin{alltt}
\sbox{\figurebox}\bgroup
\begin{minipage}{\figurewidth}
}{
\end{minipage}
\egroup
\fbox{\usebox{\figurebox}}
\end{alltt}
}

% Ohad.
% A macro for putting scaled figures in boxes.
\newcommand{\putfigfbox}[2]     {\fbox { \scalebox{#1}{\includegraphics{#2}} } }




\title{\ensemble\ Tutorial}
\author{Mark Hayden, Ohad Rodeh \\
\small{Copyright \copyright\ 1997 Cornell University,
                             2000 Hebrew  University,
                             2002 IBM Israel Science and Technology }
}

\begin{document}
\maketitle

\begin{abstract}
\ensemble\ is a reimplementation of the \horus\ reliable group communication
system in the \caml\ programming language.  This document describes:
\begin{itemize}
\item
How to configure and execute the applications included with \ensemble.
\item
The client application interface.
\item
The Server (OCaml) \ensemble\ application interface.
\end{itemize}
\end{abstract}

\newpage
\tableofcontents
\newpage

\section{Introduction}
This documentation assumes that the reader has some familiarity with group
communication. There are many texts that describe how to use and build
group-communication system.

Ensemble is structured as a client-server system with a server
providing group-communication services through a socket based
interface. Clients can connect to the server and send/receive reliable
point-to-point and multicast messages. There should be one server
running on a host, and clients should be located on the same
host. This allows using insecure communication for client-server
traffic.  The server is written (mostly) in the OCaml programming
language, the client is a small library that has
implementations in several languages. At the time of writing there are
clients in C and Java.

Previous versions of the system did not distinguish between client and
server. The client was implemented with an internal server. This
provides good performance. However, since the server is written in ML,
in order to link with a C program written by a user the
foreign-language interface of ML needs to be used. This causes very
difficult portability issues. As of release 2.00 we decided to
separate client from server; this should improve portability at the
expense of performance. 

%*************************************************************%
%
%    Ensemble, 1_42
%    Copyright 2003 Cornell University, Hebrew University
%           IBM Israel Science and Technology
%    All rights reserved.
%
%    See ensemble/doc/license.txt for further information.
%
%*************************************************************%
\section{Quick Installation} 

Several demonstration applications are included with \ensemble.
These can give a sense of the kinds of facilities provided by group
communication to those who have not used a group communication
toolkit before.  The demos can also serve as starting points for
building new applications.  These applications are briefly described
here along with how to execute them and the various command-line
options and environment variables they use.

\subsection{Compiling}
Please see the file \sourcecode{ensemble/INSTALL.htm} for instructions on
installing \ensemble\ if you have not done so already.

\subsection{Configuration Variables}
Detailed information is given in Section~\ref{subsect:config} for
initializing environment variables. We assume that you will be using
IP-multicast as a communication substrate this means no configuration
is neccessary. However, if multicast is not supported by your system you'll
be using the gossip server for processes to locate each
other, see Section~\ref{subsect:config} for more information.

Throughout this tutorial, we assume you are using the Unix csh or
tcsh shell. To set an environment variable in the bash shell you would
do the following:
\begin{verbatim}
% export ENS_CONFIG_FILE=/etc/ensemble.conf
\end{verbatim}
If you're using a win32 system you will need to use the native
environment-setting tool
(start $\rightarrow$ setting $\rightarrow$ control-panel $\rightarrow$ system $\rightarrow$ advanced $\rightarrow$ environment-variables)
 which provides similar functionality. 

\subsection{Executing Applications}
On the same or other hosts execute several instances of an
application, such as \sourcedemo{mtalk}:
\begin{verbatim}
% mtalk
 ...
\end{verbatim}

Applications should merge together and form a group.


\newpage
\part{The Server}
This chapter describes how to build server-side programs. The reason
users should be wary of writing such programs is that the server
operates in a soft read-time environment. The server is written in the
OCaml programming language, a single thread of execution is used. To
improve performance bulk-data for user messages is not allocated on
the ML heap, which is garbage collection, it is allocated on large
chunks of memory allocated with {\tt malloc}. Bulk-data extents are
also called io-vectors and the memory used to hold them is also called
iovec-memory. To reduce server memory foot-print the size of
iovec-memory is limited, at the time of writing we are using 6
mega-bytes. Since memory is limited memory-allocation can fail. The
server handles this with flow-control protocols limiting the amount of
incoming messages to fit the amount of available iovec-memory. In
order to maintain responsiveness to incoming messages pure CPU
processing (such as search on a database) should be limited. If the
messaging engine does not receive sufficient CPU every, say, 100
milliseconds then performance is going to suffer dramatically.

The casual user will be better served by the chapter on writing client
programs that do not suffer from these limitations. If, however, you
are undaunted then this chapter is for you. 

%*************************************************************%
%
%    Ensemble, 2_00
%    Copyright 2004 Cornell University, Hebrew University
%           IBM Israel Science and Technology
%    All rights reserved.
%
%    See ensemble/doc/license.txt for further information.
%
%*************************************************************%
\section{The Programs}
Of the programs described here only mtalk, gossip, groupd, and
ensembled, are normal programs. The rest: perf, rand, fifo, socktest,
and armadillo, are internal system tests not for use by the casual user. 

Notes:
\begin{itemize}
\item
please note that warning and error messages printed by Ensemble are
not prefixed with the name of the program generating the message, but
rather the name of the module.
\end{itemize}

\subsection{Ensembled: the ensemble daemon}
This is the ensemble daemon. By running a single instance of this
program on a host all clients on that host will be able to use Ensemble services. 

\subsection{Mtalk: Multi-person Talk}
This is a multi-person talk demo.  As \mlval{mtalk} processes are created, they merge
into a single group.  Input typed at one process is broadcast to the rest of the
processes in the group.

\subsection{Gossip: Group Locator Service}
This is not really an application.  The gossip server works in
conjunction with the Ensemble \mlval{UDP} communication transport to
simulate low-bandwidth gossip broadcast for systems that do not have
IP multicast.  See the discussion on transports below.  The group
communication protocols require some ``gossipping'' mechanism in order
to detect and heal partitions in the system.  When an application
wishes to gossip with other partitions, it broadcasts a message via
the \mlval{gossip} servers.  This sends messages to the \mlval{gossip}
servers.  The \mlval{gossip} servers then forward the message to all
processes they have heard from recently to simulate a broadcast.  When
an application is using the \mlval{UDP} transport and not the
\mlval{DEERING} transport (\mlval{DEERING} is the default), it is
necessary for a \mlval{gossip} process to be running somewhere in the
system.

\subsection{Groupd: Membership Service (formerly called Domain)}
\label{section:groupd}
Normally, \ensemble\ application groups implement their own group
membership protocol.  However, they have the option of using the
\ensemble\ membership service implemented by the \mlval{groupd}
application.  \mlval{groupd} is a service for managing multiple
process groups.  It uses a \emph{core} group of \ensemble\ processes
to participate in managing these groups.  Clients connect to the
service via TCP connections, through which they request to join and
leave groups.  The service supports a simple protocol through which
the clients can obtain virtual synchronous properties.  The service
also supports weaker properties that give faster membership
notifications.

\note{We emphasize that Ensemble applications can operate independently of a
membership service.}

Some of the benefits of using this service are:
\begin{itemize}
\item
When there are no membership changes, the clients communicate directly
between themselves, so the membership service has no affect on
performance.
\item
The service implements group membership for multiple groups.  The
costs of the group membership protocols (such as failure detection)
are shared over the groups.
\item
Because applications are sharing the same membership service, they see
consistent views and failure detections.
\item
The client part of the protocol for implementing virtual synchrony is
simple.  Most of the complexity is in the server.  This allows client
programs to be implemented in languages other than ML, but save much
of the programming burden because the servers handle the ``hard''
group membership protocols.  The client TCP interface is described in
the \ensemble\ reference manual.
\item
Applications that do not need the full virtual synchrony properties
can use weaker synchronization protocols and get faster view changes.
\item
The service allows groups to scale to larger sizes.  The membership
servers do not need to run on all the hosts on which the clients run,
so clients can be on more hosts than are normally supported by
\ensemble.
\end{itemize}

\paragraph{Executing Groupd:}
In order to run Groupd, set the \mlval{ENS\_GROUPD\_PORT} environment
variable to select the TCP port for the service to use.  The
membership service is executed through the \mlval{groupd} application
program:
\begin{verbatim}
% groupd
\end{verbatim}
It takes command-line arguments similar to the other \ensemble\
demonstration programs.  Normally, each host runs a server.

Other demo applications use the service when the \mlval{-groupd}
command-line argument is selected.  For example:
\begin{verbatim}
% mtalk -groupd
\end{verbatim}
Note that you must have a \mlval{groupd} server running on the same host as
mtalk for this to work.


\subsection{Perf: Performance Tests}
This program includes a variety of performance tests for Ensemble.

\paragraph{Ring:}
This test is run with the \mlval{-prog ring} option.  Say that there
are $n$ members.  Each process first waits until there are $n$
members.  It then sends $k$ messages, and waits for $(n - 1)k$
messages from other members.  It measures the time for this, and does
so a number of times to determine the average and variance.  This can
be done for varying $n$, $k$, message size, and protocol.

The time between the rounds is a measure of latency.  The total amount
of data sent between the rounds is a measure of bandwidth.  The total
number of messages sent between rounds is a measure of throughput.
For good measurements, set the parameters as follows:

\begin{center}
\begin{tabular}{|l|c|c|}			   \hline
measure		& $k$		& message size	\\ \hline \hline
latency		&  1		& 0		\\ \hline
throughput	& large		& 0		\\ \hline
bandwidth	&  1		& large		\\ \hline
\end{tabular}
\end{center}

Additional command-line arguments (with default values in parentheses):
\begin{description}
\item [-n \#]: number of members ($2$ members)
\item [-s \#]: size in bytes of application messages ($0$ bytes)
\item [-r \#]: number of rounds ($300$ rounds)
\item [-k \#]: messages per round ($1$ message per round)
\end{description}
These values must be set by all members.  All members must use the
same values for all of the arguments except message size.

\todo{The other performance tests are undocumented.}

\subsection{Rand: Virtual Synchrony Debugging Tool}
This demo is used to test \ensemble.  It uses simulated communication and
introduces random process failures to check for proper behavior of the group
membership protocols.

\subsection{Fifo: Fifo Communication Debugging Tool}
This demo is used to test \ensemble.  It uses simulated communication
structured in such a way as to trigger bugs in FIFO, reliable communication
protocols.

\subsection{Armadillo: testing Ensemble security extensions}
This program tests \ensemble\ security features. It has several
command line options:
\begin{description}
\item[-n \#]  number of endpoints to create
\item[-t \#]  after what threshold to start the test
\item[-prog] which security to use? [policy,rekey,exchange,reg,prompt]
\item[-pa]   simulate partitions? 
\item[-net]  run everything in a single process or run throughout the  network
\item[-real\_pgp]  use PGP for authentication? otherwise, simulate it.
\item[-group]    set the group name
\end{description}

The ``exchange'' test checks that the Exchange layer functions
correctly. For example, running:
\begin{verbatim}
% armadillo -n 20 -prog exchange
\end{verbatim}
will create 20 endpoints with random intial keys. the endpoints should
merge into one group after a short while.

The ``rekey'' test creates a group and once its size is above the
threshold it start rekeying it. The test: {\tt Use: armadillo -n 7 -t
7 -prog rekey} will create a group of 7 members and once the group
reaches this size, will start to rekey it.
		
To see what happens when the group partitions use: {\tt armadillo -n 5
-t 3 -prog rekey -pa}. This will create a group of 5 members and start
partitioning and remerging the group. Everytime the membership in a
group component exceeds 3, the component leader will try rekeying it. 

The ``policy'' test checks that Ensemble respects application trust
policies. For example running:
\begin{verbatim}
% armadillo -n 7 -prog policy
\end{verbatim}
will create a static group of 7 processes, numbered 0 through 6, and
dynamically change the endpoints trust policies. Ensemble forms
subgroups according to the trust relationship. The policies are
designed to change in stages:
\begin{enumerate}
\item All endpoints trust each other.
\item 
All endpoints of the same (mod 2) trust each other. That is we
have to trust domains: $\{0,2,4,6\}$ and $\{1,3,5\}$.
\item 
All endpoints of the same (mod 3) trust each other. That is we
have three trust domains: $\{0,3,6\}$, $\{1,4\}$ $\{2,5\}$.
\end{enumerate}

The ``prompt'', and ``reg'' tests are auxillary tests not related
to security. 


\subsection{Socktest}
This is a simple application that tests the soundness of the
lower level Ensemble interface to sockets and IP-multicast.

The current menu is as follows:
\begin{itemize}
\item
{[Join \mlval{ipm\_addr} ]} Join a multicast group.
\item 
{[Leave \mlval{ipm\_addr} ]} Leave a multicast group.
\item 
{[Cast \mlval{ipm\_addr} \mlval{msg}]} Send a message to a multicast
group.
\item 
{[Ttl \mlval{num} ]} set the time-to-live.
\item 
{[Loopback \mlval{onoff} ]} set the loopack. If this is on, then
messages sent to a multicast group will also be locally received.
\item 
{[Sendbuf \mlval{size} ]} Set the send-buffer size in the kernel.
Normally, to little space is reserved in the kernel for storing IP
packets, therefore, it is common practice to increase it.
\item 
{[Recvbuf \mlval{size}]} 
Same, as Sendbuf, only for received packets.
\item 
{[Nonblock \mlval{onoff} ]} set a socket to blocking or non-blocking
mode.
\end{itemize}




\section{Configuration}

\subsection{Command-line Arguments and Environment Variables}
\label{subsect:config}
\ensemble\ applications typically support a variety of configuration
parameters.  Most of these can be configured through command-line options as
well as through setting environment variables.  In all cases, command-line
options override environment variables.  Look in \sourceappl{harg.ml} for the
authoritative list of the configuration parameters.  Some are listed below as
command-line options.  The corresponding environment variable for
\mlval{-group\_name} (for example) is \mlval{ENS\_GROUP\_NAME} (the name is
capitalized and the `-' is replace with \mlval{ENS\_}).
\begin{description}
\item
[\mlval{-modes} arg :] Set the default modes for an application to use.  The
modes are specified giving their names in all-uppercase, each separated by
single colons (`:') and no white-space.
\item
[\mlval{-udp\_port} port :] set the default UDP port for \ensemble\
applications.  For point to point UDP communication, this is the port
number \ensemble\ first tries to bind to for UDP communication (if it
is already in use Ensemble will then fail).  It can be set to any
value.  The default is to let the operating system choose a port to
use.
\item
[\mlval{-deering\_port} port :] This is the port that \ensemble\ will
use for Deering IP multicast communication (if enabled).  All
processes must use the same port number.
\item
[\mlval{-gossip\_port} port :] sets the port that the \mlval{gossip} servers
use.
\item
[\mlval{-gossip\_hosts} arg :] sets the hosts where applications using UDP
communication can look for \mlval{gossip} servers.  The value should be a
colon-separated list of hostnames.  The \mlval{gossip} server application will
only execute on these hosts.  Note that you only have to execute a gossip
server on one of these hosts: applications will try each of the hosts in turn
while looking for a gossip server.  However, multiple servers can be executed
for increased availability.
\item
[\mlval{-id} name :] used to give applications unique identifiers.
Usually this is set to be your user id.  Setting this variable
prevents \ensemble\ applications run by other users from interacting
with yours.  In case you do want them to interact, you should set
their variables to have the same value.  If using DEERING IP
multicast, their \mlval{-deering\_port} variable should also be set to
the same value.
\item
[\mlval{-groupd\_port} port :] sets the port that the membership \mlval{groupd}
servers use.
\item
[\mlval{-groupd\_hosts} arg :] sets the hosts that the membership
\mlval{groupd} servers use.  Format is the same as for \mlval{-gossip\_hosts}.
\item
[\mlval{-groupd} :] Use the membership service on the local host (see
section~\ref{section:groupd}.  This option may override others.
\item
[\mlval{-group\_name} name :] Set the name of the application's group.
\note{Currently, only the \mlval{ensemble} application supports this.}
\item
[\mlval{-key} key :] Set the key to use for a particular application.  All
messages sent and received by the application will be authenticated with this
key.
\item
[\mlval{-secure} :] Enable security enforcement.  This prevents any insecure
communication transports from being initialized.
\item
[\mlval{-add\_prop} property :] Adds a specific property to the
Ensemble protocol stack. See Section~\ref{sec:properties} for more
information on supported Ensemble properties. 
\item 
[\mlval{-remove\_prop} property :] The dual of add\_prop.
\item 
[\mlval{-sock\_buf} size :] The size of socket buffers to request
from the operating system.  The default size is $52428$ (the
traditional limit on Unix).  If you are using \ensemble\ in
high-performance setting and are experiencing message loss, this is a
parameter that should be increased.
\item
[\mlval{-refcount} :] Enables reference counting of message buffers.
The default is to rely on the garbage collector to detect when a
message is no longer needed.  Setting this will improve performance,
but may expose reference counting bugs in \ensemble.
\item 
[\mlval{-multiread} :] Enable multiple reads on sockets.  The default
is to receive and process one message from the operating system
at a time.  Setting this will cause all available messages to be read
from sockets before processing any of them, which may reduce message
loss due to buffer overflow in the operating system.
\item
[\mlval{-pollcount} count :] The number of times to query the
operating system before blocking.  \ensemble\ blocks after checking
(via the \mlval{select()} system call) the operating system for
messages and not finding any.  Setting this to $1$ will cause
\ensemble\ to block immediately when there are no more messages.
Setting this to a large number will cause \ensemble\ to busy-poll for
a longer time before blocking.
\end{description}

The following configuration parameter can only be set as an
environment variable.
\begin{description}
\item
[\mlval{ENS\_TRACE}]: enables module initialization tracing.  With this set (to
any value), modules print out their names as they initialize.  This is useful
if an exception  occurs during initialization because because it
enables you to narrow the problem down to one module.
%\item
%[\mlval{ENS\_LOG\_HOST} and \mlval{ENS\_LOG\_PORT}]: sets the port and host of the remote
%log server.  Applications can execute as the server via the \mlval{-log\_server}
%command-line option.  Running programs with the \mlval{-log} option enables the
%remote logging facility for other programs.  \note{This is currently not documented:
%you have to read the source code in \sourceappl{log} in order to use it now.}
\end{description}

\subsection{Transports}

\note{If you are only using regular UDP sockets for communication,
then you do not need to read this section.}

Perhaps the most confusing part of running \ensemble\ applications comes from
selecting communication transports.  Communication transports are the
bottom-most part of \ensemble\ and are used for sending and receiving messages
on a network.  There are several ways this can be confusing and often
\ensemble\ cannot detect that there is a problem, so you do not get a warning.
For instance, if you configure an application so that one process is using UDP
sockets for communication and another is using ATM, then the two processes will
stall waiting for other processes to communicate with them on their selected
medium.

A confusing aspect of transport is that an application typically uses two
different kinds of transports: a primary transport and a gossip transport.
Normal application communication is all done over the primary transport, which
must support point-to-point communication and may also support multicast
communication.  Communication between different partitions of a group of
applications uses the gossip transport which must support ``anonymous''
multicast communication.

Applications occasionally send ``gossip'' messages with their gossip transport
to the rest of the ``world'' in order to inform other partitions about their
presence.  When two partitions learn of each other, they can then merge the
partitions together.  After they have merged together, they communicate over
their primary transport.  This gossip-and-merge mechanism is used when
applications first start up: an application creates its own singleton group and
then merges with any other already existing partitions through gossiping and
merging.  Thus, if there is a problem with the gossip transport, you will tend
to have a bunch of applications in singleton groups that never merge.  If there
is a problem with the primary transport, the merging will occur, but then the
various members will be unable to communicate.  This will cause them to
repeatedly break into partitions (when they decide that the other members must
have failed) and then re-merge again.

The various primary and gossip transports are presented in the following table.
The ``P'' and ``G'' columns specify whether a transport can be used for primary
communication and/or gossip communication.
\begin{center}
\begin{tabular}{|l|c|c|l|}					   \hline
transport	& P	& G	& description			\\ \hline \hline
\mlval{UDP}	& \chk	& \chk	& UDP (+ gossip server)		\\ \hline
\mlval{DEERING}	& \chk	& \chk	& UDP/IP multicast		\\ \hline
\mlval{TCP}	& \chk	& 	& TCP/IP			\\ \hline
\mlval{ATM}	& \chk	& 	& ATM/UNET (if installed)	\\ \hline
\mlval{ETH}	& \chk	& \chk	& raw Ethernet			\\ \hline
\mlval{NETSIM}	& \chk	& \chk	& network simulator		\\ \hline
\end{tabular}
\end{center}

The \mlval{NETSIM} transports are used only in applications that are simulating
the behavior of a group inside a single process.  The \mlval{rand} and
\mlval{fifo} demos use this, for instance.

The \mlval{ETH} transport is only available on Linux.  It uses raw
Ethernet packets to communicate (note that other modes such as
\mlval{UDP}, \mlval{DEERING}, and \mlval{TCP} may also use Ethernet,
but not directly).  The kernel must support raw Ethernet packet
sockets (\mlval{CONFIG\_PACKET} must by 'y' or 'm').  Opening a raw
Ethernet packet typically requires super-user priviledges.  Current
limitations of the ETH transport are: the \mlval{frag\_max\_frag}
integer parameter must be set to a small value (such as $1000$,
``-pint frag\_max\_frag=1000'') otherwise the default fragmentation
size will be too large.  Also, the \ensemble\ raw Ethernet the
messages are not routed, so all processes communicating with the ETH
transport must be on the same subnet.

There are several ways to change the communication transports that \ensemble\
uses.  These are listed below in order of highest ``precedence.''
\begin{enumerate}
\item
Command-line argument: with the \mlval{-modes} argument (see the command-line
argument documentation).
\item
Application setting: a particular application may differ from the \ensemble\
defaults.
\item
Environment variable: \mlval{ENS\_MODES} variable (see the environment variable
documentation below).
\item
\ensemble\ defaults: \mlval{UDP}.
\end{enumerate}

\subsection{Using Deering IP Multicast}

The method described above for running the mtalk demo is the best way
to first run mtalk because it uses \mlval{UDP} for both
transports. \mlval{UDP} does not use IP multicast communication,
which can be a source of problems because of variations in how it is
configured at different sites.  \note{IP multicast is only available
when using the Socket library on Unix.  It is not currently supported
by Ensemble on Windows NT.}  If your machines support Deering IP
multicast communication, it is preferable to use \mlval{DEERING}
transports because you will then not have to run the gossip server
with \ensemble\ applications.  You can try out the IP multicast
transport by using the command-line arguments.  (Note the problems
section at the end of this section, however, which describes some of
the problems you may have.)  This is done by executing the
applications with the \mlval{-modes} command-line arguments.  With IP
multicast you no longer need to have a gossip server running.  Run
the application on the hosts with these arguments (see below for a
description of the arguments):
\begin{verbatim}
% mtalk -modes DEERING
\end{verbatim}
To always use IP multicast by default, modify the ENS\_MODES
environment variable so that it includes DEERING.  Also set the
ENS\_DEERING\_PORT environment variable to an unused port number.
You will probably wish to add these to your standard shell
environment:
\begin{verbatim}
setenv ENS_DEERING_PORT 1234
setenv ENS_MODES DEERING:UDP
\end{verbatim}

\subsection{Notes and Problems}

See also the problems mentioned in the \ensemble\ reference manual.
\begin{description}
\item
[IP Multicast problems :] Some problems may occur with IP Multicast.
The time-to-live value for multicast messages may be too small in
some environments, preventing multicast messages from reaching all
members.  The TTL value can be adjusted by editing the file
\sourcecode{socket/multicasts.c}.
\end{description}

\section{Server ML Application Interface}
\label{section:applintf}
\todo{add example handlers from mtalk}

We present a simple interface for building single-group applications.  This
interface is intended to make small applications easy to build, and to protect
users from complications in the internals of the system.

The interface is implemented as a set of callbacks the application
provides to \ensemble.  The application is notified through these
callbacks (in a similar fashion to callbacks with Motif widgets) of
events that occur in the system, such as message receipts and
membership changes.

The interface for a member of a group is always in one of two states,
\emph{blocked} or \emph{unblocked}.  While unblocked, only the
\mlval{recv\_send}, \mlval{recv\_cast}, and \mlval{heartbeat}
callbacks are enabled.  This is the normal state of the system. While
block, the application \emph{should} refrain from sending messages. However,
it can send messages, causing the system to fail with the notification
``sending while blocked''.

Messages are sent by returning from these callbacks lists of actions to
take.  An action is usually a message send: either a \mlval{Cast} (group
broadcast) or a \mlval{Send} (point-to-point message).  Thus, messages are
delivered by callbacks from \ensemble\ and further messages are sent by
returning values from these callbacks.

\subsection{Compilation}
Compiling ML applications is easy.  You can use \sourcedemo{Makefile} as a
skeleton for your own applications.

\subsection{Interface Definition and Initialization}
Below is the full ML interface type definition for the application
interface described here.  A group member is initialized by creating
an interface record which defines a set of callback handlers for the
application.  This is then passed to one of the \ensemble\ stack
initialization functions exported by \sourcecode{appl/appl.mli}.

\begin{codebox}
(* Some type aliases.
 *)
type rank	= int
type view 	= Endpt.id list
type origin 	= rank
type dests 	= rank array

type control =
  | Leave
  | Prompt
  | Suspect of rank list

  | XferDone
  | Rekey of bool 
  | Protocol of Proto.id
  | Migrate of Addr.set
  | Timeout of Time.t            (* not supported *)

  | Dump
  | Block of bool                (* not for casual use *)
  | No_op

type ('cast_msg,'send_msg) action =
  | Cast of 'cast_msg
  | Send of dests * 'send_msg
  | Send1 of rank * 'send_msg
  | Control of control



\end{codebox}
\begin{codebox}
(* APPL_INTF.New.full: The record interface for applications.  An
 * application must define all the following callbacks and
 * put them in a record.
 *)

  type cast_or_send = C | S
  type blocked = U | B

  type 'msg naction = ('msg,'msg) action

  type 'msg handlers = {
    flow_block : rank option * bool -> unit ;
    block : unit -> 'msg naction array ;
    heartbeat : Time.t -> 'msg naction array ;
    receive : origin -> blocked -> cast_or_send -> 'msg -> 'msg naction array ;
    disable : unit -> unit
  } 

  type 'msg full = {
    heartbeat_rate : Time.t ;
    install : View.full -> ('msg naction array) * ('msg handlers) ;
    exit : unit -> unit
  } 
\}
\end{codebox}

\subsection{Actions}
Some callbacks allow a (possibly empty) array of actions to be 
returned.  There are 4 different kinds of actions:
\begin{description}
\item
[Cast(msg)] : Causes \mlval{msg} to be broadcast to the group.
\item
[Send(dests,msg)] : Causes \mlval{msg} to be sent to a subset of the
group specified in \mlval{dests}.  \mlval{dests} is an array of ranks.
\item
[Send1(dest,msg)] : Same as \mlval{Send}, but sends \mlval{msg} to a
single destination. This is slightly more efficient for single destinations.
\item
[Control c] : This bundles together all control actions. There 
are several of these:
\begin{description}
\item
[Leave] : Causes the member to leave the group.  There should always
be at most one \mlval{Leave} action returned in an action array.
\item
[Prompt] : Ask the system to perform a view-change immediately.
\item
[XferDone] : Signals that this member has completed its state
transfer.  If a state transfer layer is in the protocol stack, this
will trigger a new non-state transfer view after all members have
taken an \mlval{XferDone} action.
\item
[Rekey opt] : Ask the system to rekey itself. This should be done in case
the current key may have been compromised, for example, if a
previously trusted member should be expelled. The \mlval{opt}
parameter describes whether previously constructed pt-2-pt session
keys can be used to optimize this operation, or whether this is
disallowd. For the casual user, the optimized version (opt = false)
should be used.
\item
[Protocol(protocol)] : Requests a protocol switch.  If the stack supports
protocol switches, a new view will be triggered.
\item
[Dump] : Causes some debugging output to be printed by the stack in use.
The output depends greatly on the protocol stack.
\item
The rest of the actions are not intended for the casual user, they
are either not supported, badly supported, or used by system internals.
\end{description}
\end{description}


\subsection{The install callback}
Whenever a new view is installed, the application install callback is 
called. This handler describes several callbacks:
\begin{codebox}
  type 'msg handlers = {
    flow_block : rank option * bool -> unit ;
    block : unit -> 'msg naction array ;
    heartbeat : Time.t -> 'msg naction array ;
    receive : origin -> blocked -> cast_or_send -> 'msg -> 'msg naction array ;
    disable : unit -> unit
  } 
\end{codebox}

\mlval{flow\_block source onoff} is called whenever there are flow control
issues. The \mlval{onoff} value describes whether communication on the
specific channel can resume, or should be held back momentarily until 
communication problems are resolved. If the \mlval{source} is None,
then the problematic channel is multicast, if it is
\mlval{Some(rank)} then there are issues with the point-to-point
connection between this endpoint, and endpoint \mlval{rank}.

\mlval{block ()} is called to notify the application to stop sending
messages, because a view change is pending. It is an error to send
messages from now on, until a new view is installed, and
\mlval{install} will be called again.

\mlval{heartbeat current\_time} is regularly called by \ensemble\ when
the application is unblocked.  The expected rate of heartbeats is
specified through the \mlval{heartbeat\_rate} field of the interface
record.  The return values for all of these callbacks is an action
array.

\mlval{receive origin bk cs msg} is called when a message
has been received. The callback is made with the origin of the 
message, the current block state (bk), if this is a Cast of Send
message (cs) and the message itself.

The install callback is called with the current view state, it returns
a set of 5 handlers, and also a set of actions to be performed
immediatly. It is wrapped up in a structure bundling the heartbeat
rate, exit function (see below), and itself.

%*************************************************************%
%
%    Ensemble, 1_42
%    Copyright 2003 Cornell University, Hebrew University
%           IBM Israel Science and Technology
%    All rights reserved.
%
%    See ensemble/doc/license.txt for further information.
%
%*************************************************************%
\subsection{View state}

Several callbacks receive as an argument a pair of records with
information about the new view.  The information is split into two
parts, a \mlval{View.state} and a \mlval{View.local} record.  The
first contains information that is common to all the members in the
view, such as the \mlval{view} of the group.  The same record is
delivered to all members.  The second record contains information
local to the member that receives it.  These fields include the
\mlval{Endpt.id} of the member and its \mlval{rank} in the view.  It
also contains information that is derived from the \mlval{View.state}
record, such as \mlval{nmembers} with is merely the length of the
\mlval{view} field.

\begin{codebox}
(* VIEW.STATE: a record of information kept about views.
 * This value should be common to all members in a view.
 *)
type state = {
  (* Group information.
   *)
  version       : Version.id ;		(* version of Ensemble *)
  group		: Group.id ;		(* name of group *)
  proto_id	: Proto.id ;		(* id of protocol in use *)
  coord         : rank ;		(* initial coordinator *)
  ltime         : ltime ;		(* logical time of this view *)
  primary       : primary ;		(* primary partition? (only w/some protocols) *)
  groupd        : bool ;		(* using groupd server? *)
  xfer_view	: bool ;		(* is this an XFER view? *)
  key		: Security.key ;	(* keys in use *)
  prev_ids      : id list ;             (* identifiers for prev. views *)
  params        : Param.tl ;		(* parameters of protocols *)
  uptime        : Time.t ;		(* time this group started *)

  (* Per-member arrays.
   *)
  view 		: t ;			(* members in the view *)
  clients	: bool Arrayf.t ;	(* who are the clients in the group? *)
  address       : Addr.set Arrayf.t ;	(* addresses of members *)
  out_of_date   : ltime Arrayf.t	; (* who is out of date *)
  lwe           : Endpt.id Arrayf.t Arrayf.t ; (* for light-weight endpoints *)
  protos        : bool Arrayf.t  	(* who is using protos server? *)
}
\end{codebox}

\begin{codebox}
(* VIEW.LOCAL: information about a view that is particular to 
 * a member.
 *)
type local = {
  endpt	        : Endpt.id ;		(* endpoint id *)
  addr	        : Addr.set ;		(* my address *)
  rank 	        : rank ;		(* rank in the view *)  
  name		: string ;		(* my string name *)
  nmembers 	: nmembers ;		(* # members in view *)
  view_id 	: id ;			(* unique id of this view *)
  am_coord      : bool ;  		(* rank = vs.coord? *)
  falses        : bool Arrayf.t ;       (* all false: used to save space *)
  zeroes        : int Arrayf.t ;        (* all zero: used to save space *)
  loop          : rank Arrayf.t ;      	(* ranks in a loop, skipping me *)
  async         : (Group.id * Endpt.id) (* info for finding async *)
}  

(* LOCAL: create local record based view state and endpt.
 *)
val local : debug -> Endpt.id -> state -> local
\end{codebox}

Most of the fields are moderately self-explanatory.  If
\mlval{xfer\_view} is true, then this view is only for state transfer
and all members should take an \mlval{XferDone} action when the state
transfer is complete.  The view field is defined as \mlval{View.t},
which is:
\begin{codebox}
(* VIEW.T: an array of endpt id's.
 *)
type t = Endpt.id Arrayf.t
\end{codebox}



\subsection{Asynchronous operation}
The application can only send messages when handling a callback.
Under some circumstances (such as when receiving input from another
source), it is necessary to send messages immediately rather than
waiting for the next regularly scheduled heartbeat to occur.  Call the
function \mlval{Appl.async} with the group and endpoint of the group.
This returns a function that can be called whenever an immediate
hearbeat is desired.  \note{This replaces the previous
\mlval{heartbeat\_now} callback.}
\begin{codebox}
  let async = Appl.async (group,endpt) in
  async ()
\end{codebox}


\subsection{Exit notice}
Called when the member has left the group (through a previous \mlval{Leave}
action).  This is the last callback the group member will receive.
\begin{codebox}
  exit                  : unit -> unit ;
\end{codebox}

\subsection{Properties}
\label{sec:properties}

The \ensemble\ \mlval{Property} module is used to construct protocols based on
desired properties the application wants.  You can look at \sourceappl{property.mli}
for the various properties that are supported by \ensemble:
\begin{codebox}
type id =
  | Agree				(* agreed (safe) delivery *)
  | Gmp					(* group-membership properties *)
  | Sync				(* view synchronization *)
  | Total				(* totally ordered messages *)
  | Heal				(* partition healing *)
  | Switch		  (* protocol switching *)
  | Auth				(* authentication *)
  | Causal			(* causally ordered broadcasts *)
  | Subcast			(* subcast pt2pt messages *)
  | Frag				(* fragmentation-reassembly *)
  | Debug				(* adds debugging layers *)
  | Scale				(* scalability *)
  | Xfer				(* state transfer *)
  | Cltsvr			(* client-server management *)
  | Suspect			(* failure detection *)
  | Evs					(* extended virtual synchrony *)
  | Flow				(* flow control *)
  | Migrate			(* process migration *)
  | Privacy			(* encryption of application data *)
  | Rekey				(* support for rekeying the group *)
  | Primary			(* primary partition detection *)
  | Local				(* local delivery of messages *)
  | Slander			(* members share failure suspiciions *)

    (* The following are not normally used.
     *)
  | Drop				(* randomized message dropping *)
  | Pbcast			(* Hack: just use pbcast prot. *)
  | Dbg         (* on-line modification of network topology *)
\end{codebox}

Here is a short description of some of the properties:
\begin{itemize}
\item {Gmp:} Group Membership Properties.
\item {Sync:} Synchronizes messages on view changes to ensure view synchrony.
\item {Total:} Broadcast messages are totally ordered in the group.
\item {Heal:} Group partitions are healed.
\item {Switch:} Allows on-the-fly protocol switching.
\item {Auth:} Allows only authenticated and authorized members into
the group. Creates secure agreement in the group on a mutual group
key. This key is used to sign and verify, using keyed-MD5, all group
messages. This protects the group from outisde attack. 
\item {Rekey:} Allows rekeying the group.  
\item {Privacy:} Encrypts all user messages. 
\item {Causal:} Broadcasts are causally ordered.
\item {Subcast:} Point-to-point messages are sent using filtered broadcasts.
Guarantees FIFO ordering between broadcasts and point-to-point messages.
\item {Frag:} Message fragmentation.  Allows messages of any size to be sent.
\item {Debug:} Inserts a variety of ``assertion'' protocols that check that
other properties are being met.
\item {Scale:} Switches some protocols with more scalable versions.
\item {Xfer:} Causes the state transfer field (\mlval{xfer}) of view states to
be set.
\item {Cltsvr:} Causes the clients field of view states to be set according to
whether members are ``clients'' or ``servers''.
\item {Suspect:} Members watch other members for suspected failures.
\end{itemize}

The \mlval{Property.choose} function selects a protocol stack based on a list
of desired properties (you can examine the implementation to see exactly how
this is done):
\begin{codebox}
(* Create protocol with desired properties.
 *)
val choose : id list -> Proto.id
\end{codebox}

The default properties used for \ensemble\ applications is \mlval{Property.vsync}.
This is one of a variety of predefined protocol property lists defined in the
\mlval{Property} module:
\begin{codebox}
let vsync = [Gmp;Sync;Heal;Migrate;Switch;Frag;Suspect;Flow]
let total = vsync @ [Total]
let scale = vsync @ [Scale]
let fifo = [Frag]
\end{codebox}


In order to set the properties used by an application, you would use the
following code:
\begin{codebox}
  (* Choose default view state.
   *)
  let vs = Appl.default_info "my-appl" in

  (* Select desired properties.
   *)
  let properties = [ (* list of properties *) ] in

  (* Choose corresponding protocol stack.
   *)
  let proto_id = Property.choose properties in

  (* Set proto_id of the view state record.
   *)
  let vs = View.set vs [Vs_proto_id proto_id] in

  (* Configure the application
   *)
  Appl.config_new my_interface vs ;
\end{codebox}

As described in the reference manual, each of these protocols are derived by
combining a set of protocol layers together to get a full protocol stack with
application-level properties.  Anyway, here we describe the behavior of the
\mlval{vsync} protocol stack.
\begin{itemize}
\item
The first callback a protocol stack receives is an
\mlval{install} with a singleton view.
\item
All members in the same partition of a group receive the same
\mlval{View.state} records (excepting the \mlval{rank} field, of
course).
\item
\mlval{Send} messages are delivered reliably and in FIFO order.  It is
an error for a member to send a message to itself.
\item
\mlval{Cast} messages are delivered reliably and in FIFO order.  FIFO
order for \mlval{Cast} messages means that members receive the
messages in the order they were sent by the sender.  \mlval{Cast}
messages are usually not delivered to the sender (the primary
exceptions are stacks with total-ordering layers in them).
\item
There is no ordering relationship \emph{between} \mlval{Send} and
\mlval{Cast} messages.
\item
Messages are delivered in the same view they were sent in (the
protocol stack ``blocks'' so that the protocols can flush all the
current messages out of the system before advancing to the next view).
\item
\mlval{Cast} messages are delivered atomically.  This means that
either all members (excepting the sender) or none will receive a
\mlval{Cast} message.  If the sender of a \mlval{Cast} message fails,
other members who received the message will retransmit it for the
failed member.  When there is more than one member in a group, a
\mlval{Cast} message may be delivered to no members only if the sender
fails.
\item
All members that receive the same consecutive views (they get the same
\mlval{install upcalls} will have delivered the same set of
\mlval{Cast} messages between the upcalls (but not necessarily in the
same order).  Thus views can be considered as synchronization points
where all members agree on what has been done so far.
\end{itemize}

\subsection{Initializing \ensemble\ Applications}

This is a description of how simple applications are initialized with
\ensemble.  The source code presented here is extracted from the
\mlval{mtalk} demo, which is distributed with \ensemble.  The source
can be found in \sourcedemo{mtalk.ml} which compiles and links with
the \ensemble\ library to form the \sourcedemo{mtalk} executable.

An application consists of two parts, initialization and an interface.
The initialization involves setting up \ensemble\ and the
communication framework.  An interface consists of a set of callback
handlers that manage application events that \ensemble\ generates for
messages and membership changes.  The initialization code tends to be
similar across applications, and the handlers tend to contain most of
the application-specific functionality.  We present a sample set of
initialization code, which can easily be adapted for other simple
applications.  We do not describe the callback handlers here; they are
described in section~\ref{section:applintf}.  For specific examples,
see \sourcedemo{mtalk.ml} and \sourcedemo{rand.ml}.

\begin{codebox}
let run () =
  (*
   * Parse command line arguments.
   *)
  Arge.parse [
    (*
     * Extra arguments can go here.
     *)
  ] (Arge.badarg name) "mtalk: multiperson talk program" ;

  (*
   * Get default transport and alarm info.
   *)
  let view_state = Appl.default_info "mtalk" in

  let alarm = Alarm.get_hack () in
\end{codebox}
The initialization must do several things, all of which can be
contained in a single function, as shown here with the function
\mlval{run}.  First parse the command-line arguments as is done above.
In addition to arguments provided by the applicatoin, this parses the
standard \ensemble\ arguments.  Then, \mlval{default\_info} is called.
This initializes a \mlval{View.state} record (which contains all the
information other modules need to initialize your application).

\begin{codebox}    
  (*
   * Choose a string name for this member.  Usually
   * this is "userlogin@host".
   *)
  let name =
    try
      let host = gethostname () in

      (* Get a prettier name if possible.
       *)
      let host = string_of_inet (inet_of_string host) in
      sprintf "%s@%s" (getlogin ()) host
    with _ -> view_state.name
  in

  (*
   * Initialize the application interface.
   *)
  let interface = intf name alarm in
\end{codebox}    
Next we initialize the interface record that contains the
application's handlers and which does the actual work of the
application.  How the interface is initialized is application
dependent.  For example, \mlval{interface} will usually require
several arguments.  In the \mlval{mtalk} application, the interface
takes the endpoint identifier of the application and a string name to
use for this member of the talk group.  Other applications will use
different arguments.

\begin{codebox}    
  (*
   * Initialize the protocol stack, using the interface and
   * view state chosen above.  
   *)
  Appl.config_new interface view_state ;
\end{codebox}    
The code above initializes the protocol stack.  In this case we use
the \mlval{vsync} protocol properties, which provide FIFO,
virtually-synchronous communication and an automatic merging facility
for healing partitions.  There are several different sets of
properties by the \sourceappl{property.mli} module, each of which
provides different properties or performance characteristics (for
more information about properties, see section~\ref{sec:properties}).

\begin{codebox}    
  (*
   * Enter a main loop
   *)
  Appl.main_loop ()
  (* end of run function *)


(* Run the application, with exception handlers to catch any
 * problems that might occur.
 *)
let _ = Appl.exec ["mtalk"] run
\end{codebox}    
The initialization is complete and we enter a main loop.  The main
loop never returns.  The final code calls the \mlval{run} function
with some standard exception handlers to catch any exceptions that
should not, but may, occur.

This is all that is required for initializing simple, single-group Ensemble
server applications.  
%The main part of the work required for an application is in
%building the handlers for sending and receiving messages, described in the
%following section.

\section{Using PGP}

\ensemble\ supports the use of PGP for authenticating members of
groups.  This work is still underway and the security currently
provided is not bullet proof.  Most of the \ensemble\ demo
applications support the use of PGP, including \mlval{mtalk},
\mlval{wbml}, and \mlval{ensemble}.

These are the instructions for using PGP.  Note that PGP is currently
only supported for Unix platforms.

\begin{itemize}
\item
The \mlval{pgp} binary must be in your path.  \ensemble\ executes PGP
as a subprocess for authenticating remote members.  If you do not yet
have a PGP keyring, read the PGP documentation on how to set all this
up.
\item
You must set the \mlval{PGPPASS} environment variable to contain your
secret key pass phrase.  See the PGP documentation for more
information.
\item
\mlval{-pgp user} : command line argument.  This tells \ensemble\ what
this user's name is for PGP other processes will use this name to
select the public key to use for authenticating you.
\item
\mlval{-key sharedkey}: command line argument.  This sets the shared
key conversation key that \ensemble\ will use initially.  It should
be at least $8$ characters long (for DES).
\item
\mlval{-add\_prop Auth}: command line argument.  This adds the
\mlval{Auth} property to the default Ensemble properties.  This then
causes the \mlval{EXCHANGE} protocol to be used in the protocol stack
for exchanging shared keys.
\end{itemize}

Now when you run an application only members that start with the same
shared key or who can authenticate each other through PGP will
merge into the same group.

If you run into problems, you can access PGP's debugging output
through the additional command-line arguments, \mlval{-trace PGP}.

\newcommand{\sourcefile}[1]    {{\tt {#1}}}


\section{Heterogeneous Transports}

{\bf  Complete this section}

\ensemble\ provides a flexible infrastructure for sending communication
across a variety of different communication transports.  Not only can
different groups use different communication transports, but a single group
can support communication on multiple transports at the same time.

The design of the transport module is split into three parts: 
\begin{description}
\item[The socket module:] ~\newline
   Low-level system calls: {\tt send, sendto, recv} etc.,
  implemented in a system-independent fashion. The {\tt socket}
  directory contains the code. {\tt socket/u} is a simple-minded
  implementation that uses the Ocaml Unix library directly. A more
  efficient version is located in {\tt socket/s}, where native OS
  io-vector send/recv facilities are used. 

\item[Transports:] ~\newline
  Self registering {\it transports}:  Deering, UDP, TCP, NETSIM. These
  use the low-level socket module calls to provide an abstract {\it transport}.
	
\item[Routers:] ~\newline
  Uses a communication transport to
  build Ensemble specific send/recv capabilities. Length field, 
  group id, and endpoint rank are added to each outgoing
  message. Basic parsing is performed on received messages and sender
  rank, group, and message length are extracted. 
 
  There are several {\it routers} in the {\tt route}
  subdirectory. \sourcefile{signed.ml} adds a 16-byte MD5 checksum to
  each outgoing message. An agreed group-secret is used to key MD5,
  providing group authentication. Incoming messages are stripped of
  this header, and verified. \sourcefile{unsigned.ml} is the vanilla router.

\end{description}

The user can choose to use either one of the socket module
implementations. The socket module interface is defined in
\sourcefile{socket/socket.mli}. The unoptimized socket implementation
(usocket) represents message data as a Caml string and benefits from
native garbage collection. Its disadvantage is reduced
performance. The optimized socket library (ssocket) uses native C
io-vectors, and native operating-system scatter-gather message
send/receive facilities. This provides much better performance, and
zero-copy integration with C applications. The disadvantage is more
difficult integration with native ML values. 

The transports are defined the \sourcefile{trans} subdirectory. 
UDP in \sourcefile{trans/udp.ml}, TCP in \sourcefile{trans/tcp.ml},
DEERING in \sourcefile{trans/ipmc}, and NETSIM in
\sourcefile{trans/netsim}.

The \sourcefile{route} subdirectory contains three routes: signed,
unsigned, and bypass.

\subsection{Code walk-through}
To provide better understanding of the design this section walks
through a configuration of the unsigned router, UDP transport, 
and optimized socket library. We shall start from the bottom
and work our way up. 

In file \sourcefile{server/socket/s/nt/sendrecv.c}, there is code for sending an
array of C io-vectors and part of an ML string for win32. The function takes
five arguments:
\begin{itemize}
\item info\_v : a structure describing a list of remote targets and a
socket through which to send messages.
\item prefix\_v : an ML string that prefixes the data
\item ofs\_v, len\_v: the offset and length of the prefix to send
\item iova\_a : an array of io-vectors wrapped in an ML representation
\end{itemize}

\begin{codebox}
value skt_udp_mu_sendsv(
	value info_v,
	value prefix_v,
	value ofs_v,
	value len_v,
	value iova_v
) \{
    int naddr=0, i, ret=0, len=0;
    ocaml_skt_t sock=0 ;
    skt_sendto_info_t *info ;
    int nvecs = Wosize_val(iova_v) ;

    // Extract the set of addresses
    info = skt_Sendto_info_val(info_v);

    // Prepare the header
    skt_prepare_send_header(send_iova, peek_buf, Int_val(len_v), skt_iovl_len(iova_v));

    // Prepare the iovectors
    skt_add_ml_hdr(send_iova, 1, prefix_v, ofs_v, len_v);
    skt_gather(send_iova, 2, iova_v) ;
    
    sock = info->sock ;
    naddr = info->naddr ;

    for (i=0;i<naddr;i++) {
	// Send the message.  Assume we don't block or get interrupted.  
	ret = WSASendTo(sock, send_iova, nvecs+2, &len, 0,
			&info->sa[i], info->addrlen, 0, NULL);
	if (SOCKET_ERROR == ret) skt_udp_error("skt_udp_mu_sendsv");
    }
    return Val_unit;
\}
\end{codebox}

The {\tt \_mu\_} prefix is added to this function
because it uses the Ml/User convention for sending data. Each data
packet is split into:
\begin{description}
\item [ML header length:] Describes the length of the ML header. of length four bytes. 
\item [User data length:] Describes the length of the user data. of length four bytes. 
\item [ML header:] the ML header itself. Variable size.
\item [User data:] user data. Variable size.
\end{description}

The function builds a header of size eight that includes two integers: (a)
ml-header length (b) io-vector length in network byte order. The
header is the first in an array of io-vectors that includes in second
place the ML-header, and then the array of user io-vectors. Once the
io-array is assembled it is sent to each destination in the list using
the native OS API. 

{\tt skt\_udp\_mu\_sendsv} is hidden inside the socket library, and can
safely be used using {\tt Socket.udp\_mu\_sendsv}. The {\tt sendto\_info}
structure can be created from an array of target socket addresses, and
a sending socket.

\begin{codebox}
type sendto_info
val sendto_info : socket -> Unix.sockaddr array -> sendto_info

val udp_mu_sendsv : sendto_info -> buf -> ofs -> len -> Iov.t array -> unit
\end{codebox}


The Hsys module makes access to sendtovs safer, and changes its type:
\begin{codebox}
  val udp_mu_sendsv : sendto_info -> Buf.t -> ofs -> len -> Iovecl.t -> unit

  (* Implementation *)
  Socket.udp_mu_sendsv info 
    (Buf.string_of buf) (Buf.int_of_len ofs) (Buf.int_of_len len) 
    (Iovecl.to_iovec_array iovl) 
\end{codebox}
Core Ensemble code, including the routers, does not use Socket calls
directly. Rather, it uses the Hsys module which wraps all calls with a
more type safe interface. Separate types are used for length, offset,
io-vector, and buffer.

The UDP implementation at \sourcefile{trans/udp.ml} uses Hsys in the 
transmit function called {\tt x}.

\begin{codebox}
  let x hdr ofs len iovl = 
    Hsys.sendtosv dests hdr ofs len iovl;
    Iovecl.free iovl
\end{codebox}

The io-vector array is freed after the message is transmitted. The
reference count for an iovec-array is decremented on two occasions:
(1) it is sent on the network (2) it is handed to an application, and
the callback has completed. The iovec refcount is initially set to one
when the application sends it, and it is henceforth incremented
whenever a copy of it created. Ultimately, the refcount will be
decremented when the stability detection protocol determines that all
group members received the message.

\subsection{Design of the routers}
Many endpoints belonging to different groups can coexist in a single
Ensemble process. Each endpoint is identified by its connection
identifier. The internal representation of this id is given in module
{\tt Conn}:

\begin{codebox}
type id = \{
  version       : Version.id ;
  group 	: Group.id ;
  stack 	: Stack_id.t ;
  proto 	: Proto.id option ;
  view_id 	: View.id option ;
  sndr_mbr 	: sndr_mbr ;
  dest_mbr 	: dest_mbr ;
  dest_endpt 	: dest_endpt option
\}
\end{codebox}

The id is mapped into a string using the {\tt Route.pack\_of\_conn}
function. Ensemble uses MD5 for this mapping. The probability of a
collision, i.e., for two different endpoints to map onto a single
string, is $2^{-64}$ which is sufficient for our purposes. 

\begin{codebox}
val pack_of_conn : Conn.id -> Buf.t
\end{codebox}

The purpose of the route module is to create a single interface to
these various endpoints. The main type exported is {\tt
handlers}. This is essentially a large array holding the set of
connection identifiers and the delivery function for each of
them. When a message is received by the bottom-most part of the
system, it is parsed by the socket code into an ML header that is a
string, and the rest of the message which is received into a
C-iovector. This information is later fed into the {\tt deliver}
function.

\begin{codebox}
val deliver : handlers -> Buf.t -> Buf.ofs -> Buf.len -> Iovecl.t -> unit
\end{codebox}

Deliver takes the current set of handlers, and a message, figures out
which endpoints need to receive this message and calls the appropriate
handlers. 

A transmission function is abstracted as a type {\tt xmitf}:
\begin{codebox}
(* transmit an Ensemble packet, this includes the ML part, and a
 * user-land iovecl.
 *)
type xmitf = Buf.t -> Buf.ofs -> Buf.len -> Iovecl.t -> unit
\end{codebox}

The Router module has an API allowing the creation of send/recv
functions for connection-ids. It also allows installing and deleting
such functions. The unsigned router is a simple example of
using this functionality to create the basic, insecure,
router. It defines function {\tt f}: 
\begin{codebox}
val f : unit -> 
  (Trans.rank -> Obj.t option -> Trans.seqno -> Iovecl.t -> unit) Route.t
\end{codebox}

This router will allow users to send (1) sender rank (2) ML object (3)
sequence number and (4) a user iovector array. The body of the code
calls {\tt Route.create} where it mainly needs to define how it plans
on handling {\tt blast} and {\tt merge}. Blast is how to send
messages, merge is how to receive a message on behalf of several
connection ids. 



\newpage
\part{The Client}
The client library (or simply the ``client'') implements a message-passing
protocol between server and user. The protocol used is described in
the reference manual. The client-library has no internal threads. No
message-memory is allocated by the client, all messages are allocated
and freed by the user. This gives the user complete control on its
memory foot-print. The client is thread-safe, several threads can
send/recv messages concurrently. Blocking socket operations are used
to simplify client semantics.

In order to use the client-library the user application must first
connect to the server. It can then create group members and perform
a subset of Ensemble actions: Leave, Cast, Send, Send1, Suspect.
There are other Ensemble operations that we decided not to support
since they add more complexity than value. 

The application must poll Ensemble periodically to see if there are any
pending messages, and receive them. In the past, it was possible
for the application not to receive messages while continuing to create
new actions. This is now not possible. The application will be blocked
at some point before flooding the server. 

\section{Native Java Application Interface (CEJAVA)}

The CEJAVA interface is built upon the CE interface, using the Java
Native Interface (JNI). This allows a user to tap the power of the
Ensemble messaging system from Java. Performance of this interface is
similar to the native ML and C interfaces. 

Not surprisingly, the API is similar to the CE API. Here, we walk
through an overview of the interface, and then point out the major
differences with respect to CE. Method and constructor documentation
can be found either in the code itself, or through the javadoc
generated HTML files.

\subsection{Overview}
The basic concept is that of a {\it Group}. A group is constructed by: 
\begin{codebox}
    public Group(Callbacks cb);
\end{codebox}

A group has to define the following set of callbacks: 

\begin{codebox}
public class Callbacks \{
    public abstract void install(View view);
    public abstract void exit();
    public abstract void recv_cast(int origin, byte[] msg);
    public abstract void recv_send(int origin, byte[] msg);
    public abstract void flow_block(int rank, boolean onoff);
    public abstract void block();
    public abstract void heartbeat(double time);
\}
\end{codebox}

The callbacks define behavior when events, such as message receipt or
view changes, occur. 

Within the context of a group the user can perform eleven actions: 

\begin{codebox}
    public void join(JoinOps jops);
    public void leave();
    public void cast(byte[] msg);
    public void send(int[] dests, byte[] msg) ;
    public void send1(int dest, byte[] msg);
    public void prompt();
    public void suspect(int[] suspects);
    public void xferDone();
    public void rekey();
    public void changeProtocol(String protocol_name);
    public void changeProperties(String properties);
\end{codebox}

Before starting to use the package, the initialization function must
be called: 

\begin{codebox}
    static public void init(String args []);
\end{codebox}

Any command line argument to CE and Ensemble should be passed here. 

To run a Java program that uses CEJAVA, the classpath must point to
the {\tt ensemble.jar} java-archive file, and the library path should
point to the matching native library {\tt libcejava.so}. For example,
to run java program {\tt prog.java}, assuming the jar file is in the
local directory, and that the native library is under {\tt lib} use:
\begin{codebox}
    java -Djava.class.path=;:ensemble.jar -Djava.library.path=lib prog
\end{codebox}
A simpler option is to set the {\tt CLASSPATH} environment variable to
include {\tt ensemble.jar} and set the dynamic library path to include
{\tt libcejava.so}. On Unix systems this means setting then
{\tt LD\_LIBRARY\_PATH} and on win32 setting the {\tt PATH}.

\subsection{Notes}
There are two major difference with respect to CE: zero-copy, and
synchronization. 

CEJAVA is not ``zero-copy'', the receive callbacks copy message data
from C to Java, and the send actions copy data from Java to C. This
costs extra copying but allows the Java application to do whatever it
likes with message data, without being bound by underlying reference
counting and memory management. 

Synchronization is very different between C and Java because the Java
has language support for locking. A group can be in one of six phases: 
\begin{codebox}
    public static final int PRE     = 0;
    public static final int JOINING = 1;
    public static final int NORMAL  = 2;
    public static final int BLOCKED = 3;
    public static final int LEAVING = 4;
    public static final int LEFT    = 5;
\end{codebox}

\begin{description}
\item[PRE:] preliminary phase, the group has not been fully
  constructed yet. 
\item[JOINING:] this endpoint is joining the group. 
\item[NORMAL:] the group is in stable state. 
\item[BLOCKED:] group is blocked prior to a pending view change. 
\item[LEAVING:] this member is leaving the group. 
\item[LEFT:] member has left the group. 
\end{description}

The only phase in which actions are allowed is the NORMAL state.
Multiple threads can perform actions on the same group, furthermore,
the Ensemble main-loop executing in a separate thread invokes group
callbacks when events arrive from the network. The group object is a
synchronization point for these threads of execution. The
implementation must ensure that group-state does not change during an
action. To this end, whenever an action is performed on group $g$: (1) the
group object is locked (2) status is checked (3) if it is NORMAL, the
action is performed. For example, the code for send looks like this:
\begin{codebox}
    public void send(int[] dests, byte[] msg) \{
        synchronized(this) \{
	    check_normal();
	    natSend(nat_env, dests, msg);
        \}
    \}
\end{codebox}

An application can have critical sections in which it must ensure
group state is NORMAL. It can also lock group state, and ensure it
does not change using a similar technique. A {\tt getStatus}
call is provided that returns the group state for group $g$. An example
of coding a critical section is:
\begin{codebox}
    synchronized(group) \{
        int stat = group.getStatus ();
        if (stat == NORMAL) \{
             /* Perform critical code here */
        \}
    \}
\end{codebox}

To maximize performance, the send/recv callbacks can also invoke
Ensemble actions. The callbacks enjoy the best latency since they do
not incur a thread-switch. 


%%*************************************************************%
%
%    Ensemble, 2_00
%    Copyright 2004 Cornell University, Hebrew University
%           IBM Israel Science and Technology
%    All rights reserved.
%
%    See ensemble/doc/license.txt for further information.
%
%*************************************************************%
\section{Csharp Application Interface}
Apart for minor language differences this API is the same as the Java
API. 

Locking is slightly different in C-sharp compared with Java. The {\tt
  lock} keyword is used instead of {\tt synchronized}. An application
that wants to protect access by several threads to the
Ensemble-connection should use something like this: 
\begin{codebox}
  lock (conn) 
  \{
      if (memb.current_status == Member.Status.Normal)
          memb.Cast("hello world");
       else
           Console.WriteLine("Blocked currently, please try again later");
  \}
\end{codebox}


\section{Native C \ensemble\ Application Interface (CE)}

The C application interface is very similar in design to the ML
interface. It is located in directory \sourcecode{ce}. It has been
modified from the original ML interface, so as to fit better into
the C language (type-system and native data structures).

There are seven callbacks a C application needs to define in order
to work with Ensemble. These are:

\begin{itemize}
\item 
{\tt install(env,ls,vs) }: called whenever a new view is installed.

\item 
{\tt exit()} :called when the member leaves. 

\item 
{\tt receive\_cast(env, origin, num, iovl)} :
called with the origin, an iovec array (and its length)
whenever a mulicast message arrives. 

\item
{\tt receive\_send(env, origin, num, iovl)} : 
called with the origin, an iovec array (and its length)
whenever a point-to-point message arrives. 

\item
{\tt flow\_block(env, origin, onoff)} : 
called whenever there are flow-control problems, and
the application should refrain from sending messages until further
notice.

\item
{\tt block(env) } : 
called whenever a view change is forthcoming. All
applications are blocked,  the old view is stabilized,
cleaned, and way is made for the new view. 

\item 
{\tt heartbeat(env, time)} : 
called every timeout. The timeout is specified in the \cval{jops}
structure. Timers are not exact, this callback may be called at
inaccurate times, or more often than neccessary. If accuracy is
required, the application should check the {\it time} argument. 
\end{itemize}

The environment argument which is the first argument in all seven
callbacks is registered when a {\it C-application interface} is created.

The types of the callbacks are as follows:
\begin{codebox}
typedef int         ce_rank_t ;
typedef int         ce_len_t ;
typedef void       *ce_env_t ;
typedef double      ce_time_t ;

typedef void (*ce_appl_install_t)(ce_env_t, ce_local_state_t*, ce_view_state_t*);

typedef void (*ce_appl_exit_t)(ce_env_t) ;

typedef void (*ce_appl_receive_cast_t)(ce_env_t, ce_rank_t, int, ce_iovec_array_t) ;

typedef void (*ce_appl_receive_send_t)(ce_env_t, ce_rank_t, int, ce_iovec_array_t) ;

typedef void (*ce_appl_flow_block_t)(ce_env_t, ce_rank_t, ce_bool_t) ;

typedef void (*ce_appl_block_t)(ce_env_t) ;

typedef void (*ce_appl_heartbeat_t)(ce_env_t, ce_time_t) ;
\end{codebox}


A {\tt ce\_appl\_intf\_t} is the type of a C application interface
({\it cappl}).  It can be created by the constructor {\tt
ce\_create\_intf}. There is no need for a destructor because Ensemble
frees the interface-structure and all related memory after the \cval{exit}
callback is invoked. An application interface is opaque, it can be
used to create an endpoint, and join a group. It cannot be used to 
join more than a single group.
\begin{codebox}
typedef struct ce_appl_intf_t ce_appl_intf_t ;
\end{codebox}


The constructor takes the above handlers as parameters, as well as
an environment variable. 

\begin{codebox}
ce_appl_intf_t*
ce_create_intf(
    ce_env_t env, 
    ce_appl_exit_t exit,
    ce_appl_install_t install,
    ce_appl_flow_block_t flow_block,
    ce_appl_block_t block,
    ce_appl_receive_cast_t cast,
    ce_appl_receive_send_t send,
    ce_appl_heartbeat_t heartbeat
);
\end{codebox}

The initial operation used to initiate a CE application is
\cval{ce\_Init}. It initializes the internal Ensemble data structures, and
processes command line arguments.

\begin{codebox}
void ce_Init(int argc, char **argv) ;
\end{codebox}


After a C application completes initialization it should pass control 
the Ensemble main loop via \cval{ce\_Main\_loop}.

\begin{codebox}
void ce_Main_loop ();
\end{codebox}
			 
In order to join a group, the \cval{ce\_Join} operation should be used.
\begin{codebox}
void ce_Join(ce_jops_t *ops, ce_appl_intf_t *c_appl) ;
\end{codebox}

\subsection{Group operations}
Similarly to the ML interface, the set of supported operations is:
Leave, Cast, Send, Send1, Prompt, Suspect, XferDone, Rekey,
ChangeProtocol, and ChangeProperties. Messages are arrays of
IO-vectors ({\it iovecs}), or C memory chunks.  The application can
send and receive iovec-arrays.

Multicast an iovec-array to the group. 
\begin{codebox}
void ce_Cast(
    ce_appl_intf_t *c_appl,
    int num,
    ce_iovec_array_t iovl
) ;
\end{codebox}

Send a point-to-point message to a set of group members.
\begin{codebox}
void ce_Send(
    ce_appl_intf_t *c_appl,
    int num_dests,
    ce_rank_array_t dests,
    int num,
    ce_iovec_array_t iovl
) ;
\end{codebox}

Send a point-to-point message to the specified group member.
\begin{codebox}
void ce_Send1(
    ce_appl_intf_t *c_appl,
    ce_rank_t dest,
    int num,
    ce_iovec_array_t iovl
) ;
\end{codebox}

The control actions are the same as the ML actions.

Leave a group. Following this downcall, \cval{exit} will be called, 
freeing the cappl. 
\begin{codebox}
void ce_Leave(ce_appl_intf_t *c_appl) ;
\end{codebox}

Ask for a new View.
\begin{codebox}
void ce_Prompt(
    ce_appl_intf_t *c_appl
);
\end{codebox}

 Report specified group members as failure-suspected.
\begin{codebox}
void ce_Suspect(
    ce_appl_intf_t *c_appl,
    int num,
    ce_rank_array_t suspects
);
\end{codebox}
	
Inform Ensemble that the state-transfer is complete. 
\begin{codebox}
void ce_XferDone(
    ce_appl_intf_t *c_appl
) ;
\end{codebox}

Ask the system to rekey.
\begin{codebox}
void ce_Rekey(
    ce_appl_intf_t *c_appl
) ;
\end{codebox}


Request a protocol change.  The \cval{protocol\_name} is a string
specifying the exact set of layers to use. The string is a colon
separated list of layers, for example:
Top:Heal:Switch:Leave:Inter:Intra:Elect:Merge:Sync:Suspect:Stable:\
Vsync:Frag\_Abv:Top\_appl:Frag:Pt2ptw:Mflow:Pt2pt:Mnak:Bottom
\begin{codebox}
void ce_ChangeProtocol(
    ce_appl_intf_t *c_appl,
    char *protocol_name
) ;
\end{codebox}


Request a protocol change, specifying properties.
\cval{properties} is a string containing a colon separated list of
properties. For example:
"Gmp:Sync:Heal:Switch:Frag:Suspect:Flow:Xfer".
The system deduces a protocol stack that abides by these properties.
\begin{codebox}
void ce_ChangeProperties(
    ce_appl_intf_t *c_appl,
    char *properties
) ;
\end{codebox}

All arguments to the group-action calls are copied into CE, hence the
application can use the run-time stack to create the arguments. There
is no need to allocate nor free the arguments. There are also
limitations on the sizes of arguments: 
\begin{itemize}
\item The maximal size of an iovector array is {\tt MAX\_SIZE\_IOVL}.
\item The maximal number of destinations for {\tt send/send1/suspect}
  is MAX\_NUM\_DESTS
\item Tha maximal size of a protocol string is: {\tt CE\_PROTOCOL\_MAX\_SIZE}.
\item Tha maximal size of a properties string is: {\tt CE\_PROPERTIES\_MAX\_SIZE}.
\end{itemize}

\subsection{Integration of other sockets into the main loop}
Ensemble works in an event driven fashion, where events can either
come from the network or the user. The system runs a loop that is
split between (1) waiting for input on incoming sockets using a
\cval{select} system call (2) Processing local
application send/recv and internal events. 

The application hands over control to Ensemble after initialization.
The application may wish to wait on its own sockets, e.g., \cval{stdin} (on
Unix). To this end, we also support adding, removing, and putting
handlers on sockets.

{\tt ce\_handler\_t} is the type of handler called when there is input
to process on a socket.
\begin{codebox}
typedef void (*ce_handler_t)(void*);
\end{codebox}

{\tt ce\_AddSockRecv} adds a socket to the list Ensemble listens to. 
When input on the socket occurs, this handler will be invoked
on the specified environment variable.
\begin{codebox}
void ce_AdddSockRecv(
    CE_SOCKET socket,
    ce_handler_t handler,
    ce_env_t env
);
\end{codebox}

{\tt ce\_RmvSockRecv} is called to remove a socket from the list 
Ensemble listens to.
\begin{codebox}
void ce_RmvSockRecv(
    CE_SOCKET socket
);
\end{codebox}


\subsection{Memory management}
The convention used throughout is that all data-structures passed from
the application CE are copied by CE, and all data-structures passed
from CE to the application are owned by the CE side (hence must not be
freed nor cached). This rule holds for all structures and data apart
from the iovec-arrays.

Ensemble does not copy messages from C to the ML heap, rather, it
separates C-memory and ML memory completely. Messages are received
from the network and read directly into C-buffers. Sent iovecs are
fragmented and sent directly on the network. Messages must be buffered
until all group members reliably receive them. To this end, a
reference counting scheme is used to track iovec liveness. When an
iovec's reference count reaches zero, it is freed. In other words,
iovec's are owned by Ensemble. They are received either from the 
user, or the network.

On linux, the type of an iovec is:
\begin{codebox}
typedef struct iovec ce_iovec_t ;
typedef ce_iovec_t *ce_iovec_array_t;
\end{codebox}

To get better control of the iovec memory system, the \cval{alloc} and 
\cval{free} functions can be set by the user. The definitions are in
\sourcecode{lib/mm.h} and  \sourcecode{lib/mm\_basic.h}.

These define the types of alloc and free functions.
\begin{codebox}
typedef void* (*mm_alloc_t)(int);
typedef void  (*mm_free_t)(char*);
\end{codebox}

The actual functions called to free and allocate iovec's.
\begin{codebox}
mm_alloc_t mm_alloc_fun;
mm_free_t mm_free_fun;
\end{codebox}

Use these functions to set \cval{alloc} and \cval{free}. Be careful to
do this exactly once at application initialization, before
starting Ensemble.
\begin{codebox}
void set_alloc_fun(mm_alloc_t f);
void set_free_fun(mm_free_t f);
\end{codebox}

The upshot of this is that when a user sends or casts a message,
Ensemble takes over the message body. When a message is
delivered to the application, the user may copy it, or perform any
read-only operation while in the receive callback. The application may
{\it not} modify a received iovec, or assume it owns it.

To use the CE library as a win32 DLL the user {\bf must} set the alloc
and free functions. This makes the CE library use the application's
allocation and deallocation functions. The simplest choices here are
the standard LIBC {\tt malloc} and {\tt free} functions. 

\subsection{The flat interface}
Using iovecs is a little complex for simple applications,
therefore, a simplified ``flat'' interface is also provided.

The flat\_receive callbacks take a C memory chunk, with it's length as
arguments. This releases the application from merging together the
set of buffers that consist an iovec-array, as well as releasing that
array.
\begin{codebox}
typedef void (*ce_appl_flat_receive_cast_t)(ce_env_t, ce_rank_t, ce_len_t, ce_data_t) ;

typedef void (*ce_appl_flat_receive_send_t)(ce_env_t, ce_rank_t, ce_len_t, ce_data_t) ;
\end{codebox}


Create a standard application interface using flat receive callbacks.
\begin{codebox}
ce_appl_intf_t*
ce_create_flat_intf(
    ce_env_t env, 
    ce_appl_exit_t exit,
    ce_appl_install_t install,
    ce_appl_flow_block_t flow_block,
    ce_appl_block_t block,
    ce_appl_flat_receive_cast_t cast,
    ce_appl_flat_receive_send_t send,
    ce_appl_heartbeat_t heartbeat
);
\end{codebox}


Cast and Send operations that work with buffers instead of iovec-arrays.
\begin{codebox}
void ce_flat_Cast(
    ce_appl_intf_t *c_appl,
    ce_len_t len, 
    ce_data_t buf
) ;

void ce_flat_Send(
    ce_appl_intf_t *c_appl,
    int num_dests,
    ce_rank_array_t dests,
    ce_len_t len, 
    ce_data_t buf
) ;

void ce_flat_Send1(
    ce_appl_intf_t *c_appl,
    ce_rank_t dest,
    ce_len_t len, 
    ce_data_t buf
) ;
\end{codebox}



\subsection{An example}

This section shows how to use the CE interface to write applications. 
We walk through the \sourcecode{ce/ce\_mtalk.c} demo program.

\sourcecode{ce/ce\_mtalk.c}, similarly to \sourcecode{demo/mtalk.ml},
is a multi-person talk program. Messages are read from the user via {\tt
stdin}, and multicasted to the network.

\cval{state\_t} is the state structure used by the program. It is the
environment variable registered in the C-interface. The state contains
the current view information, a pointer to its cappl, and a flag
indicating if we are blocked.
\begin{codebox}
typedef struct state_t \{
    int rank;
    int nmembers;
    ce_endpt_t endpt;
    ce_appl_intf_t *intf ;
    int blocked;
\} state_t;
\end{codebox}


A helper function to multicast a message if we are not blocked. 
We use the flat interface, to save the messy handling of iovec's.
\begin{codebox}
void cast(state_t *s, char *msg)\{
  if (s->blocked == 0)
    ce_flat_Cast(s->intf, strlen(msg)+1, msg);
\}
\end{codebox}
 

A handler for stdin. This callback is called whenever there is input
on the socket. The handler multicasts any message the user types on the
screen. Be careful not to send messages if we are blocked.
\begin{codebox}
void
stdin_handler(void *env)
\{
    state_t *s = (state_t*)env;
    char buf[100], *msg;
    int len ;
    
    TRACE("stdin_handler");
    fgets(buf, 100, stdin);
    len = strlen(buf);
    if (len>=100)
    
    /* string too long, dumping it.
     */
    return;

    msg = (char*) malloc(len+1);
    memcpy(msg, buf, len);
    msg[len] = 0;
    cast(s, msg);
\}
\end{codebox}


There is nothing special to do if we leave the group, the application
essentially halts.
\begin{codebox}
void main_exit(void *env){}
\end{codebox}

When a new view arrives, update the environment structure. The view
structures are owned by the CE library and may not be freed nor taken. To
maintain knowledge of the view by the rank, number of members, and
endpoint named are copied locally. 
\begin{codebox}
void main_install(void *env, ce_local_state_t *ls, ce_view_state_t *vs) 
\{
    state_t *s = (state_t*) env;
    
    s->rank = ls->rank;
    s->nmembers = ls->nmembers;
    s->blocked =0;
    memcpy(s->endpt.name, ls->endpt.name, CE\_ENDPT\_MAX\_SIZE);

    printf("\%s nmembers=\%d", ls->endpt.name, ls->nmembers);
\}
\end{codebox}

Ignore flow control problems. We are not suppose to have any of 
these, we are very low bandwidth.
\begin{codebox}
void main_flow_block(void *env, ce_rank_t rank, ce_bool_t onoff){}
\end{codebox}

Mark our blocked flag. 
\begin{codebox}
void main_block(void *env) \{
  state_t *s = (state_t*) env;

  s->blocked=1;
\}
\end{codebox}

Print out any message that we receive. Be careful not to free the
received message.
\begin{codebox}
void main_recv_cast(void *env, int rank, ce_len_t len, char *msg) \{
  state_t *s = (state_t*) env;

  printf("recv_cast <- \%d msg=\%s", rank, msg);
\}
\end{codebox}

Ignore send messages, we are not supposed to get any of these.
\begin{codebox}
void main_recv_send(void *env, int rank, ce_len_t len, char *msg) \{
\}
\end{codebox}

Ignore heartbeats.
\begin{codebox}
void main_heartbeat(void *env, double time) \{ \}
\end{codebox}


Create a join options structure, and join the group ``ce\_mtalk''.
Use a regular virtually-synchronous stack. Put a handler on {\tt
stdin} such that whenever there is input, it will be called. 
The join options structure can be allocated on the stack since it is
copied internally by CE. 

There is no need to set the transport in the join-options structure,
the system uses the environment variable ENS\_MODES in this case.
\begin{codebox}
void join() \{
    ce_jops_t jops; 
    ce_appl_intf_t *main_intf;
    state_t *s;
    
    /* The rest of the fields should be zero. The
     * conversion code should be able to handle this. 
     */
    memset(&jops, 0, sizeof(ce_jops_t));
    jops.hrtbt_rate=10.0;
    strcpy(jops.group_name, "ce_mtalk");
    strcpy(jops.properties, CE_DEFAULT_PROPERTIES);
    jops.use_properties = 1;
    
    s = (state_t*) malloc(sizeof(state_t));
    memset(s, 0, sizeof(state_t));
    
    main_intf = ce_create_flat_intf(s,
				    main_exit, main_install, main_flow_block,
				    main_block, main_recv_cast, main_recv_send,
				    
				    main_heartbeat);
    
    s->intf= main_intf;
    ce_Join (&jops, main_intf);
    
    ce_AddSockRecv(0, stdin_handler, s);
\}
\end{codebox}

The main entry point, initialize the ML side, process command line 
arguments, join the {\it ce\_mtalk} group, and turn control over
to the Ensemble event loop.
\begin{codebox}
int main(int argc, char **argv) \{
  
  ce_set_alloc_fun((mm_alloc_t)malloc);
  ce_set_free_fun((mm_free_t)free);

  ce_Init(argc, argv); /* Call Arge.parse, and appl_process_args */

  join();
  
  ce_Main_loop ();
  return 0;
\}
\end{codebox}

\subsection{Outboard mode}
It is possible to run any CE application through a remote Ensemble
server. Such a configuration is called an ``outboard'' configuration.
The idea is to run a daemon on the local host that listens to
TCP connections on a specific port, the daemon provides Ensemble
services to connected clients. Such services include joining/leaving groups, 
and sending/receiving multicast and point-to-point
messages on these groups. 

A CE application can be configured to run in outboard mode by linking
with the {\tt libceo} library (suffix {\tt .a} on Unix,  {\tt .lib}
on WIN32). The user must then make sure that the Ensemble daemon is
running, simply run the ce\_outboard executable.

Using a daemon configuration has several benefits as well as some
drawbacks. The advantages are:
\begin{itemize}
\item The library to link with is orders
  of a magnitude smaller than the full (inboard) Ensemble library.
\item The user-process is completely separated from the Ensemble
  server. This allows better debugging, and also facilitates writing simple
  interfaces to other languages (e.g., Java, Ada, ...). 
\end{itemize}

The disadvantage is performance loss. Each message now has to travel
through a socket and another process before being sent on the network;
vice-versa for received messages. This may outweigh the benefits of
simple client code, and a minimal sized library.

The current port used by the outboard mode is 5002. This is
configurable by running {\tt ce\_outboard} with the command line
argument {\tt -tcp\_port <port\_num>}, and modifying the 
{\tt OUTBOARD\_TCP\_PORT} parameter in {\tt ce/ce\_outboard\_comm.h}.

Care was taken to optimize memory consumption. Messages are sent
zero-copy from the client, and they are copied once only into the
server's buffers. A sent io-vector is consumed by the send
function. Received messages are allocated at the client's buffers and
handed to the application. After the application's receive callback,
io-vectors are released. It was possible at this point to allow the
application to take control of the io-vector, yet we chose to conform
with the memory convections of the inboard mode. 

\subsection{Thread-safety}
A thread-safe version of the library is also provided, it exports the
exact same interface as the basic library. To use it link with {\tt
libce\_mt.so}, or {\tt libceo\_mt.so}.  For WIN32 systems link with
{\tt .lib} instead. The thread-safe library requires the application
to synchronize its threads so they will not perform actions (send,
cast, prompt, etc.) on a group while it is stabilizing. There are
several thread-safe applications under the {\tt ce} directory: {\tt
ce\_rand\_mt.c, ce\_perf\_mt.c}, and {\tt ce\_mtalk\_mt.c}. These applications
use a lock to ensure that sensitive group-state is accessed safely.
Threads atomically check group-state before performing an Ensemble
action.

The thread-safe library is designed as a wrapper around the basic
library. A single thread runs both Ensemble main-loop and application
callback handlers; this thread is known as the {\it Ensemble
thread}. Other threads are refered to as {\it user-threads}. When a
user-thread performs an action outside of a handler, the action is
stored in a pending queue. A byte is sent through a socket to the
Ensemble thread, notifying it that there is pending work to do. 
Asynchronously, the Ensemble thread ``wakes up'', consumes the queue,
and performs all pending actions. Any actions invoked in the interim
will also be stored in the pending queue; to be consumed along with
the rest.

Any action invoked from within a callback is performed directly when
the callback is completed and control returns to Ensemble. 

Since a single thread performs the Ensemble main-loop as well as all user
callbacks, callbacks must be short. Long-term computations should {\bf
not} be performed in the context of a callback.

There are three sensitive periods in which issuing Ensemble actions is
not allowed, these are when {\it joining, leaving}, and {\it
blocking}. A group is in:
\begin{description}
\item {\it joining} state: between {\tt ce\_Join} and
the first {\tt install} callback.
\item {\it leaving} state: between {\tt ce\_Leave} and the {\tt exit}
callback.
\item {\it blocking} state: between the {\tt block} callback and the
succeeding {\tt install} callback.
\end{description}

An example of a simple multi-threaded application is provided in {\tt
ce/ce\_mtalk\_mt.c}.

The overhead of adding thread-safety is 10\% in the worst case, and
normally much less than that. This should be acceptable for most 
applications. 

\subsection{A multi-threaded multi-person chat program}
This program is a multi-threaded version of {\tt ce\_mtalk.c}
Here, we walk through it and explain the interface and how to 
use it. 

Include the system-independent thread header file, so we'll be
able to use locks. 
\begin{codebox}
#include "ce_trace.h"
#include "ce.h"
#include "ce_threads.h"
#include <stdio.h>
#include <memory.h>
#include <malloc.h>
\end{codebox}


The {\tt NAME} variable is used for internal tracing purposes of
CE. There is no need to set it for standard user programs.
\begin{codebox}
#define NAME "CE_MTALK_MT"
\end{codebox}


Apart for standard view state, the state structure keeps track
of the current status of the group: blocked, joining, or leaving.
\begin{codebox}
typedef struct state_t \{
    int rank;
    int nmembers;
    ce_endpt_t endpt;
    ce_appl_intf_t *intf ;
    int blocked;
    int joining;
    int leaving;
    ce_lck_t *mutex;
\} state_t;
\end{codebox}

Although we must define these callbacks, they do nothing in this
program.
\begin{codebox}
void main_exit(void *env)
\{\}

void
main_flow_block(void *env, ce_rank_t rank, ce_bool_t onoff)
\{\}

void
main_recv_send(void *env, int rank, ce_len_t len, char *msg)
\{\}

void
main_heartbeat(void *env, double time)
\{\}
\end{codebox}



{\tt main\_install} updates the view state. A lock must be taken to
protect view state, as other threads may concurrently read the state.
\begin{codebox}
void
main_install(void *env, ce_local_state_t *ls, ce_view_state_t *vs)
\{
    state_t *s = (state_t*) env;
    
    ce_lck_Lock(s->mutex); \{
        s->rank = ls->rank;
        s->nmembers = ls->nmembers;
        s->blocked =0;
        memcpy(s->endpt.name, ls->endpt.name, CE\_ENDPT\_MAX\_SIZE);
        s->blocked =0;
        s->joining =0;
	
        printf("\%s nmembers=\%d", ls->endpt.name, ls->nmembers);
        TRACE2("main_install",ls->endpt.name); 
     \} ce_lck_Unlock(s->mutex);
\}
\end{codebox}


The group is blocked, lock the state structure, and update the blocked
flag. This notifies other threads not to attempt sending messages
until the upcoming install callback. A lock must be taken to protect view
state, as other threads may read it.
\begin{codebox}
void
main_block(void *env)
\{
    state_t *s = (state_t*) env;
    
    ce_lck_Lock(s->mutex); \{
        s->blocked=1;
   \} ce_lck_Unlock(s->mutex);
\}
\end{codebox}

Received a message, print who sent it and its content.
\begin{codebox}
void
main_recv_cast(void *env, int rank, ce_len_t len, char *msg)
\{
    printf("\%d -> msg=\%s", rank, msg); fflush(stdout);
\}
\end{codebox}

{\tt get\_input} is a non-terminating function performed by the user-thread of
this program. In an infinite loop, read a line from stdin, 
and multicast it to the group. Prior to sending, check that the group is not
blocked/joining/leaving. Status flags are shared information, and
may be updated concurrently by an {\tt install} or {\tt block}
callback. Hence, a lock is taken to protect access to the flags.
\begin{codebox}
void
get_input(void *env)
\{
    state_t *s = (state_t*)env;
    char buf[100], *msg;
    int len ;

    while (1) \{
        TRACE("stdin_handler");
        fgets(buf, 100, stdin);
        len = strlen(buf);
        if (len>=100)
            /* string too long, dumping it.
             */
            return;
        	
        msg = ce_copy_string(buf);
        TRACE2("Read: ", msg);
	
        ce_lck_Lock(s->mutex); \{
            if (s->joining || s->leaving || s->blocked)
               	printf("Cannot send while group is joining/leaving/blocked");
            else \{
               	ce_flat_Cast(s->intf, strlen(msg)+1, msg);
            \}
        \} ce_lck_Unlock(s->mutex);
    \}
\}
\end{codebox}


Initialize the state structure, and join the ``ce\_mtalk'' Ensemble group.
Take care to initialize the lock, and set the joining flag. The flag
will be unset, allowing sending messages, in the first install callback.
\begin{codebox}
state_t *
join(void)
\{
    ce_jops_t jops; 
    ce_appl_intf_t *main_intf;
    state_t *s;
    
    /* The rest of the fields should be zero. The
     * conversion code should be able to handle this. 
     */
    memset(&jops, 0, sizeof(ce_jops_t));
    jops.hrtbt_rate=3.0;
    strcpy(jops.group_name, "ce_mtalk_mt");
    strcpy(jops.properties, CE_DEFAULT_PROPERTIES);
    jops.use_properties = 1;
    
    s = (state_t*) malloc(sizeof(state_t));
    memset(s, 0, sizeof(state_t));
    
    main_intf = ce_create_flat_intf(s,
				    main_exit, main_install, main_flow_block,
				    main_block, main_recv_cast, main_recv_send,
				    
				    main_heartbeat);
    
    s->intf= main_intf;
    s->mutex = ce_lck_Create();
    s->joining = 1;
    ce_Join (&jops, main_intf);
    return s;
\}
\end{codebox}


Initialize Ensemble, start the reader thread, and go to sleep.
\begin{codebox}
int
main(int argc, char **argv)
\{
    state_t *s;
    
    ce_set_alloc_fun((mm_alloc_t)malloc);
    ce_set_free_fun((mm_free_t)free);

    ce_Init(argc, argv); /* Call Arge.parse, and appl_process_args */

    /* Join the group
     */
    s = join();
    
    /* Create a thread to read input from the user.
     */
    ce_thread_Create(get_input, s, 10000);
    
    ce_Main_loop ();
    return 0;
\}
\end{codebox}

\subsection{The Join Options structure}
The {\tt ce\_jops\_t} structure contains all the options an application
joining Ensemble wishes requests of the created endpoint. All string
arguments use a fixed sized char array which should look like a
C-string: initial set of readable ASCII characters followed by zeros. 
There has to be a terminating NULL character. To simplify C memory
management all string arguments use a fixed sized char arrays.
Boolean arguments use zero for false and one for true. 

\begin{codebox}
typedef struct ce_jops_t \{
    ce_time_t hrtbt_rate ;                     
    char transports[CE_TRANSPORT_MAX_SIZE] ;   
    char protocol[CE_PROTOCOL_MAX_SIZE] ;      
    char group_name[CE_GROUP_NAME_MAX_SIZE] ;  
    char properties[CE_PROPERTIES_MAX_SIZE] ;  
    ce_bool_t use_properties ;                 
    ce_bool_t groupd ;                         
    char params[CE_PARAMS_MAX_SIZE] ;          
    ce_bool_t client;                          
    ce_bool_t debug ;                          
    ce_endpt_t endpt;                          
    char princ[CE_PRINCIPAL_MAX_SIZE] ;       
    char key[CE_KEY_SIZE] ;                   
    ce_bool_t secure ;                        
\} ce_jops_t ;
\end{codebox}

\begin{description}
\item{hrtbt\_rate:} The rate of heartbeat callbacks. A resonable
  setting would be seconds, or hundreds of milliseconds. 
\item{transports:} Which transports to use want the endpoint to
  use. For example:  ``DEERING'', or ``UDP:TCP''.
\item{protocol:} Which protocol to use. This allows setting by hand
  the stack (set of layers) used by the endoint. Not for casual use.
\item{group\_name:} What is the name of the group to join. 
\item{properties:} What is the set of properties the endpoint stack
  should adhere to. This is the prefered way of creating endpoints,
  set the {\tt use\_properties} flag to 1 in this case. The default
  setting is "Gmp:Sync:Heal:Switch:Frag:Suspect:Flow:Slander". This
  includes Group membership (Gmp:Heal).
\item{groupd:} Should the endpoint use an external groupd membership
  manager?
\item {debug:} Set this flag to use a debugging stack. This
  considerably degrades performance. This is only useful if there is a
  possiblity that the stack itself has a bug. 
\item {endpt:}
  Normally, Ensemble generates a unique endpoint name for each
  group an application joins (this is what happens if you leave
  'endpt' unmodified).  The application can optionally provide
  its own endpoint name.  It can, for instance, reuse an
  endpoint name generated by Ensemble for another group (the
  same endpoint name can be used to join any number of groups).
  The application can even generate an endpoint on its own.
  Such names should be unique.  It is best if they contain only
  printable characters and do not contain spaces because
  Ensemble my print them out in debugging or error messages.
  (The names generated by Ensemble fit these characteristics.)
  See ensemble/type/endpt.mli for more information.

\item{princ:} The principal name of this endpoint. Used by the
  security code. 

\item{key:} The encryption and MAC keys used by Ensemble. If the stack
  is secure (authenticated and encrypted), then it uses two symmetric
  keys: one for encryptions and one for MACing messages. The {\tt key}
  field allows the user to set these initial keys. The key is of size
  (exactly) 32 bytes, the first 16bytes are used as the initial encryption
  key, and the last 16bytes are used as the initial MAC key.

\item{secure:} Should we use a secure stack. A secure stack is one
  that authenticates endpoints and MACs and encrypts messages. The
  default encryption mechanism is RC4, the default MAC algorithm is
  keyed MD5. 
\end{description}


\subsection{The view structure}
The view is split into pieces: the local view, describing the local
state of the endpoint, and the group view, describing the global state
of the group. 

The local view is defined as:
\begin{codebox}
typedef struct ce_local_state_t {
    ce_endpt_t endpt ;             
    ce_addr_t addr ;               
    ce_rank_t rank ;               
    char name[CE_NAME_MAX_SIZE];   
    int nmembers ;                 
    ce_view_id_t view_id ;         
    ce_bool_t am_coord ;           
} ce_local_state_t ;

\end{codebox}
\begin{description}
\item{endpt:} The endpoint name.
\item{addr:} The address of this endpoint. This is of form ``Udp(128.25.4.98)''.
\item{rank:} The rank of the member in the group. Member ranks are
  between 0 and the (number of group members)-1. Ranks are
  decided internally by Ensemble, and each member has a unique rank within
  a group. 
\item{name:} The full name of the endpoint: includes the rank,
  endpoint name and the logical time (internal to Ensemble).
\item{nmembers:} The number of members in the group. 
\item{view\_id:} A unique id for the group, provided by Ensemble. 
\item{am\_coord:} Is this endpoint the group coordinator? 
\end{description}

The group view is defined as: 
\begin{codebox}
typedef struct ce_view_state_t {
    char version[CE_VERSION_MAX_SIZE] ;     
    char group[CE_GROUP_NAME_MAX_SIZE] ;    
    char proto[CE_PROTOCOL_MAX_SIZE] ;      
    ce_rank_t coord ;              
    int ltime ;                    
    ce_bool_t primary ;            
    ce_bool_t groupd ;             
    ce_bool_t xfer_view ;          
    char key[CE_KEY_SIZE] ;        
    int num_ids ;                  
    ce_view_id_t *prev_ids ;       
    char params[CE_PARAMS_MAX_SIZE];
    ce_time_t uptime ;             
    ce_endpt_t *view ;             
    ce_addr_t  *address ;          
} ce_view_state_t ;
\end{codebox}

\begin{description}
\item{version:} The distribution version of this Ensemble library.
\item{group:} The group name. 
\item{proto:} The protocol stack used.
\item{coord:} What is the rank of the group coordinator. Currently,
  this is always member 0. We do not expect this to change in the
  future.               
\item{ltime:} The logical time of the view. Ensemble maintains this
  counter for virtual-synchrony purposes.                      
\item{primary:} Is this a primary partition? This flag matters only if
  the stack includes a primary partition layer.             
\item{groupd:} Are we using the group-daemon? 
\item{xfer\_view:} Is this view a state-transfer view? this flag
  matters only if the stack includes a state-transfer layer (Xfer). 
\item{key:}  What is the security key. This field is non-zero only
  when the stack is a secure stack. 
\item{num\_ids:} The number of previous views that merged together to
  create the current view. 
\item{prev\_ids:} An array of view ids, one for each of the views that
  merged to create the current view. 
\item{params:} The parameters provided for this endpoint. 
\item{uptime:} The amount of time this endpoint has been running.              
\item{view:} A array of size {\tt nmembers} of the endpoint ids in the group.
\item{address:} A array of size {\tt nmembers} of the addresses of
  endpoints in the group.
\end{description}


\subsection{Notes}
Of the four transports supported by Ensemble : NETSIM, UDP, TCP, and
DEERING, NETSIM is not supported for the thread-safe library. A socket
is used internally, and NETSIM does not allow any
external communication. Hence, it is unsupported.

To maintain compatibility with the non-thread-safe version,
{\tt ce\_Main\_loop} should send the calling thread to sleep forever. It
creates a semaphore and sleeps on it. The actual Ensemble thread is
spawned by the {\tt ce\_Init} call. An application that wants to run
the Ensemble main loop in a separate thread needs to call
{\tt ce\_Init} without calling the {\tt ce\_Main\_Loop}.






\section*{Acknowledgments}
Thanks to Greg Sharp for comments on previous versions of this document.
\end{document}
