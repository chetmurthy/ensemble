%*************************************************************%
%
%    Ensemble, 2_00
%    Copyright 2004 Cornell University, Hebrew University
%           IBM Israel Science and Technology
%    All rights reserved.
%
%    See ensemble/doc/license.txt for further information.
%
%*************************************************************%
\documentclass[11pt]{article}
\usepackage{fullpage}
\usepackage{alltt}
\usepackage{hyperref}

%%%%%%%%%%%%%%%%%%%%%%%%%%%%%%%%%%%%%%%%%%%%%%%%%%
%%pdf-latex stuff
%
\newif\ifpdf
\ifx\pdfoutput\undefined
   \pdffalse
\else
   \pdfoutput=1
   \pdftrue
\fi

\ifpdf
  \usepackage[pdftex]{graphicx}
  \pdfcompresslevel=9
  \DeclareGraphicsExtensions{.jpg,.pdf,.mps,.png}
\else
  \usepackage[dvips]{graphics}
\fi
%%%%%%%%%%%%%%%%%%%%%%%%%%%%%%%%%%%%%%%%%%%%%%%%%%

%*************************************************************%
%
%    Ensemble, 1_42
%    Copyright 2003 Cornell University, Hebrew University
%           IBM Israel Science and Technology
%    All rights reserved.
%
%    See ensemble/doc/license.txt for further information.
%
%*************************************************************%
\newcommand {\ensemble}		{Ensemble}
\newcommand {\horus}		{Horus}
\newcommand {\caml}		{Objective Caml}
\newcommand {\chk}		{$\surd$}

\newcommand{\hide}[1]           {~}
\newcommand{\note}[1]           {{\bf [#1]}}
\newcommand{\todo}[1]           {{\note{TODO: #1}}}
\newcommand{\mlval}[1]          {{\bf {#1}}}
\newcommand{\cval}[1]           {{\tt {#1}}}

\newcommand {\incgraphics}[1]   {\includegraphics{fig/#1}}

\newcommand {\sourcecode}[1]    {{\bf {#1}}}
%\newcommand {\sourcelayer}[1]  {{\htmladdnormallink{#1}{../../layers/#1}}}
%\newcommand {\sourceutil}[1]   {{\htmladdnormallink{#1}{../../util/#1}}}
\newcommand {\sourcelayer}[1]   {\sourcecode{layers/#1}}
\newcommand {\sourcedemo}[1]    {\sourcecode{demo/#1}}
\newcommand {\sourceutil}[1]    {\sourcecode{util/#1}}
\newcommand {\sourcetrans}[1]   {\sourcecode{trans/#1}}
\newcommand {\sourcetype}[1]    {\sourcecode{type/#1}}
\newcommand {\sourceappl}[1]    {\sourcecode{appl/#1}}
\newcommand {\sourcece}[1]      {\sourcecode{ce/#1}}
\newcommand {\sourcehot}[1]     {\sourcecode{hot/#1}}

\newcommand {\ECast}[0]         {\mlval{ECast}}
\newcommand {\ESend}[0]         {\mlval{ESend}}
\newcommand {\ESubCast}[0]      {\mlval{ESubCast}}
\newcommand {\ECastUnrel}[0]    {\mlval{ECastUnrel}}
\newcommand {\ESendUnrel}[0]    {\mlval{ESendUnrel}}
\newcommand {\EMergeRequest}[0] {\mlval{EMergeRequest}}
\newcommand {\EMergeGranted}[0] {\mlval{EMergeGranted}}
\newcommand {\EOrphan}[0]       {\mlval{EOrphan}}
\newcommand {\EAccount}[0]      {\mlval{EAccount}}
\newcommand {\EAsync}[0]        {\mlval{EAsync}}
\newcommand {\EBlock}[0]        {\mlval{EBlock}}
\newcommand {\EBlockOk}[0]      {\mlval{EBlockOk}}
\newcommand {\EDump}[0]         {\mlval{EDump}}
\newcommand {\EElect}[0]        {\mlval{EElect}}
\newcommand {\EExit}[0]         {\mlval{EExit}}
\newcommand {\EFail}[0]         {\mlval{EFail}}
\newcommand {\EGossipExt}[0]     {\mlval{EGossipExt}}
\newcommand {\EGossipExtDir}[0]    {\mlval{EGossipExtDir}}
\newcommand {\EInit}[0]     {\mlval{EInit}}
\newcommand {\ELeave}[0]     {\mlval{ELeave}}
\newcommand {\ELostMessage}[0]     {\mlval{ELostMessage}}
\newcommand {\EMergeDenied}[0]     {\mlval{EMergeDenied}}
\newcommand {\EMergeFailed}[0]     {\mlval{EMergeFailed}}
\newcommand {\EMigrate}[0]         {\mlval{EMigrate}}
\newcommand {\EPresent}[0]         {\mlval{EPresent}}
\newcommand {\EPrompt}[0]          {\mlval{EPrompt}}
\newcommand {\EProtocol}[0]        {\mlval{EProtocol}}
\newcommand {\ERekey}[0]           {\mlval{ERekey}}
\newcommand {\ERekeyPrcl}[0]       {\mlval{ERekeyPrcl}}
\newcommand {\EStable}[0]          {\mlval{EStable}}
\newcommand {\EStableReq}[0]       {\mlval{EStableReq}}
\newcommand {\ESuspect}[0]         {\mlval{ESuspect}}
\newcommand {\ESystemError}[0]     {\mlval{ESystemError}}
\newcommand {\ETimer}[0]           {\mlval{ETimer}}
\newcommand {\EView}[0]            {\mlval{EView}}
\newcommand {\EXferDone}[0]        {\mlval{EXferDone}}
\newcommand {\ESyncInfo}[0]        {\mlval{ESyncInfo}}
\newcommand {\ESecureMsg}[0]       {\mlval{ESecureMsg}}
\newcommand {\EChannelList}[0]     {\mlval{EChannelList}}
\newcommand {\EFlowBlock}[0]       {\mlval{EFlowBlock}}
\newcommand {\EAuth}[0]            {\mlval{EAuth}}
\newcommand {\ESecChannelList}[0]  {\mlval{ESecChannelList}}
\newcommand {\ERekeyCleanup}[0]    {\mlval{ERekeyCleanup}}
\newcommand {\ERekeyCommit}[0]     {\mlval{ERekeyCommit}}

\newcommand {\Dn}[1]               {\mlval{Dn(E{#1})}}
\newcommand {\Up}[1]               {\mlval{Up(E{#1})}}

\newlength{\figurewidth}
\newsavebox{\figurebox}
\newenvironment{codebox}{
\figurewidth\hsize
\addtolength{\figurewidth}{-4\fboxsep}
\addtolength{\figurewidth}{-4\fboxrule}

\begin{alltt}
\sbox{\figurebox}\bgroup
\begin{minipage}{\figurewidth}
}{
\end{minipage}
\egroup
\fbox{\usebox{\figurebox}}
\end{alltt}
}

% Ohad.
% A macro for putting scaled figures in boxes.
\newcommand{\putfigfbox}[2]     {\fbox { \scalebox{#1}{\includegraphics{#2}} } }




\title{\ensemble\ Tutorial}
\author{Mark Hayden, Ohad Rodeh \\
\small{Copyright \copyright\ 1997 Cornell University,
                             2000 Hebrew  University,
                             2002 IBM Israel Science and Technology }
}

\begin{document}
\maketitle

\begin{abstract}
\ensemble\ is a reimplementation of the \horus\ reliable group communication
system in the \caml\ programming language.  This document describes:
\begin{itemize}
\item
How to configure and execute the applications included with \ensemble.
\item
The client application interface.
\item
The Server (OCaml) \ensemble\ application interface.
\end{itemize}
\end{abstract}

\newpage
\tableofcontents
\newpage

\section{Introduction}
This documentation assumes that the reader has some familiarity with group
communication. There are many texts that describe how to use and build
group-communication system.

Ensemble is structured as a client-server system with a server
providing group-communication services through a socket based
interface. Clients can connect to the server and send/receive reliable
point-to-point and multicast messages. There should be one server
running on a host, and clients should be located on the same
host. This allows using insecure communication for client-server
traffic.  The server is written (mostly) in the OCaml programming
language, the client is a small library that has
implementations in several languages. At the time of writing there are
clients in C and Java.

Previous versions of the system did not distinguish between client and
server. The client was implemented with an internal server. This
provides good performance. However, since the server is written in ML,
in order to link with a C program written by a user the
foreign-language interface of ML needs to be used. This causes very
difficult portability issues. As of release 2.00 we decided to
separate client from server; this should improve portability at the
expense of performance. 

\section{Quick Installation} 

Several demonstration applications are included with \ensemble.
These can give a sense of the kinds of facilities provided by group
communication to those who have not used a group communication
toolkit before.  The demos can also serve as starting points for
building new applications.  These applications are briefly described
here along with how to execute them and the various command-line
options and environment variables they use.

\subsection{Compiling}
Please see the file \sourcecode{ensemble/INSTALL} for instructions on
installing \ensemble\ if you have not done so already.

\subsection{Environment Variables}
Detailed information is given in Section~\ref{subsect:config} for
initializing environment variables.  Here we give the bare minimum
you need in order to get started.  We assume that you will (at least
initially) be using the gossip server for processes to locate each
other.  \mlval{ENS\_GOSSIP\_PORT} must be set to a port number that
is not used by other applications.  Normally, user applications
cannot use port numbers below 1000.  \mlval{ENS\_GOSSIP\_HOSTS} must
be set to a list of colon-separated host names where the gossip
server may be found.  If you wish to use port 7500 for the gossip
server on hosts ``ely'' and ``natasha,'' you would set these
environment variables as follows (in Unix csh):
\begin{verbatim}
% setenv ENS_GOSSIP_PORT 7500
% setenv ENS_GOSSIP_HOSTS ely:natasha
\end{verbatim}

Throughout this tutorial, we assume you are using the Unix csh or
tcsh shell.  To set an environment variable in the bash shell, you
would do the following:
\begin{verbatim}
% export ENS_GOSSIP_PORT=7500
% export ENS_GOSSIP_HOSTS=ely:natasha
\end{verbatim}

Remember that these must be set for all Ensemble programs you run.
You may wish to add the configuration to your \sourcecode{.cshrc} or
equivalent.

\subsection{Executing Applications}
Applications by default require a gossip server process to be running
in order to contact each other.  Before executing an application, a
gossip server must be started one of the hosts listed in
\mlval{ENS\_GOSSIP\_HOSTS}:
\begin{verbatim}
% gossip
 ...
\end{verbatim}
On the same or other hosts execute several instances of an
application, such as \sourcedemo{mtalk}:
\begin{verbatim}
% mtalk
 ...
\end{verbatim}


\newpage
\part{The Server}
This chapter describes how to build server-side programs. The reason
users should be wary of writing such programs is that the server
operates in a soft read-time environment. The server is written in the
OCaml programming language, a single thread of execution is used. To
improve performance bulk-data for user messages is not allocated on
the ML heap, which is garbage collection, it is allocated on large
chunks of memory allocated with {\tt malloc}. Bulk-data extents are
also called io-vectors and the memory used to hold them is also called
iovec-memory. To reduce server memory foot-print the size of
iovec-memory is limited, at the time of writing we are using 6
mega-bytes. Since memory is limited memory-allocation can fail. The
server handles this with flow-control protocols limiting the amount of
incoming messages to fit the amount of available iovec-memory. In
order to maintain responsiveness to incoming messages pure CPU
processing (such as search on a database) should be limited. If the
messaging engine does not receive sufficient CPU every, say, 100
milliseconds then performance is going to suffer dramatically.

The casual user will be better served by the chapter on writing client
programs that do not suffer from these limitations. If, however, you
are undaunted then this chapter is for you. 

%*************************************************************%
%
%    Ensemble, 1.10
%    Copyright 2001 Cornell University, Hebrew University
%    All rights reserved.
%
%    See ensemble/doc/license.txt for further information.
%
%*************************************************************%
\section{The Programs}

Notes:
\begin{itemize}
\item
please note that warning and error messages printed by Ensemble are
not prefixed with the name of the program generating the message, but
rather the name of the module.
\end{itemize}

\subsection{Mtalk: Multi-person Talk}
This is a multi-person talk demo.  As \mlval{mtalk} processes are created, they merge
into a single group.  Input typed at one process is broadcast to the rest of the
processes in the group.

\subsection{Wbml: Distributed Whiteboard}
This is a graphical white board demo.  It uses the CamlTk library to implement a
graphical user interface\footnote{not supported under Windows/NT}.  When members are
in the same group, lines drawn on one instance are broadcast to the rest of the
group, who also draw the lines.  It supports the switching of protocols.  Initially,
wbml has an auto-merge protocol in its stack so the members merge together.  This can
be removed to disable partition healing.  Adding the XFER protocol, causes members to
transfer their state on view changes; the TOTAL protocol enable totally ordered
communication.  Initially, these extra protocols are not included in the protocol
stack.

\subsection{Ensemble: Text-based Interface}
\label{section:ensemble-demo}
This program provides a text-based interface to the Ensemble group
communication facilities.  You can run it to directly see what happens in an
Ensemble application.  You start up the program and it prints out messages
decribing changes to the membership of the group.  You can type in commands
such as ``cast hello'' which causes Ensemble to broadcast ``hello'' to the
other members of the group, who get ``cast hello'' printed out.  This program
can be used as a subprocess of an application for doing basic group
communication.  The normal usage is to set up pipes to and from the standard
input and output of the \mlval{ensemble} process.

In order to distinguish different applications that are using this interface to
communicate, you may wish to use the \mlval{-group\_name} option to set the
name of the group.

The input of the program must be formatted in text lines as follows:
\begin{itemize}
\item
{[cast \mlval{msg}]} broadcasts the following message to the group, where
\mlval{msg} is the remainder of the input line.  (Normally, the broadcaster
will not receive the message).
\item
{[send \mlval{dest} \mlval{msg}]} sends a point-to-point message to the rest of the
group.  \mlval{dest} is the endpoint identifier of the group member you wish to send
the message to.  \mlval{msg} is the remainder of the input line.
\item
{[leave]} causes the member to leave the group.  This will eventually result in
an exit message being output and then the \mlval{ensemble} process will exit.
\end{itemize}

The output of the program consists of lines in one of the following formats:
\begin{itemize}
\item
{[endpt \mlval{endpoint\_id}]} is output as the first line and only appears once.  It
gives the name of this application as it will appear in views.
\item
{[view \mlval{nmembers} \mlval{my\_rank} \mlval{view}]} describes a new view of the
group.  Initially, every member begins in its own singleton view.  Other members are
added through automatic merging with other views.  Members are removed through
failure detections.  \mlval{nmembers} is the number of members of the group.
\mlval{my\_rank} is this member's rank in the new view.  \mlval{protocol} is the name
of the protocol being used.  \mlval{view} is a space-separated list of the endpoint
identifiers of the members of the group.
\item 
{[cast \mlval{origin} \mlval{msg}]} displays a broadcast received from the member of
rank \mlval{origin}.
\item
{[send \mlval{origin} \mlval{msg}]} displays a point-to-point message received
from member of rank \mlval{origin}.
\item
{[exit]} notifies that this member has left the group as a result of a previous
\mlval{leave}.  This is the last line output by \mlval{ensemble}.
\end{itemize}

\subsection{Gossip: Group Locator Service}
This is not really an application.  The gossip server works in conjunction with
the Ensemble \mlval{UDP} communication transport to simulate low-bandwidth
gossip broadcast for systems that do not have IP multicast.  See the discussion
on transports below.  The group communication protocols require some
``gossipping'' mechanism in order to detect and heal partitions in the system.
When an application wishes to gossip with other partitions, it broadcasts a
message via the \mlval{gossip} servers.  This sends messages to the
\mlval{gossip} servers.  The \mlval{gossip} servers then forward the message to
all processes they have heard from recently to simulate a broadcast.  When an
application is using the \mlval{UDP} transport and not the \mlval{DEERING}
transport (\mlval{UDP} is the default), it is necessary for a \mlval{gossip}
process to be running somewhere in the system.

\subsection{Groupd: Membership Service (formerly called Domain)}
\label{section:groupd}
Normally, \ensemble\ application groups implement their own group membership
protocol.  However, they have the option of using the \ensemble\ membership
service implemented by the \mlval{groupd} application.  \mlval{groupd} is a
service for managing multiple process groups.  It uses a \emph{core} group of
\ensemble\ processes to participate in managing these groups.  Clients connect
to the service via TCP connections, through which they request to join and
leave groups.  The service supports a simple protocol through which the clients
can obtain virtual synchronous properties.  The service also supports weaker
properties that give faster membership notifications.

\note{We emphasize that Ensemble applications can operate independently of a
membership service.}

Some of the benefits of using this service are:
\begin{itemize}
\item
When there are no membership changes, the clients communicate directly between
themselves, so the membership service has no affect on performance.
\item
The service implements group membership for multiple groups.  The costs of the
group membership protocols (such as failure detection) are shared over the
groups.
\item
Because applications are sharing the same membership service, they see
consistent views and failure detections.
\item
The client part of the protocol for implementing virtual synchrony is simple.
Most of the complexity is in the server.  This allows client programs to be
implemented in languages other than ML, but save much of the programming burden
because the servers handle the ``hard'' group membership protocols.  The client
TCP interface is described in the \ensemble\ reference manual.
\item
Applications that do not need the full virtual synchrony properties can use
weaker synchronization protocols and get faster view changes.
\item
The service allows groups to scale to larger sizes.  The membership servers do
not need to run on all the hosts on which the clients run, so clients can be on
more hosts than are normally supported by \ensemble.
\end{itemize}

\paragraph{Executing Groupd:}
In order to run Groupd, set the \mlval{ENS\_GROUPD\_PORT} environment
variable to select the TCP port for the service to use.  The membership service
is executed through the \mlval{groupd} application program:
\begin{verbatim}
% groupd
\end{verbatim}
It takes command-line arguments similar to the other \ensemble\ demonstration
programs.  Normally, each host runs a server.

Other demo applications use the service when the \mlval{-groupd} command-line
argument is selected.  For example:
\begin{verbatim}
% mtalk -groupd
\end{verbatim}
Note that you must have a \mlval{groupd} server running on the same host as
mtalk for this to work.


\subsection{Perf: Performance Tests}
This program includes a variety of performance tests for Ensemble.

\paragraph{Ring:}
This test is run with the \mlval{-prog ring} option.  Say that there are $n$
members.  Each process first waits until there are $n$ members.  It then sends
$k$ messages, and waits for $(n - 1)k$ messages from other members.  It
measures the time for this, and does so a number of times to determine the
average and variance.  This can be done for varying $n$, $k$, message size, and
protocol.

The time between the rounds is a measure of latency.  The total amount of data
sent between the rounds is a measure of bandwidth.  The total number of
messages sent between rounds is a measure of throughput.  For good
measurements, set the parameters as follows:

\begin{center}
\begin{tabular}{|l|c|c|}			   \hline
measure		& $k$		& message size	\\ \hline \hline
latency		&  1		& 0		\\ \hline
throughput	& large		& 0		\\ \hline
bandwidth	&  1		& large		\\ \hline
\end{tabular}
\end{center}

Additional command-line arguments (with default values in parentheses):
\begin{description}
\item [-n \#]: number of members ($2$ members)
\item [-s \#]: size in bytes of application messages ($0$ bytes)
\item [-r \#]: number of rounds ($300$ rounds)
\item [-k \#]: messages per round ($1$ message per round)
\end{description}
These values must be set by all members.  All members must use the same values for
all of the arguments except message size.

\todo{The other performance tests are undocumented.}

\subsection{Rand: Virtual Synchrony Debugging Tool}
This demo is used to test \ensemble.  It uses simulated communication and
introduces random process failures to check for proper behavior of the group
membership protocols.

\subsection{Fifo: Fifo Communication Debugging Tool}
This demo is used to test \ensemble.  It uses simulated communication
structured in such a way as to trigger bugs in FIFO, reliable communication
protocols.

\subsection{Armadillo: testing Ensemble security extensions}
This program tests \ensemble\ security features. It has several
command line options:
\begin{description}
\item[-n \#]  number of endpoints to create
\item[-t \#]  after what threshold to start the test
\item[-prog] which security to use? [policy,rekey,exchange,reg,prompt]
\item[-pa]   simulate partitions? 
\item[-net]  run everything in a single process or run throughout the  network
\item[-real\_pgp]  use PGP for authentication? otherwise, simulate it.
\item[-group]    set the group name
\end{description}

The ``exchange'' test checks that the Exchange layer functions
correctly. For example, running:
\begin{verbatim}
% armadillo -n 20 -prog exchange
\end{verbatim}
will create 20 endpoints with random intial keys. the endpoints should
merge into one group after a short while.

The ``rekey'' test creates a group and once its size is above the
threshold it start rekeying it. The test: {\tt Use: armadillo -n 7 -t
7 -prog rekey} will create a group of 7 members and once the group
reaches this size, will start to rekey it.
		
To see what happens when the group partitions use: {\tt armadillo -n 5
-t 3 -prog rekey -pa}. This will create a group of 5 members and start
partitioning and remerging the group. Everytime the membership in a
group component exceeds 3, the component leader will try rekeying it. 

The ``policy'' test checks that Ensemble respects application trust
policies. For example running:
\begin{verbatim}
% armadillo -n 7 -prog policy
\end{verbatim}
will create a static group of 7 processes, numbered 0 through 6, and
dynamically change the endpoints trust policies. Ensemble forms
subgroups according to the trust relationship. The policies are
designed to change in stages:
\begin{enumerate}
\item All endpoints trust each other.
\item 
All endpoints of the same (mod 2) trust each other. That is we
have to trust domains: $\{0,2,4,6\}$ and $\{1,3,5\}$.
\item 
All endpoints of the same (mod 3) trust each other. That is we
have three trust domains: $\{0,3,6\}$, $\{1,4\}$ $\{2,5\}$.
\end{enumerate}

The ``prompt'', and ``reg'' tests are auxillary tests not related
to security. 


\subsection{Life: Game of Life Demo}
This is a graphical version of the \emph{Game of Life} that was
originally ``invented'' by J.H.~Conway in 1969 (and was first
reported in \emph{Scientific American}, October 1970).  The Game of
Life is not actually a game: ``it is the study of phenomena which can
be observed in evolving configurations of populations.  One can think
of a population as a generation of living and non-living beings.  A
generation can be modeled by a rectangular grid of cells in which
each being occupies exactly one cell and each cell can be either on
or off.  If a being is alive, then the corresponding cell is on; if
the being is dead, then the cell is off.  From this point on we refer
to beings of a population and cells of the rectangular grid
interchangeably.''  In this implementation, each cell of the grid is
implemented by a separate endpoint and all communication is through
asynchronous Ensemble communication.  Anyway, this program requires
the CamlTk library and should be self-explanatory to run (note that
you only need to run one Life process: it creates multiple endpoints
within the process).

\emph{This application was written by Samuel Weber.}





\section{Configuration}

\subsection{Command-line Arguments and Environment Variables}
\label{subsect:config}
\ensemble\ applications typically support a variety of configuration
parameters.  Most of these can be configured through command-line options as
well as through setting environment variables.  In all cases, command-line
options override environment variables.  Look in \sourceappl{harg.ml} for the
authoritative list of the configuration parameters.  Some are listed below as
command-line options.  The corresponding environment variable for
\mlval{-group\_name} (for example) is \mlval{ENS\_GROUP\_NAME} (the name is
capitalized and the `-' is replace with \mlval{ENS\_}).
\begin{description}
\item
[\mlval{-modes} arg :] Set the default modes for an application to use.  The
modes are specified giving their names in all-uppercase, each separated by
single colons (`:') and no white-space.
\item
[\mlval{-udp\_port} port :] set the default UDP port for \ensemble\
applications.  For point to point UDP communication, this is the port
number \ensemble\ first tries to bind to for UDP communication (if it
is already in use Ensemble will then fail).  It can be set to any
value.  The default is to let the operating system choose a port to
use.
\item
[\mlval{-deering\_port} port :] This is the port that \ensemble\ will
use for Deering IP multicast communication (if enabled).  All
processes must use the same port number.
\item
[\mlval{-gossip\_port} port :] sets the port that the \mlval{gossip} servers
use.
\item
[\mlval{-gossip\_hosts} arg :] sets the hosts where applications using UDP
communication can look for \mlval{gossip} servers.  The value should be a
colon-separated list of hostnames.  The \mlval{gossip} server application will
only execute on these hosts.  Note that you only have to execute a gossip
server on one of these hosts: applications will try each of the hosts in turn
while looking for a gossip server.  However, multiple servers can be executed
for increased availability.
\item
[\mlval{-id} name :] used to give applications unique identifiers.
Usually this is set to be your user id.  Setting this variable
prevents \ensemble\ applications run by other users from interacting
with yours.  In case you do want them to interact, you should set
their variables to have the same value.  If using DEERING IP
multicast, their \mlval{-deering\_port} variable should also be set to
the same value.
\item
[\mlval{-groupd\_port} port :] sets the port that the membership \mlval{groupd}
servers use.
\item
[\mlval{-groupd\_hosts} arg :] sets the hosts that the membership
\mlval{groupd} servers use.  Format is the same as for \mlval{-gossip\_hosts}.
\item
[\mlval{-groupd} :] Use the membership service on the local host (see
section~\ref{section:groupd}.  This option may override others.
\item
[\mlval{-group\_name} name :] Set the name of the application's group.
\note{Currently, only the \mlval{ensemble} application supports this.}
\item
[\mlval{-key} key :] Set the key to use for a particular application.  All
messages sent and received by the application will be authenticated with this
key.
\item
[\mlval{-secure} :] Enable security enforcement.  This prevents any insecure
communication transports from being initialized.
\item
[\mlval{-add\_prop} property :] Adds a specific property to the
Ensemble protocol stack. See Section~\ref{sec:properties} for more
information on supported Ensemble properties. 
\item 
[\mlval{-remove\_prop} property :] The dual of add\_prop.
\item 
[\mlval{-sock\_buf} size :] The size of socket buffers to request
from the operating system.  The default size is $52428$ (the
traditional limit on Unix).  If you are using \ensemble\ in
high-performance setting and are experiencing message loss, this is a
parameter that should be increased.
\item
[\mlval{-refcount} :] Enables reference counting of message buffers.
The default is to rely on the garbage collector to detect when a
message is no longer needed.  Setting this will improve performance,
but may expose reference counting bugs in \ensemble.
\item 
[\mlval{-multiread} :] Enable multiple reads on sockets.  The default
is to receive and process one message from the operating system
at a time.  Setting this will cause all available messages to be read
from sockets before processing any of them, which may reduce message
loss due to buffer overflow in the operating system.
\item
[\mlval{-pollcount} count :] The number of times to query the
operating system before blocking.  \ensemble\ blocks after checking
(via the \mlval{select()} system call) the operating system for
messages and not finding any.  Setting this to $1$ will cause
\ensemble\ to block immediately when there are no more messages.
Setting this to a large number will cause \ensemble\ to busy-poll for
a longer time before blocking.
\item
[\mlval{-ranking} trans:] this is used to set the priority of
transport mechanisms. This is useful if you wish to use TCP as a
transport instead of UDP (the default). Using {\tt -ranking TCP} will
cause TCP to be ranked higher than any other transport. 
\end{description}

The following configuration parameter can only be set as an
environment variable.
\begin{description}
\item
[\mlval{ENS\_TRACE}]: enables module initialization tracing.  With this set (to
any value), modules print out their names as they initialize.  This is useful
if an exception  occurs during initialization because because it
enables you to narrow the problem down to one module.
%\item
%[\mlval{ENS\_LOG\_HOST} and \mlval{ENS\_LOG\_PORT}]: sets the port and host of the remote
%log server.  Applications can execute as the server via the \mlval{-log\_server}
%command-line option.  Running programs with the \mlval{-log} option enables the
%remote logging facility for other programs.  \note{This is currently not documented:
%you have to read the source code in \sourceappl{log} in order to use it now.}
\end{description}

\subsection{Transports}

\note{If you are only using regular UDP sockets for communication,
then you do not need to read this section.}

Perhaps the most confusing part of running \ensemble\ applications
comes from selecting communication transports.  Communication
transports are the bottom-most part of \ensemble\ and are used for
sending and receiving messages on a network.  There are several ways
this can be confusing and often \ensemble\ cannot detect that there is
a problem, so you do not get a warning.  For instance, if you
configure an application so that one process is using UDP sockets for
communication and another is using NETSIM, then the two processes will
stall waiting for other processes to communicate with them on their
selected medium.

A confusing aspect of transport is that an application typically uses
two different kinds of transports: a primary transport and a gossip
transport.  Normal application communication is all done over the
primary transport, which must support point-to-point communication and
may also support multicast communication.  Communication between
different partitions of a group of applications uses the gossip
transport which must support ``anonymous'' multicast communication.

Applications occasionally send ``gossip'' messages with their gossip
transport to the rest of the ``world'' in order to inform other
partitions about their presence.  When two partitions learn of each
other, they can then merge the partitions together.  After they have
merged together, they communicate over their primary transport.  This
gossip-and-merge mechanism is used when applications first start up:
an application creates its own singleton group and then merges with
any other already existing partitions through gossiping and merging.
Thus, if there is a problem with the gossip transport, you will tend
to have a bunch of applications in singleton groups that never merge.
If there is a problem with the primary transport, the merging will
occur, but then the various members will be unable to communicate.
This will cause them to repeatedly break into partitions (when they
decide that the other members must have failed) and then re-merge
again.

The various primary and gossip transports are presented in the
following table.  The ``P'' and ``G'' columns specify whether a
transport can be used for primary communication and/or gossip
communication.
\begin{center}
\begin{tabular}{|l|c|c|l|}					   \hline
transport	& P	& G	& description			\\ \hline \hline
\mlval{UDP}	& \chk	& \chk	& UDP (+ gossip server)		\\ \hline
\mlval{DEERING}	& \chk	& \chk	& UDP/IP multicast		\\ \hline
\mlval{TCP}	& \chk	& 	& TCP/IP			\\ \hline
\mlval{NETSIM}	& \chk	& \chk	& network simulator		\\ \hline
\end{tabular}
\end{center}

The \mlval{NETSIM} transports are used only in applications that are
simulating the behavior of a group inside a single process.  The
\mlval{rand} and \mlval{fifo} demos use this, for instance. All other
currently supported modes run over IP. UDP requires running the gossip
server.  With TCP as a transport, a fully connected mesh of TCP
connections is used to move messages between group members. This is
not very efficient for multicast messages, however, it can
sometimes be useful. 

There are several ways to change the communication transports that \ensemble\
uses.  These are listed below in order of highest ``precedence.''
\begin{enumerate}
\item
Command-line argument: with the \mlval{-modes} argument (see the
command-line argument documentation).
\item
Application setting: a particular application may differ from the
\ensemble\ defaults.
\item
Environment variable: \mlval{ENS\_MODES} variable (see the environment
variable documentation below).
\item
\ensemble\ defaults: \mlval{UDP}.
\end{enumerate}

2\subsection{Using Deering IP Multicast}

The method described above for running the mtalk demo is the best way
to first run mtalk because it uses \mlval{UDP} for both
transports. \mlval{UDP} does not use IP multicast communication,
which can be a source of problems because of variations in how it is
configured at different sites.  \note{IP multicast is only available
when using the Socket library on Unix.  It is not currently supported
by Ensemble on Windows NT.}  If your machines support Deering IP
multicast communication, it is preferable to use \mlval{DEERING}
transports because you will then not have to run the gossip server
with \ensemble\ applications.  You can try out the IP multicast
transport by using the command-line arguments.  (Note the problems
section at the end of this section, however, which describes some of
the problems you may have.)  This is done by executing the
applications with the \mlval{-modes} command-line arguments.  With IP
multicast you no longer need to have a gossip server running.  Run
the application on the hosts with these arguments (see below for a
description of the arguments):
\begin{verbatim}
% mtalk -modes DEERING
\end{verbatim}
To always use IP multicast by default, modify the ENS\_MODES
environment variable so that it includes DEERING.  Also set the
ENS\_DEERING\_PORT environment variable to an unused port number.
You will probably wish to add these to your standard shell
environment:
\begin{verbatim}
setenv ENS_DEERING_PORT 1234
setenv ENS_MODES DEERING:UDP
\end{verbatim}

\subsection{Notes and Problems}

See also the problems mentioned in the \ensemble\ reference manual.
\begin{description}
\item
[IP Multicast problems :] Some problems may occur with IP Multicast.
The time-to-live value for multicast messages may be too small in
some environments, preventing multicast messages from reaching all
members.  The TTL value can be adjusted by editing the file
\sourcecode{socket/multicasts.c}.
\end{description}

%*************************************************************%
%
%    Ensemble, 1.10
%    Copyright 2001 Cornell University, Hebrew University
%    All rights reserved.
%
%    See ensemble/doc/license.txt for further information.
%
%*************************************************************%
\section{\ensemble\ ML Application Interface}
\label{section:applintf}
\todo{add example handlers from mtalk}

We present a simple interface for building single-group applications.  This
interface is intended to make small applications easy to build, and to protect
users from complications in the internals of the system.

The interface is implemented as a set of callbacks the application provides to
\ensemble.  The application is notified through these callbacks (in a similar
fashion to callbacks with Motif widgets) of events that occur in the system,
such as message receipts and membership changes.

The interface for a member of a group is always in one of two states,
\emph{blocked} or \emph{unblocked}.  While unblocked, only the
\mlval{recv\_send}, \mlval{recv\_cast}, and \mlval{heartbeat} callbacks are
enabled.  This is the normal state of the system.  While blocked, another set
of callbacks can occur that notify the application of membership changes in the
group.  In addition there are callbacks that notify the application of
transitions between the blocked and unblocked states.

Messages are sent by returning from these callbacks lists of actions to
take.  An action is usually a message send: either a \mlval{Cast} (group
broadcast) or a \mlval{Send} (point-to-point message).  Thus, messages are
delivered by callbacks from \ensemble\ and further messages are sent by
returning values from these callbacks.

\subsection{Compilation}
Compiling ML applications is easy.  You can use \sourcedemo{Makefile} as a
skeleton for your own applications.

\subsection{Interface Definition and Initialization}
Below is the full ML interface type definition for the application
interface described here.  A group member is initialized by creating an
interface record which defines a set of callback handlers for the
application.  This is then passed to one of the \ensemble\ stack initialization
functions exported by \sourcecode{appl/appl.mli}.

\begin{codebox}
(* Some type aliases.
 *)
type rank	= int
type view 	= Endpt.id list
type mergers 	= Endpt.id list
type contact 	= Endpt.id
type origin 	= rank
type dests 	= rank array

type ('cast_msg, 'send_msg) action =
  | Cast of 'cast_msg
  | Send of dests * 'send_msg
  | Leave
  | XferDone
  | Protocol of Proto.id
  | Migrate of Addr.tl
  | Timeout of Time.t		        (* not supported *)
  | Dump
  | Block of bool			(* not for casual use *)
\end{codebox}
\begin{codebox}
(* APPL_INTF.full: The record interface for applications.  An
 * application must define all the following callbacks and
 * put them in a record.
 *)

type (
  'cast_msg,
  'send_msg,
  'merg_msg,
  'view_msg
) full = \{

  recv_cast             : origin -> 'cast_msg ->
    ('cast_msg,'send_msg) action array ;

  recv_send             : origin -> 'send_msg ->
    ('cast_msg,'send_msg) action array ;

  heartbeat_rate        : Time.t ;

  heartbeat             : Time.t ->
    ('cast_msg,'send_msg) action array ;

  block                 : unit ->
    ('cast_msg,'send_msg) action array ;

  block_recv_cast       : origin -> 'cast_msg -> unit ;
  block_recv_send       : origin -> 'send_msg -> unit ;
  block_view            : View.state -> (rank * 'merg_msg) list ;
  block_install_view    : View.state -> 'merg_msg list -> 'view_msg ;
  unblock_view          : View.state -> 'view_msg ->
    ('cast_msg,'send_msg) action array ;

  exit                  : unit -> unit
\}
\end{codebox}

\subsection{Actions}
Some callbacks allow a (possibly empty) array of actions to be 
returned.  There are 4 different kinds of actions:
\begin{description}
\item
[Cast(msg)] : Causes \mlval{msg} to be broadcast to the group.
\item
[Send(dests,msg)] : Causes \mlval{msg} to be sent to a subset of the group
specified in \mlval{dests}.  \mlval{dests} is an array of ranks, but most
protocol stacks only support sends with a single destination.  In order to send
to multiple destinations in a single action, some layer that supports subset
broadcasts must be in the stack.
\item
[Leave] : Causes the member to leave the group.  There should always
be at most one \mlval{Leave} action returned in an action array.
\item
[XferDone] : Signals that this member has completed its state transfer.  If a
state transfer layer is in the protocol stack, this will trigger a new
non-state transfer view after all members have taken an \mlval{XferDone}
action.
\item
[Protocol(protocol)] : Requests a protocol switch.  If the stack supports
protocol switches, a new view will be triggered.
\item
[Dump] : Causes some debugging output to be printed by the stack in use.
The output depends greatly on the protocol stack.
\end{description}

\subsection{Normal Operation: Unblocked}
Under normal operation only the following callbacks are called.
\mlval{recv\_cast} is called when a broadcast message has been received.  The
callback is made with the origin of the sender and the message.
\mlval{recv\_send} is called when a point-to-point message has been received.
The callback is made with the origin of the sender and the message.
\mlval{heartbeat} is regularly called by \ensemble\ when the application is
unblocked.  The expected rate of heartbeats is specified through the
\mlval{heartbeat\_rate} field of the interface record.  The return values for
all of these callbacks is an action array.
\begin{codebox}
  recv_cast             : origin -> 'cast_msg ->
    ('cast_msg,'send_msg) action array ;

  recv_send             : origin -> 'send_msg ->
    ('cast_msg,'send_msg) action array ;

  heartbeat_rate        : Time.t ;

  heartbeat             : Time.t ->
    ('cast_msg,'send_msg) action array ;
\end{codebox}

\subsection{Asynchronous operation}
The application can only send messages when handling a callback.  Under some
circumstances (such as when receiving input from another source), it is necessary to
send messages immediately rather than waiting for the next regularly scheduled
heartbeat to occur.  Call the function \mlval{Appl.async} with the group and endpoint
of the group.  This returns a function that can be called whenever an immediate
hearbeat is desired.
\note{This replaces the previous \mlval{heartbeat\_now} callback.}
\begin{codebox}
  let async = Appl.async (group,endpt) in
  async ()
\end{codebox}

\subsection{Blocking}
When a membership change is about to happen, the group blocks and only the
``block'' callbacks occur until the group is unblocked by the installation
of a new view.  A member is notified of the blocking with a block callback:
\begin{codebox}
  block                 : unit ->
    ('cast_msg,'send_msg) action array ;
\end{codebox}

\subsection{Receiving messages while blocked}
While a group is blocked, cast and send messages are delivered through the
following callbacks.  The arguments are the same as above, but callbacks cannot
initiate new actions.
\begin{codebox}
  block_recv_cast       : origin -> 'cast_msg -> unit ;
  block_recv_send       : origin -> 'send_msg -> unit ;
\end{codebox}

\subsection{New views}
When \ensemble\ installs a new view of the group, the application goes through a
simple protocol in order to transfer state between members.  This begins by
\ensemble\ calling the \mlval{block\_view} callback with the state of the new view
(see below).  This callback returns a list of pairs.  Each pair is a rank and a state
for that member of the group.  Each member can return a pair just for itself, or
members can specify the states of other members.  When non-empty, these lists are
sent back to the coordinator.  The coordinator collects them, waiting until it has a
state from every member.  If there is a failure, then the whole process may begin
again with a \mlval{block\_view}.

When the coodinator has all of the states, the \mlval{block\_install\_view} callback
is called, again with the view state and also with a list of states, one for each
member in the group.  The ranks are not included with the states, as the states are
ordered according to the view.  The callback then returns a common state that will be
sent to all the members.  Every member will then receive an \mlval{unblock\_view}
callback with the view state and the state returned by the coordinator.  At this
point, each member is unblocked and may begin sending messages again.
\begin{codebox}
  block_view            : View.state -> (rank * 'merg_msg) list ;
  block_install_view    : View.state -> 'merg_msg list -> 'view_msg ;
  unblock_view          : View.state -> 'view_msg -> 
    ('cast_msg,'send_msg) action array ;
\end{codebox}

\subsection{View state}

Several callbacks receive as an argument a pair of records with
information about the new view.  The information is split into two
parts, a \mlval{View.state} and a \mlval{View.local} record.  The
first contains information that is common to all the members in the
view, such as the \mlval{view} of the group.  The same record is
delivered to all members.  The second record contains information
local to the member that receives it.  These fields include the
\mlval{Endpt.id} of the member and its \mlval{rank} in the view.  It
also contains information that is derived from the \mlval{View.state}
record, such as \mlval{nmembers} with is merely the length of the
\mlval{view} field.

\begin{codebox}
(* VIEW.STATE: a record of information kept about views.
 * This value should be common to all members in a view.
 *)
type state = {
  (* Group information.
   *)
  version       : Version.id ;		(* version of Ensemble *)
  group		: Group.id ;		(* name of group *)
  proto_id	: Proto.id ;		(* id of protocol in use *)
  coord         : rank ;		(* initial coordinator *)
  ltime         : ltime ;		(* logical time of this view *)
  primary       : primary ;		(* primary partition? (only w/some protocols) *)
  groupd        : bool ;		(* using groupd server? *)
  xfer_view	: bool ;		(* is this an XFER view? *)
  key		: Security.key ;	(* keys in use *)
  prev_ids      : id list ;             (* identifiers for prev. views *)
  params        : Param.tl ;		(* parameters of protocols *)
  uptime        : Time.t ;		(* time this group started *)

  (* Per-member arrays.
   *)
  view 		: t ;			(* members in the view *)
  clients	: bool Arrayf.t ;	(* who are the clients in the group? *)
  address       : Addr.set Arrayf.t ;	(* addresses of members *)
  out_of_date   : ltime Arrayf.t	; (* who is out of date *)
  lwe           : Endpt.id Arrayf.t Arrayf.t ; (* for light-weight endpoints *)
  protos        : bool Arrayf.t  	(* who is using protos server? *)
}
\end{codebox}

\begin{codebox}
(* VIEW.LOCAL: information about a view that is particular to 
 * a member.
 *)
type local = {
  endpt	        : Endpt.id ;		(* endpoint id *)
  addr	        : Addr.set ;		(* my address *)
  rank 	        : rank ;		(* rank in the view *)  
  name		: string ;		(* my string name *)
  nmembers 	: nmembers ;		(* # members in view *)
  view_id 	: id ;			(* unique id of this view *)
  am_coord      : bool ;  		(* rank = vs.coord? *)
  falses        : bool Arrayf.t ;       (* all false: used to save space *)
  zeroes        : int Arrayf.t ;        (* all zero: used to save space *)
  loop          : rank Arrayf.t ;      	(* ranks in a loop, skipping me *)
  async         : (Group.id * Endpt.id) (* info for finding async *)
}  

(* LOCAL: create local record based view state and endpt.
 *)
val local : debug -> Endpt.id -> state -> local
\end{codebox}

Most of the fields are moderately self-explanatory.  If
\mlval{xfer\_view} is true, then this view is only for state transfer
and all members should take an \mlval{XferDone} action when the state
transfer is complete.  The view field is defined as \mlval{View.t},
which is:
\begin{codebox}
(* VIEW.T: an array of endpt id's.
 *)
type t = Endpt.id Arrayf.t
\end{codebox}


\subsection{Exit notice}
Called when the member has left the group (through a previous \mlval{Leave}
action).  This is the last callback the group member will receive.
\begin{codebox}
  exit                  : unit -> unit ;
\end{codebox}

%*************************************************************%
%
%    Ensemble, 1_42
%    Copyright 2003 Cornell University, Hebrew University
%           IBM Israel Science and Technology
%    All rights reserved.
%
%    See ensemble/doc/license.txt for further information.
%
%*************************************************************%
\subsection{Properties}
\label{sec:properties}

The \ensemble\ \mlval{Property} module is used to construct protocols based on
desired properties the application wants.  You can look at \sourceappl{property.mli}
for the various properties that are supported by \ensemble:
\begin{codebox}
type id =
  | Agree				(* agreed (safe) delivery *)
  | Gmp					(* group-membership properties *)
  | Sync				(* view synchronization *)
  | Total				(* totally ordered messages *)
  | Heal				(* partition healing *)
  | Switch				(* protocol switching *)
  | Auth				(* authentication *)
  | Causal				(* causally ordered broadcasts *)
  | Subcast				(* subcast pt2pt messages *)
  | Frag				(* fragmentation-reassembly *)
  | Debug				(* adds debugging layers *)
  | Scale				(* scalability *)
  | Xfer				(* state transfer *)
  | Cltsvr				(* client-server management *)
  | Suspect				(* failure detection *)
  | Flow				(* flow control *)
  | Migrate				(* process migration *)
  | Privacy				(* encryption of application data *)
  | Rekey				(* support for rekeying the group *)
  | OptRekey				(* optimized rekeying protocol *)
  | DiamRekey                           (* Diamond rekey algorithm *)
  | Primary				(* primary partition detection *)
  | Local				(* local delivery of messages *)
  | Slander				(* members share failure suspiciions *)
  | Asym			        (* overcome asymmetry *)

    (* The following are not normally used.
     *)
  | Drop				(* randomized message dropping *)
  | Pbcast				(* Hack: just use pbcast prot. *)
  | Zbcast                              (* Use Zbcast protocol. *)
  | Gcast                               (* Use gcast protocol. *)
  | Dbg                                 (* on-line modification of network topology *)
  | Dbgbatch                            (* batch mode network emulation *)
  | P_pt2ptwp                           (* Use experimental pt2pt flow-control protocol *)
\end{codebox}

Here is a short description of some of the properties:
\begin{itemize}
\item {Gmp:} Group Membership Properties.
\item {Sync:} Synchronizes messages on view changes to ensure view synchrony.
\item {Total:} Broadcast messages are totally ordered in the group.
\item {Heal:} Group partitions are healed.
\item {Switch:} Allows on-the-fly protocol switching.
\item {Auth:} Allows only authenticated and authorized members into
the group. Creates secure agreement in the group on a mutual group
key. This key is used to sign and verify, using keyed-MD5, all group
messages. This protects the group from outisde attack. 
\item {Rekey:} Allows rekeying the group.  
\item {Privacy:} Encrypts all user messages. 
\item {Causal:} Broadcasts are causally ordered.
\item {Subcast:} Point-to-point messages are sent using filtered broadcasts.
Guarantees FIFO ordering between broadcasts and point-to-point messages.
\item {Frag:} Message fragmentation.  Allows messages of any size to be sent.
\item {Debug:} Inserts a variety of ``assertion'' protocols that check that
other properties are being met.
\item {Scale:} Switches some protocols with more scalable versions.
\item {Xfer:} Causes the state transfer field (\mlval{xfer}) of view states to
be set.
\item {Cltsvr:} Causes the clients field of view states to be set according to
whether members are ``clients'' or ``servers''.
\item {Suspect:} Members watch other members for suspected failures.
\item {Zbcast:} A probabilistic multicast protocol, does not guaranty
virtual syncrhony. Has been used for experimental studies. See the
Cornell Spinglass system for more details.
\item {Gcast:} A protocol that simulates IP-multicast useing a binary
tree of pt-2-pt connections between group members.
\end{itemize}

The \mlval{Property.choose} function selects a protocol stack based on a list
of desired properties (you can examine the implementation to see exactly how
this is done):
\begin{codebox}
(* Create protocol with desired properties.
 *)
val choose : id list -> Proto.id
\end{codebox}

The default properties used for \ensemble\ applications is \mlval{Property.vsync}.
This is one of a variety of predefined protocol property lists defined in the
\mlval{Property} module:
\begin{codebox}
let vsync = [Gmp;Sync;Heal;Migrate;Switch;Frag;Suspect;Flow]
let total = vsync @ [Total]
let scale = vsync @ [Scale]
let fifo = [Frag]
\end{codebox}


In order to set the properties used by an application, you would use the
following code:
\begin{codebox}
  (* Choose default view state.
   *)
  let vs = Appl.default_info "my-appl" in

  (* Select desired properties.
   *)
  let properties = [ (* list of properties *) ] in

  (* Choose corresponding protocol stack.
   *)
  let proto_id = Property.choose properties in

  (* Set proto_id of the view state record.
   *)
  let vs = View.set vs [Vs_proto_id proto_id] in

  (* Configure the application
   *)
  Appl.config my_interface vs ;
\end{codebox}

As described in the reference manual, each of these protocols are derived by
combining a set of protocol layers together to get a full protocol stack with
application-level properties.  Anyway, here we describe the behavior of the
\mlval{vsync} protocol stack.
\begin{itemize}
\item
The first callback a protocol stack receives is an \mlval{unblock\_view} with a
singleton view.
\item
All members in the same partition of a group receive the same \mlval{View.state}
records (excepting the \mlval{rank} field, of course).
\item
\mlval{Send} messages are delivered reliably and in FIFO order.  It is an error
for a member to send a message to itself.
\item
\mlval{Cast} messages are delivered reliably and in FIFO order.  FIFO order for
\mlval{Cast} messages means that members receive the messages in the order they
were sent by the sender.  \mlval{Cast} messages are usually not delivered to
the sender (the primary exceptions are stacks with total-ordering layers in
them).
\item
There is no ordering relationship \emph{between} \mlval{Send} and \mlval{Cast}
messages.
\item
Messages are delivered in the same view they were sent in (the protocol stack
``blocks'' so that the protocols can flush all the current messages out of the system
before advancing to the next view).
\item
\mlval{Cast} messages are delivered atomically.  This means that either all
members (excepting the sender) or none will receive a \mlval{Cast} message.  If
the sender of a \mlval{Cast} message fails, other members who received the
message will retransmit it for the failed member.  When there is more than one
member in a group, a \mlval{Cast} message may be delivered to no members only
if the sender fails.
\item
All members that receive the same consecutive views (they get the same
\mlval{block\_view} upcalls) will have delivered the same set of \mlval{Cast}
messages between the upcalls (but not necessarily in the same order).  Thus views can
be considered as synchronization points where all members agree on what has been done
so far.
\end{itemize}

\subsection{Initializing \ensemble\ Applications}

This is a description of how simple applications are initialized with \ensemble.  The
source code presented here is extracted from the \mlval{mtalk} demo, which is
distributed with \ensemble.  The source can be found in \sourcedemo{mtalk.ml} which
compiles and links with the \ensemble\ library to form the \sourcedemo{mtalk}
executable.

An application consists of two parts, initialization and an interface.  The
initialization involves setting up \ensemble\ and the communication framework.
An interface consists of a set of callback handlers that manage application
events that \ensemble\ generates for messages and membership changes.  The
initialization code tends to be similar across applications, and the handlers
tend to contain most of the application-specific functionality.  We present a
sample set of initialization code, which can easily be adapted for other simple
applications.  We do not describe the callback handlers here; they are
described in section~\ref{section:applintf}.  For specific examples, see
\sourcedemo{mtalk.ml} and \sourcedemo{ring.ml}.

\begin{codebox}
let run () =
  (*
   * Parse command line arguments.
   *)
  Arge.parse [
    (*
     * Extra arguments can go here.
     *)
  ] (Arge.badarg name) "mtalk: multiperson talk program" ;

  (*
   * Get default transport and alarm info.
   *)
  let view_state = Appl.default_info "mtalk" in
  let alarm = Alarm.get () in
\end{codebox}
The initialization must do several things, all of which can be contained in a
single function, as shown here with the function \mlval{run}.  First parse the
command-line arguments as is done above.  In addition to arguments provided by
the applicatoin, this parses the standard \ensemble\ arguments.  Then,
\mlval{default\_info} is called.  This initializes a \mlval{View.state} record
(which contains all the information other modules need to initialize your
application).

\begin{codebox}    
  (*
   * Choose a string name for this member.  Usually
   * this is "userlogin@host".
   *)
  let name =
    try
      let host = gethostname () in

      (* Get a prettier name if possible.
       *)
      let host = string_of_inet (inet_of_string host) in
      sprintf "%s@%s" (getlogin ()) host
    with _ -> view_state.name
  in

  (*
   * Initialize the application interface.
   *)
  let interface = intf view_state name alarm in
\end{codebox}    
Next we initialize the interface record that contains the
application's handlers and which does the actual work of the
application.  How the interface is initialized is application
dependent.  For example, \mlval{interface} will usually require
several arguments.  In the \mlval{mtalk} application, the interface
takes the endpoint identifier of the application and a string name to
use for this member of the talk group.  Other applications will use
different arguments.

\begin{codebox}    
  (*
   * Initialize the protocol stack, using the interface and
   * view state chosen above.  
   *)
  Appl.config interface view_state ;
\end{codebox}    
The code above initializes the protocol stack.  In this case we use
the \mlval{vsync} protocol properties, which provide FIFO,
virtually-synchronous communication and an automatic merging facility
for healing partitions.  There are several different sets of
properties by the \sourceappl{property.mli} module, each of which
provides different properties or performance characteristics (for
more information about properties, see section~\ref{sec:properties}).

\begin{codebox}    
  (*
   * Enter a main loop
   *)
  Appl.main_loop ()
  (* end of run function *)


(* Run the application, with exception handlers to catch any
 * problems that might occur.
 *)
let _ = Appl.exec ["mtalk"] run
\end{codebox}    
The initialization is complete and we enter a main loop.  The main
loop never returns.  The final code calls the \mlval{run} function
with some standard exception handlers to catch any exceptions that
should not, but may, occur.

This is all that is required for initializing simple, single-group Ensemble
applications.  
%The main part of the work required for an application is in
%building the handlers for sending and receiving messages, described in the
%following section.

\section{Using PGP}

\ensemble\ supports the use of PGP for authenticating members of
groups.  This work is still underway and the security currently
provided is not bullet proof.  Most of the \ensemble\ demo
applications support the use of PGP, including \mlval{mtalk},
\mlval{wbml}, and \mlval{ensemble}.

These are the instructions for using PGP.  Note that PGP is currently
only supported for Unix platforms.

\begin{itemize}
\item
The \mlval{pgp} binary must be in your path.  \ensemble\ executes PGP
as a subprocess for authenticating remote members.  If you do not yet
have a PGP keyring, read the PGP documentation on how to set all this
up.
\item
You must set the \mlval{PGPPASS} environment variable to contain your
secret key pass phrase.  See the PGP documentation for more
information.
\item
\mlval{-pgp user} : command line argument.  This tells \ensemble\ what
this user's name is for PGP other processes will use this name to
select the public key to use for authenticating you.
\item
\mlval{-key sharedkey}: command line argument.  This sets the shared
key conversation key that \ensemble\ will use initially.  It should
be at least $8$ characters long (for DES).
\item
\mlval{-add\_prop Auth}: command line argument.  This adds the
\mlval{Auth} property to the default Ensemble properties.  This then
causes the \mlval{EXCHANGE} protocol to be used in the protocol stack
for exchanging shared keys.
\end{itemize}

Now when you run an application only members that start with the same
shared key or who can authenticate each other through PGP will
merge into the same group.

If you run into problems, you can access PGP's debugging output
through the additional command-line arguments, \mlval{-trace PGP}.

%*************************************************************%
%
%    Ensemble, 1_42
%    Copyright 2003 Cornell University, Hebrew University
%           IBM Israel Science and Technology
%    All rights reserved.
%
%    See ensemble/doc/license.txt for further information.
%
%*************************************************************%
\newcommand{\sourcefile}[1]    {{\tt {#1}}}


\section{Heterogeneous Transports}

{\bf  Complete this section}

\ensemble\ provides a flexible infrastructure for sending communication
across a variety of different communication transports.  Not only can
different groups use different communication transports, but a single group
can support communication on multiple transports at the same time.

The design of the transport module is split into three parts: 
\begin{description}
\item[The socket module:] ~\newline
   Low-level system calls: {\tt send, sendto, recv} etc.,
  implemented in a system-independent fashion. The {\tt socket}
  directory contains the code. {\tt socket/u} is a simple-minded
  implementation that uses the Ocaml Unix library directly. A more
  efficient version is located in {\tt socket/s}, where native OS
  io-vector send/recv facilities are used. 

\item[Transports:] ~\newline
  Self registering {\it transports}:  Deering, UDP, TCP, NETSIM. These
  use the low-level socket module calls to provide an abstract {\it transport}.
	
\item[Routers:] ~\newline
  Uses a communication transport to
  build Ensemble specific send/recv capabilities. Length field, 
  group id, and endpoint rank are added to each outgoing
  message. Basic parsing is performed on received messages and sender
  rank, group, and message length are extracted. 
 
  There are several {\it routers} in the {\tt route}
  subdirectory. \sourcefile{signed.ml} adds a 16-byte MD5 checksum to
  each outgoing message. An agreed group-secret is used to key MD5,
  providing group authentication. Incoming messages are stripped of
  this header, and verified. \sourcefile{unsigned.ml} is the vanilla router.

\end{description}

The user can choose to use either one of the socket module
implementations. The socket module interface is defined in
\sourcefile{socket/socket.mli}. The unoptimized socket implementation
(usocket) represents message data as a Caml string and benefits from
native garbage collection. Its disadvantage is reduced
performance. The optimized socket library (ssocket) uses native C
io-vectors, and native operating-system scatter-gather message
send/receive facilities. This provides much better performance, and
zero-copy integration with C applications. The disadvantage is more
difficult integration with native ML values. 

The transports are defined the \sourcefile{trans} subdirectory. 
UDP in \sourcefile{trans/udp.ml}, TCP in \sourcefile{trans/tcp.ml},
DEERING in \sourcefile{trans/ipmc}, and NETSIM in
\sourcefile{trans/netsim}.

The \sourcefile{route} subdirectory contains three routes: signed,
unsigned, and bypass.

\subsection{Code walk-through}
To provide better understanding of the design this section walks
through a configuration of the unsigned router, UDP transport, 
and optimized socket library. We shall start from the bottom
and work our way up. 

In file \sourcefile{socket/s/sendrecv.c}, there is code for sending an
array of C io-vectors and part of an ML string. The function takes
five arguments:
\begin{itemize}
\item info\_v : a structure describing a list of remote targets and a
socket through which to send messages.
\item prefix\_v : an ML string that prefixes the data
\item ofs\_v, len\_v: the offset and length of the prefix to send
\item iova\_a : an array of io-vectors wrapped in an ML representation
\end{itemize}

\begin{codebox}
value skt_sendtosv(
	value info_v,
	value prefix_v,
	value ofs_v,
	value len_v,
	value iova_v
) \{
  int naddr=0, i, ret=0;
  ocaml_skt_t sock=0 ;
  skt_sendto_info_t *info ;

  info = skt_Sendto_info_val(info_v);
  send_msghdr.msg_iovlen = prefixed_gather(prefix_v, ofs_v, len_v, iova_v); 

  send_msghdr.msg_namelen = info->addrlen ;
  sock = info->sock ;
  naddr = info->naddr ;

  for (i=0;i<naddr;i++) \{
    /* Send the message.  Assume we don't block or get interrupted.  
     */
    send_msghdr.msg_name = (char*) &info->sa[i] ;
    ret = sendmsg(sock, &send_msghdr, 0) ;
  \}

  return Val_unit;
\}
\end{codebox}


{\tt skt\_sendtosv} is hidden inside the socket library, and can
safely be used using {\tt Socket.sendtosv}. The {\tt sendto\_info}
structure can be created from an array of target socket addresses, and
a sending socket.

\begin{codebox}
type sendto_info
val sendto_info : socket -> Unix.sockaddr array -> sendto_info

val sendtosv : sendto_info -> buf -> ofs -> len -> Basic_iov.t array -> unit
\end{codebox}


The Hsys module makes access to sendtovs safer, and changes its type:
\begin{codebox}
  val sendtosv : sendto_info -> Buf.t -> ofs -> len -> Iovecl.t -> unit

(* Implementation *)
  Iovec.Priv.sendtosv info 
    (Buf.string_of buf) (Buf.int_of_len ofs) (Buf.int_of_len len) 
    (Iovecl.to_iovec_array iovl) 
\end{codebox}
Core Ensemble code, including the routers, does not use Socket calls
directly. Rather, it uses the Hsys module which wraps all calls with a
more type safe interface. Separate types are used for length, offset,
io-vector, and buffer.

The UDP implementation at \sourcefile{trans/udp.ml} uses Hsys in the 
transmit function called {\tt x}.

\begin{codebox}
  let x hdr ofs len iovl = 
    Hsys.sendtosv dests hdr ofs len iovl;
    Iovecl.free iovl
\end{codebox}

The io-vector array is freed after the message is transmitted. The
reference count for an iovec-array is decremented on two occasions:
(1) it is sent on the network (2) it is handed to an application, and
the callback has completed. The iovec refcount is initially set to one
when the application sends it, and it is henceforth incremented
whenever a copy of it created. Ultimately, the refcount will be
decremented when the stability detection protocol determines that all
group members received the message.

\subsection{Design of the routers}
Many endpoints belonging to different groups can coexist in a single
Ensemble process. Each endpoint is identified by its connection
identifier. The internal representation of this id is given in module
{\tt Conn}:

\begin{codebox}
type id = \{
  version       : Version.id ;
  group 	: Group.id ;
  stack 	: Stack_id.t ;
  proto 	: Proto.id option ;
  view_id 	: View.id option ;
  sndr_mbr 	: sndr_mbr ;
  dest_mbr 	: dest_mbr ;
  dest_endpt 	: dest_endpt option
\}
\end{codebox}

The id is mapped into a string using the {\tt Route.pack\_of\_conn}
function. Ensemble uses MD5 for this mapping. The probability of a
collision, i.e., for two different endpoints to map onto a single
string, is $2^{-64}$ which is sufficient for our purposes. 

\begin{codebox}
val pack_of_conn : Conn.id -> Buf.t
\end{codebox}

The purpose of the route module is to create a single interface to
these various endpoints. The main type exported is {\tt
handlers}. This is essentially a large array holding the set of
connection identifiers and the delivery function for each of
them. When a message is received by the bottom-most part of the
system, it is parsed by the socket code into an ML header that is a
string, and the rest of the message which is received into a
C-iovector. This information is later fed into the {\tt deliver}
function.

\begin{codebox}
val deliver : handlers -> Buf.t -> Buf.ofs -> Buf.len -> Iovecl.t -> unit
\end{codebox}

Deliver takes the current set of handlers, and a message, figures out
which endpoints need to receive this message and calls the appropriate
handlers. 

A transmission function is abstracted as a type {\tt xmitf}:
\begin{codebox}
(* transmit an Ensemble packet, this includes the ML part, and a
 * user-land iovecl.
 *)
type xmitf = Buf.t -> Buf.ofs -> Buf.len -> Iovecl.t -> unit
\end{codebox}

The Router module has an API allowing the creation of send/recv
functions for connection-ids. It also allows installing and deleting
such functions. The unsigned router is a simple example of
using this functionality to create the basic, insecure,
router. It defines function {\tt f}: 
\begin{codebox}
val f : unit -> 
  (Trans.rank -> Obj.t option -> Trans.seqno -> Iovecl.t -> unit) Route.t
\end{codebox}

This router will allow users to send (1) sender rank (2) ML object (3)
sequence number and (4) a user iovector array. The body of the code
calls {\tt Route.create} where it mainly needs to define how it plans
on handling {\tt blast} and {\tt merge}. Blast is how to send
messages, merge is how to receive a message on behalf of several
connection ids. 



\newpage
\part{The Client}
The client library (or simply the ``client'') implements a message-passing
protocol between server and user. The protocol used is described in
the reference manual. The client-library has no internal threads. No
message-memory is allocated by the client, all messages are allocated
and freed by the user. This gives the user complete control on its
memory foot-print. The client is thread-safe, several threads can
send/recv messages concurrently. Blocking socket operations are used
to simplify client semantics.

In order to use the client-library the user application must first
connect to the server. It can then create group members and perform
a subset of Ensemble actions: Leave, Cast, Send, Send1, Suspect.
There are other Ensemble operations that we decided not to support
since they add more complexity than value. 

The application must poll Ensemble periodically to see if there are any
pending messages, and receive them. In the past, it was possible
for the application not to receive messages while continuing to create
new actions. This is now not possible. The application will be blocked
at some point before flooding the server. 

\section{Native Java Application Interface (CEJAVA)}

The CEJAVA interface is built upon the CE interface, using the Java
Native Interface (JNI). This allows a user to tap the power of the
Ensemble messaging system from Java. Performance of this interface is
similar to the native ML and C interfaces. 

Not surprisingly, the API is similar to the CE API. Here, we walk
through an overview of the interface, and then point out the major
differences with respect to CE. Method and constructor documentation
can be found either in the code itself, or through the javadoc
generated HTML files.

\subsection{Overview}
The basic concept is that of a {\it Group}. A group is constructed by: 
\begin{codebox}
    public Group(Callbacks cb);
\end{codebox}

A group has to define the following set of callbacks: 

\begin{codebox}
public class Callbacks \{
    public abstract void install(View view);
    public abstract void exit();
    public abstract void recv_cast(int origin, byte[] msg);
    public abstract void recv_send(int origin, byte[] msg);
    public abstract void flow_block(int rank, boolean onoff);
    public abstract void block();
    public abstract void heartbeat(double time);
\}
\end{codebox}

The callbacks define behavior when events, such as message receipt or
view changes, occur. 

Within the context of a group the user can perform eleven actions: 

\begin{codebox}
    public void join(JoinOps jops);
    public void leave();
    public void cast(byte[] msg);
    public void send(int[] dests, byte[] msg) ;
    public void send1(int dest, byte[] msg);
    public void prompt();
    public void suspect(int[] suspects);
    public void xferDone();
    public void rekey();
    public void changeProtocol(String protocol_name);
    public void changeProperties(String properties);
\end{codebox}

Before starting to use the package, the initialization function must
be called: 

\begin{codebox}
    static public void init(String args []);
\end{codebox}

Any command line argument to CE and Ensemble should be passed here. 

To run a Java program that uses CEJAVA, the classpath must point to
the {\tt ensemble.jar} java-archive file, and the library path should
point to the matching native library {\tt libcejava.so}. For example,
to run java program {\tt prog.java}, assuming the jar file is in the
local directory, and that the native library is under {\tt lib} use:
\begin{codebox}
    java -Djava.class.path=;:ensemble.jar -Djava.library.path=lib prog
\end{codebox}
A simpler option is to set the {\tt CLASSPATH} environment variable to
include {\tt ensemble.jar} and set the dynamic library path to include
{\tt libcejava.so}. On Unix systems this means setting then
{\tt LD\_LIBRARY\_PATH} and on win32 setting the {\tt PATH}.

\subsection{Notes}
There are two major difference with respect to CE: zero-copy, and
synchronization. 

CEJAVA is not ``zero-copy'', the receive callbacks copy message data
from C to Java, and the send actions copy data from Java to C. This
costs extra copying but allows the Java application to do whatever it
likes with message data, without being bound by underlying reference
counting and memory management. 

Synchronization is very different between C and Java because the Java
has language support for locking. A group can be in one of six phases: 
\begin{codebox}
    public static final int PRE     = 0;
    public static final int JOINING = 1;
    public static final int NORMAL  = 2;
    public static final int BLOCKED = 3;
    public static final int LEAVING = 4;
    public static final int LEFT    = 5;
\end{codebox}

\begin{description}
\item[PRE:] preliminary phase, the group has not been fully
  constructed yet. 
\item[JOINING:] this endpoint is joining the group. 
\item[NORMAL:] the group is in stable state. 
\item[BLOCKED:] group is blocked prior to a pending view change. 
\item[LEAVING:] this member is leaving the group. 
\item[LEFT:] member has left the group. 
\end{description}

The only phase in which actions are allowed is the NORMAL state.
Multiple threads can perform actions on the same group, furthermore,
the Ensemble main-loop executing in a separate thread invokes group
callbacks when events arrive from the network. The group object is a
synchronization point for these threads of execution. The
implementation must ensure that group-state does not change during an
action. To this end, whenever an action is performed on group $g$: (1) the
group object is locked (2) status is checked (3) if it is NORMAL, the
action is performed. For example, the code for send looks like this:
\begin{codebox}
    public void send(int[] dests, byte[] msg) \{
        synchronized(this) \{
	    check_normal();
	    natSend(nat_env, dests, msg);
        \}
    \}
\end{codebox}

An application can have critical sections in which it must ensure
group state is NORMAL. It can also lock group state, and ensure it
does not change using a similar technique. A {\tt getStatus}
call is provided that returns the group state for group $g$. An example
of coding a critical section is:
\begin{codebox}
    synchronized(group) \{
        int stat = group.getStatus ();
        if (stat == NORMAL) \{
             /* Perform critical code here */
        \}
    \}
\end{codebox}

To maximize performance, the send/recv callbacks can also invoke
Ensemble actions. The callbacks enjoy the best latency since they do
not incur a thread-switch. 


%%*************************************************************%
%
%    Ensemble, 1_42
%    Copyright 2003 Cornell University, Hebrew University
%           IBM Israel Science and Technology
%    All rights reserved.
%
%    See ensemble/doc/license.txt for further information.
%
%*************************************************************%
\section{Csharp Application Interface}

The C-sharp API allows applications written in C\# to use the Ensemble
messaging system. Currently, only an outboard interface is
supported. This means that in order to use Ensemble a CE-outboard
server must be running on the local machine. A C\# program can then
connect to this server through the client library provided. 

An attempt was made to improve the original CE/HOT APIs and therefore
the C\# interface has a very different flavour. It is based on an
event-queue abstraction rather than callbacks. 

In order to use the client-library the user application must first
connect to the server. It can then create group members and perform
Ensemble {\it actions}: Leave, Cast, Send, Send1, Prompt, Suspect,
XferDone, Rekey, ChangeProtocol, and ChangeProperties.

The application must poll Ensemble periodically to see if there are any
pending messages, and receive them. Currently, it is possible
for the application not to receive messages while continuing to create
new actions. This is considered {\it non-friendly} behaviour and is something we
intend to prevent in the future. 

%%%%%%%%%%%%%%%%%%%%%%%%%%%%%%%%%%%%%%%%%%%%%%%%%%%%%%%%%%%%%%%%%%%%%%%%%
% The API abstractions. 

The API is constructed from a namespace named Ensemble and several
public classes the major of which are: {\tt View, JoinOps, Message,
  Connection}, and {\tt Member}. 

\begin{description}
\item[View:] The {\tt View} class describes a
group membership view. 

\item[JoinOps:] The {\tt JoinOps} class contains a
specification for a new member for Ensemble to create. 

\item[Message:] The {\tt Message} class describes a message received
  from Ensemble. A Message can be: a new View, a multicast message, a
  point-to-point message, a heartbeat, a block notification, or an
  exit notification. 

\item[Connection:] The {\tt Connection} class implements the actual
  socket communication between the client and the server. It has three
  public methods the application has to use.

  \begin{codebox}
   public class Connection \{
       public bool Poll();	  
       public Message Recv();
       public void Connect ();
  \}
  \end{codebox}

  The application can open several Ensemble connections, however, a
  single connection should suffice. No action can be taken on a
  connection that has not connected to the server through the {\tt
  Connect} call. Once connected the application can receive messages
  through the {\tt Recv} method. {\tt Recv} is a blocking call, in
  order to check first that there are pending messages the
  non-blocking {\tt Poll} method should be used. 
  

\item[Member:] The {\tt Member} class embodies an Ensemble group
  member. A member can join a single group, no more. A Member can be in several states: 
  \begin{description}
    \item[Pre:] The initial status in which all members are
      created. The class constructor sets this as the default.
    \item[Joining:] Joining a group is an asynchronous operation. A
      member is in the {\tt Joining} state from the time it attempts to join,
      until when it receives a View message with the initial group
      membership. 
    \item[Normal:] Normal operation state. The member is a regular
      resident in the group. It can send/mcast messages and
      perform all other Ensemble operations. 
    \item[Blocked:] The member is currently blocked, and temporarily cannot 
      perfrom any action. This state is achieved by sending a {\tt BlockOk}
      in response to a {\tt Block} request. 
    \item[Leaving:] The member has requested to leave the group. {\tt Leave}
      is an asynchronous operation, the {\tt Leaving} state captures
      the time between the {\tt Leave} request and the final leaving
      of the group.
    \item[Left:] The member has left the group and are in an invalid state
  \end{description}

  \begin{codebox}
  public class Member \{
      public enum Status \{
          Pre,        // the initial status
          Joining,    // we joining the group
          Normal,     // Normal operation state, can send/mcast messages
          Blocked,    // we are blocked
          Leaving,    // we are leaving
          Left        // We have left the group and are in an invalid state
      \};

      public View current\_view ;		       // The current view
      public Status current\_status = Status.Pre; // Our current status
  \end{codebox}

  The member state can be learned by examining the {\tt
  current\_status} field. The current view the member is part of is in 
  the {\tt current\_view} field. 

  Prior to any action, the member has to join a group.
  \begin{codebox}
  // Join a group with the specified options. 
  public void Join(JoinOps ops);
  \end{codebox}

  The set of operations allowed on a member in Normal state is:
  \begin{description}
    \item[Leave:] Leave a group. This should be the last call made to the member
    It is possible for messages to be delivered to this member after the call
    returns. However, it is illegal to initiate new actions on this member.
    \begin{codebox}
      public void Leave();
    \end{codebox}

    \item[Cast:]
      Send a multicast message to the group.
      \begin{codebox}
	public void Cast(byte[] data);
      \end{codebox}
    
    \item[Send:]
      Send a point-to-point message to a list of members.
      \begin{codebox}
	public void Send(int[] dests, byte[] data);
      \end{codebox}
    
    \item[Send1:]
      Send a point-to-point message to the specified group member.
      \begin{codebox}
	public void Send1(int dest, byte[] data);
      \end{codebox}
      
    \item[Suspect:]
      Report group members as failure-suspected.
      \begin{codebox}
	public void Suspect(int[] suspects);
      \end{codebox}
      
    \item[XferDone:]
      Inform Ensemble that the state-transfer is complete. 
      \begin{codebox}
	public void XferDone();
      \end{codebox}
      
    \item[ChangeProtocol:]
      Request a protocol change.
      \begin{codebox}
	public void ChangeProtocol(string protocol\_name);
      \end{codebox}
      
    \item[ChangeProperties:]
      Request a protocol change specifying properties.
      \begin{codebox}
	public void ChangeProperties(string properties);
      \end{codebox}
      
    \item[Prompt:]
      Request a new view to be installed.
      \begin{codebox}
	public void Prompt();
      \end{codebox}
      
    \item[Rekey:]
      Request a Rekey operation.
      \begin{codebox}
	public void Rekey();
      \end{codebox}
      
      \item[BlockOk:]
	Send a BlockOk
	\begin{codebox}
	  public void BlockOk();
	\end{codebox}
  \end{description}
\end{description}


% The client state-machine. 
\subsection{The client state-machine}

Each group member moved inside a state-machine that has a very clear set
of rules. Initially, it is in the {\tt Pre} state. After asking to
join a group, it is in the {\tt Joining} state. When the first view
arrives it is in the {\tt Normal} state. In the {\tt Normal} state and
prior to a view-change Ensemble will send a {\tt Block} message to the
member. The member has to reply with a {\tt BlockOk} action. After the {\tt BlockOk} the
member is in the {\tt Blocked} state until the next view. When the
next view is received is moves back to the {\tt Normal} state. Upon a
{\tt Leave} request the member moves to the {\tt Leaving} state which
turns into {\tt Left} after the {\tt Exit} message arrives. 

% FIXME: Insert a figure here. 

\subsection{Locking}
All {\tt Connection} and {\tt Member} method calls are
thread-safe. However, accessing the public Member fields such as the
{\tt current\_view} and {\tt current\_status} should be done while
holding the connection lock. Most Ensemble actions can be performed
only on a group-member that is in the {\tt Normal} state. For example, the
application may need to lock the member to ensure it's state does not
change while multicasting a message. To do this, the connection
instance must be locked using the {\tt lock} C\# idiom.

For example:
\begin{codebox}
  lock (conn) 
  \{
      if (memb.current_status == Member.Status.Normal)
          memb.Cast("hello world");
       else
           Console.WriteLine("Blocked currently, please try again later");
  \}
\end{codebox}

Replying to {\tt Block} message with a {\tt BlockOk} and moving to the
{\tt Blocked} state can be done asynchronously. This gives the
application a chance to send any pending messages 
prior to moving to the {\tt Blocked} state. Depending on the
application, this may allow the programmer to avoid locking. 



\subsection{Current limitations}
Currently, the maximal message size is limited to 32K. We intend to
review this in the future with respect to application needs.











%*************************************************************%
%
%    Ensemble, 1.10
%    Copyright 2001 Cornell University, Hebrew University
%    All rights reserved.
%
%    See ensemble/doc/license.txt for further information.
%
%*************************************************************%
\section{C \ensemble\ Application Interface}

The C application interface is very similar in design to the ML interface.  It is
still in the preliminary stage.  What does that mean?  That basically everything is
there, but some the design details have not been totally worked out yet.  And also
that the documentation is not complete.  If you wish to write an application in C, we
recommend you first read section~\ref{section:applintf} to understand how the
interface works.  Then look at the files in the \sourcecappl{} directory, in
particular \sourcecappl{mtalk.c}, to see how to write and compile a C application
with \ensemble.

\note{The C application interface is now in the \sourcehot{} directory and is under
development.}


\section*{Acknowledgments}
Thanks to Greg Sharp for comments on previous versions of this document.
\end{document}
