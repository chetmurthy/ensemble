%*************************************************************%
%
%    Ensemble, 2_00
%    Copyright 2004 Cornell University, Hebrew University
%           IBM Israel Science and Technology
%    All rights reserved.
%
%    See ensemble/doc/license.txt for further information.
%
%*************************************************************%
\section{C Application Interface}
The C-language API is similar in nature to the Java/C-sharp APIs. This
section assumes the reader is already familiar with that material. 

The main addition for C is the careful description of who allocates
which memory. 

In order for an application to start using Ensemble it needs to
initialize a connection structure:
\begin{codebox}
ens_conn_t *ens_Init(void);
\end{codebox}

The {\tt ens\_Init} call allocates and initializes an {\tt ens\_conn\_t}
structure that encapsulates a local socket connection to the Ensemble
Server. All operations require the connection structure. Only one
connection is needed for an application. 

The application needs to poll the connection periodically to see if it
has pending messages. 

\begin{codebox}
typedef enum ens_rc_t {
    ENS_OK = 0,        
    ENS_ERROR = 1
} ens_rc_t;

ens_rc_t ens_Poll(ens_conn_t *conn, int milliseconds, /*OUT*/ int *data_available);
\end{codebox}

{\tt ens\_Poll} returns an {\tt ens\_rc\_t} return type which conveys
whether the operations was successful or not. Poll takes a connection,
and a number of milliseconds to wait for input on the socket. It is a
blocking call, as is the rest of the API. If data is available the out
argument: {\tt data\_available} will be set to 1; if no data is
available it will be set to 0. 

In order to Join a group the {\tt ens\_Join} call should be used. 
\begin{codebox}
ens_rc_t ens_Join(
    ens_conn_t *conn,
    ens_member_t *memb,
    ens_jops_t *ops,
    void *user_ctx
    ) ;
\end{codebox}

Join takes a connection, a member structure, a set of join-options and
a opaque pointer for the user's use. It initializes the member
structure, attaches the user-context to it, and sends a Join request
to Ensemble. The set of allowed join-options is: 

\begin{codebox}
typedef struct ens_jops_t {
    char group_name[ENS_GROUP_NAME_MAX_SIZE] ; /* The group name */
    char properties[ENS_PROPERTIES_MAX_SIZE] ; /* The set of properties */
    char params[ENS_PARAMS_MAX_SIZE] ;         /* The set of parameters */
    char princ[ENS_PRINCIPAL_MAX_SIZE] ;       /* My principal name (security) */
    int secure ;                               /* Do we want a secure stack (encryption + authentication? */
} ens_jops_t ;
\end{codebox}

The user can choose the group-name, set of properties expected from
the group, and a set of additional configuration parameters. The
default values for properties is {\tt ENS\_DEFAULT\_PROPERTIES}, the
additional set of parameters should be empty for normal use. There are
two parameters used for security: the principal name, and the secure
flag. If one is not intersted in security, simply set the secure flag
to zero. If the flag is set to one, then the principal name in PGP 
should be specified. The principal names for users are used for
authentication purposes; only authenticated users are allowed into a
secure group. 

After joining a group there are several operations that are allowed on
it: Leave, Send1, Send, Cast, Suspect, and BlockOk. All these calls
are (a) blocking, they return only after the whole data has been
written to the socket (b) thread-safe: they may be used from any
application thread.

To leave a group use the {\tt ens\_Leave} call. 
\begin{codebox}
ens_rc_t ens_Leave(
    ens_member_t *memb
    ) ;
\end{codebox}
After the call the member structure becomes invalid and cannot be used
for any other operations. 

Point-to-point and multicast messages can be send with three different calls: 
\begin{codebox}
ens_rc_t ens_Cast(
    ens_member_t *memb,
    int len, 
    char *buf
    ) ;

ens_rc_t ens_Send(
    ens_member_t *memb,
    int num_dests,
    int *dests,
    int len, 
    char* buf
    ) ;

ens_rc_t ens_Send1(
    ens_member_t *memb,
    int dest,
    int len, 
    char* buf
    ) ;
\end{codebox}
For the above three calls The data is not freed nor allocated by the
Ensemble client. The user needs to manager memory that is
sent. Maximal message size is 32K. The maximal number of destinations
is 10.

Report specified group members as failure-suspected. The maximal
number of suspects is 10. 
\begin{codebox}
ens_rc_t ens_Suspect(
    ens_member_t *memb,
    int num,
    int *suspects
     );
\end{codebox}

Tell the system we will no longer send messages in this view. Should
be sent as a reponse to a Block message.
\begin{codebox}
ens_rc_t ens_BlockOk(
    ens_member_t *memb
    ) ;
\end{codebox}

Messages can be received by the {\tt ens\_RecvMetaData} call together
with {\tt ens\_RecvView} and {\tt ens\_RecvMsg}. 

When there is data on the connection the {\tt ens\_RecvMetaData} tells
which type of message has arrived and how much memory needs to be
allocated for receiving it. 

\begin{codebox}
typedef enum ens_up_msg_t {
    VIEW = 1,      /* A new view has arrived from the server. */
    CAST = 2,      /* A multicast message */
    SEND = 3,      /* A point-to-point message */
    BLOCK = 4,     /* A block requeset, prior to the installation of a new view */
    EXIT = 5       /* A final notification that the member is no longer valid */
} ens_up_msg_t;

typedef struct ens_msg_t {
    ens_member_t *memb;        /* endpoint this message blongs to */
    ens_up_msg_t mtype ;       /* message type */
    union {
	struct {               /* The variant for VIEW: */
	    int     nmembers;  /* the number of members in a view */    
	} view;
	struct {               /* The variant for a point-to-point message */
	    int     msg_size;  /* length of a bulk-data */
	} send;
	struct {               /* The variant for multicast message */
	    int     msg_size;  /* length of a bulk-data */
	} cast;
    } u;
} ens_msg_t;


ens_rc_t ens_RecvMetaData(ens_conn_t *conn, ens_msg_t *msg);
\end{codebox}

{\tt ens\_RecvMetaData} is a blocking call, therefore, it needs to be
executed only after the user knows there is pending data on the
socket. The user needs to pre-allocate an {\tt ens\_msg\_t} structure
which is used to store meta-information about the in-coming message: 
\begin{itemize}
\item which member is this message for? 
\item what type is the message?
  view, point-to-point message, multicast message, block, or exit.
\item For each type, how much memory is required to receive it. 
\end{itemize}

The next step is to call {\tt ens\_RecvView} for a view-message, and
{\tt ens\_RecvMsg} for a bulk-data message. 

\begin{codebox}
ens_rc_t ens_RecvView(ens_conn_t *conn, 
                      ens_member_t *memb,
                      /*OUT*/ ens_view_t *view);

ens_rc_t ens_RecvMsg(ens_conn_t *conn, 
                     /*OUT*/ int *origin, char *buf);
\end{codebox}















