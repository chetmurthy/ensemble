\newenvironment{FormatTable}{%
\begin{quote}\begin{tabular}{|l|l|} \hline
}{\end{tabular}\end{quote}
}
\newcommand {\formatentry}[2]	{#1 & #2 \\ \hline}

\section{\ensemble\ Membership Service TCP Interface}
\label{section:memership}

\note{This is intended as an appendix to the Maestro paper (Maestro:
A Group Structuring Tool For Applications With Multiple Quality of
Service Requirements).  It describes the exact TCP messaging
interface to the group membership service described in that paper.}

The description here is of the nuts-and-bolts TCP interface to the
\mlval{maestro} membership service service described in the
\ensemble\ tutorial.  Ensemble also supports a direct interface to
this service in ML.  Developers using ML should probably use this
interface instead.  See \sourceappl{maestro/*.mli} for the source
code for the interface to this service.

\subsection{Locating the service}
The membership service uses the environment variable
\mlval{ENS\_GROUPD\_PORT} to select a TCP port number to use.  Client
processes connect to this port in the normal fashion for TCP
services.  Client processes can join any number of groups over a
single connection to a server, so they normally only connect once to
the servers.

If you run \mlval{groupd} on all the hosts from which your clients
will be using the service, then processes can connect to the local
port on their host.  However, clients are not limited to using local
servers, and can connect to any membership server on their system.

If the TCP connection breaks, the membership service will fail the
member from all groups that it joined.  However, a client can
reconnect to the same server and rejoin the groups it was in.  If
client's membership server crashes, it can reconnect to a different
server.

\subsection{Communicating with the service}
Communication with the service is done through specially formatted
messages.  We describe the message types and their format here.

\begin{description}
\item
[messages:] Messages in both directions are formatted as follows.
Both directions of the TCP streams are broken into variable-length
messages.  All messages are multiples of 4 bytes long and all fields
are aligned on $4$ byte boundaries.  The first $4$ bytes of a message
is a header.  The header consists of an unsigned integer in network
byte order (NBO), giving the length of the body of the message (not
including the header).  The next message follows immediately after
the body.
\item
[integers:] Integers are unsigned and are formatted as $4$ bytes in
NBO.
\item
[strings:] Strings have a somewhat more complex format.  The total
size of a string in bytes is multiple of $4$.  The first $4$ bytes
are an integer length (unsigned, NBO).  The length is not necessarily
a multiple of $4$ bytes.  The body of the string immidiately follows
the length.  Immediately following the body are $0-3$ bytes of
padding to bring the total length to a multiple of $4$ bytes.
\item
[endpoint and group identifiers:] These types have the same format as
strings.  For non-Ensemble applications, the contents can contain
whatever the transport service you are using requires.  \ensemble\
only tests the contents of endpoint and group identifiers for
equality with other endpoints and groups.
\item
[lists:] Lists have two parts.  The first is an integer giving the
number of elements in the list.  Immediately following that are the
elements in the list, one after the other and adjacent to
one-another.  It is assumed that the application knows the formats of
the items in the list in order to break them up.
\end{description}

\begin{figure}[tb]
\begin{center}
\resizebox{!}{3in}{\incgraphics{member_state}}
\end{center}
\caption{\em Client state machine diagram of the client-server membership protocol.}
\label{fig:state}
\end{figure}

The actual messages sent between the client and the servers are composed of integers
and strings.  The first field of a message is an integer \emph{tag} value from which
the format of the remainder of the message can be determined.

\begin{itemize}
\item 

\mlval{Coord\_View} : A new view is being installed.  The view is a
list of Endpt.id's.  A member who just sent a Join message may not be
included in the view, in which case it should await the next View
message.  The ltime is the logical time of the view.  The first entry
in the view and the ltime uniquely identify the view.  The ltime's
that a member sees grow monotonicly.  In addition, a boolean value is
sent specifying whether this view is a primary view.  The primary bit
is based on the primary bit of the group daemon's being used.
\begin{FormatTable}
\formatentry{integer}{Coord\_View = 0}
\formatentry{group}{my group}
\formatentry{endpoint}{my endpoint}
\formatentry{integer}{logical time}
\formatentry{boolean}{primary view}
\formatentry{endpoint list}{view of the group}
\end{FormatTable}
\item 
\mlval{Coord\_Sync} : All members should "synchronize" (usually this
means waiting for messages to stabilize) and then reply with a SyncOk
message.  The next view will not be sent until all members have
replied.
\begin{FormatTable}
\formatentry{integer}{Coord\_Sync = 1}
\formatentry{group}{my group}
\formatentry{endpoint}{my endpoint}
\end{FormatTable}
\item 
\mlval{Coord\_Failed} : Fail a member is being reported as having
failed.  This is done because members may need to know about failures
in order to determine when they are synchronized.
\begin{FormatTable}
\formatentry{integer}{Coord\_Failed = 2}
\formatentry{group}{my group}
\formatentry{endpoint}{my endpoint}
\formatentry{endpoint list}{failed endpoints}
\end{FormatTable}
\item 
\mlval{Member\_Join} : Request to join the group.  Replied with a
View message.
\begin{FormatTable}
\formatentry{integer}{Member\_Join = 3}
\formatentry{group}{my group}
\formatentry{endpoint}{my endpoint}
\formatentry{bool}{logical time}
\end{FormatTable}
\item 
\mlval{Member\_Sync} : This member is synchronized.  Is a reply to a
Sync message.  Will be replied with a View message.
\begin{FormatTable}
\formatentry{integer}{Member\_Sync = 4}
\formatentry{group}{my group}
\formatentry{endpoint}{my endpoint}
\end{FormatTable}
\item 
\mlval{Member\_Fail} : Fail other members in the group (or leave the
group by failing self).
\begin{FormatTable}
\formatentry{integer}{Member\_Fail = 5}
\formatentry{group}{my group}
\formatentry{endpoint}{my endpoint}
\formatentry{endpoint list}{failed endpoints}
\end{FormatTable}
\item 
\mlval{Client\_Version} : This is sent by the client process to tell
the server its version.  If the server's version is incompatible, the
server will send an Error message and close the client's connection.
The value to use for the version field can be found by running any of
the Ensemble demonstration programs with the \mlval{-v} flag.
\begin{FormatTable}
\formatentry{integer}{Member\_Version= 6}
\formatentry{string}{service name (``ENSEMBLE:groupd'')}
\formatentry{string}{my version (``0.40'')}
\end{FormatTable}
\item
\mlval{Server\_Error} : This is sent by the server.  An error has 
occurred.  Usually this means the client's connection will be closed.
\begin{FormatTable}
\formatentry{integer}{Server\_Error = 7}
\formatentry{string}{explanation}
\end{FormatTable}

\end{itemize}
