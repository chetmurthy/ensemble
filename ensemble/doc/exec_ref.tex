\section{Sessvr and Procsvr: Remote execution service}

\mlval{sessvr} together with \mlval{procsvr} provide
coordinated execution of commands on multiple machines\footnote{this
description was written by Takako Hickey}.
\mlval{sessvr} forms the front end of the execution service.
It maintains a database of machines in a cluster, and serves
as the interface to clients of the cluster.
\mlval{procsvr} forms the back end of the execution service.
It handles the execution of programs on the machine it is running.
To assist \mlval{sessvr} in selecting machines, \mlval{procsvr}
periodically informs \mlval{sessvr} of its resource usage status.

A client submits to \mlval{sessvr} a set of programs to be executed
along with criteria used to select machines on which programs
are executed (e.g., machine architecture, load, etc.)
\mlval{sessvr} selects machines that are appropriate (to balance load
ties are broken randomly) and forwards the
request to the procsvr of each machine.
\mlval{procsvr} performs the request and returns the response to sessvr,
which forwards the request to the client.

The execution service currently tolerates message losses.
It will tolerate crash failures in near future.


\b{Execution service interface}

This section describes the rpc interface client must use to
communicate with the execution server.
An example client program is \mlval{dsh}.


\begin{itemize}
\item request: SessCreate of string\\
   response: SessCreateSuccess of string * string\\

Before requesting execsvr to execute any commands, a session must
be created using SessCreate.  The argument to SessCreate is a name hint.
A unique session name is generated from it and returned as the second
argument to SessCreateSuccess.


\item request: ProcCreate of string * command array array * string array
     * (dbop * attrval) list array * (string * string) list * sessop\_flag\\
   response: ProcCreateSuccess of string * sessproc array\\

\begin{center}
\begin{tabular}{|l|}	   \hline
dbop	 \\ \hline
DBeq \\
DBlt \\
DBgt \\
DBlteq \\
DBgteq \\
DBmin of int \\
DBrandom of int \\
DBinclude of string \\ \hline
\end{tabular}
\end{center}

\begin{center}
\begin{tabular}{|l|}	   \hline
dbval	 \\ \hline
String of string \\
Int of int \\
Float of float \\
Endpt of Ensemble.Endpt.id \\
Addr of Unix.inet\_addr \\
StrList of string list \\
Noval \\ \hline
\end{tabular}
\end{center}

\begin{center}
attrval = string * dbval
\end{center}
   
\begin{center}
\begin{tabular}{|l|l|}  \hline
sessop\_flag		& description \\ \hline
AllOrNothing of unit	& if not enough machines, abort \\
MultLimit of int	& run upto n programs per machine \\
Unlimited of unit	& infinite programs per machine \\ \hline
\end{tabular}
\end{center}
   
\begin{center}
\begin{tabular}{|l|l|l|}  \hline
sessproc fields & type & description \\ \hline
seprocname	& string &		process name \\
seprogram 	& string &		program  \\
seenv		& string array &	environment \\
seprocsvr	& Ensemble.Endpt.id & 	procsvr address \\ \hline
\end{tabular}
\end{center}

Once session is created, processes can be added to it using ProcCreate.
Arguments to ProcCreate are session name, programs, environment,
machine specification, process properties, and flag.  Machine
specification is an array of list of criteria.  Select operation on
machine database is performed on each list of criteria until enough
machines are selected.  Programs are arrays of arrays of commands.
Each component of the outer array contains commands to be executed in
parallel.  Each of these components contains an array of the size
same as that of machine specification and corrensponds to command
to be executed based on machine selection criteria.
This would, for example, allow execution of different version of
a program based on machine architecture.
Flag indicates what to do when not enough machines are selected.


\item request: SessWait of string\\
   response: SessWaitSuccess of string\\
   request: ProcWait of string * string\\
   response: ProcWaitSuccess of string * string\\

Processes created can be waited using SessWait or ProcWait operations.
SessWait on session waits on all processes and returns SessWaitSuccess
when the last process finishes.  ProcWait on a process in a session
waits for a specific process in a session.


\item request: SessSig of string * sess\_sig \\
   response: SessSigSuccess of string * sess\_sig \\
   request: ProcSig of string * string * sess\_sig \\
   response: ProcSigSuccess of string * string * sess\_sig \\

\begin{center}
\begin{tabular}{|l|}  \hline
sess\_sig \\ \hline
SessSigKill \\
SessSigSuspend \\
SessSigResume \\ \hline
\end{tabular}
\end{center}

Processes created can be killed using SessSig or ProcSig operations.
Currently only the kill signal is supported.
\end{itemize}
