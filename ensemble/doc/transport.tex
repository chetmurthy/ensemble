\newcommand{\sourcefile}[1]    {{\tt {#1}}}


\section{Heterogeneous Transports}

{\bf  Complete this section}

\ensemble\ provides a flexible infrastructure for sending communication
across a variety of different communication transports.  Not only can
different groups use different communication transports, but a single group
can support communication on multiple transports at the same time.

The design of the transport module is split into three parts: 
\begin{description}
\item[The socket module:] ~\newline
   Low-level system calls: {\tt send, sendto, recv} etc.,
  implemented in a system-independent fashion. The {\tt socket}
  directory contains the code. {\tt socket/u} is a simple-minded
  implementation that uses the Ocaml Unix library directly. A more
  efficient version is located in {\tt socket/s}, where native OS
  io-vector send/recv facilities are used. 

\item[Transports:] ~\newline
  Self registering {\it transports}:  Deering, UDP, TCP, NETSIM. These
  use the low-level socket module calls to provide an abstract {\it transport}.
	
\item[Routers:] ~\newline
  Uses a communication transport to
  build Ensemble specific send/recv capabilities. Length field, 
  group id, and endpoint rank are added to each outgoing
  message. Basic parsing is performed on received messages and sender
  rank, group, and message length are extracted. 
 
  There are several {\it routers} in the {\tt route}
  subdirectory. \sourcefile{signed.ml} adds a 16-byte MD5 checksum to
  each outgoing message. An agreed group-secret is used to key MD5,
  providing group authentication. Incoming messages are stripped of
  this header, and verified. \sourcefile{unsigned.ml} is the vanilla router.

\end{description}

The user can choose to use either one of the socket module
implementations. The socket module interface is defined in
\sourcefile{socket/socket.mli}. The unoptimized socket implementation
(usocket) represents message data as a Caml string and benefits from
native garbage collection. Its disadvantage is reduced
performance. The optimized socket library (ssocket) uses native C
io-vectors, and native operating-system scatter-gather message
send/receive facilities. This provides much better performance, and
zero-copy integration with C applications. The disadvantage is more
difficult integration with native ML values. 

The transports are defined the \sourcefile{trans} subdirectory. 
UDP in \sourcefile{trans/udp.ml}, TCP in \sourcefile{trans/tcp.ml},
DEERING in \sourcefile{trans/ipmc}, and NETSIM in
\sourcefile{trans/netsim}.

The \sourcefile{route} subdirectory contains three routes: signed,
unsigned, and bypass.

\subsection{Code walk-through}
To provide better understanding of the design this section walks
through a configuration of the unsigned router, UDP transport, 
and optimized socket library. We shall start from the bottom
and work our way up. 

In file \sourcefile{server/socket/s/nt/sendrecv.c}, there is code for sending an
array of C io-vectors and part of an ML string for win32. The function takes
five arguments:
\begin{itemize}
\item info\_v : a structure describing a list of remote targets and a
socket through which to send messages.
\item prefix\_v : an ML string that prefixes the data
\item ofs\_v, len\_v: the offset and length of the prefix to send
\item iova\_a : an array of io-vectors wrapped in an ML representation
\end{itemize}

\begin{codebox}
value skt_udp_mu_sendsv(
	value info_v,
	value prefix_v,
	value ofs_v,
	value len_v,
	value iova_v
) \{
    int naddr=0, i, ret=0, len=0;
    ocaml_skt_t sock=0 ;
    skt_sendto_info_t *info ;
    int nvecs = Wosize_val(iova_v) ;

    // Extract the set of addresses
    info = skt_Sendto_info_val(info_v);

    // Prepare the header
    skt_prepare_send_header(send_iova, peek_buf, Int_val(len_v), skt_iovl_len(iova_v));

    // Prepare the iovectors
    skt_add_ml_hdr(send_iova, 1, prefix_v, ofs_v, len_v);
    skt_gather(send_iova, 2, iova_v) ;
    
    sock = info->sock ;
    naddr = info->naddr ;

    for (i=0;i<naddr;i++) {
	// Send the message.  Assume we don't block or get interrupted.  
	ret = WSASendTo(sock, send_iova, nvecs+2, &len, 0,
			&info->sa[i], info->addrlen, 0, NULL);
	if (SOCKET_ERROR == ret) skt_udp_error("skt_udp_mu_sendsv");
    }
    return Val_unit;
\}
\end{codebox}

The {\tt \_mu\_} prefix is added to this function
because it uses the Ml/User convention for sending data. Each data
packet is split into:
\begin{description}
\item [ML header length:] Describes the length of the ML header. of length four bytes. 
\item [User data length:] Describes the length of the user data. of length four bytes. 
\item [ML header:] the ML header itself. Variable size.
\item [User data:] user data. Variable size.
\end{description}

The function builds a header of size eight that includes two integers: (a)
ml-header length (b) io-vector length in network byte order. The
header is the first in an array of io-vectors that includes in second
place the ML-header, and then the array of user io-vectors. Once the
io-array is assembled it is sent to each destination in the list using
the native OS API. 

{\tt skt\_udp\_mu\_sendsv} is hidden inside the socket library, and can
safely be used using {\tt Socket.udp\_mu\_sendsv}. The {\tt sendto\_info}
structure can be created from an array of target socket addresses, and
a sending socket.

\begin{codebox}
type sendto_info
val sendto_info : socket -> Unix.sockaddr array -> sendto_info

val udp_mu_sendsv : sendto_info -> buf -> ofs -> len -> Iov.t array -> unit
\end{codebox}


The Hsys module makes access to sendtovs safer, and changes its type:
\begin{codebox}
  val udp_mu_sendsv : sendto_info -> Buf.t -> ofs -> len -> Iovecl.t -> unit

  (* Implementation *)
  Socket.udp_mu_sendsv info 
    (Buf.string_of buf) (Buf.int_of_len ofs) (Buf.int_of_len len) 
    (Iovecl.to_iovec_array iovl) 
\end{codebox}
Core Ensemble code, including the routers, does not use Socket calls
directly. Rather, it uses the Hsys module which wraps all calls with a
more type safe interface. Separate types are used for length, offset,
io-vector, and buffer.

The UDP implementation at \sourcefile{trans/udp.ml} uses Hsys in the 
transmit function called {\tt x}.

\begin{codebox}
  let x hdr ofs len iovl = 
    Hsys.sendtosv dests hdr ofs len iovl;
    Iovecl.free iovl
\end{codebox}

The io-vector array is freed after the message is transmitted. The
reference count for an iovec-array is decremented on two occasions:
(1) it is sent on the network (2) it is handed to an application, and
the callback has completed. The iovec refcount is initially set to one
when the application sends it, and it is henceforth incremented
whenever a copy of it created. Ultimately, the refcount will be
decremented when the stability detection protocol determines that all
group members received the message.

\subsection{Design of the routers}
Many endpoints belonging to different groups can coexist in a single
Ensemble process. Each endpoint is identified by its connection
identifier. The internal representation of this id is given in module
{\tt Conn}:

\begin{codebox}
type id = \{
  version       : Version.id ;
  group 	: Group.id ;
  stack 	: Stack_id.t ;
  proto 	: Proto.id option ;
  view_id 	: View.id option ;
  sndr_mbr 	: sndr_mbr ;
  dest_mbr 	: dest_mbr ;
  dest_endpt 	: dest_endpt option
\}
\end{codebox}

The id is mapped into a string using the {\tt Route.pack\_of\_conn}
function. Ensemble uses MD5 for this mapping. The probability of a
collision, i.e., for two different endpoints to map onto a single
string, is $2^{-64}$ which is sufficient for our purposes. 

\begin{codebox}
val pack_of_conn : Conn.id -> Buf.t
\end{codebox}

The purpose of the route module is to create a single interface to
these various endpoints. The main type exported is {\tt
handlers}. This is essentially a large array holding the set of
connection identifiers and the delivery function for each of
them. When a message is received by the bottom-most part of the
system, it is parsed by the socket code into an ML header that is a
string, and the rest of the message which is received into a
C-iovector. This information is later fed into the {\tt deliver}
function.

\begin{codebox}
val deliver : handlers -> Buf.t -> Buf.ofs -> Buf.len -> Iovecl.t -> unit
\end{codebox}

Deliver takes the current set of handlers, and a message, figures out
which endpoints need to receive this message and calls the appropriate
handlers. 

A transmission function is abstracted as a type {\tt xmitf}:
\begin{codebox}
(* transmit an Ensemble packet, this includes the ML part, and a
 * user-land iovecl.
 *)
type xmitf = Buf.t -> Buf.ofs -> Buf.len -> Iovecl.t -> unit
\end{codebox}

The Router module has an API allowing the creation of send/recv
functions for connection-ids. It also allows installing and deleting
such functions. The unsigned router is a simple example of
using this functionality to create the basic, insecure,
router. It defines function {\tt f}: 
\begin{codebox}
val f : unit -> 
  (Trans.rank -> Obj.t option -> Trans.seqno -> Iovecl.t -> unit) Route.t
\end{codebox}

This router will allow users to send (1) sender rank (2) ML object (3)
sequence number and (4) a user iovector array. The body of the code
calls {\tt Route.create} where it mainly needs to define how it plans
on handling {\tt blast} and {\tt merge}. Blast is how to send
messages, merge is how to receive a message on behalf of several
connection ids. 

