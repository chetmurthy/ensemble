%*************************************************************%
%
%    Ensemble, 2_00
%    Copyright 2004 Cornell University, Hebrew University
%           IBM Israel Science and Technology
%    All rights reserved.
%
%    See ensemble/doc/license.txt for further information.
%
%*************************************************************%
\section{Java Application Interface}

%%%%%%%%%%%%%%%%%%%%%%%%%%%%%%%%%%%%%%%%%%%%%%%%%%%%%%%%%%%%%%%%%%%%%%%%%
% The API abstractions. 

The API is constructed from a namespace named Ensemble and several
public classes the major of which are: {\tt View, JoinOps, Message,
  Connection}, and {\tt Member}. 

\begin{description}
\item[View:] The {\tt View} class describes a group membership view. 

\item[JoinOps:] The {\tt JoinOps} class contains a
  specification for a new member for Ensemble to create. 

\item[Message:] The {\tt Message} class describes a message received
  from Ensemble. A Message can be: a new View, a multicast message, a
  point-to-point message, a block notification, or an exit notification. 

\item[Connection:] The {\tt Connection} class implements the actual
  socket communication between the client and the server. It has three
  public methods the application has to use.

  \begin{codebox}
   public class Connection \{
       public bool Poll();	  
       public Message Recv();
       public void Connect ();
  \}
  \end{codebox}

  The application can open several Ensemble connections, however, a
  single connection should suffice. No action can be taken on a
  connection that has not connected to the server through the {\tt
  Connect} call. Once connected the application can receive messages
  through the {\tt Recv} method. {\tt Recv} is a blocking call, in
  order to check first that there are pending messages the
  non-blocking {\tt Poll} method should be used. 
  

\item[Member:] The {\tt Member} class embodies an Ensemble group
  member. A member can join a single group, no more. A Member can be in several states: 
  \begin{description}
    \item[Pre:] The initial status in which all members are
      created. The class constructor sets this as the default.
    \item[Joining:] Joining a group is an asynchronous operation. A
      member is in the {\tt Joining} state from the time it attempts to join,
      until when it receives a View message with the initial group
      membership. 
    \item[Normal:] Normal operation state. The member is a regular
      resident in the group. It can send/mcast messages and
      perform all other Ensemble operations. 
    \item[Blocked:] The member is currently blocked, and temporarily cannot 
      perfrom any action. This state is achieved by sending a {\tt BlockOk}
      in response to a {\tt Block} request. 
    \item[Leaving:] The member has requested to leave the group. {\tt Leave}
      is an asynchronous operation, the {\tt Leaving} state captures
      the time between the {\tt Leave} request and the final leaving
      of the group. It is possible for a member to receive a {\tt
      Block} message after it has requested to leave the group. The
      member should not respond with a {\tt BlockOk}, it is in the
      process of being removed from the group. 
    \item[Left:] The member has left the group and is in an invalid state
  \end{description}

  \begin{codebox}
  public class Member \{
      public enum Status \{
          Pre,        // the initial status
          Joining,    // we joining the group
          Normal,     // Normal operation state, can send/mcast messages
          Blocked,    // we are blocked
          Leaving,    // we are leaving
          Left        // We have left the group and are in an invalid state
      \};

      public View current\_view ;		       // The current view
      public Status current\_status = Status.Pre; // Our current status
  \end{codebox}

  The member state can be learned by examining the {\tt
  current\_status} field. The current view the member is part of is in 
  the {\tt current\_view} field. 

  Prior to any action, the member has to join a group.
  \begin{codebox}
  // Join a group with the specified options. 
  public void Join(JoinOps ops);
  \end{codebox}
  
  The set of operations allowed on a member in Normal state is:
  \begin{description}
  \item[Leave:] Leave a group. This should be the last call made to the member
    It is possible for messages to be delivered to this member after the call
    returns. However, it is illegal to initiate new actions on this member.
    \begin{codebox}
      public void Leave();
    \end{codebox}
    
  \item[Cast:]
    Send a multicast message to the group.
    \begin{codebox}
      public void Cast(byte[] data);
    \end{codebox}
    
  \item[Send:]
    Send a point-to-point message to a list of members.
    \begin{codebox}
      public void Send(int[] dests, byte[] data);
    \end{codebox}
    
  \item[Send1:]
    Send a point-to-point message to the specified group member.
    \begin{codebox}
      public void Send1(int dest, byte[] data);
    \end{codebox}
    
  \item[Suspect:]
    Report group members as failure-suspected.
    \begin{codebox}
      public void Suspect(int[] suspects);
    \end{codebox}
    
  \item[BlockOk:]
    Send a BlockOk
    \begin{codebox}
      public void BlockOk();
    \end{codebox}
  \end{description}
\end{description}


% The client state-machine. 
\subsection{The client state-machine}

Each group member moved inside a state-machine that has a very clear set
of rules. Initially, it is in the {\tt Pre} state. After asking to
join a group, it is in the {\tt Joining} state. When the first view
arrives it is in the {\tt Normal} state. In the {\tt Normal} state and
prior to a view-change Ensemble will send a {\tt Block} message to the
member. The member has to reply with a {\tt BlockOk} action. After the {\tt BlockOk} the
member is in the {\tt Blocked} state until the next view. When the
next view is received is moves back to the {\tt Normal} state. Upon a
{\tt Leave} request the member moves to the {\tt Leaving} state which
turns into {\tt Left} after the {\tt Exit} message arrives. 

% FIXME: Insert a figure here. 

\subsection{Locking}
All {\tt Connection} and {\tt Member} method calls are
thread-safe. However, accessing the public Member fields such as the
{\tt current\_view} and {\tt current\_status} should be done while
holding the connection lock. All Ensemble actions, except Join, can be performed
only on a group-member that is in the {\tt Normal} state. A
multi-threaded application may need to synchronize its access to the
client library so as to avoid a situation where one thread sends a
{\tt BlockOk} and another thread sends message on the group later. 
The connection lock should be used for synchronization purposes. 

For example:
\begin{codebox}
  synchronization (conn) 
  \{
      if (memb.current_status == Member.Status.Normal)
          memb.Cast("hello world");
       else
           System.out.Println("Blocked currently, please try again later");
  \}
\end{codebox}

Replying to {\tt Block} message with a {\tt BlockOk} and moving to the
{\tt Blocked} state can be done asynchronously. This gives the
application a chance to send any pending messages 
prior to moving to the {\tt Blocked} state. Depending on the
application, this may allow the programmer to avoid locking. 

\subsection{The View structure}
The view structure is composed of several fields describing the
members of the current membership. 

\begin{codebox}
public class View \{
    public int nmembers;
    public String version;     /** The Ensemble version */
    public String group;       /** group name  */
    public String proto;       /** protocol stack in use  */
    public int ltime;          /** logical time  */
    public boolean primary;    /** this a primary view?  */
    public String parameters;  /** parameters used for this group  */
    public String[] address;   /** list of communication addresses  */
    public String[] endpts;    /** list of endpoints in this view  */
    public String endpt;       /** local endpoint name  */
    public String addr;        /** local address  */
    public int rank;           /** local rank  */
    public String name;        /** My name. */
    public ViewId view_id;     /** view identifier */
\}
\end{codebox}

\begin{description}
\item[nmembers:] The number of members. 
\item[version:] The Ensemble version
\item[group:] The name of this group. This was chosen by the user when
  he joined the group. 
\item[proto:] The names of all the layers in the Ensemble stack. 
\item[ltime:] The logical time of the view.
\item[primary:] It this a primary view? 
\item[parameters:]  The set of additional parameters used for this
  stack. These were chosen by the user when he joined the group. 
\item[address:] The list of communication addresses of group
  members. Currently, this lists the addresses of {\it servers} not
  the clients. 
\item[endpts:] list of endpoints in this view.
\item[endpt:] my endpoint name.
\item[addr:] my communication address.
\item[rank:] my rank.
\item[name:] My name. Does not change through the lifetime of this member. 
\item[view\_id:] The view identifier.
\end{description}

\subsection{Join options}
The join-options structure is used to specify which group an endpoint
should join. 

\begin{codebox}
public class JoinOps \{
    public String group_name = null;  /** group name.   */

    /** The default set of properties */
    public final String DEFAULT_PROPERTIES = "Vsync";
    /** requsted list of properties.  */    
    public String properties = DEFAULT_PROPERTIES; 

    public String parameters = null;  /** parameters to pass to Ensemble.  */
    public String princ = null;       /** principal name  */
    public boolean secure = false;    /** a secure stack?  */
\}
\end{codebox}

\begin{description}
\item[group\_name:] The group name to join.
\item[properties:] The set of properties requested. The most oftenly
  used property is Vsync which ensures virtually-synchronous
  point-to-point communication. The other commonly used properties
  are:
  \begin{description}
    \item{Total:} totally ordered multicast messages
    \item{Auth:} Authenticate members and MAC all group-messages. 
    \item{Scale:} use scalable protocols. Useful if there are more
      than 12 machines in the group. 
  \end{description}

  See section~\ref{sec:properties} for more information. 

\item[parameters:]  An additional set of parameters controlling the
  stack. 
\item[princ:] The client principal name. This is used only for secure
  stacks. It can be left {\tt null} otherwise. 
\item[secure:] Should the stack be secure? 
\end{description}

\subsection{Limitations}
There are several limitations on argument sizes. 

\begin{codebox}
    private final int ENS_DESTS_MAX_SIZE=      10;
    private final int ENS_GROUP_NAME_MAX_SIZE= 64;
    private final int ENS_PROPERTIES_MAX_SIZE= 128;
    private final int ENS_PROTOCOL_MAX_SIZE=   256;
    private final int ENS_PARAMS_MAX_SIZE=     256;
    private final int ENS_ENDPT_MAX_SIZE=      48;
    private final int ENS_ADDR_MAX_SIZE=       48;
    private final int ENS_PRINCIPAL_MAX_SIZE=  32;
    private final int ENS_NAME_MAX_SIZE=       ENS_ENDPT_MAX_SIZE+24;
    private final int ENS_VERSION_MAX_SIZE=    8;
    private final int ENS_MSG_MAX_SIZE=        32 *1024;
\end{codebox}

The important limitations are: 
\begin{description}
\item[Number of destinations:] The user can send a point-to-point
  message to a set of targets. The size of the target set is limited
  to ENS\_DESTS\_MAX\_SIZE. The same goes for suspecting sets of
  members. 
\item[Message size:] The maximal message size is limited to 32K. 
\end{description}

\subsection{Code examples}
Take a look at the Mtalk.java program in the {\tt client/java} subdirectory.
