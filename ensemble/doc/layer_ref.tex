%\newcommand {\layerdef}[1]     {\subsection{{#1} Protocol Layer}}
%\newcommand {\stackdef}[1]     {\subsection{{#1} Protocol Stack}}
%\newcommand {\parameters}[0]   {\paragraph{Parameters} ~\par}
%\newcommand {\generates}[0]    {\paragraph{Generated Events} ~\par}
%\newcommand {\properties}[0]   {\paragraph{Properties} ~\par}
%\newcommand {\protocol}[0]     {\paragraph{Protocol} ~\par}
%\newcommand {\notes}[0]        {\paragraph{Notes} ~\par}
%\newcommand {\sources}[0]      {\paragraph{Sources} ~\par}
%\newcommand {\testing}[0]      {\paragraph{Testing} ~\par}

\newenvironment{Layer}[1]       {\subsection{{#1}}}{}
\newenvironment{Stack}[1]       {\subsection{{#1}}}{}

\newenvironment{GenEvent}{%
\paragraph{Generated Events} ~\par
\begin{quote}\begin{tabular}{|l|} \hline
}{\end{tabular}\end{quote}
}
\newcommand {\genevent}[1]      {#1 \\ \hline}

\newenvironment{Protocol}{%
\paragraph{Protocol} ~\par
\begin{quote}
}{\end{quote}}

\newenvironment{Sources}{%
\paragraph{Sources} ~\par
\begin{quote}\begin{tabular}{|l|} \hline
}{\end{tabular}\end{quote}
}
\newcommand {\sourcesfile}[1]   {layers/#1 \\ \hline}

\newenvironment{Parameters}{%
\paragraph{Parameters}~\par\begin{itemize}
}{\end{itemize}}

\newenvironment{Properties}{%
\paragraph{Properties} ~\par\begin{itemize}
}{\end{itemize}}

\newenvironment{Notes}{%
\paragraph{Notes} ~\par\begin{itemize}
}{\end{itemize}}

\newenvironment{Testing}{%
\paragraph{Testing} ~\par\begin{itemize}
}{\end{itemize}}

\section{Layers and Stacks}

We document a subset of the \ensemble\ layers and stacks (compositions of
layers) in this section.  This documentation is intended to be largely
independent of the implementation language.  They are currently listed in
order, bottom-up, of their use in the VSYNC layer.

Each layer (or stack) has these items in its documentation:

\begin{Layer}{ANYLAYER}
The name of the layer follwed by a general description of its purpose.

\begin{Protocol}
A description of the protocol implemented by the layer.
\end{Protocol}

\begin{Parameters}
\item The list of parameters required to initialize the layer, along with
descriptions of their purpose.
\item \note{should also specify reasonable values}
\item If a layers takes no arguments, the documentation specifies
``None.''
\end{Parameters}

\begin{Properties}
\item A list of informal properties of the layer.
\end{Properties}

\begin{Notes}
\item General notes about the layer.
\end{Notes}

\paragraph{Sources} ~\par
The source files for the ML implementation of the layer.

\paragraph{Generated Events} ~\par
A list of event types generated by the layer.  In the future, this field
will contain more information, such as what event types are examined by
the layer (instead of being blindly passed on).  Hopefully, this
information will eventually be generated automatically.

\begin{Testing}
\item Information about the status of the layer regarding testing.
\item Testing information should always be documented: if the layer has
  not been tested, that should be stated.
\item What testing has been completed on the layer (along with version
  information).
\item What infrastracture is in place for testing the layer.
\item Known bugs for a layer are listed in the ML source code.
\end{Testing}
\end{Layer}

\begin{Layer}{CREDIT}

This layer implements a credit based flow control.

\begin{Protocol}
On initialization, sender informs receivers how many credits it wants to
keep in stock.  Receivers sends credits whenever it finds that the sender
is low on credits, either explicitly through a sender's request or
implicitly through its local accounting.  A credit is one time use only.
Sender is allowed to send a message only if it has a credit available.  If
the sender does not have a credit, the message is buffered.  Buffered
messages are sent when new credits arrive.  Credits are piggybacked to data
messages whenever there is an opportunity of doing so to save bandwidth.
\end{Protocol}

\begin{Parameters}
\item 
rtotal: the total number of credits that this member can give out.  Should
be set according to the number of receive buffers that the machine the
member is running has.
\item 
ntoask: the number of credits that this member likes to keep in stock.
\item 
whentoask: the threshold number of credits remaining at sender before the
receiver consider sending out more credits.
\item 
pntoask: like ntoask for piggyback style of credit giving.
\item 
pwhentoask: like nwhentoask for piggyback style of credit giving.
\item 
sweep: frequency at which periodic sweep routine, which give out credits to
senders, should run.
\end{Parameters}

%\begin{Properties}
%\end{Properties}

\begin{Notes}
\item Future implementation should support dynamic credit adjustment.
\item Alternative flow control layers include RATE and WINDOW.
\end{Notes}

\begin{Sources}
\sourcesfile{credit.ml}
\end{Sources}

Last updated: Fri Mar 29, 1996

\end{Layer}

%*************************************************************%
%
%    Ensemble, 1_42
%    Copyright 2003 Cornell University, Hebrew University
%           IBM Israel Science and Technology
%    All rights reserved.
%
%    See ensemble/doc/license.txt for further information.
%
%*************************************************************%
\begin{Layer}{RATE}

This layer implements a sender rate based flow control.  Multicast messages
from each sender are sent at a rate not exceeding some prescribed value.

\begin{Protocol}
All the messages to be sent are buffered initially.  Buffered messages are
sent on periodic timeouts that are set based on the sender's rate.
\end{Protocol}

\begin{Parameters}
\item
rate\_n, rate\_t: the pair determines the rate.  At most \mlval{rate\_n}
messages are allowed to sent over any time period of \mlval{rate\_t}.  This
is ensured by having two consecutive messages sent with a inter-send time
of at least $(rate\_t / rate\_n)$ apart.
\end{Parameters}

%\begin{Properties}
%\end{Properties}

\begin{Notes}
\item Future implementation should support dynamic rate adjustment.
\item Alternative flow control layers include CREDIT and WINDOW.
\end{Notes}

\begin{Sources}
\sourcesfile{rate.ml}
\end{Sources}

\emph{This layer and its documentation were written by Takako Hickey.}
\end{Layer}

\begin{Layer}{BOTTOM} 

Not surprisingly, the BOTTOM layer is the bottommost layer in a \ensemble\
protocol stack.  It interacts directly with the communication transport by
sending/receiving messages and scheduling/handling timeouts.  The properties
implemented are all \emph{local} to the protocol stack in which the layer
exists: ie., a (dn)Fail event causes failed members to be removed from the local
view of the group, but no failure message to be sent out--it is assumed that
some other layer actually informs the other members of the failure.

\begin{Protocol}
None
\end{Protocol}

\begin{Parameters}
\item None
\end{Parameters}

\begin{Properties}
\item
Requires messages be appropriately fragmented for the transport in use.
\item
\Dn{Timer}\{time\} events cause an alarm to be scheduled with the transport so
that an \Up{Timer} event is later delivered some time after $time$.
\item
\Dn{Block}, \Dn{View}, \Dn{Stable}, and \Dn{Fail} events cause an \Up{Block},
\Up{View}, \Up{Stable}, and \Up{Fail} event (respectively) to be locally
``bounced'' up the protocol stack.  No communication results from these
events.
\item
In addition, \Dn{Fail} events cause further Send and Cast messages from the
failed members to be dropped.
\item
\Dn{View} events do not affect the membership in the current protocol stack.
The view in the resulting \Up{View} event is merely a proposal for the next
view of the group.  (It is expected that a new protocol stack will be
created for that view.)
\item
\Dn{Send}, \Dn{Cast} events cause messages to be sent (unreliably) to other
members in the group.  The resulting \Up{Send} and \Up{Cast} events are delivered
with the origin field set with the rank of the sender and the time field
set with the time that the messages was received (according to the
transport).
\item
\Dn{Merge}, \Dn{MergeDenied}, and \Dn{MergeGranted} causes messages to be sent
(unreliably) to members outside of the group.  These result in
\Up{MergeRequest}, \Up{MergeDenied}, and \Up{MergeGranted} messages at the
destination, respectively.
\item
\Dn{Leave} events disable the transport instance and bounce up an \Up{Exit}
event.  No further events are delivered after the \Up{Exit}. \note{currently,
this may not be true}
\end{Properties}

\begin{Sources}
\sourcesfile{bottom.ml}
\end{Sources}

\begin{GenEvent}
\genevent{\Up{Block}}
\genevent{\Up{Cast}}
\genevent{\Up{Exit}}
\genevent{\Up{Fail}}
\genevent{\Up{Stable}}
\genevent{\Up{MergeDenied}}
\genevent{\Up{MergeGranted}}
\genevent{\Up{MergeRequest}}
\genevent{\Up{Send}}
\genevent{\Up{Suspect}}
\genevent{\Up{Timer}}
\genevent{\Up{View}}
\end{GenEvent}

\begin{Testing}
\item see the VSYNC stack
\end{Testing}
\end{Layer}

%*************************************************************%
%
%    Ensemble, 2_00
%    Copyright 2004 Cornell University, Hebrew University
%           IBM Israel Science and Technology
%    All rights reserved.
%
%    See ensemble/doc/license.txt for further information.
%
%*************************************************************%
\begin{Layer}{CAUSAL}

The CAUSAL layer implements causally order multicast.  It assumes
reliable, FIFO ordered reliable messaging from layers below.

\begin{Protocol}
The protocol has two versions: full and compressed vectors.  First,
we explain the simple version which uses full vectors. Then, we
explain how these vectors are compressed.

Each outgoing message is appended with a \emph{causal vector}. This
vector contains the last causally delivered message from each member
in the group.  Each received message is checked for deliverability.
It may be delivered only if all messages which it causally follows,
according to its causal vector, have been delivered.  If it is not
yet deliverable, it is delayed in the layer until delivery is
possible.  A view change erases all delayed messages, since they can
never become deliverable.

Causal vectors become large with the group size, so they must be
compressed in order for this protocol to scale.  The compression we
use is derived from the Transis system.  We demonstrate with an
example: assume the membership includes three processes $p,q$ and
$r$. Process $p$ sends message $m_{p,1}$, $q$ sends $m_{q,1}$,
causally following $m_{p,1}$ and $r$ sends $m_{r,1}$ causally
following $m_{q,1}$. The causal vector for $m_{r,1}$ is
$[1|1|1]$. There is redundancy in the causal vector since it is clear
that $m_{r,1}$ follows $m_{r,0}$. Furthermore, since $m_{q,1}$
follows $m_{p,1}$ we may omit stating that $m_{r,1}$ follows
$m_{p,1}$. To conclude, it suffices to state that $m_{r,1}$ follows
$m_{q,1}$.  Using such optimizations causal vectors may be compressed
considerably.
\end{Protocol}

\begin{Sources}
\sourcesfile{causal.ml}
\end{Sources}

\begin{Testing}
\item
The CHK\_CAUSAL protocol layer checks for CAUSAL delivery.
\end{Testing}

\emph{This layer and its documentation were written by Ohad Rodeh.}
\end{Layer}

\begin{Layer}{ELECT}

This layer implements a leader election protocol.  It determines when a member
should become the coordinator.  Election is done by delivering an \DnElect\
event at the new coordinator.

\begin{Protocol}
When a member suspects all lower ranked members of being faulty, that member elects
itself as coordinator.
\end{Protocol}

\begin{Parameters}
\item None
\end{Parameters}

\begin{Properties}
\item
\UpSuspect\ events may cause a \DnElect\ event to be generated.
\end{Properties}

\begin{Sources}
\sourcesfile{elect.ml}
\end{Sources}

\begin{GenEvent}
\genevent{\DnElect}
\end{GenEvent}

\begin{Testing}
\item
see also the VSYNC stack
\end{Testing}
\end{Layer}


%*************************************************************%
%
%    Ensemble, 1.10
%    Copyright 2001 Cornell University, Hebrew University
%    All rights reserved.
%
%    See ensemble/doc/license.txt for further information.
%
%*************************************************************%
\begin{Layer}{ENCRYPT}

This layer encrypts application data for privacy.  Uses keys in the view state
record.  Authentication needs to be provided by the lower layers in the system.
The protocol headers are not encrypted.  This layer must reside above FIFO
layers for sending and receiving because it uses encryption contexts whereby
the encryption of a message is dependent on the previous messages from this
member.  These contexts are dropped at the end of a view.  A smarter protocol
would try to maintain them, as they improve the quality of the encryption.

\begin{Protocol}
Does chained encryption on the message payload in the \mlval{iov} field of
events.  Each member keeps track of the encryption state for all incoming and
outgoing point-to-point and multicast channels.  Messages marked
\mlval{Unreliable} are not encrypted (these should not be application
messages).
\end{Protocol}

\begin{Parameters}
\item None
\end{Parameters}

\begin{Properties}
\item
Guarantees (modulo encryption being broken) that only processes that know the
shared group key can read the contents of the application portion of data
messages.
\item
Requires FIFO ordering on point-to-point and multicast messages.
\end{Properties}

\begin{Sources}
\sourcesfile{encrypt.ml}
\end{Sources}

\begin{GenEvent}
\genevent{None}
\end{GenEvent}

\begin{Testing}
\item
see the VSYNC stack
\end{Testing}
\end{Layer}

%*************************************************************%
%
%    Ensemble, 1_42
%    Copyright 2003 Cornell University, Hebrew University
%           IBM Israel Science and Technology
%    All rights reserved.
%
%    See ensemble/doc/license.txt for further information.
%
%*************************************************************%
\begin{Layer}{HEAL}

This protocol is used to merge partitions of a group.

\begin{Protocol}
The coordinator occasionally broadcasts the existence of this partition via
\Dn{GossipExt} events.  These are delivered unreliably to coordinators of other
partitions.  If a coordinator decides to merge partitions, then it prompts a
view change and inserts the name of the remote coordinator in the \Up{BlockOk}
event.  The INTER protocol takes over from there.  Merge cycles are prevented
by only allowing merges to be made from smaller view id's to larger view id's.
\end{Protocol}

\begin{Parameters}
\item
heal\_wait\_stable : whether or not to wait for a first broadcast message to
become stable before starting the protocol.  This ensures that all the members
are in the group.
\end{Parameters}

\begin{Properties}
\item \todo{}
\end{Properties}

\begin{Sources}
\sourcesfile{heal.ml}
\end{Sources}

\begin{GenEvent}
\genevent{\Up{Prompt}}
\genevent{\Dn{GossipExt}}
\end{GenEvent}

\begin{Testing}
\item see the VSYNC stack
\end{Testing}
\end{Layer}

%*************************************************************%
%
%    Ensemble, 1_42
%    Copyright 2003 Cornell University, Hebrew University
%           IBM Israel Science and Technology
%    All rights reserved.
%
%    See ensemble/doc/license.txt for further information.
%
%*************************************************************%
\begin{Layer}{INTER}

This protocol handles view changes that involve more than one
partition (see also INTRA).

\begin{Protocol}
Group merges are the more complicated part of the group membership
protocol.  However, we constrain the problem so that:
\begin{itemize}
\item
Groups cannot be both merging and accepting mergers at the same time.
This eliminates the potential for cycles in the ``merge-graph.''
\item
A view (i.e. view\_id) can only attempt to merge once, and only if no
failures have occured.  Each merge attempt is therefore uniquely
identified by the view\_id of the merging group.  Note also that by
requiring no failures to have occured for a merge to happen, this
prevents a member from being failed in one view and then reappearing
in the next view.  There has to be an intermediate view without the
failed member: this is a desirable property.
\end{itemize}
The merge protocol works as follows:
\begin{enumerate}
\item
The merging coordinator blocks its group,
\item
The merging coordinator sends a merge request to the remote group's
coordinator.
\item
The remote coordinator blocks its group,
\item
The remote coordinator installs a new view (with the mergers in it) and
sends the view to the merging coordinator (through a merge-granted
message).
\item
The merging coordinator installs the view in its group.
\end{enumerate}
If the merging coordinator times out on the merged coordinator then it
immediately installs a new view in its partition (without the other members
even finding out about the merge attempt).
\end{Protocol}

\begin{Parameters}
\item None
\end{Parameters}

\begin{Properties}
\item
When another partition is merging, a View message is also sent to the
coordinator of the merging group, which then forwards the message to
the rest of its group.
\item 
Requires that \Dn{Merge} events only be delivered by the original
coordinator of views (in which no failures have yet occured).
Otherwise, the partition should first form a new view and then attempt
the merge.
\item
\Dn{Merge} causes a \Dn{Merge} event to be delivered to the layer below.  This
will be replied with either an \Up{View}, \Up{MergeFailed}, or \Up{MergeDenied}
event, depending on the outcome of the merge attempt.
\item
\Up{MergeRequest}'s are only delivered at the coordinator.  And only if the
group is not currently blocking and only if the mergers list does not
contain members that are/were in this view or in previous merge requests in
this view.
\end{Properties}

\begin{Sources}
\sourcesfile{inter.ml}
\end{Sources}

\begin{GenEvent}
\genevent{\Dn{Merge}}
\genevent{\Dn{MergeDenied}}
\genevent{\Dn{Suspect}}
\end{GenEvent}

\begin{Testing}
\item
see the VSYNC stack
\end{Testing}
\end{Layer}


\begin{Layer}{INTRA}

This layer manages group membership within a view (see also the INTER layer).
There are three related tasks here:
\begin{itemize}
\item
Forwarding of group membership events to the rest of the group
(without INTRA, normally \DnView\ and \DnFail\ events have only local
effect).
\item 
Filtering of group membership events from remote members (for
example, when two other group members think they are the coordinator
and fail each other, the INTRA layer choose one of them and ignores
the other member).
\item
Determining the view\_id of the following view.
\end{itemize}

\begin{Protocol}
This is a relatively simple group membership protocol.  We have done our
best to resist the temptation to ``optimize'' special cases under which the
group is ``unnecessarily'' partitioned.  We also constrain the conditions
under which operations such as merges can occur.  The implementation does
not ``touch'' any data messages: it only handles group membership changes.
Furthermore, this protocol does not use any timeouts.

Views and failures are forwarded via broadcast to the rest of the members.
Other members accept the view/failure if they are consistent with their
current representation of the group's state.  Otherwise, the view/failure
message is dropped and the sender is suspected of being problematic. 
\end{Protocol}

\begin{Parameters}
\item None
\end{Parameters}

\begin{Properties}
\item
\DnView\ events are passed on to the layer below.  They also cause a View message to
be broadcast to the other members.  On receipt of this View message, the other
members either accept it (and deliver a \DnView\ event to layer below) or mark the
sender of the View as problematic, and possibly deliver a \DnSuspect\ event to the
layer below.
\item
Requires FIFO, atomic broadcast delivery from layers below.
\item
\DnFail\ events are passed on to the layer below.  They also cause a Fail
message to be broadcast to the other members.  On receipt of this Fail
message, the other INTRA instances will either accept it (and deliver a
\DnFail\ event to the layer beneath them) or mark the sender of the Fail
message as problematic, and possibly deliver an \DnSuspect\ event to the
layer below.
\item
View and Fail messages from a particular coordinator are delivered in FIFO
order to the members.
\item
Not all members may see same set of \UpFail\ events.  However, the set of
failed members grows monotonicly with each failure notification.
\end{Properties}

\begin{Sources}
\sourcesfile{intra.ml}
\end{Sources}

\begin{GenEvent}
\genevent{\DnAck}
\genevent{\DnCast}
\genevent{\DnFail}
\genevent{\DnSuspect}
\genevent{\DnView}
\end{GenEvent}

\begin{Testing}
\item
see the VSYNC stack
\end{Testing}
\end{Layer}


%*************************************************************%
%
%    Ensemble, 1.10
%    Copyright 2001 Cornell University, Hebrew University
%    All rights reserved.
%
%    See ensemble/doc/license.txt for further information.
%
%*************************************************************%
\begin{Layer}{LEAVE}

This protocol has two tasks.  (1) When a member really wants to leave a group,
the LEAVE protocol tells the other members to suspect this member.  (2) The
leave protocol garbage collects old protocol stacks by initiating a \DnLeave\
after getting an \UpView\ and then getting an \UpStable\ where everything is
marked as being stable.

\begin{Protocol}
Both protocols are simple.

For leaving the group, a member broacasts a Leave message to the group
which causes the other members to deliver a \DnSuspect\ event.  Note that
the other members will get the Leave message only after receiving all the
prior broadcast messages.  This member should probably stick around,
however, until these messages have stabilized.

Garbage collection is done by waiting until all broadcast message are
stable before delivering a local \DnLeave\ event.
\end{Protocol}

\begin{Parameters}
\item
leave\_wait\_stable : whether or not to wait for the leave announcment to
become stable before leaving
\end{Parameters}

\begin{Properties}
\item \todo{}
\end{Properties}

\begin{Sources}
\sourcesfile{leave.ml}
\end{Sources}

\begin{GenEvent}
\genevent{\DnLeave}
\end{GenEvent}

\begin{Testing}
\item see the VSYNC stack
\end{Testing}
\end{Layer}

%*************************************************************%
%
%    Ensemble, 1.10
%    Copyright 2001 Cornell University, Hebrew University
%    All rights reserved.
%
%    See ensemble/doc/license.txt for further information.
%
%*************************************************************%
\begin{Layer}{MERGE}

This protocol provides reliable retransmissions of merge messages and failure
detection of remote coordinators when merging.

\begin{Protocol}
Simple retransmission protocol.  A hash table is used to detect copied
merge requests, which are dropped.
\end{Protocol}

\begin{Parameters}
\item
merge\_sweep: how often to retransmit merge requests
\item
merge\_timeout: how long before timing out on merges
\end{Parameters}

\begin{Properties}
\item
\DnMerge, \DnMergeGranted, and \DnMergeDenied\ events are buffered for later
retransmission.
\item
\UpMergeRequest, \UpMergeGranted, and \UpMergeDenied\ events are filtered
so that each event is delivered at most once by this layer (i.e., so that
retransmissions are dropped).
\item
After $timeout$ time (a parameter listed above) an \UpMergeFailed\ event is
delivered with the problems field set to be the contact of the \DnMerge\
(only) event.  (It is assumed that the merge process will normally be
complete before this timeout occurs.)
\end{Properties}

\begin{Notes}
\item
Removal of this protocol layer only makes the merges unreliable, and
stops the failure detection of the new coordinator.
\end{Notes}

\begin{Sources}
\sourcesfile{merge.ml}
\end{Sources}

\begin{GenEvent}
\genevent{\UpSuspect}
\genevent{\DnMerge}
\genevent{\DnTimer}
\end{GenEvent}

\begin{Testing}
\item see the VSYNC stack
\end{Testing}
\end{Layer}

\begin{Layer}{MFLOW} 

This layer implements window-based flow control for multicast messages.
Multicast messages from each sender are transmitted only if the number of send
credit left is greater than zero.  The protocol attempts to avoid situations
where all recievers send credit at the same time, so that a sender is not
flooded with credit messages.

\begin{Protocol}
Whenever the amount of send credits drops to zero, messages are buffered without
being sent.  On receipt of acknowledgement credit, the amount of send credits
are recalculated and buffered messages are sent based on the new credit.
\end{Protocol}

\begin{Parameters}
\item mflow\_window : the maximum amount on unacknowledged messages or the size of the
window.
\item mflow\_ack\_thresh : The acknowledge threshold.  After receiving this number of
bytes of data from a sender, the receiver acknowledged previous credit.
\end{Parameters}

\begin{Properties}
\item
This protocol bounds the number of unrecieved multicast messages a member has
sent.
\item
The amount of received credits are initialized to different values for
avoiding many members sending back acknowledge at the same time. 
\item
This protocol requires reliable multicast and point-to-point properties from
underlying protocol layers.
\end{Properties}

\begin{Notes}
\item
As opposed to most of the \ensemble\ protoocols, this protocol implements flow
control on bytes and not on messages.  It only considers the data in the
application payload portion of the message (the \mlval{iov} field of the
event).
\item
Because of the EBlockOk events, this layer needs to be below the
broadcast stability layer.
\end{Notes}

\begin{Sources}
\sourcesfile{mflow.ml}
\end{Sources}

\begin{Testing}
\item
Some testing has been carried out.
\end{Testing}

\emph{This layer and its documentation were written with Zhen Xiao.}
\end{Layer}

\begin{Layer}{MNAK}

The MNAK (Multicast NAK) layer implements a reliable, agreed, FIFO-ordered
broadcast protocol.  Broadcast messages from each sender are delivered in
FIFO-order at their destinations.  Messages from live members are delivered
reliably and messages from failed members are retransmitted by the coordinator
of the group.  When all failed members are marked as such, the protocol
guarantees that eventually all live members will have delivered the same set of
messages.

\begin{Protocol}
Uses a negative acknowledgment (NAK) protocol: when messages are detected
to be out of order (or the $NumCast$ field in an \UpStable\ event detects
missing messages), a NAK is sent.  The NAK is sent in one of three ways,
chosen in the following order:
\begin{enumerate}
\item Pt2pt to the sender, if the sender is not failed.
\item Pt2pt to the coordinator, if the reciever is not the coordinator.
\item Broadcast to the rest of the group if the receiver is the
  coordinator.
\end{enumerate}
All broadcast messages are buffered until stable.
\end{Protocol}

\begin{Parameters}
\item mnak\_allow\_lost : boolean that determines whether the MNAK layer will
check for lost messages.  Lost messages are only possible when using an
inaccurate stability protocol.
\end{Parameters}

\begin{Properties}
\item
Requires stability and $NumCast$ information (equivalent to that provided
by the STABLE layer).
\item
\DnCast\ events cause \UpCast\ events to be delivered at all other members
(but not locally) in the group in FIFO order.
\item
\UpCast\{ack = Ack(rank,seqno)\} events from this layer require a later
\DnAck\{dn\_ack = Ack(rank,seqno)\} event to be delivered from above.
\item
\UpStable\ events from the layer below cause stable messages to be garbage
collected and may cause NAK messages to be sent to other members in the
group.
\end{Properties}

\begin{Sources}
\sourcesfile{mnak.ml}
\end{Sources}

\begin{GenEvent}
\genevent{\DnCast}
\genevent{\DnSend}
\end{GenEvent}

\begin{Testing}
\item
The CHK\_FIFO protocol layer checks for FIFO  safety conditions.
\item
The FIFO application generates bursty communication in which a token traces
its way through each burst.  If a reliable communication layer drops the
token, communication comes to an abrupt halt.  This is intended to capture
the liveness conditions of FIFO layers.
\item 
see also the VSYNC stack
\end{Testing}
\end{Layer}

\begin{Layer}{PRIMARY}

Detect primary partition in a group. Usually a primary partition has
the majority of members or holds some important resources.
\begin{Protocol}
Upon \UpInit\ event, a member sends a message to the coordinator,
claiming that it is in the current view. When a view has the majority
of members, its coordinator prompts a view change to make itself the
primary partition if it is not yet.  When a new view is ready, it
decides whether it is primary and mark it as so.
\end{Protocol}

\begin{Parameters}
\item primary\_quorum: how many servers (non-client member) are
needed to form the primary partition.
\end{Parameters}

\begin{Properties}
\item
Guarantees no two primary partitions can have the same logical timestamp.
\item
Optimal in the normal case: no addtional view change is necessary.
\item
This protocol requires group membership management from underlying
protocol layers. 
\end{Properties}

\begin{Sources}
\sourcesfile{primary.ml}
\end{Sources}

\begin{GenEvent}
\genevent{\DnPrompt}
\genevent{\DnSend}
\end{GenEvent}

\begin{Testing}
\item
TODO
\end{Testing}

\emph{This layer and its documentation were written with Zhen Xiao.}
\end{Layer}

%*************************************************************%
%
%    Ensemble, 2_00
%    Copyright 2004 Cornell University, Hebrew University
%           IBM Israel Science and Technology
%    All rights reserved.
%
%    See ensemble/doc/license.txt for further information.
%
%*************************************************************%
\begin{Layer}{PT2PT} 

This layer implements reliable point-to-point message delivery.

\todo{finish this documentation}

\begin{Parameters}
\item
pt2pt\_sweep : how often to retransmit messages and send out acknowledgments
\item 
pt2pt\_ack\_rate : determines how many messages will be received before an
acknowledgement is generated.
\item 
pt2pt\_sync : boolean determining if point-to-point messages should be
synchronized with view changes
\end{Parameters}

\begin{Testing}
\item see the VSYNC stack
\end{Testing}
\end{Layer}

%*************************************************************%
%
%    Ensemble, 1.10
%    Copyright 2001 Cornell University, Hebrew University
%    All rights reserved.
%
%    See ensemble/doc/license.txt for further information.
%
%*************************************************************%
\begin{Layer}{PT2PTW} 

This layer implements window-based flow control for point to point messages.
Point-to-point messages from each sender are transmitted only if the window is
not yet full.

\begin{Protocol}
Whenever the amount of send credits drops to zero, messages are buffered
without being sent.  On receipt of acknowledgement credit, the amount of send
credits are recalculated and buffered messages are sent based on the new
credit.  Acknowledgements are sent whenever a speicified threshhold is passed.
\end{Protocol}

\begin{Parameters}
\item pt2ptw\_window : the maximum amount on unacknowledged messages or the size of the
window.
\item pt2ptw\_ack\_thresh : The acknowledge threshold.  After receiving this
number of bytes of data from a sender, the receiver acknowledges previous
credit.
\end{Parameters}

\begin{Properties}
\item
This protocol bounds the number of unrecieved point-to-point messages a member
can send.
\item
This protocol requires reliable point-to-point properties from underlying
protocol layers.
\end{Properties}

\begin{Notes}
\item
As opposed to most of the \ensemble\ protocols, this protocol implements flow
control on bytes and not on messages.  It only considers the data in the
application payload portion of the message (the \mlval{iov} field of the
event).
\end{Notes}

\begin{Sources}
\sourcesfile{pt2ptw.ml}
\end{Sources}

\begin{Testing}
\item
Some testing has been carried out.
\end{Testing}

Last updated: March 21, 1997

\end{Layer}

'\begin{Layer}{PT2PTWP} 

This layer implements an adaptive window-based flow control protocol 
for point-to-point communication between the group members. 

In this protocol the receiver's buffer space is shared between all group
members. This is accomplished by dividing the receiver's window among 
the senders according to the bandwidth of the data being received from each 
sender. Such way of sharing attempts to minimize the number of ack messages, 
i.e.\ to increase message efficiency.

\begin{Protocol}
In the following, the term acknowledgement is used with the meaning 
of flow control protocols and not that of reliable communication protocols.

This protocol uses \emph{credits} to measure the available buffer space 
at the receiver's side. Each sender maintains a \emph{window} per each 
destination, which is used to bound the unacknowledged data a process 
can send point-to-point to the given destination. For each message it 
sends, the process deducts a certain amount of credit based on the size 
of the message. Messages are transmitted only if the sender has enough 
credit for them. Otherwise, messages are buffered at the sender.

A receiver keeps track of the amount of unacknowledged data it has 
received from each sender. Whenever it decides to acknowledge a sender,
it sends a message containing new amount of credit for this sender.
On receipt of an acknowledgement message, sender recalculates the amount 
of credit for this receiver, and the buffered messages are sent based 
on the new credit.

The receiver measures the bandwidth of the data being received from each 
sender. It starts with zero bandwidth, and adjusts it periodically with 
timeout \emph{pt2ptwp\_sweep}. 

On receipt of a point-to-point message, the receiver checks if the sender 
has passed threshold of its window, i.e. if the amount of data in
point-to-point messages received from this sender since the last ack was
sent to it has exceeded a certain ratio, \emph{pt2ptwp\_ack\_thresh}, 
of the sender's window. If it is, an ack with some credit has to be sent 
to the sender. In order to adjust processes' windows according to their 
bandwidth, the receiver attempts to steal some credit from an appropriate 
process and add it to the sender's window. The receiver looks for a process 
with maximal \( \frac{window}{\sqrt{bandwidth}} \) ratio, decreases its window 
by certain amount of credit and increases the window of the sender appropriately. 
Then the receiver sends the sender ack with the new amount of credit. When the 
process from which the credit was stolen passes theshold of its new, smaller 
window, the receiver sends ack to it.
\end{Protocol}

\begin{Parameters}
\item pt2ptwp\_window : size of the receiver's window, reflects the receiver's 
buffer space.
\item pt2ptwp\_ack\_thresh : the ratio used by the receiver while deciding 
to acknowledge senders.
\item pt2ptwp\_min\_credit : minimal amount of credit each process must have.
\item pt2ptwp\_bw\_thresh : credit may be stolen for processes with greater 
bandwidth only.
\item pt2ptwp\_sweep : the timeout of periodical adjustment of bavdwidth.
\end{Parameters}

\begin{Properties}
\item
This protocol requires reliable point-to-point properties from underlying 
protocol layers.
\end{Properties}

\begin{Notes}
\item
As opposed to most of the \ensemble\ protocols, this protocol implements flow
control on bytes and not on messages. It only considers the data in the
application payload portion of the message (the \mlval{iov} field of the
event).
\end{Notes}

\begin{Sources}
\sourcesfile{pt2ptwp.ml}
\end{Sources}

\begin{Testing}
\item
Correctness and performance testing has been carried out.
\end{Testing}

\end{Layer}

\begin{Layer}{REALKEYS}
\label{layer:realkeys}
This layer is part of the dWGL suite. Together with OptRekey is
implements the dWGL protocol. This layer's task is to actually
perform the instructions passed to it from OptRekey, generate and
pass securely all group subkeys, and finally the group key.

\begin{Protocol}
When a Rekey operation is performed a complex set of layers and
protocols is set into motion. Eventually, each group member receives a
new keygraph and a set of instructions describing how to merge its
partial keytree with the rest of the group keytrees to achieve
a unified group tree. The head of the keytree is the group key.

The instructions are implemented in several stages by the subleaders:
\begin{enumerate}
\item Choose new keys, and send them securely to peer subleaders
using secure channels.
\item Get new keys through secure channels. Disseminate these keys
by encrypting them with the top subtree key, and sending pt-2-pt to the leader.
\item When the leader gets all 2nd stage messages, it bundles them
into a single multicast and sends to the group. 
\item A member $p$ that receives the multicast, extracts the set of
keys it should know. Member $p$ creates an \mlval{ERekeyPrcl} event
with the new group key attached. The event it send down to PerfRekey
notifing it that the protocol is complete.
\end{enumerate}

\end{Protocol}

\begin{Properties}
\item Requires VSYNC properties.
\end{Properties}

\begin{Sources}
\sourcesfile{realkeys.ml}
\sourcesfile{type/tdefs.ml,mli}
\end{Sources}

\begin{GenEvent}
\genevent{\mlval{ESecureMsg}}
\genevent{\DnCast}
\genevent{\DnSend}
\end{GenEvent}

\begin{Testing}
\item 
The armadillo program (in the demo subdirectory) tests the security properties
of \ensemble.
\end{Testing}

\end{Layer}




\begin{Layer}{REKEY}

This layers switches the group key upon request. There may be several 
reasons for switching the key:
\begin{itemize}
\item The key's lifetime has expired --- it is now possible that some
dedicated attacker has cracked it.
\item The key has been compromised. 
\item Application authorization policies have changed and previously trusted
members need to be excluded from the group. 
\end{itemize}

This layer also relies on the Secchan layer to create secure
channels when required. A secure channel is essentially a way
to pass confidential information between two endpoints. The Secchan
layer creates secure channels upon demand and caches them for future
use. This allows the new group key to be disseminated efficiently and
confidentially through the tree. 

\begin{Protocol}
When a member layer gets an \ERekeyPrcl\ event, it sends a message to the
coordinator to start the rekeying process.  The coordinator generates
a new key and sends it to its children using secure channels. The
children pass it down the tree. Once a member receives the new key is
passes it down to PerfRekey using an \mlval{ERekeyPrcl} event.

The PerfRekey layer is responsible for collecting acknowledgments from the
members and performing a view change with the new key once
dissemination is complete. 
\end{Protocol}

\begin{Parameters}
\item {rekey\_degree:} The degree of the dissemination tree. By
default it is 2.
\end{Parameters}

\begin{Properties}
\item
Guarantees during a view change, either all members switch to the new
shared key or none of them do.
\end{Properties}

\begin{Sources}
\sourcesfile{rekey.ml}
\end{Sources}

\begin{GenEvent}
\genevent{\DnCast}
\genevent{\DnSend}
\end{GenEvent}

\begin{Testing}
\item 
The armadillo.ml file in the demo directory tests the security properties
of \ensemble.
\end{Testing}

\emph{This layer was originally written by Mark Hadyen with Zhen Xiao.
Ohad Rodeh later rewrote the security layers and related infrastructure.}
\end{Layer}



%*************************************************************%
%
%    Ensemble, 1.10
%    Copyright 2001 Cornell University, Hebrew University
%    All rights reserved.
%
%    See ensemble/doc/license.txt for further information.
%
%*************************************************************%
\begin{Layer}{REKEY\_DT}
\label{layer:rekey_dt}

This is the default rekeying layer. The basic 
data structure used is a tree of secure channels. This tree
changes every view-change, therefore the name of the layer. 
Dynamic Tree REKEY. 

The basic problem in obtaining efficient rekeying is the high cost of
constructing secure channels. A secure channel is established using a
two-way handshake using a Diffie-Hellman exchange.  At the time of
writing, a PentiumIII 500Mhz can perform one side of a Diffie-Hellman
exchange (using the OpenSSL cryptographic library) in 40 milliseconds.
This is a heavyweight operation. 

To discuss the set of channels in a group, we shall view it as a graph
where the nodes are group members, and the edges are secure channels
connecting them. The strategy employed by REKEY\_DT is to use a tree
graph. When a rekey request is made by a user, in some view $V$, the
leader multicasts a tree structure that uses, as much as possible, the
existing set of edges. 

For example, if the view is composed of several previous
components, then the leader attempts to merge together existing
key-trees. If a single member joins, then it is located as close to the
root as possible, for better tree-balancing. If a member leaves, then
the tree may, in the worst case, split into three pieces. The leader
fuses them together using (at most) 2 new secure channels. 

The leader chooses a new key and passes it to its
children. The key is passed recursively down the tree until it reaches
the leaves. The leaf nodes send acknowledgments back to the leader.

This protocol has very good performance. It is even possible, that a
rekey will not require any new secure-channels. For example, in case
of member leave, where the node was a tree-leaf. 

\begin{Protocol}
When a member layer gets an \ERekeyPrcl\ event, it sends a message to the
coordinator to start the rekeying process.  The coordinator checks if
the view is composed of a single tree-component. If not, it multicasts
a {\it Start} message. All members that are tree-roots, sends their
tree-structure to the leader. The leader merges the trees together,
and multicasts the group-tree. It then chooses a new key and sends it
down the tree. 

Once a member receives the new key is passes it down to PerfRekey
using an \mlval{ERekeyPrcl} event.

The PerfRekey layer is responsible for collecting acknowledgments from the
members and performing a view change with the new key once
dissemination is complete. 
\end{Protocol}

\begin{Sources}
\sourcesfile{rekey\_dt.ml}
\end{Sources}

\begin{GenEvent}
\genevent{\DnCast}
\genevent{\DnSend}
\end{GenEvent}

\begin{Testing}
\item 
The armadillo.ml file in the demo directory tests the security properties
of \ensemble.
\end{Testing}

\end{Layer}



\begin{Layer}{SECCHAN}
\label{layer:secchan}
This layer is responsible for sending and receiving private messages 
to/from group members. Privacy is guaranteed through the creation and
maintenance of {\it secure channels}. 

A secure channel is, essentially, a symmetric key (unrelated to the group key)
agreed upon between two members. This key is used to encrypt any
confidential message sent between them. We allow layers above 
Secchan to send/receive confidential information using
{\it SecureMsg} events. When a SecureMsg($dst,data$) event arrives at
Secchan, a secure channel to member $dst$ is created (if one does
not already exist). Then, the $data$ is encrypted using the secure channel key
and reliably sent to {\it dst}.

This layer relies on an authentication engine - this is provided in
system independent form by the Auth module. Currently, PGP is used for
authentication. New random shared keys are generated by the
Security module. The Security module also provides hashing and
symmetric encryption functions. Currently RC4 is used for encryption
and MD5 is used for hashing. 

\begin{Protocol}
A secure channel between members $p$ and $q$ is created using the
following basic protocol:
\begin{enumerate}
\item 
Member $p$ chooses a new random symmetric key $k_{pq}$.  It creates a
ticket to $q$ that includes $k_{pq}$ using the Auth module ticket
facility. Essentially, Auth encrypts $k_{pq}$ with $q$'s public key and
signs it using $p$'s private key. Member $p$ then sends the ticket to $q$.
\item 
Member $q$ authenticates and decrypts the message, and sends an
acknowledgment ({\it Ack}) back to $p$.
\end{enumerate}

This two-phase protocol is used to prevent the occurrence of a {\it
double channel}. By this we mean the case where $p$ and $q$ open
secure channels to each other at the same time. We augment the Ack phase;
$q$ discards $p$'s ticket if:
\begin{enumerate} 
\item $q$ has already started opening a channel to $p$
\item $q$ has a larger\footnote{Polymorphic comparison is used here.} name
than $p$.
\end{enumerate}

Secchan also keeps the number of open channels, per member, below the
{\it secchan\_cache\_size} configuration parameter. Regardless, a
channel is closed if it's lifetime exceeds 8 hours (the setable {\it
secchan\_ttl} parameter).  A two-phase protocol is used to close a
channel. If members $p$ and $q$ share channel, assuming $p$ created
it, then $p$ sends a {\it CloseChan} message to $q$.  Member $q$
responds by sending a {\it CloseChanOk} to $p$.

It typically happens that many secure channels are created
simultaneously group wide. For example, in the first Rekey of a
group. If we tear down all these channels exactly 8 hours from their
inception, the group will experience an explosion of management
information. To prevent this, we stagger channel tear down times.
Upon creation, a channel's maximal lifetime is set to $8 hours + I
seconds$ where $I$ is a random integer in the range [0 ..{\it
secchan\_rand}] . {\it secchan\_rand} is set by default to 200 seconds,
which we view as enough.

\end{Protocol}

\begin{Properties}
\item Requires VSYNC properties.
\end{Properties}

\begin{Parameters}
\item{secchan\_cache\_size:} determines size of secure channel cache. 
\item{secchan\_ttl:} Time To Live of a channel.
\item{secchan\_rand:} Used to stagger channel refresh times.
\item {secchan\_causal\_flag:} for performance evaluations.
\end{Parameters}

\begin{Sources}
\sourcesfile{secchan.ml}
\sourcesfile{msecchan.ml}
\end{Sources}

\begin{GenEvent}
\genevent{EChannelList}
\genevent{ESecureMsg}
\genevent{\DnCast}
\genevent{\DnSend}
\end{GenEvent}

\begin{Testing}
\item 
The armadillo program (in the demo subdirectory) tests the security properties
of \ensemble.
\end{Testing}

\end{Layer}




%*************************************************************%
%
%    Ensemble, 1_42
%    Copyright 2003 Cornell University, Hebrew University
%           IBM Israel Science and Technology
%    All rights reserved.
%
%    See ensemble/doc/license.txt for further information.
%
%*************************************************************%
\begin{Layer}{SEQUENCER}

This layer implements a sequencer based protocol for total ordering.

\begin{Protocol}
One member of the group serves as the sequencer. Any member that wishes to
send messages, send them point-to-point to the sequencer. The sequencer then
delivers the message localy, and cast it to the rest of the group. Other
members, as soon as they receive a cast from the sequencer, they deliver
the message.

If a view change occurs, messages are tagged as unordered and are send as
such.
When the \Up{View} event arrives, indicating that the group has successfully
been flushed, these messages are delivered in a deterministic order everywhere
(according to the ranks of their senders, breaking ties using FIFO).
\end{Protocol}

\begin{Parameters}
\item None
\end{Parameters}

\begin{Properties}
\item
Requires VSYNC properties.
\end{Properties}

\begin{Sources}
\sourcesfile{sequencer.ml}
\end{Sources}

\begin{GenEvent}
\genevent{\Dn{Cast}}
\genevent{\Dn{Send}}
\end{GenEvent}

\begin{Testing}
\item
\todo{}
\end{Testing}

\emph{This layer and its documentation were written by Roy Friedman.}
\end{Layer}

\begin{Layer}{SLANDER}

This protocol is used to share suspicions between members of a partition.  This
way, if one member suspects another member of being faulty, the coordinator is
informed so that the faulty member is removed, even if the coordinator does not
detect the failure.  This ensures that partitions will occur even in the case
of asymmetric network failures.  Without the protocol, only when the
coordinator notices the faulty member will the member be removed.

\begin{Protocol}
The protocol works by broadcasting slander messages to other members whenever
it recieves a new Suspect event.  On the receipt of such a message, DnSuspect
events are generated.
\end{Protocol}

\begin{Parameters}
\item None
\end{Parameters} 

\begin{Properties}
\item
If any member suspects another member of being faulty, all members will
eventually suspect that member.
\item
\Up{Suspect} events may cause a Slander message to be generated.
\end{Properties}

\begin{Sources}
\sourcesfile{slander.ml}
\end{Sources}

\begin{GenEvent}
\genevent{\Dn{Suspect}}
\end{GenEvent}

\begin{Testing}
\item
see also the VSYNC stack
\end{Testing}

\emph{This layer and its documentation were written by Zhen Xiao.}
\end{Layer}

\begin{Layer}{STABLE}

This layer tracks the stability of broadcast messages and does
failure detection.  It keeps track of and gossips about an
acknowledgement matrix, from which it occasionally computes the
number of messages from each member that are stable and delivers this
information in an \Dn{Stable} event to the layer below (which will be
bounced back up by a layer such as the BOTTOM layer).

\begin{Protocol}
The stability protocol consists of each member keeping track of its
view of an acknowledgement matrix.  In this matrix, each entry,
(A,B), corresponds to the number of member B's messages member A has
acknowledged (the diagonal entries, (A,A), contain the number of
broadcast messages sent by member A).  The minimum of column A
(disregarding entries for failed members) is the number of broadcast
messages from A that are stable.  The vector of these minimums is
called the stability vector.  The maximum of column A (disregarding
entries of failed members) is the number of broadcast messages member
A has sent that are held by at least one live member.  The vector of
the maximums is called the $NumCast$ vector \note{there has got to be
a better name}.  Occasionally, each member gossips its row to the
other members in the group.  Occasionally, the protocol layer
recomputes the stability and $NumCast$ vectors and delivers them up
in an \Dn{Stable} event.
\end{Protocol}

\begin{Parameters}
\item 
stable\_sweep: how often to (1) gossip and (2) deliver stability (if
it has changed)
\item 
stable\_explicit\_ack: whether to request end-to-end acknowledgements
for messages
\end{Parameters}

\begin{Properties}
\item
Unless it is marked with the \mlval{Unreliable} option all
DnCast events are counted by the STABLE layer and require eventual
acknowledgement by the other members in the group in order to achieve
stability.
\item
\Dn{Stable} events from the stability layer have two extension fields
set.  The first is the \mlval{StableVect} extension, which is the
vector of stability number of messages from each of the members in
the group which are known to be stable.  The second is the
\mlval{NumCast} extension which is a vector with the number of
broadcast messages each member in the group is known to have sent.
\item
\Dn{Stable} events are never delivered before all live members have
acknowledged at least the number of messages noted in the stability
event. (safety)
\item
\Dn{Stable} event will eventually be delivered after live members have
acknowledged message \mlval{seqno} from member A, where the entry in
the stable vector for member A is at least \mlval{seqno+1}. (liveness)
\item
The stability vectors in \Dn{Stable} events from the STABLE layer are
monotonically increasing.
\end{Properties}

\begin{Notes}
\item
\mlval{NumCast} entries are not monotonicly increasing.  For example,
consider the case of member A broadcasting some messages (which are
all dropped by the network), then broadcasting its gossip information
(which are recieved), then failing.  The other members may deliver
some UpStable events with the number of known broadcasts from member
A, in which the dropped broadcasts are counted.  However, after the
other members detect member A's failure, the \mlval{NumCast} entry
for member A will be lowered to be the number of messages from A that
the live members have recieved, which will be lower than when A was
not failed.
\item
\Up{Cast} events do not need to be acknowledged individually: an
acknowledgment, \mlval{Ack(from,seqno)}, is taken to acknowledge all
of the first \mlval{seqno} messages from the member with rank
\mlval{from}.
\item
An attempt has been made to speed up stability detection during view
changes by sending extra gossip messages when failures have occurred.
\end{Notes}

\begin{Sources}
\sourcesfile{stable.ml}
\end{Sources}

\begin{GenEvent}
\genevent{\Up{Stable}}
\genevent{\Dn{Cast}}
\genevent{\Dn{Timer}}
\end{GenEvent}

\begin{Testing}
\item see the VSYNC stack
\end{Testing}
\end{Layer}

%*************************************************************%
%
%    Ensemble, 1.10
%    Copyright 2001 Cornell University, Hebrew University
%    All rights reserved.
%
%    See ensemble/doc/license.txt for further information.
%
%*************************************************************%
\begin{Layer}{SUSPECT}

This layer regularly pings other members to check for suspected
failures.  Suspected failures are announce in a \DnSuspect\ event to
the layer below (which will be bounced back up by a layer such as the
BOTTOM layer).

\begin{Protocol}
Simple pinging protocol.  Uses a sweep interval.  On each sweep, Ping
messages are broadcast unreliably to the entire group.  Also, the
number of sweep rounds since the last Ping was received from other
members is checked and if it exceed the \mlval{max\_idle} threshold
then a \DnSuspect\ event is generated.  \hide{Suspicions are repeated
until a Ping message is received from the suspected member.}
\end{Protocol}

\begin{Parameters}
\item 
suspect\_sweep : how often to Ping other members and check for suspicions
\item
suspect\_max\_idle : number of unacknowledged Ping messages before generating
failure suspicions.
\end{Parameters}

\begin{Properties}
\item
Suspicions are no guarantee that an actual failure has occured, only a guess.
\end{Properties}

\begin{Notes}
\item None
\end{Notes}

\begin{Sources}
\sourcesfile{suspect.ml}
\end{Sources}

\begin{GenEvent}
\genevent{\DnSuspect}
\genevent{\DnCast}
\genevent{\DnTimer}
\end{GenEvent}

\begin{Testing}
\item see the VSYNC stack
\end{Testing}
\end{Layer}

\begin{Layer}{SYNC}

This layer implements a protocol for blocking a group during view changes.  One
member initiates the SYNC protocol by delivering a \DnBlock\ event from above.
Other members will receive an \UpBlock\ event.  After replying with a
\DnBlockOk, the layers above the SYNC layer should not broadcast any further
messages.  Eventually, after all members have responded to the \UpBlock\ and
all broadcast messages are stable, the member that delivered the \DnBlock\
event will recieve an \UpBlockOk\ event.

\begin{Protocol}
This protocol is very inefficient and needs to be reimplemented at some
point.  The Block request is broadcast by the coordinator.  All members
respond with another broadcast.  When the coordinator gets all replies, it
delivers up an \UpBlockOk.
\end{Protocol}

\begin{Parameters}
\item None
\end{Parameters}

\begin{Properties}
\item
Requires FIFO, reliable broadcasts with stability detection.
\item
Expects at most one \DnBlock\ from above.
\item
Always delivers at most one \UpBlockOk\ event.  Only delivers an \UpBlockOk\ if
a \DnBlock\ was recieved from above.
\item
When at least one member recieves a \DnBlock\ event, all live members will
eventually deliver an \UpBlock\ event.
\item
Expects at most one \DnBlockOk\ event from above.  Expects a \DnBlockOk\ from
above only if an \UpBlock\ event was previously delivered by this layer.
\item
Expects a \DnBlock\ to the layers below will be replied with an \UpBlock\ from
below.
\item
When all members have delivered a \DnBlockOk\ event from above and all
broadcast messages have been acknowledged (by non-failed members),
eventually all members who delivered a \DnBlock\ event will receive an
\UpBlockOk\ event from this layer.
\end{Properties}

\begin{Sources}
\sourcesfile{sync.ml}
\end{Sources}

\begin{GenEvent}
\genevent{\UpBlockOk}
\genevent{\DnAck}
\genevent{\DnBlock}
\genevent{DnCast}
\end{GenEvent}

\begin{Testing}
\item
The CHK\_SYNC protocol layer checks for SYNC safety conditions.
\item 
see also the VSYNC stack
\end{Testing}
\end{Layer}

%*************************************************************%
%
%    Ensemble, 1_42
%    Copyright 2003 Cornell University, Hebrew University
%           IBM Israel Science and Technology
%    All rights reserved.
%
%    See ensemble/doc/license.txt for further information.
%
%*************************************************************%
\begin{Layer}{TOPS}

This layer implements a lexicographic total ordering protocol.  (This is
a variation on the protocol developed as part of the Transis project.)

\begin{Protocol}
The protocol works by lexigraphically ordering messages. For example,
if group members are $\{ A, B, C \}$ and they send messages $A_1,
B_1,$ and $C_1,$ then the ordering will be $A_1 < B_1 < C_1$. Since
the ordering is fixed the protocol can get stuck if a member does not
send a message every timeout. For example, if the application messages
are $A_1$ and $C_1$ then the protocol would wait indefinitely for
$B_1$ and never deliver $C_1$. Hence, every member multicasts a null
message every timeout to maintain liveness. Currently, the timeout is
hardcoded to one second.

This protocol is not normally used because it has high latency 
if members do not multicast messages often. 
\end{Protocol}

\begin{Parameters}
\item None
\end{Parameters}

\begin{Properties}
\item
Requires VSYNC properties, implements the AGREE property. 
\end{Properties}

\begin{Sources}
\sourcesfile{tops.ml}
\end{Sources}

\end{Layer}
%*************************************************************%
%
%    Ensemble, 1_42
%    Copyright 2003 Cornell University, Hebrew University
%           IBM Israel Science and Technology
%    All rights reserved.
%
%    See ensemble/doc/license.txt for further information.
%
%*************************************************************%
\begin{Layer}{TOTEM}

This layer implements the rotating token protocol for total ordering.  (This is
a variation on the protocol developed as part of the Totem project.)

\begin{Protocol}
The protocol here is fairly simple: As soon as the stack becomes valid, the
lowest ranked member starts rotating a token in the group. In order to send a
message, a process must wait for the token. When the token arrives, all
buffered messages are broadcast, and the token is passed to the next member.
The token must be passed on even if there are no buffered messages.

If a view change occurs, messages are tagged as unordered and are send as
such.
When the \Up{View} event arrives, indicating that the group has successfully
been flushed, these messages are delivered in a deterministic order everywhere
(according to the ranks of their senders, breaking ties using FIFO).
\end{Protocol}

\begin{Parameters}
\item None
\end{Parameters}

\begin{Properties}
\item
Requires VSYNC properties and local delivery.
\end{Properties}

\begin{Sources}
\sourcesfile{totem.ml}
\end{Sources}

\begin{GenEvent}
\genevent{\Dn{Cast}}
\end{GenEvent}

\begin{Testing}
\item
\todo{}
\end{Testing}

\emph{This layer and its documentation were written by Roy Friedman.}
\end{Layer}

%*************************************************************%
%
%    Ensemble, 2_00
%    Copyright 2004 Cornell University, Hebrew University
%           IBM Israel Science and Technology
%    All rights reserved.
%
%    See ensemble/doc/license.txt for further information.
%
%*************************************************************%
\begin{Layer}{WINDOW} 

This layer implements window-based flow control based on stability information.
Multicast messages from each sender are sent only if the number of
unacknowledged messages from the sender is smaller than the window.

\begin{Protocol}
Whenever the number of unstable messages goes above the window, messages
are buffered without being sent.  On receipt of a stability update, the
number of unstable messages are recalculated and buffered messages are sent
as allowed by the window.
\end{Protocol}

\begin{Parameters}
\item window\_window : the window size in number of messages
\end{Parameters}

\begin{Properties}
\item
Requires stability information in the form of \Up{Stable} events.
\end{Properties}

\begin{Notes}
\item 
Future implementation should support dynamic window adjustment.
\item 
Performance with the WINDOW layer depends in part with the frequency of
stability updates.  The WINDOW flow control works the best when the
frequency is based on the number of unstable messages rather than on
periodic timeouts.
\item 
Alternative flow control layers include RATE and CREDIT.
\end{Notes}

\begin{Sources}
\sourcesfile{window.ml}
\end{Sources}

\emph{This layer and its documentation were written by Takako Hickey.}
\end{Layer}

\begin{Layer}{XFER}

This protocol facilitates application based state-transfer. 
The view structure contains a boolean field {\tt xfer\_view}
conveying whether the current view is one where
state-transfer is taking place ({\tt xfer\_view = true}) or whether it 
is a regular view ({\tt xfer\_view = false}).

\begin{Protocol}
It is assumed that an application initiates state-transfer after a view
change occurs. In the initial view, {\tt xfer\_view = true}. 
In a fault free run, 
each application sends pt-2-pt and multicast messages, according
to its state-transfer protocol. Once the application-protocol is
complete, an {\tt XferDone} action is sent to Ensemble. 
This action is caught by the Xfer layer, where each member sends a pt-2-pt
message {\tt XferMsg} to the leader. When the leader
collects {\tt XferMsg} from all members, the state-transfer is
complete, and a new view is installed with the {\tt xfer\_view} field
set to false. 

When faults occur, and members fail during the state-transfer
protocol, new views are installed with {\tt xfer\_view} set to {\tt
true}. This informs applications that state-transfer was not
completed, and they can restart the protocol. 
\end{Protocol}

\begin{Notes}
\item 
This layer allows the application to choose
the state-transfer protocol it wishes to use, the only constrain being
the XferDone actions. 

\item 
In the normal case, (a fault free run) the protocol should take a 
single view to complete. 
\end{Notes}

\begin{Parameters}
\item None
\end{Parameters}

\begin{Properties}
\item
Requires VSYNC properties.
\end{Properties}

\begin{Sources}
\sourcesfile{xfer.ml}
\end{Sources}

\end{Layer}


\begin{Layer}{ZBCAST}

The ZBCAST layer implements a gossip-style probabilistically reliable
multicast protocol.  Unlike most other protocols in \ensemble, this
protocol admits a small, but non-zero probability of message loss: a
message might be garbage collected even though some operational member
in the group has not received it yet.  We found that doing so can
offer dramatic improvements in the performance and scalability of the
protocol.


\begin{Protocol}
This protocol is composed of two sub-protocols structured roughly as in the
Internet MUSE protocol.  The first protocol is an unreliable multicast
protocol which makes a best-effort attempt to efficiently deliver each
message to its destinations.  The second protocol is a 2-phase
anti-entropy protocol that operates in a series of unsynchronized
rounds.  During each round, the first phase detects message loss; the
second phase corrects such losses and runs only if needed.
\end{Protocol}

\begin{Parameters}
\item zbcast\_fanout : the fanout of gossip messages.  This determines
how many destinations a member gossips to during each round.

\item zbcast\_sweep: the interval of each round.

\item zbcast\_idle: how many rounds to wait after the last
retransmission request of a message before that message can be garbage
collected.

\item zbcast\_max\_polls: the maximum number of destinations a member
can poll for missing messages during one round.

\item zbcast\_max\_reqs: the maximum number of retransmission requests
a member can make during one round.

\item zbcast\_max\_entropy: the maximum amount of data a member can
retransmit during one round.

\item zbcast\_req\_limit: the threshold for message retransmission
request before that request is multicasted to the whole group.

\item zbcast\_reply\_limit: the threshold for message retransmission
before that retransmission is multicasted to the whole group.
\end{Parameters}

\begin{Properties}
\item
Under some conservative assumptions about the network properties,
message delivery can be proved to have a bimodal distribution under
this protocol: with a very small probability the message will be
delivered to a small number of destinations(including failed ones);
with very high probability the message will be delivered to almost all
destinations; and with vanishingly low probability the message will be
delivered to \emph{many} but not \emph{most} destinations.

\item
Using this protocol for multicast transmissions, virtual synchrony
cannot be guaranteed since it admits a non-zero probability of message
losses at some operational members.  Message losses (if any) are
reported to the application.  If the message loss is deemed to
compromise correct behavior, the application may decide to leave the
group and then rejoins them, triggering state transfer -- a separate
feature provided by \ensemble.

\item
This protocol needs multicast support from underlying layers.  If
IP-multicast is not available, GCAST protocol is needed to simulate
the effect of multicast by a series of unicasts.

\item
As its current implementation, this protocol requires \emph{groupd}
membership services.

\item
This protocol assumes that the application is able to control its
message transmission within a certain rate (rate-based flow control).
If the load injected into the network is heavier than what it can
sustain, the failure probability and latency guarantees of the
protocol may no longer hold.
\end{Properties}

\begin{Sources}
\sourcesfile{zbcast.ml}
\end{Sources}

\begin{GenEvent}
\genevent{\Dn{Cast}}
\genevent{\Dn{Send}}
\genevent{\Up{LostMessage}}
\end{GenEvent}

\begin{Testing}
\item
Extensive experiments have been conducted on a SP2 parallel machine
(used as a network of UNIX workstations) with group size ranging from
$8$ to $128$ nodes.  The protocol scales gracefully and maintains
stable throughput.
\item
The protocol has been tested on a multicast-capable LAN with 30
Solaris workstations.  One of the member is the sender and the rest
are receivers.  The sender is sending at a rate of 200 messages per
second.  Each message is $1000$ bytes.  Most of the receivers are able
to maintain a steady throughput of $200$ msgs/sec.  Message losses are
very rare.

We emphasize that in both tests we have reached the limit of the
largest group of machines which we have access to.  We believe that
our protocol can scale far more than what is indicated above.
\item
In the next step of our work, we will investigate its performance on WAN.
\end{Testing}

\emph{This layer and its documentation were written by Zhen Xiao.  It
is based on the \emph{PBCAST} protocol implemented by Mark Hayden.
This documentation is based the \emph{Bimodal Multicast} paper.}
\end{Layer}

%*************************************************************%
%
%    Ensemble, 1_42
%    Copyright 2003 Cornell University, Hebrew University
%           IBM Israel Science and Technology
%    All rights reserved.
%
%    See ensemble/doc/license.txt for further information.
%
%*************************************************************%
\begin{Stack}{VSYNC}

Virtual synchrony is decomposed into a set of 8 independent protocol
layers, listed in figure~\ref{vsynclayers}.  The layers in this stack are
decribed in the layer section.

\begin{table}[b]
\begin{center}
\begin{tabular}{|l|l|l|}			   \hline
name		& purpose			\\ \hline
LEAVE		& reliable group leave		\\ \hline
INTER		& inter-group view management	\\ \hline
INTRA		& intra-group view management	\\ \hline
ELECT		& leader election		\\ \hline
MERGE		& reliable group merge		\\ \hline
SYNC		& view change synchronization	\\ \hline
PT2PT		& FIFO, reliable pt2pt		\\ \hline
SUSPECT		& failure suspcions		\\ \hline
STABLE		& broadcast stability		\\ \hline
MNAK		& FIFO, agreed broadcast	\\ \hline
BOTTOM		& bare-bones communication	\\ \hline
\end{tabular}
\end{center}
\caption{Virtual synchrony protocol stack}
\label{vsynclayers}
\end{table}

\todo{here describe the overall protocol created by composing all
the protocol layers}

\begin{Parameters}
\item \todo{composition of parameters below}
\end{Parameters}

\begin{Protocol}
\todo{composition of protocols below}
\end{Protocol}

\begin{Properties}
\item
\note{some form of composition of properties in layers}
\end{Properties}

\begin{Notes}
\item
Causal ordering can be introduced by replacing the MNAK layer with a
causal implementation of same protocol.
\item
Weak virtual synchrony can be implemented by removing the SYNC layer
and adding application support for managing multiple live protocol
stacks.
\end{Notes}

\begin{Testing}
\item
Use the various testing code described in the component layers.
\item
Version of Jan 12, 1996, tested with $>100000$ random failure scenarios.
\item
Version of April 10, 1997, tested with $>100000$ random failure scenarios.
\item
Random testing is done nightly on debugged VSYNC protocol stack.
\end{Testing}
\end{Stack}

