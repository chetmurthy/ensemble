\section{Quick Installation} 

Several demonstration applications are included with \ensemble.
These can give a sense of the kinds of facilities provided by group
communication to those who have not used a group communication
toolkit before.  The demos can also serve as starting points for
building new applications.  These applications are briefly described
here along with how to execute them and the various command-line
options and environment variables they use.

\subsection{Compiling}
Please see the file \sourcecode{ensemble/INSTALL.htm} for instructions on
installing \ensemble\ if you have not done so already.

\subsection{Configuration Variables}
Detailed information is given in Section~\ref{subsect:config} for
initializing environment variables. We assume that you will be using
IP-multicast as a communication substrate this means no configuration
is neccessary. However, if multicast is not supported by your system you'll
be using the gossip server for processes to locate each
other, see Section~\ref{subsect:config} for more information.

Throughout this tutorial, we assume you are using the Unix csh or
tcsh shell. To set an environment variable in the bash shell you would
do the following:
\begin{verbatim}
% export ENS_CONFIG_FILE=/etc/ensemble.conf
\end{verbatim}
If you're using a win32 system you will need to use the native
environment-setting tool
(start $\rightarrow$ setting $\rightarrow$ control-panel $\rightarrow$ system $\rightarrow$ advanced $\rightarrow$ environment-variables)
 which provides similar functionality. 

\subsection{Executing Applications}
On the same or other hosts execute several instances of an
application, such as \sourcedemo{mtalk}:
\begin{verbatim}
% mtalk
 ...
\end{verbatim}

Applications should merge together and form a group. 

Mtalk is a {\it server-based} program, this means that it is really on
the server-side, not a client application. Mtalk is useful for testing
if servers on different machines can talk to each other and merge to
form groups. We discourage users from writing server-side programs, it
is easier and safer to write client-side programs. 



