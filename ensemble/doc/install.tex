\section{Quick Installation} 

Several demonstration applications are included with \ensemble.
These can give a sense of the kinds of facilities provided by group
communication to those who have not used a group communication
toolkit before.  The demos can also serve as starting points for
building new applications.  These applications are briefly described
here along with how to execute them and the various command-line
options and environment variables they use.

\subsection{Compiling}
Please see the file \sourcecode{ensemble/INSTALL} for instructions on
installing \ensemble\ if you have not done so already.

\subsection{Environment Variables}
Detailed information is given in Section~\ref{subsect:config} for
initializing environment variables.  Here we give the bare minimum
you need in order to get started.  We assume that you will (at least
initially) be using the gossip server for processes to locate each
other.  \mlval{ENS\_GOSSIP\_PORT} must be set to a port number that
is not used by other applications.  Normally, user applications
cannot use port numbers below 1000.  \mlval{ENS\_GOSSIP\_HOSTS} must
be set to a list of colon-separated host names where the gossip
server may be found.  If you wish to use port 7500 for the gossip
server on hosts ``ely'' and ``natasha,'' you would set these
environment variables as follows (in Unix csh):
\begin{verbatim}
% setenv ENS_GOSSIP_PORT 7500
% setenv ENS_GOSSIP_HOSTS ely:natasha
\end{verbatim}

Throughout this tutorial, we assume you are using the Unix csh or
tcsh shell.  To set an environment variable in the bash shell, you
would do the following:
\begin{verbatim}
% export ENS_GOSSIP_PORT=7500
% export ENS_GOSSIP_HOSTS=ely:natasha
\end{verbatim}

Remember that these must be set for all Ensemble programs you run.
You may wish to add the configuration to your \sourcecode{.cshrc} or
equivalent.

\subsection{Executing Applications}
Applications by default require a gossip server process to be running
in order to contact each other.  Before executing an application, a
gossip server must be started one of the hosts listed in
\mlval{ENS\_GOSSIP\_HOSTS}:
\begin{verbatim}
% gossip
 ...
\end{verbatim}
On the same or other hosts execute several instances of an
application, such as \sourcedemo{mtalk}:
\begin{verbatim}
% mtalk
 ...
\end{verbatim}
