\begin{Layer}{XFER}

This protocol facilitates application based state-transfer. 
The view structure contains a boolean field {\tt xfer\_view}
conveying whether the current view is one where
state-transfer is taking place ({\tt xfer\_view = true}) or whether it 
is a regular view ({\tt xfer\_view = false}).

\begin{Protocol}
It is assumed that an application initiates state-transfer after a view
change occurs. In the initial view, {\tt xfer\_view = true}. 
In a fault free run, 
each application sends pt-2-pt and multicast messages, according
to its state-transfer protocol. Once the application-protocol is
complete, an {\tt XferDone} action is sent to Ensemble. 
This action is caught by the Xfer layer, where each member sends a pt-2-pt
message {\tt XferMsg} to the leader. When the leader
collects {\tt XferMsg} from all members, the state-transfer is
complete, and a new view is installed with the {\tt xfer\_view} field
set to false. 

When faults occur, and members fail during the state-transfer
protocol, new views are installed with {\tt xfer\_view} set to {\tt
true}. This informs applications that state-transfer was not
completed, and they can restart the protocol. 
\end{Protocol}

\begin{Notes}
\item 
This layer allows the application to choose
the state-transfer protocol it wishes to use, the only constrain being
the XferDone actions. 

\item 
In the normal case, (a fault free run) the protocol should take a 
single view to complete. 
\end{Notes}

\begin{Parameters}
\item None
\end{Parameters}

\begin{Properties}
\item
Requires VSYNC properties.
\end{Properties}

\begin{Sources}
\sourcesfile{xfer.ml}
\end{Sources}

\end{Layer}

