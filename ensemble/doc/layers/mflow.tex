%*************************************************************%
%
%    Ensemble, 1.10
%    Copyright 2001 Cornell University, Hebrew University
%    All rights reserved.
%
%    See ensemble/doc/license.txt for further information.
%
%*************************************************************%
\begin{Layer}{MFLOW} 

This layer implements window-based flow control for multicast messages.
Multicast messages from each sender are transmitted only if the number of send
credit left is greater than zero.  The protocol attempts to avoid situations
where all recievers send credit at the same time, so that a sender is not
flooded with credit messages.

\begin{Protocol}
Whenever the amount of send credits drops to zero, messages are buffered without
being sent.  On receipt of acknowledgement credit, the amount of send credits
are recalculated and buffered messages are sent based on the new credit.
\end{Protocol}

\begin{Parameters}
\item mflow\_window : the maximum amount on unacknowledged messages or the size of the
window.
\item mflow\_ack\_thresh : The acknowledge threshold.  After receiving this number of
bytes of data from a sender, the receiver acknowledged previous credit.
\end{Parameters}

\begin{Properties}
\item
This protocol bounds the number of unrecieved multicast messages a member has
sent.
\item
The amount of received credits are initialized to different values for
avoiding many members sending back acknowledge at the same time. 
\item
This protocol requires reliable multicast and point-to-point properties from
underlying protocol layers.
\end{Properties}

\begin{Notes}
\item
As opposed to most of the \ensemble\ protoocols, this protocol implements flow
control on bytes and not on messages.  It only considers the data in the
application payload portion of the message (the \mlval{iov} field of the
event).
\item
Because of the EBlockOk events, this layer needs to be below the
broadcast stability layer.
\end{Notes}

\begin{Sources}
\sourcesfile{mflow.ml}
\end{Sources}

\begin{Testing}
\item
Some testing has been carried out.
\end{Testing}

\emph{This layer and its documentation were written with Zhen Xiao.}
\end{Layer}
