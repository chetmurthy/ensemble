%*************************************************************%
%
%    Ensemble, 1.10
%    Copyright 2001 Cornell University, Hebrew University
%    All rights reserved.
%
%    See ensemble/doc/license.txt for further information.
%
%*************************************************************%
\begin{Layer}{SECCHAN}
\label{layer:secchan}
This layer is responsible for sending and receiving private messages 
to/from group members. Privacy is guaranteed through the creation and
maintenance of {\it secure channels}. 

A secure channel is, essentially, a symmetric key (unrelated to the group key)
agreed upon between two members. This key is used to encrypt any
confidential message sent between them. We allow layers above 
Secchan to send/receive confidential information using
{\it SecureMsg} events. When a SecureMsg($dst,data$) event arrives at
Secchan, a secure channel to member $dst$ is created (if one does
not already exist). Then, the $data$ is encrypted using the secure channel key
and reliably sent to {\it dst}.

This layer relies on an authentication engine - this is provided in
system independent form by the Auth module. Currently, PGP is used for
authentication. New random shared keys are generated by the
Security module. The Security module also provides hashing and
symmetric encryption functions. Currently RC4 is used for encryption
and MD5 is used for hashing. 

\begin{Protocol}
A secure channel between members $p$ and $q$ is created using the
following basic protocol:
\begin{enumerate}
\item 
Member $p$ chooses a new random symmetric key $k_{pq}$.  It creates a
ticket to $q$ that includes $k_{pq}$ using the Auth module ticket
facility. Essentially, Auth encrypts $k_{pq}$ with $q$'s public key and
signs it using $p$'s private key. Member $p$ then sends the ticket to $q$.
\item 
Member $q$ authenticates and decrypts the message, and sends an
acknowledgment ({\it Ack}) back to $p$.
\end{enumerate}

This two-phase protocol is used to prevent the occurrence of a {\it
double channel}. By this we mean the case where $p$ and $q$ open
secure channels to each other at the same time. We augment the Ack phase;
$q$ discards $p$'s ticket if:
\begin{enumerate} 
\item $q$ has already started opening a channel to $p$
\item $q$ has a larger\footnote{Polymorphic comparison is used here.} name
than $p$.
\end{enumerate}

Secchan also keeps the number of open channels, per member, below the
{\it secchan\_cache\_size} configuration parameter. Regardless, a
channel is closed if it's lifetime exceeds 8 hours (the setable {\it
secchan\_ttl} parameter).  A two-phase protocol is used to close a
channel. If members $p$ and $q$ share channel, assuming $p$ created
it, then $p$ sends a {\it CloseChan} message to $q$.  Member $q$
responds by sending a {\it CloseChanOk} to $p$.

It typically happens that many secure channels are created
simultaneously group wide. For example, in the first Rekey of a
group. If we tear down all these channels exactly 8 hours from their
inception, the group will experience an explosion of management
information. To prevent this, we stagger channel tear down times.
Upon creation, a channel's maximal lifetime is set to $8 hours + I
seconds$ where $I$ is a random integer in the range [0 ..{\it
secchan\_rand}] . {\it secchan\_rand} is set by default to 200 seconds,
which we view as enough.

\end{Protocol}

\begin{Properties}
\item Requires VSYNC properties.
\end{Properties}

\begin{Parameters}
\item{secchan\_cache\_size:} determines size of secure channel cache. 
\item{secchan\_ttl:} Time To Live of a channel.
\item{secchan\_rand:} Used to stagger channel refresh times.
\item {secchan\_causal\_flag:} for performance evaluations.
\end{Parameters}

\begin{Sources}
\sourcesfile{secchan.ml}
\sourcesfile{msecchan.ml}
\end{Sources}

\begin{GenEvent}
\genevent{EChannelList}
\genevent{ESecureMsg}
\genevent{\DnCast}
\genevent{\DnSend}
\end{GenEvent}

\begin{Testing}
\item 
The armadillo program (in the demo subdirectory) tests the security properties
of \ensemble.
\end{Testing}

\end{Layer}



