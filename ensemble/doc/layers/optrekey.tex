\begin{Layer}{OPTREKEY}
\label{layer:optrekey}
This layer is part of the dWGL suite. Together with RealKeys, it
implements the dWGL rekeying algorithm. The specific task
performed by OptRekey is computing the new group keygraph.

\begin{figure}[bht]
\begin{center}
\begin{tabular}{cc}
  \putfigfbox{0.40}{./fig/dWGL/keygraph.eps} \\
  (a) \\
  \putfigfbox{0.40}{./fig/dWGL/keygraph-sep.eps} & \putfigfbox{0.40}{./fig/dWGL/keygraph-leave.eps} \\
  (b) & (c) 
\end{tabular}
\caption{The effect of leave on a group key-graph of a group G of
eight members. 
(1) The initial keygraph. 
(2) The tree after member $p_1$ leaves. 
(3) The merged tree.
}
\label{fig:keygraph-evol}
\end{center}
\end{figure}


Briefly, a keygraph is a graph where all group members form the
leaves, and the inner nodes are shared sub-keys. A member knows all
the keys on the route from itself to the root. The top key is the
group key, and it is known by all members. For example,
Figure~\ref{fig:keygraph-evol}(a) depicts a group $G$ of eight members
$\{p_1 \dots p_8\}$ and their subkeys. When a member leaves the group,
all the keys known to it must be discarded. This splits a group into a
set of subtrees. Figure~\ref{fig:keygraph-evol}(b) shows $G$ after
member $p_1$ has left. In order to re-merge the group keygraph, the
subtrees should be merged. This can be seen in Figure~\ref{fig:keygraph-evol}(c).
A subleader is the leader of a subtree. In our example, member $p_2$
is the leader of $\{p_2\}$, $p_3$ is the leader of
$\{p_3,p_4\}$, and $p_5$ is the leader of $\{p_5,p_6,p_7,p_8\}$.

\begin{Protocol}
This layer is activated upon a Rekey action. The leader receives
an \mlval{ERekeyPrcl} event, and starts the OptRekey protocol. 
Typically, a Rekey will follow a join or a leave. Hence, the group keygraph
is initially fragmented. This layer's task is to remerge it. The
protocol employed is as follows:

\begin{enumerate}
\item The leader multicasts {\it Start}.
\item Subleaders send their keygraphs to the leader.
\item The leader computes an optimal new keygraph. 
\item The leader multicasts the new keygraph.
\item Members receive the keygraph and send it up using a \mlval{ERekeyPrcl}
event to the RealKeys layer.
\end{enumerate}

An optimal keygraph is complex to compute, an auxiliary module is used
for this task. Note that OptRekey is designed so that only
subleaders participate. In the normal case, where a single member
joins or leaves, this will include $log_2n$ members.

It is possible that a Rekey will be initiated even though
membership hasn't changed. This case is specially handled, since it can
be executed with nearly no communication.
\end{Protocol}

\begin{Properties}
\item Requires VSYNC properties.
\end{Properties}

\begin{Sources}
\sourcesfile{optrekey.ml}
\sourcesfile{util/tree.ml,mli}
\sourcesfile{type/tdefs.ml,mli}
\end{Sources}

\begin{GenEvent}
\genevent{\DnCast}
\genevent{\DnSend}
\end{GenEvent}

\begin{Testing}
\item 
The armadillo program (in the demo subdirectory) tests the security properties
of \ensemble.
\end{Testing}

\end{Layer}



