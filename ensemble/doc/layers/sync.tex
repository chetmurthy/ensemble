\begin{Layer}{SYNC}

This layer implements a protocol for blocking a group during view changes.  One
member initiates the SYNC protocol by delivering a \DnBlock\ event from above.
Other members will receive an \UpBlock\ event.  After replying with a
\DnBlockOk, the layers above the SYNC layer should not broadcast any further
messages.  Eventually, after all members have responded to the \UpBlock\ and
all broadcast messages are stable, the member that delivered the \DnBlock\
event will recieve an \UpBlockOk\ event.

\begin{Protocol}
This protocol is very inefficient and needs to be reimplemented at some
point.  The Block request is broadcast by the coordinator.  All members
respond with another broadcast.  When the coordinator gets all replies, it
delivers up an \UpBlockOk.
\end{Protocol}

\begin{Parameters}
\item None
\end{Parameters}

\begin{Properties}
\item
Requires FIFO, reliable broadcasts with stability detection.
\item
Expects at most one \DnBlock\ from above.
\item
Always delivers at most one \UpBlockOk\ event.  Only delivers an \UpBlockOk\ if
a \DnBlock\ was recieved from above.
\item
When at least one member recieves a \DnBlock\ event, all live members will
eventually deliver an \UpBlock\ event.
\item
Expects at most one \DnBlockOk\ event from above.  Expects a \DnBlockOk\ from
above only if an \UpBlock\ event was previously delivered by this layer.
\item
Expects a \DnBlock\ to the layers below will be replied with an \UpBlock\ from
below.
\item
When all members have delivered a \DnBlockOk\ event from above and all
broadcast messages have been acknowledged (by non-failed members),
eventually all members who delivered a \DnBlock\ event will receive an
\UpBlockOk\ event from this layer.
\end{Properties}

\begin{Sources}
\sourcesfile{sync.ml}
\end{Sources}

\begin{GenEvent}
\genevent{\UpBlockOk}
\genevent{\DnAck}
\genevent{\DnBlock}
\genevent{DnCast}
\end{GenEvent}

\begin{Testing}
\item
The CHK\_SYNC protocol layer checks for SYNC safety conditions.
\item 
see also the VSYNC stack
\end{Testing}
\end{Layer}
