\begin{Layer}{CREDIT}

This layer implements a credit based flow control.

\begin{Protocol}
On initialization, sender informs receivers how many credits it wants to
keep in stock.  Receivers sends credits whenever it finds that the sender
is low on credits, either explicitly through a sender's request or
implicitly through its local accounting.  A credit is one time use only.
Sender is allowed to send a message only if it has a credit available.  If
the sender does not have a credit, the message is buffered.  Buffered
messages are sent when new credits arrive.  Credits are piggybacked to data
messages whenever there is an opportunity of doing so to save bandwidth.
\end{Protocol}

\begin{Parameters}
\item 
rtotal: the total number of credits that this member can give out.  Should
be set according to the number of receive buffers that the machine the
member is running has.
\item 
ntoask: the number of credits that this member likes to keep in stock.
\item 
whentoask: the threshold number of credits remaining at sender before the
receiver consider sending out more credits.
\item 
pntoask: like ntoask for piggyback style of credit giving.
\item 
pwhentoask: like nwhentoask for piggyback style of credit giving.
\item 
sweep: frequency at which periodic sweep routine, which give out credits to
senders, should run.
\end{Parameters}

%\begin{Properties}
%\end{Properties}

\begin{Notes}
\item Future implementation should support dynamic credit adjustment.
\item Alternative flow control layers include RATE and WINDOW.
\end{Notes}

\begin{Sources}
\sourcesfile{credit.ml}
\end{Sources}

Last updated: Fri Mar 29, 1996

\end{Layer}
