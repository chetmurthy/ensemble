%*************************************************************%
%
%    Ensemble, 1.10
%    Copyright 2001 Cornell University, Hebrew University
%    All rights reserved.
%
%    See ensemble/doc/license.txt for further information.
%
%*************************************************************%
\begin{Layer}{PT2PTW} 

This layer implements window-based flow control for point to point messages.
Point-to-point messages from each sender are transmitted only if the window is
not yet full.

\begin{Protocol}
Whenever the amount of send credits drops to zero, messages are buffered
without being sent.  On receipt of acknowledgement credit, the amount of send
credits are recalculated and buffered messages are sent based on the new
credit.  Acknowledgements are sent whenever a speicified threshhold is passed.
\end{Protocol}

\begin{Parameters}
\item pt2ptw\_window : the maximum amount on unacknowledged messages or the size of the
window.
\item pt2ptw\_ack\_thresh : The acknowledge threshold.  After receiving this
number of bytes of data from a sender, the receiver acknowledges previous
credit.
\end{Parameters}

\begin{Properties}
\item
This protocol bounds the number of unrecieved point-to-point messages a member
can send.
\item
This protocol requires reliable point-to-point properties from underlying
protocol layers.
\end{Properties}

\begin{Notes}
\item
As opposed to most of the \ensemble\ protocols, this protocol implements flow
control on bytes and not on messages.  It only considers the data in the
application payload portion of the message (the \mlval{iov} field of the
event).
\end{Notes}

\begin{Sources}
\sourcesfile{pt2ptw.ml}
\end{Sources}

\begin{Testing}
\item
Some testing has been carried out.
\end{Testing}

Last updated: March 21, 1997

\end{Layer}
