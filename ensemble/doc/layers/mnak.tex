\begin{Layer}{MNAK}

The MNAK (Multicast NAK) layer implements a reliable, agreed, FIFO-ordered
broadcast protocol.  Broadcast messages from each sender are delivered in
FIFO-order at their destinations.  Messages from live members are delivered
reliably and messages from failed members are retransmitted by the coordinator
of the group.  When all failed members are marked as such, the protocol
guarantees that eventually all live members will have delivered the same set of
messages.

\begin{Protocol}
Uses a negative acknowledgment (NAK) protocol: when messages are detected
to be out of order (or the $NumCast$ field in an \UpStable\ event detects
missing messages), a NAK is sent.  The NAK is sent in one of three ways,
chosen in the following order:
\begin{enumerate}
\item Pt2pt to the sender, if the sender is not failed.
\item Pt2pt to the coordinator, if the reciever is not the coordinator.
\item Broadcast to the rest of the group if the receiver is the
  coordinator.
\end{enumerate}
All broadcast messages are buffered until stable.
\end{Protocol}

\begin{Parameters}
\item mnak\_allow\_lost : boolean that determines whether the MNAK layer will
check for lost messages.  Lost messages are only possible when using an
inaccurate stability protocol.
\end{Parameters}

\begin{Properties}
\item
Requires stability and $NumCast$ information (equivalent to that provided
by the STABLE layer).
\item
\DnCast\ events cause \UpCast\ events to be delivered at all other members
(but not locally) in the group in FIFO order.
\item
\UpCast\{ack = Ack(rank,seqno)\} events from this layer require a later
\DnAck\{dn\_ack = Ack(rank,seqno)\} event to be delivered from above.
\item
\UpStable\ events from the layer below cause stable messages to be garbage
collected and may cause NAK messages to be sent to other members in the
group.
\end{Properties}

\begin{Sources}
\sourcesfile{mnak.ml}
\end{Sources}

\begin{GenEvent}
\genevent{\DnCast}
\genevent{\DnSend}
\end{GenEvent}

\begin{Testing}
\item
The CHK\_FIFO protocol layer checks for FIFO  safety conditions.
\item
The FIFO application generates bursty communication in which a token traces
its way through each burst.  If a reliable communication layer drops the
token, communication comes to an abrupt halt.  This is intended to capture
the liveness conditions of FIFO layers.
\item 
see also the VSYNC stack
\end{Testing}
\end{Layer}
