%*************************************************************%
%
%    Ensemble, 1.10
%    Copyright 2001 Cornell University, Hebrew University
%    All rights reserved.
%
%    See ensemble/doc/license.txt for further information.
%
%*************************************************************%
'\begin{Layer}{PT2PTWP} 

This layer implements an adaptive window-based flow control protocol 
for point-to-point communication between the group members. 

In this protocol the receiver's buffer space is shared between all group
members. This is accomplished by dividing the receiver's window among 
the senders according to the bandwidth of the data being received from each 
sender. Such way of sharing attempts to minimize the number of ack messages, 
i.e.\ to increase message efficiency.

\begin{Protocol}
In the following, the term acknowledgement is used with the meaning 
of flow control protocols and not that of reliable communication protocols.

This protocol uses \emph{credits} to measure the available buffer space 
at the receiver's side. Each sender maintains a \emph{window} per each 
destination, which is used to bound the unacknowledged data a process 
can send point-to-point to the given destination. For each message it 
sends, the process deducts a certain amount of credit based on the size 
of the message. Messages are transmitted only if the sender has enough 
credit for them. Otherwise, messages are buffered at the sender.

A receiver keeps track of the amount of unacknowledged data it has 
received from each sender. Whenever it decides to acknowledge a sender,
it sends a message containing new amount of credit for this sender.
On receipt of an acknowledgement message, sender recalculates the amount 
of credit for this receiver, and the buffered messages are sent based 
on the new credit.

The receiver measures the bandwidth of the data being received from each 
sender. It starts with zero bandwidth, and adjusts it periodically with 
timeout \emph{pt2ptwp\_sweep}. 

On receipt of a point-to-point message, the receiver checks if the sender 
has passed threshold of its window, i.e. if the amount of data in
point-to-point messages received from this sender since the last ack was
sent to it has exceeded a certain ratio, \emph{pt2ptwp\_ack\_thresh}, 
of the sender's window. If it is, an ack with some credit has to be sent 
to the sender. In order to adjust processes' windows according to their 
bandwidth, the receiver attempts to steal some credit from an appropriate 
process and add it to the sender's window. The receiver looks for a process 
with maximal \( \frac{window}{\sqrt{bandwidth}} \) ratio, decreases its window 
by certain amount of credit and increases the window of the sender appropriately. 
Then the receiver sends the sender ack with the new amount of credit. When the 
process from which the credit was stolen passes theshold of its new, smaller 
window, the receiver sends ack to it.
\end{Protocol}

\begin{Parameters}
\item pt2ptwp\_window : size of the receiver's window, reflects the receiver's 
buffer space.
\item pt2ptwp\_ack\_thresh : the ratio used by the receiver while deciding 
to acknowledge senders.
\item pt2ptwp\_min\_credit : minimal amount of credit each process must have.
\item pt2ptwp\_bw\_thresh : credit may be stolen for processes with greater 
bandwidth only.
\item pt2ptwp\_sweep : the timeout of periodical adjustment of bavdwidth.
\end{Parameters}

\begin{Properties}
\item
This protocol requires reliable point-to-point properties from underlying 
protocol layers.
\end{Properties}

\begin{Notes}
\item
As opposed to most of the \ensemble\ protocols, this protocol implements flow
control on bytes and not on messages. It only considers the data in the
application payload portion of the message (the \mlval{iov} field of the
event).
\end{Notes}

\begin{Sources}
\sourcesfile{pt2ptwp.ml}
\end{Sources}

\begin{Testing}
\item
Correctness and performance testing has been carried out.
\end{Testing}

\end{Layer}
