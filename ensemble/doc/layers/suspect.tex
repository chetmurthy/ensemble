\begin{Layer}{SUSPECT}

This layer regularly pings other members to check for suspected
failures.  Suspected failures are announce in a \Dn{Suspect} event to
the layer below (which will be bounced back up by a layer such as the
BOTTOM layer).

\begin{Protocol}
Simple pinging protocol.  Uses a sweep interval.  On each sweep, Ping
messages are broadcast unreliably to the entire group.  Also, the
number of sweep rounds since the last Ping was received from other
members is checked and if it exceed the \mlval{max\_idle} threshold
then a \Dn{Suspect} event is generated.  \hide{Suspicions are repeated
until a Ping message is received from the suspected member.}

To prevent a message storm when member's sweep timers expire, each
member adds an initial time-delta to its sweep timer. The deltas are
spread between zero and {\tt suspect\_spacing} based on member rank.
For example, if there are 10 members, and suspect\_spacing is set to 1
second, then the deltas for members zero through nine are: 0.0, 0.1,
.., 0.9.
\end{Protocol}

\begin{Parameters}
\item 
suspect\_sweep : how often to Ping other members and check for suspicions
\item
suspect\_max\_idle : number of unacknowledged Ping messages before generating
failure suspicions.
\item 
suspect\_spacing : the time-interval over which to spread periodic
sending of suspicions.
\end{Parameters}

\begin{Properties}
\item
Suspicions are no guarantee that an actual failure has occured, only a guess.
\end{Properties}

\begin{Notes}
\item None
\end{Notes}

\begin{Sources}
\sourcesfile{suspect.ml}
\end{Sources}

\begin{GenEvent}
\genevent{\Dn{Suspect}}
\genevent{\Dn{Cast}}
\genevent{\Dn{Timer}}
\end{GenEvent}

\begin{Testing}
\item see the VSYNC stack
\end{Testing}
\end{Layer}
