%*************************************************************%
%
%    Ensemble, 1.10
%    Copyright 2001 Cornell University, Hebrew University
%    All rights reserved.
%
%    See ensemble/doc/license.txt for further information.
%
%*************************************************************%
\begin{Layer}{REALKEYS}
\label{layer:realkeys}
This layer is part of the dWGL suite. Together with OptRekey is
implements the dWGL protocol. This layer's task is to actually
perform the instructions passed to it from OptRekey, generate and
pass securely all group subkeys, and finally the group key.

\begin{Protocol}
When a Rekey operation is performed a complex set of layers and
protocols is set into motion. Eventually, each group member receives a
new keygraph and a set of instructions describing how to merge its
partial keytree with the rest of the group keytrees to achieve
a unified group tree. The head of the keytree is the group key.

The instructions are implemented in several stages by the subleaders:
\begin{enumerate}
\item Choose new keys, and send them securely to peer subleaders
using secure channels.
\item Get new keys through secure channels. Disseminate these keys
by encrypting them with the top subtree key, and sending pt-2-pt to the leader.
\item When the leader gets all 2nd stage messages, it bundles them
into a single multicast and sends to the group. 
\item A member $p$ that receives the multicast, extracts the set of
keys it should know. Member $p$ creates an \mlval{ERekeyPrcl} event
with the new group key attached. The event it send down to PerfRekey
notifing it that the protocol is complete.
\end{enumerate}

\end{Protocol}

\begin{Properties}
\item Requires VSYNC properties.
\end{Properties}

\begin{Sources}
\sourcesfile{realkeys.ml}
\sourcesfile{type/tdefs.ml,mli}
\end{Sources}

\begin{GenEvent}
\genevent{\mlval{ESecureMsg}}
\genevent{\DnCast}
\genevent{\DnSend}
\end{GenEvent}

\begin{Testing}
\item 
The armadillo program (in the demo subdirectory) tests the security properties
of \ensemble.
\end{Testing}

\end{Layer}



