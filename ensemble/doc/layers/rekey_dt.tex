\begin{Layer}{REKEY\_DT}
\label{layer:rekey_dt}

This is the default rekeying layer. The basic 
data structure used is a tree of secure channels. This tree
changes every view-change, therefore the name of the layer. 
Dynamic Tree REKEY. 

The basic problem in obtaining efficient rekeying is the high cost of
constructing secure channels. A secure channel is established using a
two-way handshake using a Diffie-Hellman exchange.  At the time of
writing, a PentiumIII 500Mhz can perform one side of a Diffie-Hellman
exchange (using the OpenSSL cryptographic library) in 40 milliseconds.
This is a heavyweight operation. 

To discuss the set of channels in a group, we shall view it as a graph
where the nodes are group members, and the edges are secure channels
connecting them. The strategy employed by REKEY\_DT is to use a tree
graph. When a rekey request is made by a user, in some view $V$, the
leader multicasts a tree structure that uses, as much as possible, the
existing set of edges. 

For example, if the view is composed of several previous
components, then the leader attempts to merge together existing
key-trees. If a single member joins, then it is located as close to the
root as possible, for better tree-balancing. If a member leaves, then
the tree may, in the worst case, split into three pieces. The leader
fuses them together using (at most) 2 new secure channels. 

The leader chooses a new key and passes it to its
children. The key is passed recursively down the tree until it reaches
the leaves. The leaf nodes send acknowledgments back to the leader.

This protocol has very good performance. It is even possible, that a
rekey will not require any new secure-channels. For example, in case
of member leave, where the node was a tree-leaf. 

\begin{Protocol}
When a member layer gets an \ERekeyPrcl\ event, it sends a message to the
coordinator to start the rekeying process.  The coordinator checks if
the view is composed of a single tree-component. If not, it multicasts
a {\it Start} message. All members that are tree-roots, sends their
tree-structure to the leader. The leader merges the trees together,
and multicasts the group-tree. It then chooses a new key and sends it
down the tree. 

Once a member receives the new key is passes it down to PerfRekey
using an \mlval{ERekeyPrcl} event.

The PerfRekey layer is responsible for collecting acknowledgments from the
members and performing a view change with the new key once
dissemination is complete. 
\end{Protocol}

\begin{Sources}
\sourcesfile{rekey\_dt.ml}
\end{Sources}

\begin{GenEvent}
\genevent{\DnCast}
\genevent{\DnSend}
\end{GenEvent}

\begin{Testing}
\item 
The armadillo.ml file in the demo directory tests the security properties
of \ensemble.
\end{Testing}

\end{Layer}


