\begin{Layer}{SLANDER}

This protocol is used to share suspicions between members of a partition.  This
way, if one member suspects another member of being faulty, the coordinator is
informed so that the faulty member is removed, even if the coordinator does not
detect the failure.  This ensures that partitions will occur even in the case
of asymmetric network failures.  Without the protocol, only when the
coordinator notices the faulty member will the member be removed.

\begin{Protocol}
The protocol works by broadcasting slander messages to other members whenever
it recieves a new Suspect event.  On the receipt of such a message, DnSuspect
events are generated.
\end{Protocol}

\begin{Parameters}
\item None
\end{Parameters} 

\begin{Properties}
\item
If any member suspects another member of being faulty, all members will
eventually suspect that member.
\item
\Up{Suspect} events may cause a Slander message to be generated.
\end{Properties}

\begin{Sources}
\sourcesfile{slander.ml}
\end{Sources}

\begin{GenEvent}
\genevent{\Dn{Suspect}}
\end{GenEvent}

\begin{Testing}
\item
see also the VSYNC stack
\end{Testing}

\emph{This layer and its documentation were written by Zhen Xiao.}
\end{Layer}
