\begin{Layer}{PERFREKEY}
\label{layer:perfrekey}
This layer is responsible for common management tasks related to group
rekeying. Above PerfRekey, a rekeying layer is situated. At the time
of writing there are four options: Rekey, RealKeys+OptRekey, Rekey\_dt,
and Rekey\_diam. The Rekey layer implements a very simple rekeying
protocol, RealKeys and OptRekey layers together implement the dWGL
protocol. Rekey\_dt implements a dynamic-tree based protocol, and
Rekey\_diam uses a diamond-like graph. 

\begin{Protocol}
The layer comes into effect when a Rekey operation is
initiated by the user. It is bounced by the Bottom layer as a \mlval{Rekey}
event and received at PerfRekey. From this point, following protocol is used: 

\begin{enumerate}
\item The \mlval{Rekey} action is diverted to the leader.
\item The leader initiates the rekey sequence by 
 passing the request up to Rekey/OptRekey/Rekey\_dt/Rekey\_diam.
\item Once rekeying is done, the members pass a {\bf RekeyPrcl} event with 
    the new group key back down. 
\item PerfRekey logs the new group key. A tree spanning the group 
    is computed through which acks will propagate. The leaves 
    sends Acks up the tree.
\item When Acks from all the children are received at the leader,
    it prompts the group for a view change. 
\end{enumerate}

In the upcoming view, the new key will be installed. 

Another rekeying flavor includes a {\it Cleanup} stage. Every couple
of hours, the set of cached secure channels, and other key-ing
material should be removed. This prevents an adversary from using
cryptanalysis to break the set of symmetric keys in use by the
system. To this end, PerfRekey supports an optional cleanup stage
prior to actual rekeying.  This is a sub-protocol that works as follows:
\begin{enumerate}
\item The leader multicasts a {\it Cleanup} message.
\item All members remove all their cached key-material from all
security layers. A \mlval{ERekeyCleanup} event is sent down to
Secchan, bounced up to Rekey/OptRekey+RealKeys/.., and bounced back down
to PerfRekey.
\item All members send {\it CleanupOk} to the leader through the Ack-tree.
\item When the leader receives {\it CleanupOk} from all the members, it starts
the Rekey protocol itself.
\end{enumerate}

By default, cleanup is perform every 24hours. This is a settable
parameter that the application can decide upon. 

Rekeying may fail due to member failure or due to a merge that
occurs during the execution. In this case, the new key is discarded
and the old key is kept. PerfRekey supports persistent rekeying: when
the 24hour timeout is over, a rekey will ensue no-matter how many
failures occur. 

The Top layer checks that all members in a view a trusted. Any
untrusted member is removed from the group through a \mlval{Suspicion}
event. Trust is established using the Exchange layer, and the user
access control policy.
\end{Protocol}

\begin{Properties}
\item Requires VSYNC properties.
\item
Guarantees during a view change, either all members switch to the new
shared key or none of them do.
\end{Properties}

\begin{Parameters}
\item {perfrekey\_ack\_tree\_degree:} determines the degree of the Ack
tree. The default is 6.
\item {perfrekey\_sweep:} determines the compulsory cleanup period. 
\end{Parameters}

\begin{Sources}
\sourcesfile{perfrekey.ml}
\end{Sources}

\begin{GenEvent}
\genevent{\mlval{EPrompt}}
\genevent{\mlval{ERekeyPrcl}}
\genevent{\Dn{Cast}}
\genevent{\Dn{Send}}
\end{GenEvent}

\begin{Testing}
\item 
The armadillo program (in the demo subdirectory) tests the security properties
of \ensemble.
\end{Testing}

\end{Layer}



