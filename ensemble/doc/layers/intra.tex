\begin{Layer}{INTRA}

This layer manages group membership within a view (see also the INTER layer).
There are three related tasks here:
\begin{itemize}
\item
Forwarding of group membership events to the rest of the group
(without INTRA, normally \Dn{View} and \Dn{Fail} events have only local
effect).
\item 
Filtering of group membership events from remote members (for
example, when two other group members think they are the coordinator
and fail each other, the INTRA layer choose one of them and ignores
the other member).
\item
Determining the view\_id of the following view.
\end{itemize}

\begin{Protocol}
This is a relatively simple group membership protocol.  We have done our
best to resist the temptation to ``optimize'' special cases under which the
group is ``unnecessarily'' partitioned.  We also constrain the conditions
under which operations such as merges can occur.  The implementation does
not ``touch'' any data messages: it only handles group membership changes.
Furthermore, this protocol does not use any timeouts.

Views and failures are forwarded via broadcast to the rest of the members.
Other members accept the view/failure if they are consistent with their
current representation of the group's state.  Otherwise, the view/failure
message is dropped and the sender is suspected of being problematic. 
\end{Protocol}

\begin{Parameters}
\item None
\end{Parameters}

\begin{Properties}
\item
\Dn{View} events are passed on to the layer below.  They also cause a
View message to be broadcast to the other members.  On receipt of this
View message, the other members either accept it (and deliver a
\Dn{View} event to layer below) or mark the sender of the View as
problematic, and possibly deliver a \Dn{Suspect} event to the layer
below.
\item
Requires FIFO, atomic broadcast delivery from layers below.
\item
\Dn{Fail} events are passed on to the layer below.  They also cause a Fail
message to be broadcast to the other members.  On receipt of this Fail
message, the other INTRA instances will either accept it (and deliver a
\Dn{Fail} event to the layer beneath them) or mark the sender of the Fail
message as problematic, and possibly deliver an \Dn{Suspect} event to the
layer below.
\item
View and Fail messages from a particular coordinator are delivered in FIFO
order to the members.
\item
Not all members may see same set of \Up{Fail} events.  However, the set of
failed members grows monotonicly with each failure notification.
\end{Properties}

\begin{Sources}
\sourcesfile{intra.ml}
\end{Sources}

\begin{GenEvent}
\genevent{\Dn{Cast}}
\genevent{\Dn{Fail}}
\genevent{\Dn{Suspect}}
\genevent{\Dn{View}}
\end{GenEvent}

\begin{Testing}
\item
see the VSYNC stack
\end{Testing}
\end{Layer}

