\begin{Layer}{REKEY}

This layers switches the group key upon request. There may be several 
reasons for switching the key:
\begin{itemize}
\item The key's lifetime has expired --- it is now possible that some
dedicated attacker has cracked it.
\item The key has been compromised. 
\item Application authorization policies have changed and previously trusted
members need to be excluded from the group. 
\end{itemize}

This layer also relies on the Secchan layer to create secure
channels when required. A secure channel is essentially a way
to pass confidential information between two endpoints. The Secchan
layer creates secure channels upon demand and caches them for future
use. This allows the new group key to be disseminated efficiently and
confidentially through the tree. 

\begin{Protocol}
When a member layer gets an \ERekeyPrcl\ event, it sends a message to the
coordinator to start the rekeying process.  The coordinator generates
a new key and sends it to its children using secure channels. The
children pass it down the tree. Once a member receives the new key is
passes it down to PerfRekey using an \mlval{ERekeyPrcl} event.

The PerfRekey layer is responsible for collecting acknowledgments from the
members and performing a view change with the new key once
dissemination is complete. 
\end{Protocol}

\begin{Parameters}
\item {rekey\_degree:} The degree of the dissemination tree. By
default it is 2.
\end{Parameters}

\begin{Properties}
\item
Guarantees during a view change, either all members switch to the new
shared key or none of them do.
\end{Properties}

\begin{Sources}
\sourcesfile{rekey.ml}
\end{Sources}

\begin{GenEvent}
\genevent{\Dn{Cast}}
\genevent{\Dn{Send}}
\end{GenEvent}

\begin{Testing}
\item 
The armadillo.ml file in the demo directory tests the security properties
of \ensemble.
\end{Testing}

\emph{This layer was originally written by Mark Hadyen with Zhen Xiao.
Ohad Rodeh later rewrote the security layers and related infrastructure.}
\end{Layer}


