\begin{Layer}{INTER}

This protocol handles view changes that involve more than one
partition (see also INTRA).

\begin{Protocol}
Group merges are the more complicated part of the group membership
protocol.  However, we constrain the problem so that:
\begin{itemize}
\item
Groups cannot be both merging and accepting mergers at the same time.
This eliminates the potential for cycles in the ``merge-graph.''
\item
A view (i.e. view\_id) can only attempt to merge once, and only if no
failures have occured.  Each merge attempt is therefore uniquely
identified by the view\_id of the merging group.  Note also that by
requiring no failures to have occured for a merge to happen, this
prevents a member from being failed in one view and then reappearing
in the next view.  There has to be an intermediate view without the
failed member: this is a desirable property.
\end{itemize}
The merge protocol works as follows:
\begin{enumerate}
\item
The merging coordinator blocks its group,
\item
The merging coordinator sends a merge request to the remote group's
coordinator.
\item
The remote coordinator blocks its group,
\item
The remote coordinator installs a new view (with the mergers in it) and
sends the view to the merging coordinator (through a merge-granted
message).
\item
The merging coordinator installs the view in its group.
\end{enumerate}
If the merging coordinator times out on the merged coordinator then it
immediately installs a new view in its partition (without the other members
even finding out about the merge attempt).
\end{Protocol}

\begin{Parameters}
\item None
\end{Parameters}

\begin{Properties}
\item
When another partition is merging, a View message is also sent to the coordinator of
the merging group, which then forwards the message to the rest of its group.
\item 
Requires that \DnMerge\ events only be delivered by the original coordinator of views
(in which no failures have yet occured).  Otherwise, the partition should first form
a new view and then attempt the merge.
\item
\DnMerge\ causes a \DnMerge\ event to be delivered to the layer below.  This
will be replied with either an \UpView, \UpMergeFailed, or \UpMergeDenied\
event, depending on the outcome of the merge attempt.
\item
\UpMergeRequest's are only delivered at the coordinator.  And only if the
group is not currently blocking and only if the mergers list does not
contain members that are/were in this view or in previous merge requests in
this view.
\end{Properties}

\begin{Sources}
\sourcesfile{inter.ml}
\end{Sources}

\begin{GenEvent}
\genevent{\DnMerge}
\genevent{\DnMergeDenied}
\genevent{\DnSuspect}
\end{GenEvent}

\begin{Testing}
\item
see the VSYNC stack
\end{Testing}
\end{Layer}

