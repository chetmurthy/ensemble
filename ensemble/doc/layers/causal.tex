\begin{Layer}{CAUSAL}

The CAUSAL layer implements causally order multicast.  It assumes
reliable, FIFO ordered reliable messaging from layers below.

\begin{Protocol}
The protocol has two versions: full and compressed vectors.  First,
we explain the simple version which uses full vectors. Then, we
explain how these vectors are compressed.

Each outgoing message is appended with a \emph{causal vector}. This
vector contains the last causally delivered message from each member
in the group.  Each received message is checked for deliverability.
It may be delivered only if all messages which it causally follows,
according to its causal vector, have been delivered.  If it is not
yet deliverable, it is delayed in the layer until delivery is
possible.  A view change erases all delayed messages, since they can
never become deliverable.

Causal vectors become large with the group size, so they must be
compressed in order for this protocol to scale.  The compression we
use is derived from the Transis system.  We demonstrate with an
example: assume the membership includes three processes $p,q$ and
$r$. Process $p$ sends message $m_{p,1}$, $q$ sends $m_{q,1}$,
causally following $m_{p,1}$ and $r$ sends $m_{r,1}$ causally
following $m_{q,1}$. The causal vector for $m_{r,1}$ is
$[1|1|1]$. There is redundancy in the causal vector since it is clear
that $m_{r,1}$ follows $m_{r,0}$. Furthermore, since $m_{q,1}$
follows $m_{p,1}$ we may omit stating that $m_{r,1}$ follows
$m_{p,1}$. To conclude, it suffices to state that $m_{r,1}$ follows
$m_{q,1}$.  Using such optimizations causal vectors may be compressed
considerably.
\end{Protocol}

\begin{Sources}
\sourcesfile{causal.ml}
\end{Sources}

\begin{Testing}
\item
The CHK\_CAUSAL protocol layer checks for CAUSAL delivery.
\end{Testing}

\emph{This layer and its documentation were written by Ohad Rodeh.}
\end{Layer}
