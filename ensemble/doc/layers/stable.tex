%*************************************************************%
%
%    Ensemble, 2_00
%    Copyright 2004 Cornell University, Hebrew University
%           IBM Israel Science and Technology
%    All rights reserved.
%
%    See ensemble/doc/license.txt for further information.
%
%*************************************************************%
\begin{Layer}{STABLE}

This layer tracks the stability of broadcast messages and does
failure detection.  It keeps track of and gossips about an
acknowledgement matrix, from which it occasionally computes the
number of messages from each member that are stable and delivers this
information in an \Dn{Stable} event to the layer below (which will be
bounced back up by a layer such as the BOTTOM layer).

\begin{Protocol}
The stability protocol consists of each member keeping track of its
view of an acknowledgement matrix.  In this matrix, each entry,
(A,B), corresponds to the number of member B's messages member A has
acknowledged (the diagonal entries, (A,A), contain the number of
broadcast messages sent by member A).  The minimum of column A
(disregarding entries for failed members) is the number of broadcast
messages from A that are stable.  The vector of these minimums is
called the stability vector.  The maximum of column A (disregarding
entries of failed members) is the number of broadcast messages member
A has sent that are held by at least one live member.  The vector of
the maximums is called the $NumCast$ vector \note{there has got to be
a better name}.  Occasionally, each member gossips its row to the
other members in the group.  Occasionally, the protocol layer
recomputes the stability and $NumCast$ vectors and delivers them up
in an \Dn{Stable} event.

To prevent a message storm when members gossip their stability
vectors, each member adds an initial time-delta to its timer. The
deltas are spread between zero and {\tt stable\_spacing} based on
member rank.  For example, if there are 10 members, and
suspect\_spacing is set to 1 second, then the deltas for members zero
through nine are: 0.0, 0.1, .., 0.9. 
\end{Protocol}

\begin{Parameters}
\item 
stable\_sweep: how often to (1) gossip and (2) deliver stability (if
it has changed)
\item 
stable\_explicit\_ack: whether to request end-to-end acknowledgements
for messages
\item
stable\_spacing : the time-interval over which to spread periodic
sending of stability vectors.
\end{Parameters}

\begin{Properties}
\item
Unless it is marked with the \mlval{Unreliable} option all
DnCast events are counted by the STABLE layer and require eventual
acknowledgement by the other members in the group in order to achieve
stability.
\item
\Dn{Stable} events from the stability layer have two extension fields
set.  The first is the \mlval{StableVect} extension, which is the
vector of stability number of messages from each of the members in
the group which are known to be stable.  The second is the
\mlval{NumCast} extension which is a vector with the number of
broadcast messages each member in the group is known to have sent.
\item
\Dn{Stable} events are never delivered before all live members have
acknowledged at least the number of messages noted in the stability
event. (safety)
\item
\Dn{Stable} event will eventually be delivered after live members have
acknowledged message \mlval{seqno} from member A, where the entry in
the stable vector for member A is at least \mlval{seqno+1}. (liveness)
\item
The stability vectors in \Dn{Stable} events from the STABLE layer are
monotonically increasing.
\end{Properties}

\begin{Notes}
\item
\mlval{NumCast} entries are not monotonicly increasing.  For example,
consider the case of member A broadcasting some messages (which are
all dropped by the network), then broadcasting its gossip information
(which are recieved), then failing.  The other members may deliver
some UpStable events with the number of known broadcasts from member
A, in which the dropped broadcasts are counted.  However, after the
other members detect member A's failure, the \mlval{NumCast} entry
for member A will be lowered to be the number of messages from A that
the live members have recieved, which will be lower than when A was
not failed.
\item
\Up{Cast} events do not need to be acknowledged individually: an
acknowledgment, \mlval{Ack(from,seqno)}, is taken to acknowledge all
of the first \mlval{seqno} messages from the member with rank
\mlval{from}.
\item
An attempt has been made to speed up stability detection during view
changes by sending extra gossip messages when failures have occurred.
\end{Notes}

\begin{Sources}
\sourcesfile{stable.ml}
\end{Sources}

\begin{GenEvent}
\genevent{\Up{Stable}}
\genevent{\Dn{Cast}}
\genevent{\Dn{Timer}}
\end{GenEvent}

\begin{Testing}
\item see the VSYNC stack
\end{Testing}
\end{Layer}
