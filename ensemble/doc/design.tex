\section{\ensemble\ Design Goals}

\item
Achieve modularity of protocols by separation of intra-stack communication
(which uses events) from inter-stack communication (which uses headers
pushed onto messages).  Protocol layers should communicate in the same
protocol stack only by passing/receiving events directly to/from adjacent
layers.  Protocol layers should communicate with other protocol stacks only
by use of headers on messages which are sent to the equivalent protocol
layer in the remote protocol stacks.
\item
Use simple model of execution for protocol layers.  This model should be
simple in order to allow flexibility in the design of the infrastructure
surrounding it.
\item
Attempt to make protocol layers configurable in different stacks with
different configurations of other protocol layers.
\item
Separation of messages and membership parts of protocols.  For the
membership protocols, this does not apply.  For the non-membership
protocols, this means attempting to isolate aspects of the protocols that
deal with membership from those that do not, so that they have the
appearance of being two layers right next to each other.
\item
Avoid bouncing of events.  Events should in general traverse the protocol
stack without interruption.  Membership-related events should be only
bounced off of the bottom of the protocol stack and actions taken as a
result of membership changes should only be made on the bounced up event.
