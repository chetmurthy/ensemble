\documentclass[12pt]{article}
\usepackage{alltt}

\newsavebox{\figurebox}

\newenvironment{codebox}{
\begin{alltt}
\sbox{\figurebox}\bgroup
\begin{minipage}{\hsize}
}{
\end{minipage}
\egroup
\fbox{\usebox{\figurebox}}
\end{alltt}
}

\begin{document}


\begin{figure}
\begin{codebox}
(*
 * UP\_TYPE: These are the enumerated types for up events.
 * The default value for the types of up events is
 * UpInvalid.  It is usually an error to pass around
 * UpInvalid events. 
 *)

type up\_type =
 | UpBlock				(* Prepare to block for new view *)
 | UpBlockOk				(* Done blocking group *)
 | UpCast				(* Received a broadcast message *)
 | UpDump				(* Dump your state (debugging) *)
 | UpElect				(* I am now the coordinator *)
 | UpExit				(* Last up event delivered *)
 | UpFail				(* Failure notification *)
 | UpInit				(* First up event delivered *)
 | UpInvalid				(* Erroneous up event type *)
 | UpLeave				(* A member wants to leave *)
 | UpLostMessage			(* Member doesn't have a message *)
 | UpMergeDenied			(* Merge request denied *)
 | UpMergeFailed			(* Merge request failed *)
 | UpMergeGranted			(* Merge granted *)
 | UpMergeRequest			(* Other group wants to merge *)
 | UpOrphan				(* Message was orphaned *)
 | UpSend				(* Received pt2pt message *)
 | UpStable				(* Stability information *)
 | UpSuspect				(* Member is suspected to be faulty *)
 | UpSystemError			(* Something serious has happened *)
 | UpTimer				(* A timer has expired *)
 | UpView				(* A new view is ready *)
;;
\end{codebox}
\caption{The $n^2$ approach}
\label{fig:up_type}
\end{figure}
See Figure~\ref{fig:up_type} (on page~\pageref{fig:up_type}) for
further details.

\end{document}
