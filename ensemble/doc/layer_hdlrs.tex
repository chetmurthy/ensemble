%*************************************************************%
%
%    Ensemble, 2_00
%    Copyright 2004 Cornell University, Hebrew University
%           IBM Israel Science and Technology
%    All rights reserved.
%
%    See ensemble/doc/license.txt for further information.
%
%*************************************************************%
\section{Event Handlers: Standard}

Logically, a protocol has two incoming event handlers (one each above and
below) and two outgoing event handlers (one each above and below).  In
practice, because some events have messages and others do not, these
handlers are split up into several extra handlers.  The breakdown of the 4
logical handlers into 10 actual handlers is done for compatibility with the
ML typechecker.  Typechecking is used extensively to guarantee that layers
recieve messages of the same type they send.  This is a very useful
property because it prevents a large class of programming errors.

\begin{table}[tb]
\begin{center}
\begin{tabular}{|l|c|c|c|c|c|} \hline
name	& in/	& up/	& above/	& message?	& header? \\
	& out	& dn	& below		&		& 	\\ \hline \hline
upnm	& out 	& up	& above		& no		& no	\\ \hline
up	& out	& up	& above		& yes		& no	\\ \hline
dnnm	& out	& dn	& below		& no		& no	\\ \hline
dnlm	& out	& dn	& below 	& no		& yes	\\ \hline
dn	& out	& dn	& below		& yes		& yes	\\ \hline \hline
upnm	& in 	& up	& below		& no		& no	\\ \hline
uplm	& in	& up	& below		& no		& yes	\\ \hline
up	& in	& up	& below		& yes		& yes	\\ \hline
dnnm	& in 	& dn 	& above		& no		& no	\\ \hline
dn	& in	& dn	& above		& yes		& no	\\ \hline
\end{tabular}
\end{center}
\caption{\em The 10 standard event handlers.}
\label{table:handlers}
\end{table}

\begin{figure}[tb]
\begin{center}
\resizebox{!}{4in}{\incgraphics{handlers}}
%\resizebox{!}{4in}{\includegraphics*[0,0][500,700]{fig/handlers}}
\end{center}
\caption{\em Diagram of the 10 standard event handlers.  Note that the
ABOVE layer has a similar interface above it as the BELOW layer.  Likewise
with the interface beneath the BELOW layer.}
\label{fig:handlers}
\end{figure}

In the standard configuration, each layer has 10 handlers.  A handler is
uniquely specified by a set of characteristics: whether it is an incoming
or outgoing handler, a handler for up events or down events, a handler for
communication with the layer above or for the layer below, whether it has
an associated message, and whether it has an associated header.  See
table~\ref{table:handlers} for a enumeration of the 10 handlers.  Of the 10
handlers, 5 are outgoing and 5 are incoming; 5 are up event handlers and 5
are down event handlers; 4 are for event communication with the layer below
and 6 are for event communication with the layer above.  These are depicted
in fig~\ref{fig:handlers}.

The names of the handlers have two parts.  The first specifies the sort of
event the handler is called with (``up'' or ``dn'').  The second specifies
the sort of message that is associated with the event and may be either
``'' (nothing, the default case), ``lm'' (for local message), or ``nm''
(for no message), which correspond to:
\begin{description}
\item
[nothing:] Events with associated messages, where the message was created
by a layer above this layer.  This layer was not the first layer to push a
header onto the message and will not be the last layer to pop its header
off the message.
\item
[``lm'':] Events with associated messeges, where the
message was created by this layer.  This was the first layer to push a
header onto the message and is the last layer to pop its header off of the
message.
\item
[``nm'':] Corresponds to events without associated messages.
These handlers always take a single argument which is either an up event or
a down event.
\end{description}
