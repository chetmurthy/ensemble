%*************************************************************%
%
%    Ensemble, 1.10
%    Copyright 2001 Cornell University, Hebrew University
%    All rights reserved.
%
%    See ensemble/doc/license.txt for further information.
%
%*************************************************************%
\section{The Programs}

Notes:
\begin{itemize}
\item
please note that warning and error messages printed by Ensemble are
not prefixed with the name of the program generating the message, but
rather the name of the module.
\end{itemize}

\subsection{Mtalk: Multi-person Talk}
This is a multi-person talk demo.  As \mlval{mtalk} processes are created, they merge
into a single group.  Input typed at one process is broadcast to the rest of the
processes in the group.

\subsection{Wbml: Distributed Whiteboard}
This is a graphical white board demo.  It uses the CamlTk library to implement a
graphical user interface\footnote{not supported under Windows/NT}.  When members are
in the same group, lines drawn on one instance are broadcast to the rest of the
group, who also draw the lines.  It supports the switching of protocols.  Initially,
wbml has an auto-merge protocol in its stack so the members merge together.  This can
be removed to disable partition healing.  Adding the XFER protocol, causes members to
transfer their state on view changes; the TOTAL protocol enable totally ordered
communication.  Initially, these extra protocols are not included in the protocol
stack.

\subsection{Ensemble: Text-based Interface}
\label{section:ensemble-demo}
This program provides a text-based interface to the Ensemble group
communication facilities.  You can run it to directly see what happens in an
Ensemble application.  You start up the program and it prints out messages
decribing changes to the membership of the group.  You can type in commands
such as ``cast hello'' which causes Ensemble to broadcast ``hello'' to the
other members of the group, who get ``cast hello'' printed out.  This program
can be used as a subprocess of an application for doing basic group
communication.  The normal usage is to set up pipes to and from the standard
input and output of the \mlval{ensemble} process.

In order to distinguish different applications that are using this interface to
communicate, you may wish to use the \mlval{-group\_name} option to set the
name of the group.

The input of the program must be formatted in text lines as follows:
\begin{itemize}
\item
{[cast \mlval{msg}]} broadcasts the following message to the group, where
\mlval{msg} is the remainder of the input line.  (Normally, the broadcaster
will not receive the message).
\item
{[send \mlval{dest} \mlval{msg}]} sends a point-to-point message to the rest of the
group.  \mlval{dest} is the endpoint identifier of the group member you wish to send
the message to.  \mlval{msg} is the remainder of the input line.
\item
{[leave]} causes the member to leave the group.  This will eventually result in
an exit message being output and then the \mlval{ensemble} process will exit.
\end{itemize}

The output of the program consists of lines in one of the following formats:
\begin{itemize}
\item
{[endpt \mlval{endpoint\_id}]} is output as the first line and only appears once.  It
gives the name of this application as it will appear in views.
\item
{[view \mlval{nmembers} \mlval{my\_rank} \mlval{view}]} describes a new view of the
group.  Initially, every member begins in its own singleton view.  Other members are
added through automatic merging with other views.  Members are removed through
failure detections.  \mlval{nmembers} is the number of members of the group.
\mlval{my\_rank} is this member's rank in the new view.  \mlval{protocol} is the name
of the protocol being used.  \mlval{view} is a space-separated list of the endpoint
identifiers of the members of the group.
\item 
{[cast \mlval{origin} \mlval{msg}]} displays a broadcast received from the member of
rank \mlval{origin}.
\item
{[send \mlval{origin} \mlval{msg}]} displays a point-to-point message received
from member of rank \mlval{origin}.
\item
{[exit]} notifies that this member has left the group as a result of a previous
\mlval{leave}.  This is the last line output by \mlval{ensemble}.
\end{itemize}

\subsection{Gossip: Group Locator Service}
This is not really an application.  The gossip server works in conjunction with
the Ensemble \mlval{UDP} communication transport to simulate low-bandwidth
gossip broadcast for systems that do not have IP multicast.  See the discussion
on transports below.  The group communication protocols require some
``gossipping'' mechanism in order to detect and heal partitions in the system.
When an application wishes to gossip with other partitions, it broadcasts a
message via the \mlval{gossip} servers.  This sends messages to the
\mlval{gossip} servers.  The \mlval{gossip} servers then forward the message to
all processes they have heard from recently to simulate a broadcast.  When an
application is using the \mlval{UDP} transport and not the \mlval{DEERING}
transport (\mlval{UDP} is the default), it is necessary for a \mlval{gossip}
process to be running somewhere in the system.

\subsection{Groupd: Membership Service (formerly called Domain)}
\label{section:groupd}
Normally, \ensemble\ application groups implement their own group membership
protocol.  However, they have the option of using the \ensemble\ membership
service implemented by the \mlval{groupd} application.  \mlval{groupd} is a
service for managing multiple process groups.  It uses a \emph{core} group of
\ensemble\ processes to participate in managing these groups.  Clients connect
to the service via TCP connections, through which they request to join and
leave groups.  The service supports a simple protocol through which the clients
can obtain virtual synchronous properties.  The service also supports weaker
properties that give faster membership notifications.

\note{We emphasize that Ensemble applications can operate independently of a
membership service.}

Some of the benefits of using this service are:
\begin{itemize}
\item
When there are no membership changes, the clients communicate directly between
themselves, so the membership service has no affect on performance.
\item
The service implements group membership for multiple groups.  The costs of the
group membership protocols (such as failure detection) are shared over the
groups.
\item
Because applications are sharing the same membership service, they see
consistent views and failure detections.
\item
The client part of the protocol for implementing virtual synchrony is simple.
Most of the complexity is in the server.  This allows client programs to be
implemented in languages other than ML, but save much of the programming burden
because the servers handle the ``hard'' group membership protocols.  The client
TCP interface is described in the \ensemble\ reference manual.
\item
Applications that do not need the full virtual synchrony properties can use
weaker synchronization protocols and get faster view changes.
\item
The service allows groups to scale to larger sizes.  The membership servers do
not need to run on all the hosts on which the clients run, so clients can be on
more hosts than are normally supported by \ensemble.
\end{itemize}

\paragraph{Executing Groupd:}
In order to run Groupd, set the \mlval{ENS\_GROUPD\_PORT} environment
variable to select the TCP port for the service to use.  The membership service
is executed through the \mlval{groupd} application program:
\begin{verbatim}
% groupd
\end{verbatim}
It takes command-line arguments similar to the other \ensemble\ demonstration
programs.  Normally, each host runs a server.

Other demo applications use the service when the \mlval{-groupd} command-line
argument is selected.  For example:
\begin{verbatim}
% mtalk -groupd
\end{verbatim}
Note that you must have a \mlval{groupd} server running on the same host as
mtalk for this to work.


\subsection{Perf: Performance Tests}
This program includes a variety of performance tests for Ensemble.

\paragraph{Ring:}
This test is run with the \mlval{-prog ring} option.  Say that there are $n$
members.  Each process first waits until there are $n$ members.  It then sends
$k$ messages, and waits for $(n - 1)k$ messages from other members.  It
measures the time for this, and does so a number of times to determine the
average and variance.  This can be done for varying $n$, $k$, message size, and
protocol.

The time between the rounds is a measure of latency.  The total amount of data
sent between the rounds is a measure of bandwidth.  The total number of
messages sent between rounds is a measure of throughput.  For good
measurements, set the parameters as follows:

\begin{center}
\begin{tabular}{|l|c|c|}			   \hline
measure		& $k$		& message size	\\ \hline \hline
latency		&  1		& 0		\\ \hline
throughput	& large		& 0		\\ \hline
bandwidth	&  1		& large		\\ \hline
\end{tabular}
\end{center}

Additional command-line arguments (with default values in parentheses):
\begin{description}
\item [-n \#]: number of members ($2$ members)
\item [-s \#]: size in bytes of application messages ($0$ bytes)
\item [-r \#]: number of rounds ($300$ rounds)
\item [-k \#]: messages per round ($1$ message per round)
\end{description}
These values must be set by all members.  All members must use the same values for
all of the arguments except message size.

\todo{The other performance tests are undocumented.}

\subsection{Rand: Virtual Synchrony Debugging Tool}
This demo is used to test \ensemble.  It uses simulated communication and
introduces random process failures to check for proper behavior of the group
membership protocols.

\subsection{Fifo: Fifo Communication Debugging Tool}
This demo is used to test \ensemble.  It uses simulated communication
structured in such a way as to trigger bugs in FIFO, reliable communication
protocols.

\subsection{Armadillo: testing Ensemble security extensions}
This program tests \ensemble\ security features. It has several
command line options:
\begin{description}
\item[-n \#]  number of endpoints to create
\item[-t \#]  after what threshold to start the test
\item[-prog] which security to use? [policy,rekey,exchange,reg,prompt]
\item[-pa]   simulate partitions? 
\item[-net]  run everything in a single process or run throughout the  network
\item[-real\_pgp]  use PGP for authentication? otherwise, simulate it.
\item[-group]    set the group name
\end{description}

The ``exchange'' test checks that the Exchange layer functions
correctly. For example, running:
\begin{verbatim}
% armadillo -n 20 -prog exchange
\end{verbatim}
will create 20 endpoints with random intial keys. the endpoints should
merge into one group after a short while.

The ``rekey'' test creates a group and once its size is above the
threshold it start rekeying it. The test: {\tt Use: armadillo -n 7 -t
7 -prog rekey} will create a group of 7 members and once the group
reaches this size, will start to rekey it.
		
To see what happens when the group partitions use: {\tt armadillo -n 5
-t 3 -prog rekey -pa}. This will create a group of 5 members and start
partitioning and remerging the group. Everytime the membership in a
group component exceeds 3, the component leader will try rekeying it. 

The ``policy'' test checks that Ensemble respects application trust
policies. For example running:
\begin{verbatim}
% armadillo -n 7 -prog policy
\end{verbatim}
will create a static group of 7 processes, numbered 0 through 6, and
dynamically change the endpoints trust policies. Ensemble forms
subgroups according to the trust relationship. The policies are
designed to change in stages:
\begin{enumerate}
\item All endpoints trust each other.
\item 
All endpoints of the same (mod 2) trust each other. That is we
have to trust domains: $\{0,2,4,6\}$ and $\{1,3,5\}$.
\item 
All endpoints of the same (mod 3) trust each other. That is we
have three trust domains: $\{0,3,6\}$, $\{1,4\}$ $\{2,5\}$.
\end{enumerate}

The ``prompt'', and ``reg'' tests are auxillary tests not related
to security. 


\subsection{Life: Game of Life Demo}
This is a graphical version of the \emph{Game of Life} that was
originally ``invented'' by J.H.~Conway in 1969 (and was first
reported in \emph{Scientific American}, October 1970).  The Game of
Life is not actually a game: ``it is the study of phenomena which can
be observed in evolving configurations of populations.  One can think
of a population as a generation of living and non-living beings.  A
generation can be modeled by a rectangular grid of cells in which
each being occupies exactly one cell and each cell can be either on
or off.  If a being is alive, then the corresponding cell is on; if
the being is dead, then the cell is off.  From this point on we refer
to beings of a population and cells of the rectangular grid
interchangeably.''  In this implementation, each cell of the grid is
implemented by a separate endpoint and all communication is through
asynchronous Ensemble communication.  Anyway, this program requires
the CamlTk library and should be self-explanatory to run (note that
you only need to run one Life process: it creates multiple endpoints
within the process).

\emph{This application was written by Samuel Weber.}




