\section{The Programs}
Of the programs described here only mtalk, gossip, groupd, and
ensembled, are normal programs. The rest: perf, rand, fifo, socktest,
and armadillo, are internal system tests not for use by the casual user. 

Notes:
\begin{itemize}
\item
please note that warning and error messages printed by Ensemble are
not prefixed with the name of the program generating the message, but
rather the name of the module.
\end{itemize}

\subsection{Ensembled: the ensemble daemon}
This is the ensemble daemon. By running a single instance of this
program on a host all clients on that host will be able to use Ensemble services. 

\subsection{Mtalk: Multi-person Talk}
This is a multi-person talk demo.  As \mlval{mtalk} processes are created, they merge
into a single group.  Input typed at one process is broadcast to the rest of the
processes in the group.

\subsection{Gossip: Group Locator Service}
This is not really an application.  The gossip server works in
conjunction with the Ensemble \mlval{UDP} communication transport to
simulate low-bandwidth gossip broadcast for systems that do not have
IP multicast.  See the discussion on transports below.  The group
communication protocols require some ``gossipping'' mechanism in order
to detect and heal partitions in the system.  When an application
wishes to gossip with other partitions, it broadcasts a message via
the \mlval{gossip} servers.  This sends messages to the \mlval{gossip}
servers.  The \mlval{gossip} servers then forward the message to all
processes they have heard from recently to simulate a broadcast.  When
an application is using the \mlval{UDP} transport and not the
\mlval{DEERING} transport (\mlval{DEERING} is the default), it is
necessary for a \mlval{gossip} process to be running somewhere in the
system.

\subsection{Groupd: Membership Service (formerly called Domain)}
\label{section:groupd}
Normally, \ensemble\ application groups implement their own group
membership protocol.  However, they have the option of using the
\ensemble\ membership service implemented by the \mlval{groupd}
application.  \mlval{groupd} is a service for managing multiple
process groups.  It uses a \emph{core} group of \ensemble\ processes
to participate in managing these groups.  Clients connect to the
service via TCP connections, through which they request to join and
leave groups.  The service supports a simple protocol through which
the clients can obtain virtual synchronous properties.  The service
also supports weaker properties that give faster membership
notifications.

\note{We emphasize that Ensemble applications can operate independently of a
membership service.}

Some of the benefits of using this service are:
\begin{itemize}
\item
When there are no membership changes, the clients communicate directly
between themselves, so the membership service has no affect on
performance.
\item
The service implements group membership for multiple groups.  The
costs of the group membership protocols (such as failure detection)
are shared over the groups.
\item
Because applications are sharing the same membership service, they see
consistent views and failure detections.
\item
The client part of the protocol for implementing virtual synchrony is
simple.  Most of the complexity is in the server.  This allows client
programs to be implemented in languages other than ML, but save much
of the programming burden because the servers handle the ``hard''
group membership protocols.  The client TCP interface is described in
the \ensemble\ reference manual.
\item
Applications that do not need the full virtual synchrony properties
can use weaker synchronization protocols and get faster view changes.
\item
The service allows groups to scale to larger sizes.  The membership
servers do not need to run on all the hosts on which the clients run,
so clients can be on more hosts than are normally supported by
\ensemble.
\end{itemize}

\paragraph{Executing Groupd:}
In order to run Groupd, set the \mlval{ENS\_GROUPD\_PORT} environment
variable to select the TCP port for the service to use.  The
membership service is executed through the \mlval{groupd} application
program:
\begin{verbatim}
% groupd
\end{verbatim}
It takes command-line arguments similar to the other \ensemble\
demonstration programs.  Normally, each host runs a server.

Other demo applications use the service when the \mlval{-groupd}
command-line argument is selected.  For example:
\begin{verbatim}
% mtalk -groupd
\end{verbatim}
Note that you must have a \mlval{groupd} server running on the same host as
mtalk for this to work.


\subsection{Perf: Performance Tests}
This program includes a variety of performance tests for Ensemble.

\paragraph{Ring:}
This test is run with the \mlval{-prog ring} option.  Say that there
are $n$ members.  Each process first waits until there are $n$
members.  It then sends $k$ messages, and waits for $(n - 1)k$
messages from other members.  It measures the time for this, and does
so a number of times to determine the average and variance.  This can
be done for varying $n$, $k$, message size, and protocol.

The time between the rounds is a measure of latency.  The total amount
of data sent between the rounds is a measure of bandwidth.  The total
number of messages sent between rounds is a measure of throughput.
For good measurements, set the parameters as follows:

\begin{center}
\begin{tabular}{|l|c|c|}			   \hline
measure		& $k$		& message size	\\ \hline \hline
latency		&  1		& 0		\\ \hline
throughput	& large		& 0		\\ \hline
bandwidth	&  1		& large		\\ \hline
\end{tabular}
\end{center}

Additional command-line arguments (with default values in parentheses):
\begin{description}
\item [-n \#]: number of members ($2$ members)
\item [-s \#]: size in bytes of application messages ($0$ bytes)
\item [-r \#]: number of rounds ($300$ rounds)
\item [-k \#]: messages per round ($1$ message per round)
\end{description}
These values must be set by all members.  All members must use the
same values for all of the arguments except message size.

\todo{The other performance tests are undocumented.}

\subsection{Rand: Virtual Synchrony Debugging Tool}
This demo is used to test \ensemble.  It uses simulated communication and
introduces random process failures to check for proper behavior of the group
membership protocols.

\subsection{Fifo: Fifo Communication Debugging Tool}
This demo is used to test \ensemble.  It uses simulated communication
structured in such a way as to trigger bugs in FIFO, reliable communication
protocols.

\subsection{Armadillo: testing Ensemble security extensions}
This program tests \ensemble\ security features. It has several
command line options:
\begin{description}
\item[-n \#]  number of endpoints to create
\item[-t \#]  after what threshold to start the test
\item[-prog] which security to use? [policy,rekey,exchange,reg,prompt]
\item[-pa]   simulate partitions? 
\item[-net]  run everything in a single process or run throughout the  network
\item[-real\_pgp]  use PGP for authentication? otherwise, simulate it.
\item[-group]    set the group name
\end{description}

The ``exchange'' test checks that the Exchange layer functions
correctly. For example, running:
\begin{verbatim}
% armadillo -n 20 -prog exchange
\end{verbatim}
will create 20 endpoints with random intial keys. the endpoints should
merge into one group after a short while.

The ``rekey'' test creates a group and once its size is above the
threshold it start rekeying it. The test: {\tt Use: armadillo -n 7 -t
7 -prog rekey} will create a group of 7 members and once the group
reaches this size, will start to rekey it.
		
To see what happens when the group partitions use: {\tt armadillo -n 5
-t 3 -prog rekey -pa}. This will create a group of 5 members and start
partitioning and remerging the group. Everytime the membership in a
group component exceeds 3, the component leader will try rekeying it. 

The ``policy'' test checks that Ensemble respects application trust
policies. For example running:
\begin{verbatim}
% armadillo -n 7 -prog policy
\end{verbatim}
will create a static group of 7 processes, numbered 0 through 6, and
dynamically change the endpoints trust policies. Ensemble forms
subgroups according to the trust relationship. The policies are
designed to change in stages:
\begin{enumerate}
\item All endpoints trust each other.
\item 
All endpoints of the same (mod 2) trust each other. That is we
have to trust domains: $\{0,2,4,6\}$ and $\{1,3,5\}$.
\item 
All endpoints of the same (mod 3) trust each other. That is we
have three trust domains: $\{0,3,6\}$, $\{1,4\}$ $\{2,5\}$.
\end{enumerate}

The ``prompt'', and ``reg'' tests are auxillary tests not related
to security. 


\subsection{Socktest}
This is a simple application that tests the soundness of the
lower level Ensemble interface to sockets and IP-multicast.

The current menu is as follows:
\begin{itemize}
\item
{[Join \mlval{ipm\_addr} ]} Join a multicast group.
\item 
{[Leave \mlval{ipm\_addr} ]} Leave a multicast group.
\item 
{[Cast \mlval{ipm\_addr} \mlval{msg}]} Send a message to a multicast
group.
\item 
{[Ttl \mlval{num} ]} set the time-to-live.
\item 
{[Loopback \mlval{onoff} ]} set the loopack. If this is on, then
messages sent to a multicast group will also be locally received.
\item 
{[Sendbuf \mlval{size} ]} Set the send-buffer size in the kernel.
Normally, to little space is reserved in the kernel for storing IP
packets, therefore, it is common practice to increase it.
\item 
{[Recvbuf \mlval{size}]} 
Same, as Sendbuf, only for received packets.
\item 
{[Nonblock \mlval{onoff} ]} set a socket to blocking or non-blocking
mode.
\end{itemize}



