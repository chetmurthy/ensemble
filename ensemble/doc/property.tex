\subsection{Properties}
\label{sec:properties}

The \ensemble\ \mlval{Property} module is used to construct protocols based on
desired properties the application wants.  You can look at \sourceappl{property.mli}
for the various properties that are supported by \ensemble:
\begin{codebox}
type id =
  | Agree       (* agreed (safe) delivery *)
  | Gmp	        (* group-membership properties *)
  | Sync        (* view synchronization *)
  | Total       (* totally ordered messages *)
  | Heal        (* partition healing *)
  | Switch      (* protocol switching *)
  | Auth        (* authentication *)
  | Causal      (* causally ordered broadcasts *)
  | Subcast     (* subcast pt2pt messages *)
  | Frag        (* fragmentation-reassembly *)
  | Debug       (* adds debugging layers *)
  | Scale       (* scalability *)
  | Xfer        (* state transfer *)
  | Cltsvr      (* client-server management *)
  | Suspect     (* failure detection *)
  | Flow        (* flow control *)
  | Migrate     (* process migration *)
  | Privacy     (* encryption of application data *)
  | Rekey       (* support for rekeying the group *)
  | Primary     (* primary partition detection *)
  | Local       (* local delivery of messages *)
  | Slander     (* members share failure suspiciions *)
  | Asym        (* overcome asymmetry *)

    (* The following are not normally used.
     *)
  | Drop        (* randomized message dropping *)
  | Pbcast      (* Hack: just use pbcast prot. *)
  | Zbcast      (* Use Zbcast protocol. *)
  | Gcast       (* Use gcast protocol. *)
  | Dbg         (* on-line modification of network topology *)
  | Dbgbatch    (* batch mode network emulation *)
  | P_pt2ptwp   (* Use experimental pt2pt flow-control protocol *)
\end{codebox}

Here is a short description of some of the properties:
\begin{itemize}
\item {Gmp:} Group Membership Properties.
\item {Sync:} Synchronizes messages on view changes to ensure view synchrony.
\item {Total:} Broadcast messages are totally ordered in the group.
\item {Heal:} Group partitions are healed.
\item {Switch:} Allows on-the-fly protocol switching.
\item {Auth:} Allows only authenticated and authorized members into
the group. Creates secure agreement in the group on a mutual group
key. This key is used to sign and verify, using keyed-MD5, all group
messages. This protects the group from outisde attack. 
\item {Rekey:} Allows rekeying the group.  
\item {Privacy:} Encrypts all user messages. 
\item {Causal:} Broadcasts are causally ordered.
\item {Subcast:} Point-to-point messages are sent using filtered broadcasts.
Guarantees FIFO ordering between broadcasts and point-to-point messages.
\item {Frag:} Message fragmentation.  Allows messages of any size to be sent.
\item {Debug:} Inserts a variety of ``assertion'' protocols that check that
other properties are being met.
\item {Scale:} Switches some protocols with more scalable versions.
\item {Xfer:} Causes the state transfer field (\mlval{xfer}) of view states to
be set.
\item {Cltsvr:} Causes the clients field of view states to be set according to
whether members are ``clients'' or ``servers''.
\item {Suspect:} Members watch other members for suspected failures.
\item {Zbcast:} A probabilistic multicast protocol, does not guaranty
virtual syncrhony. Has been used for experimental studies. See the
Cornell Spinglass system for more details.
\item {Gcast:} A protocol that simulates IP-multicast useing a binary
tree of pt-2-pt connections between group members.
\end{itemize}

The \mlval{Property.choose} function selects a protocol stack based on a list
of desired properties (you can examine the implementation to see exactly how
this is done):
\begin{codebox}
(* Create protocol with desired properties.
 *)
val choose : id list -> Proto.id
\end{codebox}

The default properties used for \ensemble\ applications is \mlval{Property.vsync}.
This is one of a variety of predefined protocol property lists defined in the
\mlval{Property} module:
\begin{codebox}
let vsync = [Gmp;Sync;Heal;Migrate;Switch;Frag;Suspect;Flow]
let total = vsync @ [Total]
let scale = vsync @ [Scale]
let fifo = [Frag]
\end{codebox}
